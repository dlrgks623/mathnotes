\documentclass[11pt]{article}
\usepackage{ikany}
\usepackage[margin=1.2in]{geometry}

\let\realsection\section
\renewcommand\section{\newpage\realsection}

\begin{document}
\tableofcontents

\section{Fock space reading notes}
\subsection{Closed operators}
\begin{thm}
Let $A,B$ be closed operators between Banach spaces.
Then, $A+B$ is closed iff
\[\|Ax\|+\|Bx\|\les\|(A+B)x\|+\|x\|,\]
i.e. $A$ and $B$ are $A+B$-bounded.
\end{thm}
\begin{thm}
Let $A$ be a closed, and $B$ be a closable operator between Banach spaces with $D(A)\subset D(B)$.
Then, $A+B$ is closed if
\[\|Bx\|\les\|Ax\|+\|x\|,\]
i.e. $B$ is $A$-bounded.
Hmmmmmmmm
\end{thm}
Any idea to prove?

\begin{prop}
For $T\in D_{cl}(X,Y)$,
\[T\text{ is unbounded}\iff T\text{ is not everywhere defined}.\]
\end{prop}


\subsection{Decomposition of spectrum}
Note that decomposition of spectrum is usually for closed operators.
\begin{cond}
\item For closed operators, we do not have to consider cores.
\item We will not distinguish everywhere defined operators and densely defined operators.
\item We will not care whether its domain for unbounded densely defined operators. If closed, then unboundedness is equivalent to everywhere defined.
\end{cond}

Let $X=Y$ in order to see $L(X,Y)$ as a ring.
Let $L(X)\subset D(X)\subset B(X)$ be spaces of linear operators, densely defined operators, densely define bounded operators respectively.
For $T\in L(X)$,
\[\lambda\begin{cases}\text{is in }\rho(T)\\\text{is in }\sigma_c(T)\\\text{is in }\sigma_r(T)\\\text{is in }\sigma_p(T)\end{cases}\qquad\text{iff}\qquad R_\lambda(T)\begin{cases}\in B(X)\\\in D(X)\setminus B(X)\\\in L(X)\setminus D(X)\\\text{does not exist in }L(X)\end{cases}.\]

Discrete spectrum consists of scalars having finite dimensional eigenspace and is isolated from any other elements in spectrum.
\clearpage
\subsection{Adjoint}
For Banach spaces, we have
\[\adj:D(X,Y)\to cl(Y^*,X^*)\]
that is not injetcive. (I don't know it's surjective)

For reflexive $Y$, we have
\[\adj:D_{cl}(X,Y)\to D_{cl}(Y^*,X^*)\]
that is inj? surj?

For reflexive $X$, we have
\[\adj:D_{closable}(X)\ep D_{cl}(X^*).\]

\begin{thm}
The adjoint $B_{cl}(H)\to{\sim}B_{cl}(H)$ can be extended to $D_{cl}(H)\to{\sim}D_{cl}(H)$.
\end{thm}
\begin{thm}
For $T\in D_{cl}(H)$, $H=\ker T\dsum\cl{\im T^*}$.
\end{thm}
The space $D_{cl}$ is optimized when we think adjoints for reflexive spacse.

unitarily equivalence can defined for $T_1\in L(H_1)$ and $T_2\in L(H_2)$.












\section{Vector calculus on spherical coordinates}

\begin{tikzcd}[row sep=5pt, column sep=3pt]
V&=&(V_r,V_\theta,V_\phi)&&&&&&&&\\
&=&V_r&\hat{r}&+&V_\theta&\hat{\theta}&+&V_\phi&\hat{\phi}&(\text{normalized coords})\\
&=&V_r&\pd{r}&+&\frac1r\ V_\theta&\pd{\theta}&+&\frac1{r\sin\theta}\ V_\phi&\pd{\phi}&(\Gamma(TM))\\
&=&V_r&dr&+&r\ V_\theta&d\theta&+&r\sin\theta\ V_\phi&d\phi&(\Omega^1(M))\\
&=&r^2\sin\theta\ V_r&d\theta\wdg d\phi&+&r\sin\theta\ V_\theta&d\phi\wdg dr&+&r\ V_\phi&dr\wdg d\theta&(\Omega^2(M)).
\end{tikzcd}

\[
\del\cdot V=\frac1{r^2\sin\theta}\left[  \pd{r}\left(r^2\sin\theta\ V_r\right) + \pd{\theta}\left(r\sin\theta\ V_\theta\right) + \pd{\phi}\left(r\ V_\phi\right)  \right]
\]


\section{Statements in functional analysis and general topology}
Function analysis:
\begin{itemize}
\item Every seperable Banach space is linearly isomorphic and homeomorphic. But there are two non-isomorphic Banach spaces.
\item open mapping theorem -> continuous embedding is really an embedding.
\item $D(\Omega)$ is defined by a \emph{countable stict} inductive limit of $D_K(\Omega)$, $K\subset\Omega$ compact. Hence it is not metrizable by the Baire category theorem. (Here strict means that whenever $\alpha<\beta$ the induced topology by $\cT_\beta$ coincides with $\cT_\alpha$)
\item A net $(\phi_d)_d$ in $D(\Omega)$ converges if and only if there is a compact $K$ such that $\phi_d\in D_K(\Omega)$ for all $d$ and $\phi_d$ converges uniformly.
\item Th integration with a locally integrable function is a distribution. This kind of distribution is called \emph{regular}. The nonregular distribution such as $\delta$ is called \emph{singular}.
\item $D'$ is equipped with the weak$^*$ topology.
\item $\pd{x}\colon D'\to D'$ is continuous. They commute (Schwarz theorem holds).
\item $D\to S\to L^p$ are continuous (immersion) but not imply closed subspaces (embedding).
\end{itemize}
General topology:
\begin{itemize}
\item $H\subset\C$ and $H\subset\hat\C$ have distinct Cauchy structures which give a same topology. In addition, the latter is precompact while the former is not.
\end{itemize}



\section{Algebraic closure}
\begin{thm}
Every field has an algebraic closure.
\end{thm}
\begin{pf}
Let $F$ be a field.

\Step[1]{Construct an algebraically closed field containing $F$.}
Let $S$ be a set of irreducibles or nonconstants in $F[x]$.(anyone is fine)
Define $R:=F[\{x_p\}_{p\in S}]$.
Let $I$ be an ideal in $R$ generated by $p(x_p)$ as $p$ runs through all $S$.
It has a maximal ideal $\fm\supset I$.

Define $K_1:=R/\fm$.
Every nonconstant $f\in F[x]$ has a root in $K_1$.
(In fact, this $K_1$ is already algebraically closed, but it's hard to prove.)
Construct $K_2,\cdots$ such that every nonconstant $f\in K_n[x]$ has a root in $K_{n+1}$.
Define $K=\lim_{\to}K_n$.
Then, $K$ is algebraically closed.

\Step[2]{Construct the algebraic closure of $F$.}
Let $\cl F$ be the set of all algebraic elements of $K$ over $F$.
Then, this is the algebraic closure.
\end{pf}




\section{Space curve theory}
\begin{defn}
Let $\alpha$ be a curve.
\[\bT:=\frac{\alpha'}{\norm{\alpha'}},\quad\bN:=\frac{\bT'}{\norm{\bT'}},\quad\bB:=\bT\x\bN.\]
\end{defn}
\begin{prop}
$\bT',\bB',\bN$ are collinear.
\end{prop}

\begin{defn}
\[s(t):=\int_0^t\norm{\alpha'},\quad\kappa:=\dd{\bT}{s}\cdot\bN,\quad\tau:=-\dd{\bB}{s}\cdot\bN.\]
\end{defn}



\begin{thm}[Frenet-Serret formula]
Let $\alpha$ be a unit speed curve.
\[\begin{pmatrix}\bT'\\\bN'\\\bB'\end{pmatrix}=\begin{pmatrix}0&\kappa&0\\-\kappa&0&\tau\\0&-\tau&0\end{pmatrix}\begin{pmatrix}\bT\\\bN\\\bB\end{pmatrix}.\]
\end{thm}
\begin{thm}
Let $\alpha$ be a unit speed curve.
\begin{align*}
\alpha'&=\bT\\
\alpha''&=\kappa\bN\\
\alpha'''&=-\kappa^2\bT+\kappa'\bN+\kappa\tau\bB
\end{align*}
\[\kappa=\norm{\alpha''},\quad\tau\frac{[\alpha'\alpha''\alpha''']}{\kappa^2}.\]
\end{thm}

\begin{thm}
Let $\alpha$ be a curve.
\begin{align*}
\alpha'&=s'\bT\\
\alpha''&=s''\bT+s'^2\kappa\bN\\
\alpha'''&=(s'''-s'^3\kappa^2)\bT+(3s's''\kappa+s'^2\kappa')\bN+s'^3\kappa\tau\bB
\end{align*}
\[\kappa=\frac{\norm{\alpha'\x\alpha''}}{\norm{\alpha'}^3},\quad\tau=\frac{[\alpha'\alpha''\alpha''']}{\norm{\alpha'\x\alpha''}}.\]
\end{thm}

Problem solving strategy: 
\begin{itemize}
\item Represent $\alpha$ and its derivatives over the Frenet basis.
\item 
\end{itemize}

Uniqueness: The Frene-Serret formula is an ODE for the vector (of vectors) $(\bT,\bN,\bB)$.
After showing this equation preserves orthonormality, obtain $\alpha$ by integratin $\bT$.
The skew-symmetry implies that $\norm\bT^2+\norm\bN^2+\norm\bB^2$ is constant.








\section{Algebraic integer}
\subsection{Quadratic integer}
\begin{thm}
Every quadratic field is of the form $\Q(\sqrt d)$ for a square-free $d$.
\end{thm}
\begin{thm}
Let $d$ be a square-free.
\[\cO_{\Q(\sqrt d)}=\begin{cases}\Z+\sqrt d\Z&,d\equiv2,3\pmod4\\\bigmath\Z+\frac{1+\sqrt d}2\Z&,d\equiv1\pmod4\end{cases}\]
\[\Delta_{\Q(\sqrt d)}=\begin{cases}4d&,d\equiv2,3\pmod4\\ d&,d\equiv1\pmod4\end{cases}\]
\end{thm}
\begin{thm}
Let $\theta^3=hk^2$ for $h,k$ square-free's.
\[\cO_{\Q(\theta)}=\begin{cases}\bigmath\Z+\sqrt[3]{hk^2}\Z+\sqrt[3]{h^2k}\Z&,m\not\equiv\pm1\pmod9\\
\bigmath\Z+\theta\Z+\frac{\theta^2\pm\theta k+k^2}{3k}\Z&,m\equiv\pm1\pmod9\end{cases}\]
\end{thm}
\subsection{Integral basis}
\begin{thm}
Let $\alpha\in K$.
$Tr_K(\alpha)\in\Z$ if $\alpha\in\cO_K$.
$N_K(\alpha)\in\Z$ if and only if $\alpha\in\cO_K$.
\end{thm}
\begin{thm}
Let $\{\omega_1,\cdots,\omega_n\}$ be a basis of $K$ over $\Q$.
If $\Delta(\omega_1,\cdots,\omega_n)$ is square-free, then $\{\omega_1,\cdots,\omega_n\}$ is an integral basis.
\end{thm}
\begin{thm}
Let $\{\omega_1,\cdots,\omega_n\}$ be a basis of $K$ over $\Q$ consisting of algebraic integers.
If $p^2\mid\Delta$ and it is not an integral basis, then there is a nonzero algebraic integer of the form
\[\frac1p\sum_{i=1}^n\lambda_i\omega_i.\]
\end{thm}

\subsection{Fractional ideals}
\begin{thm}
Every fractional ideal of $K$ is a free $\Z$-module with rank $[K,\Q]$.
\end{thm}
\begin{pf}
This theorem holds because $K/\Q$ is separable and $\Z$ is a PID.

\end{pf}






\section{Diophantine equations}
The reciprocity laws let us know not only what primes split, but also what prime factors a polynomial has.
\subsection{The Mordell equations}
\[y^2=x^3+k\]
There are two strategies for the Mordell equations:
\begin{itemize}
\item $x^2-2x+4$ has a prime factor of the form $4k+3$
\item $x^3=N(y-a)$ for some $a$.
\end{itemize}
First case: $k=7,-5,-6,45,6,46,-24,-3,-9,-12$.
\begin{ex} Solve $y^2=x^3+7$. \end{ex}
\begin{pf}
Taking mod 8, $x$ is odd and $y$ is even.
Consider
\[y^2+1=(x+2)(x^2-2x+4).\]
Since
\[x^2-2x+4=(x-1)^2+3,\]
there is a prime $p\equiv3\pmod4$ that divides the right hand side.
Taking mod $p$, we have
\[y^2\equiv-1\pmod p,\]
which is impossible.
Therefore, the equation has no solutions.
\end{pf}
\begin{ex} Solve $y^2=x^3-2$. \end{ex}
\begin{pf}
Taking mod 8, $x$ and $y$ are odd.
Consider a ring of algebraic integers $\Z[\sqrt{-2}]$.
We have
\[N(y-\sqrt{-2})=(y-\sqrt{-2})(y+\sqrt{-2})=x^3.\]
For a common divisor $\delta$ of $y\pm\sqrt{-2}$, we have 
\[N(\delta)\mid N((y-\sqrt{-2})-(y+\sqrt{-2}))=N(2\sqrt{-2})=|(2\sqrt{-2})(-2\sqrt{-2})|=8.\]
On the other hand,
\[N(\delta)\mid x^3 \equiv1\pmod2,\]
so $N(\delta)=1$ and $\delta$ is a unit.
Thus, $y\pm\sqrt{-2}$ are relatively prime.
Since the units in $\Z[\sqrt{-2}]$ are $\pm1$, which are cubes, $y\pm\sqrt{-2}$ are cubics in $\Z[\sqrt{-2}]$.

Let
\[y+\sqrt{-2}=(a+b\sqrt{-2})^3=a(a^2-6b^2)+b(3a^2-2b^2)\sqrt{-2},\]
so that $1=b(3a^2-2b^2)$.
We can conclude $b=\pm1$.
The possible solutions are $(x,y)=(3,\pm5)$, which are in fact solutions.
\end{pf}



\section{The local-global principle}
\subsection{The local fields}
Let $f\in\Z[x]$.
\[\text{\emph{Does $f=0$ have a solution in $\Z$?}}\]
\[\text{\emph{Does $f=0$ have a solution in $\Z/(p^n)$ for all $n$?}}\]
\[\text{\emph{Does $f=0$ have a solution in $\Z_p$?}}\]
In the first place, here is the algebraic definition.
\begin{defn}
Let $p\in\Z$ be a prime.
The ring of the $p$-adic integers $\Z_p$ is defined by the inverse limit:
\begin{es}
\Z_p:={\bigmath\lim_{\substack{\longleftarrow\\n\in\N}}}\F_{p^n}  \>  \cdots  \>  \Z/(p^3)  \>  \Z/(p^2)  \>  \F_p.
\end{es}
\end{defn}
\begin{defn}
$\Q_p=\Frac\Z_p$.
\end{defn}
Secondly, here is the analytic definition.
\begin{defn}
Let $p\in\Z$ be a prime.
Define a absolute value $|\cdot|_p$ on $\Q$ by $|p^ma|_p=\frac1{p^m}$.
The local field $\Q_p$ is defined by the completion of $\Q$ with respect to $|\cdot|_p$.
\end{defn}
\begin{defn}
$\Z_p:=\{x\in\Q_p:|x|_p\le1\}$.
\end{defn}

\begin{ex}
Observe
\begin{align*}
3^{-1}&\equiv2_5\pmod5\\
&\equiv32_5\pmod{5^2}\\
&\equiv132_5\pmod{5^3}\\
&\equiv1313132_5\pmod5^7\cdots.
\end{align*}
Therefore, we can write
\[3^{-1}=\overline{13}2_5=2+3p+p^2+3p^3+p^4+\cdots\]
for $p=5$.
Since there is no negative power of 5, $3^{-1}$ is a $p$-adic integer for $p=5$.
\end{ex}
\begin{ex}

\begin{align*}
7&\equiv1_3^2\pmod3\\
&\equiv111_3^2\pmod{3^3}\\
&\equiv20111_3^2\pmod{3^5}\\
&\equiv120020111_3^2\pmod{3^9}\cdots.
\end{align*}
Therefore, we can write
\[\sqrt7=\cdots120020111_3=1+p+p^2+2p^4+2p^7+p^8+\cdots\]
for $p=3$.
Since there is no negative power of 3, $\sqrt7$ is a $p$-adic integer for $p=3$.
\end{ex}


There are some pathological and interesting phenomena in local fields.
Actually note that the values of $|\cdot|_p$ are totally disconnected.
\begin{thm}
The absolute value $|\cdot|_p$ is nonarchimedean: it satisfies $|x+y|_p\le\max\{|x|_p,|y|_p\}$.
\end{thm}
\begin{pf}
Trivial.
\end{pf}

\begin{thm}
Every triangle in $\Q_p$ is isosceles.
\end{thm}

\begin{thm}
$\Z_p$ is a discrete valuation ring: it is local PID.
\end{thm}
\begin{pf}
asdf
\end{pf}

\begin{thm}
$\Z_p$ is open and compact.
Hence $\Q_p$ is locally compact Hausdorff.
\end{thm}
\begin{pf}
$\Z_p$ is open clearly.
Let us show limit point compactness.
Let $A\subset\Z_p$ be infinite.
Since $\Z_p$ is a finite union of cosets $p\Z_p$, there is $\alpha_0$ such that $A\cap(\alpha_0+p\Z_p)$ is infinite.
Inductively, since
\[\alpha_n+p^{n+1}\Z_p=\bigcup_{1\le x<p}(\alpha_n+xp^{n+1}+p^{n+2}\Z_p),\]
we can choose $\alpha_{n+1}$ such that $\alpha_n\equiv\alpha_{n+1}\pmod{p^{n+1}}$ and $A\cap(\alpha_{n+1}+p^{n+2}\Z_p)$ is infinite.
The sequence $\{\alpha_n\}$ is Cauchy, and the limit is clearly in $\Z_p$.
\end{pf}



\subsection{Hensel's lemma}

\begin{thm}[Hensel's lemma]
Let $f\in\Z_p[x]$.
If $f$ has a simple solution in $\F_p$, then $f$ has a solution in $\Z_p$.
\end{thm}
\begin{pf}
asdf
\end{pf}

\begin{rmk}
Hensel's lemma says: for $X$ a scheme over $\Z_p$, $X$ is smooth iff $X(\Z_p)\twoheadrightarrow X(\F_p)$....???
\end{rmk}

\begin{ex}
$f(x)=x^p-x$ is factorized linearly in $\Z_p[x]$.
\end{ex}

\subsection{Sums of two squares}


\begin{thm}[Euler]
A positive integer $m$ can be written as a sum of two squares if and only if $v_p(m)$ is even for all primes $p\equiv3\pmod4$.
\end{thm}
\begin{lem}
Let $p$ be a prime with $p\equiv1\pmod4$.
Every $p$-adic integer is a sum of two squares of $p$-adic integers.
\end{lem}











\section{Ultrafilter}

\begin{thm}
Let $\cU$ be an ultrafilter on a set $S$ and $X$ be a compact space.
For $f\colon S\to X$, the limit $\cU$-$\lim f$ always exists.
\end{thm}


\begin{thm}
Let $X=\prod_{\alpha\in\cA}X_\alpha$ be a product space of compact spaces $X_\alpha$.
A net $\{f_d\}_{d\in\cD}$ on $X$ has a convergent subnet.
\end{thm}
\begin{pf}[1]
Use Tychonoff.
Compactness and net compactness are equivalent.
\end{pf}
\begin{pf}[2]
It is a proof without Tychonoff.
Let $\cU$ be a ultrafilter on a set $\cD$ contatining all $\uparrow d$.
Define a directed set $\cE=\{(d,U)\in\cD\x\cU:d\in U\}$ as $(d,U)\prec(d',U')$ for $U\supset U'$.
Let $f\colon\cE\to X$ be a net defined by $f_{(d,U)}=f_d$.

By the previous theorem, $\cU$-$\lim\pi_\alpha f_d$ exsits for each $\alpha$.
Define $f\in X$ such that $\pi_\alpha f=\cU$-$\lim\pi_\alpha f_d$.
Let $G=\prod_\alpha G_\alpha\subset X$ be any open neighborhood of $f$ where $G_\alpha=X_\alpha$ except finite.
Then $G_\alpha$ is an open neighborhood of $\pi_\alpha f$ so that we have $U_\alpha:=\{d:\pi_\alpha f_d\in G_\alpha\}\in\cU$ by definition of convergence with ultrafilter.9
Since $U_\alpha=\cD$ except finites, we can take an upper bound $U_0\in\cU$.
Then, by taking any $d_0\in U_0$, we have $f_{(d,U)}\in G$ for every $(d,U)\succ(d_0,U_0)$.
This means $f=\lim_\cE f_{(d,U)}$, so we can say $\lim_\cE f_{(d,U)}$ exists.

%WTS f_{(d,U)} is a subnet f_d
\end{pf}















\section{Universal coefficient theorem}
\begin{lem}
Suppose we have a flat resolution
\begin{es}
0 \>  P_1  \>  P_0  \>  A  \> 0.
\end{es}
Then, we have a exact sequence
\begin{es}
\cdots  \>  0  \>  \Tor_1^R(A,B)  \>  P_1\tn B  \>  P_0\tn B  \>  A\tn B  \> 0.
\end{es}
\end{lem}


\begin{thm}
Let $R$ be a PID.
Let $C_\bul$ be a chain complex of flat $R$-modules and $G$ be a $R$-module.
Then, we have a short exact sequence
\begin{es}
0 \>  H_n(C)\tn G  \>  H_n(C;G)  \>  \Tor(H_{n-1}(C),G)  \> 0,
\end{es}
which splits, but not naturally.
\end{thm}

\begin{pf}[1]
We have a short exact sequence of chain complexes
\begin{es}
0 \>  Z_\bul  \>  C_\bul  \>  B_{\bul-1}  \> 0
\end{es}
where every morphism in $Z_\bul$ and $B_\bul$ are zero.
Since modules in $B_{\bul-1}$ are flat, we have a short exact sequence
\begin{es}
0 \>  Z_\bul\tn G  \>  C_\bul\tn G  \>  B_{\bul-1}\tn G  \> 0
\end{es}
and the associated long exact sequence
\begin{es}
\cdots \>  H_n(B;G)  \>  H_n(Z;G)  \>  H_n(C;G)  \>  H_{n-1}(B;G)  \>  H_{n-1}(Z;G)  \> \cdots
\end{es}
where the connecting homomomorphisms are of the form $(i_n\colon B_n\to Z_n)\tn1_G$ (It is better to think diagram chasing than a natural construction).
Since morhpisms in $B$ and $Z$ are zero (if it is not, then the short exact sequence of chain complexes are not exact, we have
\begin{es}
\cdots \>  B_n\tn G  \>  Z_n\tn G  \>  H_n(C;G)  \>  B_{n-1}\tn G  \>  Z_{n-1}\tn G  \> \cdots.
\end{es}
Since
\begin{es}
0 \>  \Tor_1^R(H_n,G)  \>  B_n\tn G  \>  Z_n\tn G  \>  H_n\tn G  \> 0
\end{es}
for all $n$, the exact sequence splits into short exact sequence by images
\begin{es}
0 \>  H_n\tn G  \>  H_n(C;G)  \>  \Tor_1^R(H_{n-1},G)  \> 0.
\end{es}

For splitting,
\end{pf}

\begin{pf}[2]
Since $R$ is PID, we can construct a flat resolution of $G$
\begin{es}
0 \>  P_1  \>  P_0  \>  G  \> 0.
\end{es}
Since modules in $C_\bul$ are flat so that the tensor product functors are exact and $P_1\to P_0$ and $P_0\to G$ induce the chain maps, we have a short exact sequence of chain complexes
\begin{es}
0 \>  C_\bul\tn P_1  \>  C_\bul\tn P_0  \>  C_\bul\tn G  \> 0.
\end{es}
Then, we have the associated long exact sequence
\begin{es}
\cdots \>  H_n(C;P_1)  \>  H_n(C;P_0)  \>  H_n(C;G)  \>  H_{n-1}(C;P_1)  \>  H_{n-1}(C;P_0)  \> \cdots.
\end{es}
Since flat tensor product functor commutes with homology funtor from chain complexes, we have
\begin{es}
\cdots \>  H_n\tn P_1  \>  H_n\tn P_0  \>  H_n(C;G)  \>  H_{n-1}\tn P_1  \>  H_{n-1}\tn P_0  \> \cdots.
\end{es}
Since
\begin{es}
0 \>  \Tor_1^R(G,H_n)  \>  H_n\tn P_1  \>  H_n\tn P_0  \>  H_n\tn G  \> 0
\end{es}
for all $n$, the exact sequence splits into short exact sequence by images
\begin{es}
0 \>  H_n\tn G  \>  H_n(C;G)  \>  \Tor_1^R(G,H_{n-1})  \> 0.
\end{es}
\end{pf}

\begin{pf}[3](??)
By tensoring $G$, we get the following diagram.
\begin{cd}[row sep={24pt,between origins}, column sep={36pt,between origins}]
H_n\tn G  \ar{ddr}  &&&& H_{n-1}\tn G \\
&&&& \\
& \coker\pd_{n+1}\tn G  \ep{ddr} && \ker\pd_{n-1}\tn G  \ep{uur}\ar{dr} & \\
C_n\tn G  \ep{ur}\ep{drr} &&&& C_{n-1}\tn G \\
&& \im\pd_n\tn G  \ar{uur}\ar{urr} && \\
&&&& \\
&\Tor_1(H_{n-1},G)\mn{uur}&&&
\end{cd}
Every aligned set of consecutive arrows indicates an exact sequence.
Notice that epimorphisms and cokernals are preserved, but monomorphisms and kernels are not.
Especially, $\coker\pd_{n+1}\tn G=\coker(\pd_{n+1}\tn1_G)$ is important.

Consider the following diagram.
\begin{cd}[row sep={30pt,between origins}, column sep={60pt,between origins}]
H_n(C;G) \mn{dr} & H_n\tn G \ar{d}&&&\\
& \coker\pd_{n+1}\tn G \ep{dd}\ep{ddrr} && \ker\pd_{n-1}\tn G \ar{dr}{\text{monic!}} & \\
&&&& C_{n-1}\tn G \\
& \im\pd_n\tn G \ar{uurr} && \im(\pd_n\tn1_G) \mn{ur}\ar[dashed,tail]{uu} & \\
\Tor_1(H_{n-1},G) \mn{ur} &&&&
\end{cd}
Since $\ker\pd_{n-1}$ is free, 

If we show $\im(\pd_n\tn1_G)\to\ker\pd_{n-1}\tn G$ is monic, then we can get
\begin{align*}
H_n(C;G)&=\ker(\coker\pd_{n+1}\tn G\to\im(\pd_n\tn1_G))\\
&=\ker(\coker\pd_{n+1}\tn G\to\ker\pd_{n-1}\tn G).
\end{align*}



\end{pf}


\section{Estimates}

\begin{thm}
The following series diverges: \[\sum_{n=1}^\infty\frac1{n^{1+|\sin n|}}.\]
\end{thm}
\begin{pf}
Let $A_k:=[1,2^k]\cap\{x:|\sin x|<\frac1k\}$.
Divide the unit circle $\R/2\pi\Z$ by $7k$ uniform arcs.
There are at least $2^k/7k$ integers that are not exceed $2^k$ and are in a same arc.
Let $S$ be the integers and $x_0$ be the smallest element.
Since, $|x-x_0|\pmod{2\pi}<\frac{2\pi}{7k}$ for $x\in S$,
\[|\sin(x-x_0)|<|x-x_0|\pmod{2\pi}<\frac{2\pi}{7k}<\frac1k.\]
Also, $1\le x-x_0\le x\le2^k$, $x-x_0\in A_k$.
\[|A_k|\ge\frac{2^k}{7k}.\]
Therefore,
\begin{align*}
\sum_{n=1}^\infty\frac1{n^{1+|\sin n|}}
&\ge\sum_{n\in A_N}\frac1{n^{1+|\sin n|}}\\
&\ge\sum_{k=1}^N(|A_k|-|A_{k-1}|)\frac1{2^{k+1}}\\
&=\sum_{k=1}^N\frac{|A_k|}{2^{k+1}}-\sum_{k=1}^{N-1}\frac{|A_k|}{2^{k+2}}\\
&=\frac{|A_N|}{2^{N+1}}+\sum_{k=1}^{N-1}\frac{|A_k|}{2^{k+2}}\\
&>\sum_{k=1}^N\frac{2^k}{2^{k+2}}\frac1{7k}\\
&=\frac1{28}\sum_{k=1}^N\frac1k\\
&\to\infty.
\end{align*}
\end{pf}

\begin{thm}
If $|xf'(x)|\le M$ and $\frac1x\int_0^xf(y)\,dy\to L$, then $f(x)\to L$ as $x\to\infty$.
\end{thm}
\begin{pf}
Since
\begin{align*}
\abs{f(x)-\frac{F(x)-F(a)}{x-a}}
&\le\frac1{x-a}\int_a^x|f(x)-f(y)|\,dy\\
&=\frac1{x-a}\int_a^x(x-y)|f'(c)|\,dy\\
&\le\frac M{x-a}\int_a^x\frac{x-y}c\,dy\\
&\le M\frac{x-a}a
\end{align*}
by the mean value theorem and 
\[f(x)-L=\left[f(x)-\frac{F(x)-F(a)}{x-a}\right]+\frac x{x-a}\left[\frac{F(x)}x-L\right]+\frac a{x-a}\left[\frac{F(a)}a-L\right],\]
we have for any $\e>0$
\[\limsup_{x\to\infty}|f(x)-L|\le\e\]
where $a$ is defined by $\frac{x-a}a=\frac\e M$.
\end{pf}


%%%
\section{Action}

\subsection*{Definition}
\begin{itemize}
\item $G\curvearrowright X$
	\par-- fcn $G\times X\to X$ : compatibility, identity
	\par-- hom $\rho\colon G\to\text{Sym}(X)$ or $\text{Aut}(X)$
	\par-- funtor from $G$
	\par nt) $X$ is called $G$-set.
	\par nt) $\rho$ is called permutation repr.
	\par$*$ right action is a contravariant functor.
\item $\text{Stab}_G(x)=G_x$, $\text{Orb}_G(x)=G.x$
	\par-- Orbit-stabilizer theorem
		\par\quad$pf)$ quotient with $-.x\colon G\to X$.
		\par\quad$*$ this is not the first isom.
	\par$*$ stabilizer is also called isotropy group.
\item Faithfulness, Transitivity
\end{itemize}


\subsection*{Useful Actions}
$*$ these actions are on $P(G)$.
\begin{itemize}
\item Left Multiplication
	\par-- $\text{Stab}(A)=AA^{-1}$
	\par eg) $G\curvearrowright G/H$, for $H\le G$
		\par\qquad $\ker\rho=\bigcap xHx^{-1}=\text{Core}_G(H)$
\item Conjugation
	\par-- $\text{Stab}(A)=N_G(A)$
	\par eg) $G\curvearrowright\{\{h\}:h\in H\}$, for $H\lhd G$
		\par\qquad $\ker\rho=C_G(H)$, $\im\rho\subset\text{Aut}(H)$
	\par eg) $G\curvearrowright\text{Syl}_p$
	\par$*$ conjugation is an isomorphism.
\end{itemize}


\subsection*{Sylow Theorem}

\begin{itemize}
\item $\text{Syl}_p\ne\varnothing$
\item $n_p=kp+1\mid\left[G:\overline P\right]$
\item $G\curvearrowright\text{Syl}_p$ transitive
	\par $pf)$ four actions by conjugation:
	\par\quad $G\curvearrowright G$,\quad $\overline P,G,P\curvearrowright\text{Orb}_G(\overline P)$.
\end{itemize}

\bigskip
EXERCISES

\section{Some problems}

Problems I made:
\begin{enumerate}
\item Let $f$ be $C^2$ with $f''(c)\ne0$. Defined a function $\xi$ such that $f(x)-f(c)=f'(\xi(x))(x-c)$ with $|\xi-c|\le|x-c|$, show that $\xi'(c)=1/2$.
\item Let $f$ be a $C^2$ function such that $f(0)=f(1)=0$. Show that $\|f\|\le\frac18\|f''\|$.
\item Show that a measurable subset of $\R$ with positive measure contains an arbitrarily long subsequence of an arithmetic progression.
\item Show that there is no continuous bijection from $[0,1]^2\setminus\{p\}$ to $[0,1]^2$. %나도 못풂
\end{enumerate}
\bigskip
\begin{enumerate}
\item Show that for a nonnegative sequence $a_n$ if $\sum a_n$ diverges then $\sum\frac{a_n}{1+a_n}$ also diverges.
\item Show that if both limits of a function and its derivative exist at infinity then the former is 0.
\item Show that every real sequence has a monotonic subsequence that converges to the limit superior of the supersequence.
\item Show that if a decreasing nonnegative sequence $a_n$ converges to 0 and satisfies $S_n\le1+na_n$ then $S_n$ is bounded by 1.
\item Show that the set of local minima of a convex function is connected.
\item Show that a smooth function such that for each $x$ there is $n$ having the $n$th derivative vanish is a polynomial.
\item Show that if a continuously differentiable $f$ satisfies $f(x)\ne0$ for $f'(x)=0$, then in a bounded set there are only finite points at which $f$ vanishes.
\item Let a function $f$ be differentiable. For $a<a'<b<b'$ show that there exist $a<c<b$ and $a'<c'<b'$ such that $f(b)-f(a)=f'(c)(b-a)$ and $f(b')-f(a')=f'(c')(b'-a')$.
\item Show that if $xf'(x)$ is bounded and $x^{-1}\int_0^xf\to L$ then $f(x)\to L$ as $x\to\infty$.
\item Show that if a sequence of real functions $f_n\colon[0,1]\to[0,1]$ satisfies $|f(x)-f(y)|\le|x-y|$ whenever $|x-y|\ge\frac1n$, then the sequence has a uniformly convergent subsequence.
\item (Flett)
\item Let $f$ be a differentiable function with $f(0)=0$. Show that there is $c\in(0,1)$ such that $cf(c)=(1-c)f'(c)$.
\item Find the value of $\lim_{n\to\infty}\frac1n\left(\sum_{k=1}^n\frac1nf\left(\frac kn\right)-\int_0^1f(x)\,dx\right)$.

\hrule
\item Let f be a continuous function. Show that f(x)=c cannot have exactly two solutions for every c.
\item Show that a continuous function that takes on no value more than twice takes on some value exactly once.
\item Let f be a function that has the intermediate value property. Show that if the preimage of every singleton is closed, then f is continuous.

\hrule
\item Show that if a holomorphic function has positive real parts on the open unit disk then $|f'(0)|<2\Re f(0)$.
\item Show that if at least one coefficient in the power series of a holomorphic function at each point is 0 then the function is a polynomial.
\item Show that if a holomorphic function on a domain containing the closed unit disk is injective on the unit circle then so is on the disk.
\item Show that for a holomorphic function $f$ and every $z_0$ in the domain there are $z_1\ne z_2$ such that $\frac{f(z_1)-f(z_2)}{z_1-z_2}=f'(z_0)$.
\item For two linearly independent entire functions, show that one cannot dominate the other.
\item Show that uniform limit of injective holomorphic function is either constant or injective.
\item Suppose the set of points in a domain $U\subset\C$ at which a sequence of bounded holomorphic functions $(f_n)$ converges has a limit point. Show that $(f_n)$ compactly converges.

\hrule
\item Show that normal nilpotent matrix equals zero.
\item Show that two matrices $AB$ and $BA$ have same nonzero eigenvalues whose both multiplicities are coincide blabla...
\item Show that if $A$ is a square matrix whose characteristic polynomial is minimal then a matrix commuting $A$ is a polynomial in $A$.
\item Show that if two by two integer matrix is a root of unity then its order divides 12.
\item Show that a finite symmetric group has two generators.
\item Show that a nontrivial normalizer of a $p$-group meets its center out of identity.
\item Show that a proper subgroup of a finite $p$-group is a proper subgroup of its normalizer. In particular, every finite p-group is nilpotent.
\item Show that the complement of a saturated monoid in a commutative ring is a union of prime ideals.
\item Show that the Galois group of a quintic over $\Q$ having exactly three real roots is isomorphic to $S_5$.

\hrule
\item Show that if $A^\o\in B$ and $B$ is closed, then $(A\cup B)^\o\subset B$.

\hrule
\item Show that the tangent space of the unitary group at the identity is identified with the space of skew Hermitian matrices.
\item Prove the Jacobi formula for matrix.
\item Show that $S^3$ and $T^2$ are parallelizable.
\item Show that $\mathbb{R}P^n=S^n/Z_2$ is orientable if and only if $n$ is odd.
\end{enumerate}



\end{document}
최고차항의 계수가 1인 사차함수 f(x)는 다음 조건을 만족시킨다.
[각각의 양의 실수 s에 대하여, |f(t)|≤g(t)를 만족시키는 실수 t가 두 개 존재하고 그 차가 s가 되게 하는 이차 이하의 다항함수 g(x)가 유일하게 존재한다.]
또한 위의 조건에서 s=1,3에 대응하는 함수 g(x)를 각각 g₁(x), g₃(x)라 할 때, 방정식 g₁(x)+g₃(x-1/2)=0을 만족시키는 실수 x가 단 한 개 존재한다. 함수 f(x)가 최솟값 -p/q을 가질 때, p+q를 구하시오. (단, p와 q는 서로 소인 자연수이다.)

ans: 43