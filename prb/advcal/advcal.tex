\documentclass{../prb}
\usepackage{../../ikany}
\korean

\title{고급 미적분 훈련}

\begin{document}
\maketitle
\tableofcontents


\chapter{함수}

\section{증가함수}

\section{미분을 사용하지 않는 부등식}

\begin{prb}
양수 $x,y$가 $xy=1$를 만족할 때 $2x+y$의 최솟값을 구하여라.
\end{prb}
\begin{sol}
산술기하평균 부등식에 따라
\[2x+y\ge2\sqrt{2xy}=2\sqrt2.\]
또한, $x=\frac1{\sqrt2}$, $y=\sqrt2$일 때 $2x+y=2\sqrt2$이다.
따라서 답은 $2\sqrt2$.
\end{sol}

\+

\begin{prb}
실수 $x$에 대하여 다음 함수 $f:\R\to\R$의 최댓값과 최솟값을 구하여라:
\[f(x)=\frac{x^2-x+1}{x^2+x+1}.\]
\end{prb}
\begin{sol}[1]
잠시 $x\ne0$인 경우만 생각하자.
$t=x+\frac1x$라고 하면 $t$의 범위는 $(-\oo,-2)\cup(2,\oo)$이다.
\[f(x)=\frac{t-1}{t+1}=1-\frac2{1+t}\]이므로 $x\in\R-\{0\}$에 대한 $f$의 범위는 그래프를 그려 확인해보면 $\left(\tfrac13,1\right)\cup(1,3)$이다.
$f(0)=1$이므로 $f$의 범위는 $\left(\tfrac13,3\right)$이고 답은 최솟값 $\frac13$, 최댓값 3이다.
\end{sol}
\begin{sol}[2]
$f(x)=k$를 만족하는 실수 $x$가 존재하는 $k$를 구하자.
방정식 $f(x)=k$를 $x$에 대해 정리하면 $(k-1)x^2+(k+1)x+(k-1)=0$이고, 판별식 $D=(k+1)^2-4(k-1)(k-1)=-(3k-1)(k-3)$이 0 이상일 필요충분조건은 $\frac13\le k\le 3$이다.
따라서 답은 최솟값 $\frac13$, 최댓값 3이다.
\end{sol}
\begin{note}[1]
많은 학생들은 최댓값과 최솟값을 구하는 문제에서 기계적으로 미분을 하려는 경향이 있다.
하지만 의외로 미적분학의 복잡한 계산 없이 부등식 문제가 깔끔히 풀리는 경우가 있는데 대표적으로 산술기하 부등식 및 코시 슈바르츠 부등식과 같은 절대부등식을 이용하는 방법과 이차방정식의 \emph{판별식}을 사용하는 방법, 두 가지가 있다.
우선 이 문제처럼 분모분자가 이차식인 유리함수에서 유용하게 사용될 수 있고, 또한 코시 슈바르츠 부등식의 가장 표준적인 증명에도 사용된다.
\end{note}
\begin{note}[2]
함수의 최댓값과 최솟값이란 그 함수의 치역의 최댓값과 최솟값으로 정의될 수 있다.
\end{note}
\begin{note}[3]
이 문제에서 $f$의 정의역은 $\R$ 전체였지만, 유리함수의 형태를 보자마자 정의역이 어떻게 제한되는지 생각하는 습관을 가지는 것이 좋다.
여담으로, 실수 $x$에 대해 $x^2+x+1$와 $x^2-x+1$가 항상 양수라는 사실은 생각보다 많은 문제에서 이용된다.
\end{note}

\section{접선}

\section{볼록성}

\section{지수함수와 로그함수}

\section{평균값정리}

\begin{prb}
다음 극한값을 구하여라:
\[\lim_{x\to\oo}(\sin\sqrt{x+1}-\sin\sqrt x.\]
\end{prb}







\chapter{수열}

\section{선형점화식}

\section{스퀴즈정리}

\section{고정점정리}

\section{무한급수}
등비급수, 망원급수, 테일러급수, 급수판정법

\section{수렴속도와 발산속도}

\section{가합성이론}











\chapter{삼각법}


\section{특수각의 활용}

\section{복소평면과 복소지수}

\section{삼각치환}

\section{논증기하학의 삼각대칭식}

\section{특수함수열}

\section{직교성}

\section{적분변환과 진동적분}











\chapter{적분}

\section{유리함수의 부정적분}

\section{역함수}

\section{삼각치환적분}

\section{정적분의 수렴}

\section{적분부등식}







\end{document}













