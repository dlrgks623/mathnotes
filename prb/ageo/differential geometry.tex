\documentclass[11pt]{article}
\input{../../header}
\NeedsTeXFormat{LaTeX2e}
\ProvidesClass{prb}[2019/03/25 prb]
\LoadClass[11pt]{amsbook}
\makeatletter

\def\thisisprbcls{1}

\usepackage[margin = 100pt]{geometry}



\let\@@title\title
\renewcommand{\title}[1]{
  \def\ltitle{#1}
  \@@title{Problem Set : \ltitle}
}
\let\@@author\author
\renewcommand{\author}[1]{
  \def\lauthor{#1}
  \@@author{\small Written by \lauthor \\ \small Solved by \lauthor}
}

\author{Ikhan Choi}

\usepackage{fancyhdr}
\pagestyle{fancy}
\fancyhf{}
\fancyhead[L]{\nouppercase{\leftmark}}
\fancyhead[R]{\nouppercase{\rightmark}}
\fancyfoot[C]{\scriptsize{\thepage}}

\let\@@section\section
\renewcommand{\section}{\newpage\@@section}



%\setcounter{tocdepth}{2}
%\def\l@subsection{\@tocline{2}{0pt}{2pc}{4pc}{}}



\usepackage{hyperref}


\makeatother

\def\ntitle{Differential Geometry}

\begin{document}
\maketitle
\tableofcontents

\section{Manifolds}
\subsection{Lie derivatives}

\begin{prb}[Cartan's magic formula]
Cartan's magic formula is
\[L_X\alpha=\iota_Xd\alpha+d\iota_X\alpha.\]
\q Show that \[d\alpha(X,Y)=X(\alpha(Y))-Y(\alpha(X))-\alpha([X,Y]).\]
\end{prb}

\subsection{Curves}
\bigskip
Newton's dot notation will be used for only curves and scalar-valued functions.




\section{Lie groups}







\section{Connections}

\subsection{Affine connection}

\begin{prb}[Covariant derivative]
Let $E$ be a vector bundle over a manifold $M$.
A connection on $E$ is an $\R$-linear map $\del\colon\Gamma(E)\to\Gamma(T^*M\tn E)$ such that the Leibniz rule $\del(fs)=f\del s+df\tn s$ for $s\in\Gamma(E)$ and $f\in C^\infty(M)$.
\end{prb}


\begin{prb}[Affine property]
tensor
\q Show that $\del_1-\del_2$ is a tensor.
\end{prb}


\begin{prb}[Christoffel symbol]
Let $\del$ be a affine connection on a manifold $M$.
Let $\{e_i\}_i$ be a local frame on $M$.
The Christoffel symbol is a set of functions characterizing a connection on the tangent bundle, an affine connection, defined by \[\del_ie_j=\Gamma_{ij}^ke_k.\]
Let $X,Y$ be vector fields on $U$.
\q Show that \[\del_XY=X^i(\pd_iY^k+\Gamma_{ij}^kY^j)e_k.\]
\q Use a partition of unity to show that every manifold admits an affine connection.
\end{prb}
\begin{sol}
\q This is easy.
\q This is not easy.
\end{sol}

\begin{prb}[Dependnecy]
Let $M$ be a manifold.
Let $p$ be a point on $M$.
Let $X_1,\,X_2,\,Y_1,\,Y_2$ be vector fields on a neighborhood of $p$.
\q Show that if $X_1$ and $X_2$ are same at $p$, then $\del_{X_1}Y=\del_{X_2}Y$ at $p$ for every vector field $Y$.
\q Show that if $Y_1$ and $Y_2$ are same on a neighborhood of $p$, then $\del_XY_1=\del_XY_2$ at $p$ for every vector field $X$.
\end{prb}

\begin{prb}[Parallel transport of vector]
Let $\del$ be an affine connection on a manifold $M$.
A vector field $X$ is called \emph{parallel} along with $\gamma$ if $\del_{\dot\gamma}X=0$.
\q Deduce $X$ satisfies a first order ODE \[\dot X^k(t)+\Gamma_{ij}^k\dot\gamma^i(t)X^j(t)=0.\]
\q Prove that given a tangent vector $v_0\in T_{\gamma(t_0)}M$, there exists a unique parallel vector field $X$ along $\gamma$ such that $X(t_0)=v$.
\end{prb}
\begin{prb}
Parallel transport is defined by the previous subproblem
\q Show that the parallel trasport is
\q parallel frame
\q Show that \[\del_{\dot\gamma}X(t_0)=\lim_{t\to t_0}\frac{P_{t_0\to t}^{-1}X(t)-X(t_0)}{t-t_0}.\]
\end{prb}


\begin{prb}[Second covariant dervative]
The second covariant dervative is defined by
\q Note that \[\del_{X,Y}^2f=[\del(\del f)](X,Y)=\inn{\inn{\del(df)}X}Y=\inn{\del_X(df)}Y=\del_X\inn{df}Y-\inn{df}{\del_XY}=XYf-(\del_XY)f.\]
\q Prove \[\del_{X,Y}^2=\del_X\del_Y-\del_{\del_XY}.\]
\end{prb}


\begin{prb}[Torsion tensor]
Let $\del$ be an affine connection on a manifold $M$.
The torsion tensor is a tensor field of (1,2)-type defined by
\[T(X,Y):=\del_XY-\del_YX-[X,Y].\]
\q If torsion-free, then second covariant derivative is symmetric(Hessian is symmetric)
\end{prb}


\begin{prb}[Riemannian curvature tensor]
\mtprb
\end{prb}






\begin{prb}[Connection on natural constructions]
\mtprb
\end{prb}








\subsection{Ehresmann connection}

\begin{prb}[Horizontal subbundle]
Let $\pi\colon E\to M$ be a fiber bundle.
A distribution of $E$ defined by $\ker\pi$ is called vertical subbundle and denoted by $VE$.
An Ehresmann connection is a choice of a subbundle $HE$ of $TE$ called horizontal subbundle that satisfies $TE=VE\dsum HE$.
\q definitions by (1) choice of horizontal subbundle (2) projection to vericals (3) right action is adjoint + fundamental vector fields.
\end{prb}

\begin{prb}[Connection form]
Let $\pi\colon P\to M$ be a principal $G$-bundle and $VP$ be the vertical subbundle.
Let $v\colon TP\to VP$ be a projection vector bundle homomorphism so that $\ker v=:HP$ defines an Ehresmann connection on $P$.
\q Provide a vector bundle isomorphism $VP\to P\x\fg$.
\q Show that $v$ determines a $\fg$-valued one-form on $P$. This is called connection form and will be denoted by $\omega$. blablbalblablbala
\q fundamental vector field $\sigma_p\colon\fg\to T_pP$ is defined by the differential of the orbit map $g\mapsto pg$: \[\sigma_p(X)=\dd{t}(p\exp(tX))|_{t=0}.\]
\end{prb}


\begin{prb}[Parallel transport]
\mtprb
\end{prb}

\begin{prb}[Principal connection]
\mtprb
\end{prb}








\begin{prb}[Exterior covariant derivative]
de Rham
\q Note that exterior derivative can be realized as the unique connection on the trivial bundle $M\x\R\to M$.(I guess it is true, but I should check)
\end{prb}

\begin{prb}[Curvature form]
\mtprb
\end{prb}


\begin{prb}[Torsion form]
\q Show that $d\id\in\Omega^1(M;TM)$ and $d^\del d\id=T$.
\end{prb}

\begin{prb}[Lie algebra-valued differential form]
wedge product and cartan structural equation
\end{prb}
















\section{Riemannian geometry}

\subsection{Riemannian metric}

\begin{prb}[Musical isomorphism]
sharp flat
\q sharp flat
\end{prb}


\begin{prb}[Levi-civita connection]
Let $M$ be a Riemannian manifold.
A Levi-civita connection is a connection on $M$ such that
\begin{cond}
\item $\del_XY$
\end{cond}
\end{prb}


\begin{prb}[Riemannian curvature tensor]
Let $\del$ be the Levi-civita connection on a Riemannian manifold $M$.
The Riemannian curvature tensor is a tensor field of type (1,3) defined by
\[R(X,Y)Z:=\del_X\del_YZ-\del_Y\del_XZ-\del_{[X,Y]}Z.\]

\end{prb}


\subsection{Geodesics}
\begin{prb}[Geodesic equation]
Let $\del$ be an affine connection on a manifold $M$.
A smooth curve $\gamma\colon I\to M$ is called geodesic if \[\del_{\dot\gamma}\dot\gamma(t)=0\] at every $t$.
\q Deduce the geodesic equation \[\ddot\gamma^k(t)+\Gamma_{ij}^k\dot\gamma^i(t)\dot\gamma^j(t)=0.\]
\q Prove that for $p\in M$, $v\in T_pM$, and $t_0\in\R$, there exists a locally defined geodesic curve $\gamma\colon I\to M$ with $\gamma(t_0)=p$ and $\dot\gamma(t_0)=v$.
\end{prb}



\subsection{Hodge theory}

\begin{prb}[Normal bundle]
Let $\tilde M$ be an oriented Riemannian manifold.
Let $M$ be an embedded manifold in $\tilde M$.
The normal bundle is defined as the quotient...
\q Show that $M$ is orientable if and only if there is a globally nonvanishing normal vector.
\q Show show show
\end{prb}


\begin{prb}[Volume form]
Let $M$ be an oriented Riemannian manifold with boundary.
\q Show that if $\pd M=\mt$, then $d\vol_M$ is not exact.
\end{prb}

\begin{prb}[Divergence]
The divergence is defined by
  \[\div X\,d\vol:=d(\iota_Xd\vol)=L_Xd\vol.\]
\q Show that
  \[\iota_Fd\vol=F^1\,dy\wedge dz+F^2\,dz\wedge dx+F^3\,dx\wedge dy\]
for a vector field $F=F^1\pd_x+F^2\pd_y+F^3\pd_z$ on $\R^3$.
\q Deduce that
  \[\div F=\pd{F^1}{x}+\pd{F^2}{y}+\pd{F^3}{z}.\]
\end{prb}

\begin{prb}[Divergence theorem]
Let $M$ be an oriented Riemannian manifold with boundary.
Let $X$ be a vector field on $M$ and $\nu$ be a global normal vector field on $\bd M$.
Let $d\vol_M$ and $d\vol_{\bd M}$ be invariant volume forms.
\q Show that \[\inn{X}{\nu}\,d\vol_{\bd M}=\iota_X\,d\vol_M\] on $\bd M$
\q Deduce the divergence theorem: \[\int_{\bd M}\inn{X}{\nu}\,d\vol_{\pd M}=\int_M\div X\,d\vol_M.\]
\end{prb}
\end{document}