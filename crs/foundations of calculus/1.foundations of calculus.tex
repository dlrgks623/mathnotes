\documentclass{../crs}
\usepackage{../../ikany}

\title{Analysis 1 : Foundations of Calculus}

\begin{document}
\maketitle
\tableofcontents


\frontmatter

\chapter*{Preface}

Before 19th century, theory of limits, infinite series, differentiation, and integration was so focused on calculating particular values of expressions, such as infinite series, that they do not have sufficient rigor.
As an example, A. L. Cauchy, a pioneer of mathematical analysis, is known to have made many mistakes on his theses.
Thereby, establishing a rigid framework for analysis has emerged as one of the central problems of mathematics in 19th century.%% can't make sure





\mainmatter

\chapter{Real numbers}

\section{Fields}




\section{Orders}

% relation



\section{Norms}

% vector space
% banach 
% metric open closed compact



\section{Completeness}





\section{The real numbers}
\subsection{Constructions of the real numbers}

\subsection{Cardinality}

\subsection{Elementary functions}





\chapter{Sequences}


\chapter{Functions}


\chapter{Integration}




\section{Lebesgue spaces}

\subsection{$1\le p\le\oo$}

\begin{defn}
Let $f$ be a measurable real-valued function on a measure space $(X,\mu)$.
The \emph{$L^p$ norm} of $f$ is defined as a real number
\[\|f\|_{L^p(X)}:=(\int|f|^p\,d\mu)^{\frac1p}\]
for $0<p<\oo$, and
\[\|f\|_{L^\oo(X)}:=\esssup_{x\in X}|f(x)|\]
for $p=\oo$.
\end{defn}
\begin{defn}
The \emph{essential supremum} is defined by the smallest number $M$ such that
\[|f(x)|\le M\quad a.e.\]
\end{defn}
We often abbreviate $\|f\|_{L^p(X)}$ as $\|f\|_{L^p}$ or $\|f\|_p$.






\subsection{$L^2$ space}








\chapter{Multivariable calculus}
$T<\min\{1/K,R/M\}$ and $|f(t,x)|\le M$ on $[0,T]\times\cl{B(x_0,R)}$
\[x'=f(t,x),\qquad x(0)=x_0.\]

\[\phi(x)(t)=x_0+\int_0^tf(s,x(s))\,ds.\]
\begin{align*}
|\phi(x)(t)-x_0|&\le\int_0^t|f(s,x(s))|\,ds\\
\le TM
\end{align*}
\begin{align*}
|\phi(g)(t)-\phi(h)(t)|&=|\int_0^t(f(s,g(s))-f(s,h(s)))\,ds|\\
&\le\int_0^t|f(s,g(s))-f(s,h(s))|\,ds\\
&\le\int_0^tK|g(s)-h(s)|\,ds\\
&\le TK\sup_s|g(s)-h(s)|
\end{align*}

\subsection{Inverse function theorem}

\begin{pf}[1]
Take $y$.
We must first define $x$ such that $f(x)=y$.
Let
\[\f_y(x)=x+df_p^{-1}(y-f(x)).\]
Note that $f(x)=y$ if and only if $x$ is a fixed point of $\f_y$.

Since $df$ is continuous, there is an open ball $U$ such that
\[\|df(x)-df(p)\|<\frac1{2\|df(p)^{-1}\|}.\]
For $x\in U$,
\[d\f_y(x)=\id-df_p^{-1}df_x\]
\end{pf}



\end{document}













