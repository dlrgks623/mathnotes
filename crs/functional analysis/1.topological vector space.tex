\documentclass{../crs}
\usepackage{../../ikany}

\title{Functional Analysis I : Topological Vector Space}

\begin{document}
\maketitle
\tableofcontents





\chapter{Category of topological vector spaces}

\section{Elementary properties}
definition - how to use the continuity of vector space operations effectively
homeomorphism by translation and dialation: local base at 0
uniformity pseudometrics, basic classification
translation invariant metric
completely regular (up to 3.5)
boundedness and continuity


\section{Classification}
\begin{rd}[row sep={50pt,between origins}, column sep={60pt,between origins}]
& inner product \ar{d} &&& Hilbert space \ar{d} & \\
& \block{finite\\seminorms\\(= normable)} \ar{d} &&& Banach space \ar{d} & \\
& \block{countable\\seminorms} \ar{ld}\ar{rd} &&& Fr\'echet space \ar{ld}\ar{rd} & \\
  \block{countable\\pseudometrics\\(= metrizable)} \ar{rd} &
& \block{uncountable\\seminorms\\(= locally convex)} \ar{ld} &
  F-space \ar{rd} && (no name) \ar{ld} \\
& \block{uncountable\\pseudometrics} &&& (no name) &
\end{rd}

\begin{prop}
Let $\rho$ be a pseudometric.
Then,
\[B(0,1)\subset\frac{B(0,1)+B(0,1)}2\subset\frac12B(0,2).\]
If $\rho$ is a seminorm, then the equalities hold.
\end{prop}
I say this as $\frac12B(0,2)$ is ``fatter'' than $B(0,1)$.







\chapter{Barreled spaces and bornological spaces}

\subsection{The Baire category theorem}

\subsection{Uniform boundedness principle}
\begin{thm}[Uniform boundedness principle]
Let $X$ be a barreled space and $Y$ be a topological vector space.
Let $\cF\subset B(X,Y)$.
If $\cF$ is pointwise bounded, then $\cF$\ is equicontinuous.
\end{thm}



\subsection{Open mapping theorem}
\begin{thm}[Open mapping theorem]
Let $X$ be a topological vector space and $Y$ be a metrizable barreled space.
Let $T\colon X\to Y$ be linear.
If $T$ is surjective and continuous, then $T$ is open.
\end{thm}
\begin{proof}
If we let $U$ be an open neighborhood in $X$, then we want to show $TU$ is a neighborhood.
Because $T$ is surjective so that $\cl{TU}$ is absorbent, $\cl{TU}$ is a neighborhood.
Note that an open set intersects $\cl{TU}$ also intersects $TU$.

If there exist two sequences of balanced open neighborhoods $U_n\subset X$ and $V_n\subset Y$ with
\begin{cond}
\item $U_1+\cdots+U_n\subset U$,
\item $V_n\subset\cl{TU_n}$,
\item $\bigcap_{n\in\N}V_n=\{0\}$,
\end{cond}
then we can show $V_1\subset TU$.
Here is the proof:
Suppose $y\in V_1$.
Then,
\begin{cd}[row sep={30pt,between origins}, column sep={140pt,between origins}]
y\cap V_1\ne\mt\ar{r}&y\cap\cl{TU_1}\ne\mt\ar{r}&(y+V_2)\cap TU_1\ne\mt\ar{lld}\\
(y+TU_1)\cap V_2\ne\mt\ar{r}&(y+TU_1)\cap\cl{TU_2}\ne\mt\ar{r}&((y+TU_1)+V_3)\cap TU_2\ne\mt\ar{lld}\\
(y+TU_1+TU_2)\cap V_3\ne\varnothing\ar{r}&\quad\cdots.&
\end{cd}
From the first columns, and by the conditions (1) and (3), we obtain
\[(y+TU)\cap\bigcap_{n\in\N}V_n\ne\mt.\]
Therefore, the set $y+TU$ contains 0, hence $y\in TU$.

Let us show the existence of such sequences.
At first, take $U_n=2^{-n}U$ for (1).
Then we can take $\{V_n\}_n$ with (2) as we mentioned above.
Simultaneously we can have it satisfy (3) because $Y$ is metrizable.
\end{proof}
\begin{cor}
Let $X$ be metrizable and $Y$ be barreled.
Then, the open mapping theorem holds.
\end{cor}
\begin{pf}
The quotient of metrizable space is also metrizable, so $Y$ is a metrizble barreled space.
\end{pf}
\begin{cor}[The Banach Isomorphy]
A continuous linear bijection onto a metrizable barreled space is a homeomorphism, i.e. topological isomorphism.
\end{cor}
\begin{cor}[The first isomorphism theorem]
Let $T:X\to Y$ be a bounded linear operator between Banach spaces.
Then, the induced map $X/\ker T\to\im T$ is a topological isomorphism.
\end{cor}















\chapter{Locally convex spaces}
\section{Seminorms}
minkowski functional
locally boundedness
polar

\section{The Hahn-Banach theorem}

\section{Weak topology}



















\chapter{Operators on Banach space}
DO NOT contain topics tht can be generalized within Banach algebras or any other operator algebras(e.g. polar decomposition, Gelfand theory, functional calculus, spectral resolution)

\begin{thm}
Let $X$ be complete and $Y$ be complete metrizable.
The range of a continuous operator $T:X\to Y$ is closed if and only if the induced linear isomorphism
\[\frac X{\ker T}\to\im T\]
has a continuous inverse so that it becomes a topological isomorphism.
\end{thm}
\begin{pf}
One direction is easy.

For the other direction, suppose $\im T$ is closed in $Y$.
Note that the metrizability condition of $Y$ is set in order to apply the open mapping theorem.
\end{pf}
\begin{cor}
Let $T:X\to Y$ be a bounded operator between Banach spaces.
Then, $T$ is bounded below if and only if $\im T$ is closed and $T$ is injective.
\end{cor}



\section{Spectral theory}
When a Banach algebra is realized as a concrete operator space, then the spectral theory on it changes drastically.
For example we can categorize three cases for a linear operator between Banach spaces to fail the invertibility:
\begin{cond}
\item it is not injective;\hfill (point spectrum)
\item it is injective, its range is not dense;\hfill (residual spectrum)
\item it is injective, its range is dense, but not closed;\hfill (continuous spectrum)
\end{cond}



\section{Compact operators}

\section{Unbounded operators}

\section{Nuclear operators}

\section{Fredholm theory}

\section{Perturbation theory}







\end{document}














