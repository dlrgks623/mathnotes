\documentclass{../crs}
\usepackage{../../ikany}

\title{Analysis 5 : Functional Analysis}

\begin{document}
\maketitle
\tableofcontents





\chapter{Topological vector spaces}

\section{Elementary properties}
definition - how to use the continuity of vector space operations effectively
homeomorphism by translation and dialation: local base at 0
uniformity pseudometrics, basic classification
translation invariant metric
completely regular (up to 3.5)
boundedness and continuity


\section{Classification}
\begin{rd}[row sep={50pt,between origins}, column sep={60pt,between origins}]
& inner product \ar{d} &&& Hilbert space \ar{d} & \\
& \block{finite\\seminorms\\(= normable)} \ar{d} &&& Banach space \ar{d} & \\
& \block{countable\\seminorms} \ar{ld}\ar{rd} &&& Fr\'echet space \ar{ld}\ar{rd} & \\
  \block{countable\\pseudometrics\\(= metrizable)} \ar{rd} &
& \block{uncountable\\seminorms\\(= locally convex)} \ar{ld} &
  F-space \ar{rd} && (no name) \ar{ld} \\
& \block{uncountable\\pseudometrics} &&& (no name) &
\end{rd}

\begin{prop}
Let $\rho$ be a pseudometric.
Then,
\[B(0,1)\subset\frac{B(0,1)+B(0,1)}2\subset\frac12B(0,2).\]
If $\rho$ is a seminorm, then the equalities hold.
\end{prop}
I say this as $\frac12B(0,2)$ is ``fatter'' than $B(0,1)$.







\section{Barreled spaces and bornological spaces}

\subsection{The Baire category theorem}

\subsection{Uniform boundedness principle}
\begin{thm}[Uniform boundedness principle]
Let $X$ be a barreled space and $Y$ be a topological vector space.
Let $\cF\subset B(X,Y)$.
If $\cF$ is pointwise bounded, then $\cF$\ is equicontinuous.
\end{thm}



\subsection{Open mapping theorem}
\begin{thm}[Open mapping theorem]
Let $X$ be a topological vector space and $Y$ be a metrizable barreled space.
Let $T\colon X\to Y$ be linear.
If $T$ is surjective and continuous, then $T$ is open.
\end{thm}
\begin{proof}
If we let $U$ be an open neighborhood in $X$, then we want to show $TU$ is a neighborhood.
Because $T$ is surjective so that $\cl{TU}$ is absorbent, $\cl{TU}$ is a neighborhood.
Note that an open set intersects $\cl{TU}$ also intersects $TU$.

If there exist two sequences of balanced open neighborhoods $U_n\subset X$ and $V_n\subset Y$ with
\begin{cond}
\item $U_1+\cdots+U_n\subset U$,
\item $V_n\subset\cl{TU_n}$,
\item $\bigcap_{n\in\N}V_n=\{0\}$,
\end{cond}
then we can show $V_1\subset TU$.
Here is the proof:
Suppose $y\in V_1$.
Then,
\begin{cd}[row sep={30pt,between origins}, column sep={140pt,between origins}]
y\cap V_1\ne\mt\ar{r}&y\cap\cl{TU_1}\ne\mt\ar{r}&(y+V_2)\cap TU_1\ne\mt\ar{lld}\\
(y+TU_1)\cap V_2\ne\mt\ar{r}&(y+TU_1)\cap\cl{TU_2}\ne\mt\ar{r}&((y+TU_1)+V_3)\cap TU_2\ne\mt\ar{lld}\\
(y+TU_1+TU_2)\cap V_3\ne\varnothing\ar{r}&\quad\cdots.&
\end{cd}
From the first columns, and by the conditions (1) and (3), we obtain
\[(y+TU)\cap\bigcap_{n\in\N}V_n\ne\mt.\]
Therefore, the set $y+TU$ contains 0, hence $y\in TU$.

Let us show the existence of such sequences.
At first, take $U_n=2^{-n}U$ for (1).
Then we can take $\{V_n\}_n$ with (2) as we mentioned above.
Simultaneously we can have it satisfy (3) because $Y$ is metrizable.
\end{proof}
\begin{cor}
Let $X$ be metrizable and $Y$ be barreled.
Then, the open mapping theorem holds.
\end{cor}
\begin{pf}
The quotient of metrizable space is also metrizable, so $Y$ is a metrizble barreled space.
\end{pf}
\begin{cor}[The Banach Isomorphy]
A continuous linear bijection onto a metrizable barreled space is a homeomorphism, i.e. topological isomorphism.
\end{cor}
\begin{cor}[The first isomorphism theorem]
Let $T:X\to Y$ be a bounded linear operator between Banach spaces.
Then, the induced map $X/\ker T\to\im T$ is a topological isomorphism.
\end{cor}















\chapter{Locally convex spaces}
\section{Seminorms}
minkowski functional
locally boundedness
polar

\section{The Hahn-Banach theorem}

\section{Weak topology}



















\chapter{Operators on Hilbert spaces}
DO NOT contain topics that can be generalized within Banach algebras or any other operator algebras(e.g. polar decomposition, Gelfand theory, functional calculus, spectral resolution)

\begin{thm}
Let $X$ be complete and $Y$ be complete metrizable.
The range of a continuous operator $T:X\to Y$ is closed if and only if the induced linear isomorphism
\[\frac X{\ker T}\to\im T\]
has a continuous inverse so that it becomes a topological isomorphism.
\end{thm}
\begin{pf}
One direction is easy.

For the other direction, suppose $\im T$ is closed in $Y$.
Note that the metrizability condition of $Y$ is set in order to apply the open mapping theorem.
\end{pf}
\begin{cor}
Let $T:X\to Y$ be a bounded operator between Banach spaces.
Then, $T$ is bounded below if and only if $\im T$ is closed and $T$ is injective.
\end{cor}



\section{Spectral theory}


\subsection{Closed operators}
\begin{defn}
An operator $A$ is said to be \emph{closable} if
\[x_n\text{ and }Ax_n\text{ are Cauchy}\impl\lim_{n\to\oo}Ax_n=A\lim_{n\to\oo}x_n.\]
Note that the opposite direction is always true.
\end{defn}
\subsubsection{Properties of closed operators}
For closed operators, we introduce a new norm. 
\begin{thm}
Let $A,B$ be closed operators between Banach spaces.
Then, $A+B$ is closed iff
\[\|Ax\|+\|Bx\|\les\|(A+B)x\|+\|x\|\]
for $x\in D(A)\cap D(B)$, i.e. $A$ and $B$ are $A+B$-bounded.
It is paraphrased by
\[\|x\|_A+\|x\|_B\sim\|x\|_{A+B}.\]
\end{thm}
\begin{pf}
($\Leftarrow$) Suppose $(x_n,(A+B)x_n)$ is Cauchy.
Then, the inequality gives that $Ax_n$ and $Bx_n$ are Cauchy.
Since $A$ and $B$ are closed, we have $\lim Ax_n=A\lim x_n$ and $\lim Bx_n=B\lim x_n$.
So $\lim(A+B)x_n=\lim Ax_n+\lim Bx_n=A\lim x_n+B\lim x_n=(A+B)\lim x_n$.

($\Rightarrow$)
\end{pf}
\begin{thm}
Let $A$ be a closed, and $B$ be a closable operator between Banach spaces with $D(A)\subset D(B)$.
Then, $A+B$ is closed if
\[\|Bx\|\le\alpha\|Ax\|+c\|x\|\]
for some $\alpha<1$.
\end{thm}
\begin{pf}
\[\|Ax\|\le\|(A+B)x\|+\|Bx\|\le\|(A+B)x\|+\alpha\|Ax\|+c\|x\|\]
implies
\[\|Ax\|\les\|(A+B)x\|+\|x\|.\]

\end{pf}

\begin{prop}[Closed graph theorem]
For $T\in D_{cl}(X,Y)$,
\[T\text{ is unbounded}\iff T\text{ is not everywhere defined}.\]
\end{prop}

Closed operators,
\begin{cond}
\item provide with the optimal extended domain for adjoint operators,
\item have maximal essential domains,
\item are closed under invertibility,
\item do not distinguish everywhere defined denslely defined, since everywhere definedness is equivalent to boundedness.
\end{cond}


\subsubsection{Decomposition of spectrum for closed operators}

When a Banach algebra is realized as a concrete operator space, then the spectral theory on it changes drastically.

Note that since decomposition of spectrum is orginated for application to quantum mechanics, this traditional definition is usually for closed operators.
Even though the following definitions can be applied for non-closable operators, but it does not make sense in any senses.
So, every operator in this subsection is assumed to be \emph{closed}.

Let $X=Y$ in order to see $L(X,Y)$ as a ring.
Let $B(X)\subset D(X)\subset L(X)$ be the spaces of \emph{everywhere defined operators, densely defined operators, and just linear operators} respectively.
Note that $D(X)$ is not a vector space.
For $T\in L(X)$,
\[\lambda\begin{cases}\text{is in }\rho(T)\\\text{is in }\sigma_c(T)\\\text{is in }\sigma_r(T)\\\text{is in }\sigma_p(T)\end{cases}\qquad\textit{iff}\qquad R_\lambda(T)\begin{cases}\in B(X)\\\in D(X)\setminus B(X)\\\in L(X)\setminus D(X)\\\text{cannot be defined.}\end{cases}.\]

Discrete spectrum is defined to consist of scalars having finite dimensional eigenspace and is isolated from any other elements in spectrum.
\clearpage
\subsection{Densly defined operators}
\subsubsection{Adjoint}
Adjoint is defined for densely defined operators:
For Banach spaces, we have
\[\adj:D(X,Y)\to L_{cl}(Y^*,X^*)\]
that is not injetcive. (I don't know it's surjective)

For reflexive $Y$, we have
\[\adj:D_{cl}(X,Y)\to D_{cl}(Y^*,X^*)\]
that is inj? surj?

For reflexive $X$, we have
\[adj:D_{closable}(X)\Rightarrow D_{cl}(X^*).\]
For $f:X\to Y$, ``I'' define the predicate $f:A\Rightarrow B$ by
\[f(A)=B\quad\text{and}\quad A=f^{-1}(B).\]

\begin{thm}
The adjoint $B_{cl}(H)\to{\sim}B_{cl}(H)$ can be extended to $D_{cl}(H)\to{\sim}D_{cl}(H)$.
\end{thm}
\begin{thm}
For $T\in D_{cl}(H)$, $H=\ker T\dsum\cl{\im T^*}$.
\end{thm}
The space $D_{cl}$ is optimized when we think adjoints for reflexive spacse.

unitarily equivalence can defined for $T_1\in L(H_1)$ and $T_2\in L(H_2)$.


\subsection{Self-adjoint operators}
\begin{defn}
Let $T\in L(H)$ be satisfy $T\subset T^*$, i.e. $\inn{Tx,y}=\inn{x,Ty}$ for all $x,y\in D(T)$.
Then, we have definitions by the following diagram:
\begin{rd}
&& Bounded self-adjoint \ar{d}\\
Hermitian \rds{r}{densely defined}\lds{rru}{everywhere defined} & Symmetric \ar{l}\rds{r}{$D(T)=D(T^*)$} & Self-adjoint \ar{l}
\end{rd}
\end{defn}
\begin{prop}
Hermitian iff the numerical range is in $\R$.
\end{prop}
\begin{prop}
A symmetric operator is closable.
\end{prop}
\begin{pf}
Since $T$ is dense and $T\subset T^*$, $T^*$ is dense.
Therefore, $T$ is closable.
\end{pf}


\section{Compact operators}

\section{Nuclear operators}



\chapter{Operator algebra}

\section{The Gelfand-Naimark theorems}
\begin{thm}
If $x,y$ commutes, then $\sigma(xy)\subset\sigma(x)\sigma(y)$.
\end{thm}
\subsection{Commutative Banach algebras}

The Gelfand representation $A\to C_0(\hat A)$ can be defined for commutative Banach algebras.

\begin{defn}
A \emph{symmetric} Banach algebra is an involutive Banach algebra for which the Gelfand representation preserves the involution.
We will not consider non-symmetric involutive Banach algebras in this section.
\end{defn}
Notice the following implication:
\begin{cd}
& \text{symmetirc Banach algebra} \ar[dr] &\\
\text{$C^*$-algebra} \ar[ur]\ar[dr]&& \text{Banach algebra}.\\
& \text{semisimple Banach algebra} \ar[ur] &
\end{cd}

Let $A$ be a commutative Banach algebra.
\begin{thm}
If $A$ is semisimple, then the Gelfand representation is a monomorphism; it is injective.
\end{thm}
\begin{pf}
It is because the kernel is given by the Jacobson radical.
\end{pf}
\begin{thm}
If $A$ is symmetric, then the Gelfand representation is an epimorphism; it has a dense range.
\end{thm}
\begin{pf}
The image is closed under all operations except involution, separates points, and vanishes nowhere.
If $A$ is symmetric, then the image is closed under involution.
Thus, by the Stone-Weierstrass theorem, we get the result.
\end{pf}

$C^*$-algebras are semisimple and symmetric (even if it is noncommutative).
\begin{thm}
A $C^*$-algebra is semisimple.
\end{thm}
\begin{thm}
A $C^*$-algebra is symmetric.
\end{thm}
\begin{pf}[1]
It is by Arens.
\end{pf}
\begin{pf}[2]
It is by Fukamiya.
\end{pf}
Furethermore,
\begin{thm}
If $A$ is a commutative $C^*$-algebra, then the Gelfand representation is isometric.
\end{thm}
Since an isometry is injective and has a closed range, therefore, it should be isometric $^*$-isomorphism.






\end{document}














