\documentclass{../crs}
\usepackage{../../ikany}

\title{Functional Analysis II : Operator Algebra}

\begin{document}
\maketitle
\tableofcontents


\chapter{Banach algebras}

\section{Spectra}

\section{Gelfand theory}

\begin{thm}
If $x,y$ commutes, then $\sigma(xy)\subset\sigma(x)\sigma(y)$.
\end{thm}


\chapter{Operators on Hilbert spaces}

\section{Spectral theory}

\chapter{$C^*$-algebras}
\section{Basics}




\section{The Gelfand-Naimark theorems}
\subsection{Commutative Banach algebras}

The Gelfand representation $A\to C_0(\hat A)$ can be defined for commutative Banach algebras.

\begin{defn}
A \emph{symmetric} Banach algebra is an involutive Banach algebra for which the Gelfand representation preserves the involution.
We will not consider non-symmetric involutive Banach algebras in this section.
\end{defn}
Notice the following implication:
\begin{cd}
& \text{symmetirc Banach algebra} \ar[dr] &\\
\text{$C^*$-algebra} \ar[ur]\ar[dr]&& \text{Banach algebra}.\\
& \text{semisimple Banach algebra} \ar[ur] &
\end{cd}

Let $A$ be a commutative Banach algebra.
\begin{thm}
If $A$ is semisimple, then the Gelfand representation is a monomorphism; it is injective.
\end{thm}
\begin{pf}
It is because the kernel is given by the Jacobson radical.
\end{pf}
\begin{thm}
If $A$ is symmetric, then the Gelfand representation is an epimorphism; it has a dense range.
\end{thm}
\begin{pf}
The image is closed under all operations except involution, separates points, and vanishes nowhere.
If $A$ is symmetric, then the image is closed under involution.
Thus, by the Stone-Weierstrass theorem, we get the result.
\end{pf}

$C^*$-algebras are semisimple and symmetric (even if it is noncommutative).
\begin{thm}
A $C^*$-algebra is semisimple.
\end{thm}
\begin{thm}
A $C^*$-algebra is symmetric.
\end{thm}
\begin{pf}[1]
It is by Arens.
\end{pf}
\begin{pf}[2]
It is by Fukamiya.
\end{pf}
Furethermore,
\begin{thm}
If $A$ is a commutative $C^*$-algebra, then the Gelfand representation is isometric.
\end{thm}
Since an isometry is injective and has a closed range, therefore, it should be isometric $^*$-isomorphism.


\chapter{Von Neumann algebras}






\end{document}
