\documentclass{../crs}
\usepackage{../../ikany}

\title{Real Analysis II : Fourier Analysis}

\begin{document}
\maketitle
\tableofcontents

\chapter{Basic techniques}




\section{Interpolation}



\subsection{The distribution function}
\begin{defn}
Let $f$ be a measurable function on a measure space $(X,\mu)$.
The \emph{distribution function} $\lambda_f:[0,\infty)\to [0,\infty)$ is defined as:
\[\lambda_f(\alpha)=\mu(\{x:|f(x)|>\alpha\}).\]
\end{defn}


Do not use $\mu(\{x:|f(x)|\ge\alpha\})$.
The strict inequality implies the \emph{lower semi-continuity} of $\lambda_f$.

\begin{thm}[Fubini]
Denote $f_h=\chi_{|f|>\alpha}f$ and $f_l=\chi_{|f|\le\alpha}f$.
For $p>0$, we have
\[\|f\|_p^p=\int p\alpha^{p-1}\lambda_f(\alpha)\,d\alpha.\]
For $p>0$ and $n>0$, we have
\[\|f\|_{p+n}^{p+n}=\int p\alpha^{p-1}\|f_h\|_n^n\,d\alpha.\]
For $p<0$ and $n>0$, we have
\[\|f\|_{p+n}^{p+n}=\int |p|\alpha^{p-1}\|f_l\|_n^n\,d\alpha.\]
\end{thm}
\begin{thm}
For $p>1$, by the Chebyshev inequality, we have
\[\sup_\alpha\alpha^p\lambda_f(\alpha)\le\int p\alpha^{p-1}\lambda_f(\alpha)\,d\alpha.\]
In other words, $\|f\|_{p,\infty}\le\|f\|_p$.
\end{thm}





\subsection{Real interpolation}


\begin{thm}[Marcinkiewicz interpolation]
Let $X$ be a $\sigma$-finite measure space and $Y$ be a measure space.
Let $1<p_0<p<p_1<\infty$.
Let $T\colon L^{p_0}(X)+L^{p_1}(X)\to M(Y)$ be a sublinear operator.
If $T$ has a weak type estimate
\[\|T\|_{p_0\to p_0,\infty},\,\|T\|_{p_1\to p_1,\infty}<\infty,\]
then
\[\|T\|_{p\to p}<\infty.\]
\end{thm}
\begin{pf}
Let $f\in L^p$ and denote $f_h=\chi_{|f|>\alpha}f$ and $f_l=\chi_{|f|\le\alpha}f$.
It is easy to show $f_h\in L^{p_0}$ and $f_l\in L^{p_1}$.
Then,
\begin{align*}
\|Tf\|_p^p&\sim\int\alpha^{p-1}\lambda_{Tf}\,d\alpha\\
&\lesssim\int\alpha^{p-1}\lambda_{Tf_h}\,d\alpha+\int\alpha^{p-1}\lambda_{Tf_l}\,d\alpha\\
&\le\int\alpha^{p-1}\frac1{\alpha^{p_0}}\|Tf_h\|_{p_0,\infty}^{p_0}\,d\alpha+\int\alpha^{p-1}\frac1{\alpha^{q_1}}\|Tf_l\|_{p_1,\infty}^{p_1}\,d\alpha\\
&\lesssim\int\alpha^{p-p_0-1}\|f_h\|_{p_0}^{p_0}\,d\alpha+\int\alpha^{p-p_1-1}\|f_l\|_{p_1}^{p_1}\,d\alpha\\
&\sim\|f\|_p^p.
\end{align*}
by (1) Fubini, (2) Sublinearlity, (3) Chebyshev, (4) Boundedness, (5) Fubini.
\end{pf}

\begin{thm}[Hadamard's three line lemma]
Let $f$ be a bounded holomorphic function on the vertical unit stripe $\{z:0<\Re z<1\}$.
Then, for $0<\theta<1$,
\[\|f\|_{L^\infty(\Re=\theta)}\le\|f\|_{L^\infty(\Re=0)}^{1-\theta}\|f\|_{L^\infty(\Re=1)}^\theta.\]
\end{thm}
\begin{pf}
Define
\[g(z):=\frac{f(z)}{\|f\|_{L^\infty(\Re=0)}^{1-z}\|f\|_{L^\infty(\Re=1)}^z},\qquad g_n(z)=g(z)e^\frac{z^2-1}n.\]
Then we have
\begin{cond}
\item $g_n\to g$ pointwisely as $n\to\infty$,
\item $g_n(z)\to0$ uniformly as $\Im z\to\infty$.
\end{cond}
The second one is because $g$ is bounded and for $z=x+yi$ we have
\[|g_n(z)|\lesssim|e^\frac{z^2-1}n|=e^{\Re\frac{z^2-1}n}=e^\frac{x^2-y^2-1}n\le e^\frac{-y^2}n.\]

By (1), it is enough to bound $g_n$ for each $n$.
Truncating the stripe, the outer region is controlled by (2) and the interior region is controlled by the maximum modulus principle.
\end{pf}



\subsection{Complex interpolation}
\begin{thm}[Riesz-Thorin interpolation]
Let $X,Y$ be $\sigma$-finite measure spaces.
Let
\[\frac1{p_\theta}=\frac1{p_0}(1-\theta)+\frac1{p_1}\theta,\qquad\frac1{q_\theta}=\frac1{q_0}(1-\theta)+\frac1{q_1}\theta.\]
Then,
\[\|T\|_{p_\theta\to q_\theta}\le\|T\|_{p_0\to q_0}^{1-\theta}\|T\|_{p_1\to q_1}^\theta.\]
\end{thm}
\begin{pf}
Note that
\[\|T\|_{p_\theta\to q_\theta}=\sup_f\frac{\|Tf\|_{q_\theta}}{\|f\|_{p_\theta}}=\sup_{f,g}\frac{\abs{\inn{Tf}{g}}}{\|f\|_{p_\theta}\|g\|_{q'_\theta}}.\]
Consider a holomorphic function
\[z\mapsto\inn{Tf_z}{g_z}=\int\conj{g_z(y)}Tf_z(y)\,dy,\]
where $f_z$ and $g_z$ are defined as
\[f_z=|f|^{\frac{p_\theta}{p_0}(1-z)+\frac{p_\theta}{p_1}z}\frac f{|f|}\]
so that we have $f_\theta=f$ and
\[\|f\|_{p_\theta}^{p_\theta}=\|f_z\|_{p_x}^{p_x}\]
for $\Re z=x$.

Then,
\[\abs{\inn{Tf_z}{g_z}}\le\|T\|_{p_0\to q_0}\|f_z\|_{p_0}\|g_z\|_{q'_0}=\|T\|_{p_0\to q_0}\|f\|_{p_\theta}^{p_\theta/p_0}\|g\|_{q'_\theta}^{q'_\theta/q'_0}\]
for $\Re z=0$, and
\[\abs{\inn{Tf_z}{g_z}}\le\|T\|_{p_1\to q_1}\|f_z\|_{p_1}\|g_z\|_{q'_1}=\|T\|_{p_1\to q_1}\|f\|_{p_\theta}^{p_\theta/p_1}\|g\|_{q'_\theta}^{q'_\theta/q'_1}\]
for $\Re z=1$.
By Hadamard's three line lemma, we have
\[\abs{\inn{Tf_z}{g_z}}\le\|T\|_{p_0\to q_0}^{1-\theta}\|T\|_{p_1\to q_1}^\theta\|f\|_{p_\theta}\|g\|_{q'_\theta}\]
for $\Re z=\theta$.
Putting $z=\theta$ in the last inequality, we get the desired result.
\end{pf}









\section{Maximal function}

We often want to show a net of linear operators $\{T_t\}_t$ is an ``approximate identity'' in the sense of pointwise convergence, not a certain norm; in other words, say, we want to show
\[\lim_{t\to0}T_tf(x)=f(x)\for a.e.\]

If we introduce maximal function $Mf$ defined by
\[Mf(x)=\sup_t|T_tf(x)|\]
and if it satisfies a boundedness, then for a suitable seminorm or a quasinorm $\|\cdot\|$, we can apply the approximation argument
\begin{align*}
\|\lim_t T_tf-f\|&\le \|\lim_t T_t(f-g)\|+\|\lim_t T_tg-g\|+\|g-f\|\\&\le \|M(f-g)\|+\|g-f\|\\&\lesssim \|f-g\|.
\end{align*}

\subsection{The Hardy-Littlewood maximal function}
Hardy-Littlewood maximal function is the most famous maximal function.

\begin{thm}[Hardy-Littlewoord]
\[\|Mf\|_{1,\oo}\le 3^d\|f\|_1.\]
\end{thm}
\begin{pf}
Let $E_\alpha=\{|f|>\alpha\}$.
By the inner regularity of $\mu$, we may assume $E_\alpha$ is compact.
For every $x\in E_\alpha$, we can choose $B_x$ such that
\[\frac1{B_x}\int_{B_x}|f|>\alpha\quad\impl\quad\mu(B_x)<\alpha\int_{B_x}|f|.\]
By the Vitali covering and by the compactness of $E_\alpha$,
\[\lambda_f(\alpha)=\mu(E_\alpha)\le3^d\sum\mu(B_k)\le\frac{3^d}\alpha\sum_k\int_{B_k}|f|\le\frac{3^d}\alpha\|f\|_1.\]
The disjointness is important in showing the constant $3^d$ does not depend on the number of $B_k$'s.
\end{pf}

\begin{defn}
\[f^*(x):=\lim_{r\to0+}\frac1{\mu(B)}\int_B|f(y)-f(x)|\,dy.\]
\end{defn}
\begin{thm}[Lebesgue differentiation]
$f^*=0$ a.e.
\end{thm}
\begin{pf}
Note that $f^*\le Mf+|f|$ implies
\[\|f^*\|_{1,\oo}\le\|Mf\|_{1,\oo}+\|f\|_{1,\oo}\lesssim\|f\|_1.\]
Note that $g^*=0$ for $g\in C_c$.
Approximate using $f^*=(f-g)^*$.
\end{pf}








\section{Convergence of Fourier series}
\begin{defn}
The \emph{Dirichlet kernel} is a function $D_n\colon \bT\to\R$ defined by
\[D_n=\hat{\1_{|k|\le n}},\quad\text{or equivalently,}\quad\hat{D_n}=\1_{|k|\le n}.\]
This is because they are invariant under inverse, in other words, they are even.
\end{defn}

\begin{thm}
\[D_n(x)=\frac{\sin\frac{2n+1}2x}{\sin\frac12x}.\]
\end{thm}
\begin{pf}
\begin{align*}
D_n(x)&=\sum_{k=-n}^ne^{ikx}\\
&=\frac{e^{i\frac{2n+1}2x}-e^{-i\frac{2n+1}2x}}{e^{i\frac12x}-e^{-i\frac12x}}\\
&=\frac{\sin\frac{2n+1}2x}{\sin\frac12x}.
\end{align*}

\end{pf}

\begin{thm}
If $f\in\Lip(\bT)$, then $D_n*f\to f$ pointwisely as $n\to\infty$.
\end{thm}

\begin{thm}
\[\|D_n\|_{L^1(\bT)}\gtrsim\log n.\]
\end{thm}
\begin{pf}
By (2) $\sin x\le x$ for $x\in[0,\pi/2]$, (3) change of variable,
\begin{align*}
\|D_n\|_{L^1(\bT)}
&=\frac1{2\pi}\int_{-\pi}^\pi\abs{\frac{\sin\frac{2n+1}2x}{\sin\frac12x}}dx\\
&\ge\frac2\pi\int_0^\pi\frac{|\sin\frac{2n+1}2x|}x\,dx\\
&=\frac2\pi\int_0^{\frac{2n+1}2\pi}\frac{|\sin x|}x\,dx\\
&=\frac2\pi\sum_{k=0}^{2n}\int_{\frac k2\pi}^{\frac{k+1}2\pi}\frac{|\sin x|}x\,dx\\
&\ge\frac2\pi\sum_{k=0}^{2n}\int_0^{\frac12\pi}\frac{\sin x}{\frac{k+1}2\pi}\,dx\\
&\ge\frac4{\pi^2}\sum_{k=0}^{2n}\frac1{1+k}\\
&\ge\frac4{\pi^2}\log(2n+2).
\end{align*}....
\end{pf}

\begin{defn}
The \emph{Fej\'er kernel} is
\end{defn}
\begin{thm}
\[K_n(x)=\frac1{n+1}\frac{\sin^2\frac{n+1}2x}{\sin^2\frac12x}.\]
\end{thm}
\begin{pf}
Since
\begin{align*}
D_n(x)=
&=\frac{e^{i\frac{2n+1}2x}-e^{-i\frac{2n+1}2x}}{e^{i\frac12x}-e^{-i\frac12x}}\\
&=\frac{[e^{i\frac{2n+1}2x}-e^{-i\frac{2n+1}2x}][e^{i\frac12x}-e^{-i\frac12x}]}{[e^{i\frac12x}-e^{-i\frac12x}]^2}\\
&=\frac{[e^{i(n+1)x}+e^{-i(n+1)x}]-[e^{inx}+e^{-inx}]}{[e^{i\frac12x}-e^{-i\frac12x}]^2},
\end{align*}
by telescoping, we get
\begin{align*}
\sum_{k=0}^nD_k(x)
&=\frac{[e^{i(n+1)x}+e^{-i(n+1)x}]-[e^{i0x}+e^{-i0x}]}{[e^{i\frac12x}-e^{-i\frac12x}]^2}\\
&=\frac{[e^{i\frac{n+1}2x}-e^{-i\frac{n+1}2x}]^2}{[e^{i\frac12x}-e^{-i\frac12x}]^2}\\
&=\frac{\sin^2\frac{n+1}2x}{\sin^2\frac12x}.
\end{align*}
\end{pf}

Two important results from Fej\'er kernel:
\begin{cond}
\item If $f(x-)$, $f(x+)$ exist and $S_nf(x)$ converges, then $S_nf(x)\to\frac12(f(x-)+f(x+))$.
\item (If $f\in L^1(\bT)$, then $\sigma_nf\to f$ a.e.)

\item If $f\in L^1(\bT)$, then $S_nf\to f$ in $L^1$ and $L^2$.
\item If $f$ is continuous and $\hat{f}\in L^1(\Z)$, then $S_nf\to f$ uniformly.
\item Since $\sigma_nf$ is a trigonometric polynomial, the set of trigonometric polynomials are dense in $L^1(\bT)$ and $L^2(\bT)$.
\end{cond}





\chapter{Differentiation theory}






\chapter{Calderon-Zygmund theory}



\chapter{Littlewood-Paley theory}










\end{document}