\documentclass{../crs}
\usepackage{../../ikany}

\title{Real Analysis I : Measure Theory}

\begin{document}
\maketitle
\tableofcontents

\chapter{Measure spaces}


\chapter{Riesz spaces}



\chapter{Topological measures}

\section{Radon measures}
In this section, we assume every base space is locally compact Hausdorff.
In locally compact Hausdorff spaces, compact finiteness and locally finiteness are equivalent.
\begin{defn}
A \emph{Radon measure} is a Borel measure which satisfies the following three conditions:
\begin{cond}
\item outer regular on all Borel sets,
\item inner regular on all open sets,
\item locally finite.
\end{cond}
\end{defn}

Radon measures are rather simply characterized when the base space is $\sigma$-compact.
\begin{thm}
A Radon measure is inner regular on all $\sigma$-finite Borel sets.
\end{thm}
\begin{pf}
Let $E$ be a Borel set with $\mu(E)<\oo$.
By outer regularity, there is an open set $U\supset E$ such that
\[\mu(U)<\mu(E)+\frac\e2.\]
Then,
\[\mu(U\setminus E)<\frac\e2.\]
By outer regularity, there is an open set $V\supset U\setminus E$ such that
\[\mu(V)<\mu(U\setminus E)+\frac\e2.\]

By inner regularity, there is a compact set $K\subset U$ such that
\[\mu(K)\frac\e2>\mu(U).\]
Then, we have $K\setminus V\subset E$ and
\[\mu(K\setminus V)=blabla\]
\end{pf}

\begin{center}
\begin{tikzcd}[column sep={120pt,between origins}]
\begin{tabular}{c}\text{locally finite}\\\text{Borel regular}\end{tabular}\ar{r} &
\text{\quad Radon\quad} \ar{r}\ar[dashed, bend right]{l}{\scriptsize X\text{ is $\sigma$-compact}} &
\begin{tabular}{c}\text{locally finite}\\\text{Borel}\end{tabular} \ar[dashed, bend left]{ll}{\scriptsize X\text{ is second countable}}
\end{tikzcd}
\end{center}

\[L_{\text{loc}}^1=\text{absolutely continuous measures}\subset\text{Radon measures}\subset\cD'.\]

\begin{cor}
If $X$ is $\sigma$-compact, then a compact finite Borel measure is Radon if and only if it is regular.
\end{cor}

\begin{thm}
If every open set in $X$ is $\sigma$-compact, then every locally finite Borel measure is regular.
\end{thm}
\begin{prop}
In a second countable space, every open set is $\sigma$-compact.
\end{prop}




\section{The Riesz-Markov-Kakutani theorem}
In this section, we always assume $X$ is a locally compact Hausdorff space.
Hence we can use the Urysohn lemma: If $K$ is compact and $F$ is closed, then we can find a continuous function $f:X\to[0,1]$ such that $f|_K=1$ and $f|_F=0$.

There are two Riesz-Markov-Kakutani theorems: the first theorem describes the positive elements in $C_c(X)^*$ as Radon measures when LF topology is assumed, and the second theorem describes $C_c(X)^*$ as the space of finite Radon measures when uniform topology is assumed.

\subsection{The first theorem}
Positivity of linear functional itself implies a rather strong continuity property.
\begin{thm}
Let $C_c(X)$ be a space of compactly supported continuous functions on $X$.
(Give an LF topology with a directed inductive family $C_K(X)$.)
If a linear functional $I$ is positive, then continuous with respect to the topology.
\end{thm}
\begin{pf}
Let $K$ be a compact subset.
We want to show $|I(f)|\lesssim\|f\|$ for $f\in C_K(X)$.
The proof idea comes from $|\int_K f\,d\mu|\le\mu(K)\|f\|$.

Choose $\phi\in C_c(X)$ such that $\phi|_K=1$.
\end{pf}





Jordan decomposition: $(C_0(X),u)^*\subset(C_c(X),LF)^*$
converse?




\chapter{Hmmmm}

\subsection{Convergence in measure}
Since $\{f_n(x)\}_n$ diverges if and only if
\[\exists k>0,\quad\forall n_0>0.\quad\exists n>n_0:\quad |f_n(x)-f(x)|>n^{-1},\]
we have
\begin{align*}
\{x:\{f_n(x)\}_n\text{ diverges}\}&=\bigcup_{k>0}\bigcap_{n_0>0}\bigcup_{n>n_0}\{x:|f_n(x)-f(x)|>n^{-1}\}\\
&=\bigcup_{k>0}\limsup_n\{x:|f_n(x)-f(x)|>n^{-1}\}.
\end{align*}
Since for every $k$
\[\limsup_n\{x:|f_n(x)-f(x)|>k^{-1}\}\subset\limsup_n\{x:|f_n(x)-f(x)|>n^{-1}\},\]
we have
\[\{x:\{f_n(x)\}_n\text{ diverges}\}\subset\limsup_n\{x:|f_n(x)-f(x)|>n^{-1}\}.\]




\begin{thm}
Let $f_n$ be a sequence of measurable functions on a measure space $(X,\mu)$.
If $f_n$ converges to $f$ in measure, then $f_n$ has a subsequence that converges to $f$ $\mu$-a.e.
\end{thm}
\begin{pf}
Since $d_{f_n-f}(1/k)\to0$ as $n\to\infty$, we can extract a subsequence $f_{n_k}$ such that
\[\mu(\{x:|f_{n_k}(x)-f(x)|>k^{-1}\})>2^{-k}.\]
Since
\[\sum_{k=1}^\infty\mu(\{x:|f_{n_k}(x)-f(x)|>k^{-1}\})<\infty,\]
by the Borel-Canteli lemma, we get
\[\mu(\limsup_k\{x:|f_{n_k}(x)-f(x)|>k^{-1}\})=0.\]
Therefore, $f_{n_k}$ converges $\mu$-a.e.
\end{pf}




\end{document}