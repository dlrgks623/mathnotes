\documentclass{../crs}
\usepackage{../../ikany}

\title{Analysis III : Measure Theory}

\begin{document}
\maketitle
\tableofcontents

\chapter{Carath\'eodory's theory}


\chapter{Riesz spaces}



\chapter{Topological measures}

\section{Descriptive set theory}





\section{Borel measures}


\subsection{Regular measures}

\begin{thm}
A Borel measure is inner regular on $\sigma$-bounded sets if and only if outer regular on $\sigma$-bounded sets.
\end{thm}



\subsection{Radon measures}

Locally compact Hausdorff spaces have at least two important applications in abstract analysis related to measure theory: one is locally compact groups and the associated Harr measures in abstract harmonic analysis, the other is the Gelfand-Naimark theorem which states every commutative $C^*$-algebra can be represented as a function space on a locally compact Hausdorff space.
In the set of this section, we assume every base space $X$ is locally compact Hausdorff.

Note that locally finite measures are compact finite but the converser holds only if in locally compact Hausdorff spaces.
We want to consider locally finite Borel measures as the minimally compatible measures with a given topology on $X$.
For locally finite Borel measures, a set is finite-measured if and only if it is contained in a compact set.
\begin{defn}
(Folland's)A \emph{Radon measure} is a Borel measure on $X$ which satisfies the following three conditions:
\begin{cond}
\item locally finite,
\item outer regular on all Borel sets,
\item inner regular on all open sets.
\end{cond}
\end{defn}

Radon measures are rather simply characterized when the base space $X$ is $\sigma$-compact.
The following proposition proves the equivalence between regularity and Radonness of locally finite Borel measure on a $\sigma$-compact space.
\begin{prop}
A Radon measure is inner regular on all $\sigma$-finite Borel sets.(Folland's)
\end{prop}
\begin{pf}
First we approximate Borel sets of finite measure, with compact sets.
Let $E$ be a Borel set with $\mu(E)<\oo$ and $U$ be an open set containing $E$.
By outer regularity, there is an open set $V\supset U-E$ such that
\[\mu(V)<\mu(U-E)+\frac\e2.\]
By inner regularity, there is a compact set $K\subset U$ such that
\[\mu(K)>\mu(U)-\frac\e2.\]
Then, we have a compact set $K-V\subset K-(U-E)\subset E$ such that
\begin{align*}
\mu(K-V)&\ge\mu(K)-\mu(V)\\
&>\left(\mu(U)-\frac\e2\right)-\left(\mu(U-E)+\frac\e2\right)\\
&\ge\mu(E)-\e.
\end{align*}
It implies that a Radon measure is inner regular on Borel sets of finite measures.

Suppose $E$ is a $\sigma$-finite Borel set so that $E=\bigcup_{n=1}^\oo E_n$ with $\mu(E_n)<\oo$.
We may assume $E_n$ are pairwise disjoint.
Let $K_n$ be a compact subset of $E_n$ such that
\[\mu(K_n)>\mu(E_n)-\frac\e{2^n},\]
and define $K=\bigcup_{n=1}^\oo K_n\subset E$.
Then,
\[\mu(K)=\sum_{n=1}^\oo\mu(K_n)>\sum_{n=1}^\oo\left(\mu(E_n)-\frac\e{2^n}\right)=\mu(E)-\e.\]
Therefore, a Radon measure is inner regular on all $\sigma$-finite Borel sets.
\end{pf}
We get a corollary:
\begin{cor}
If $X$ is $\sigma$-compact, then a locally finite Borel measure is Radon if and only if it is regular.
\end{cor}

\begin{thm}
If every open set in $X$ is $\sigma$-compact(i.e. Borel sets and Baire sets coincide), then every locally finite Borel measure is regular.
\end{thm}
\begin{prop}
In a second countable space, every open set is $\sigma$-compact(i.e. Borel sets and Baire sets coincide).
\end{prop}


Two corollaries are presented as follows:
\begin{rd}[column sep={120pt,between origins}]
\block{locally finite \\ Borel regular} \ar{r} &
\block{Radon}   \ar{r} \rds{l}{$X$ is $\sigma$-compact} &
\block{locally finite \\ Borel}  \lds{ll}{$X$ is second countable}
\end{rd}


Many applications assume $X$ is an open subset of a Euclidean space, so $X$ is usually second countable.
In this case, the followings will be synonym: A measure is
\begin{cond}
\item 
\end{cond}



\[L_{\text{loc}}^1=\text{absolutely continuous measures}\subset\text{Radon measures}\subset\cD'.\]


\begin{thm}
Every finite Radon measure is regular.
\end{thm}

\section{Riesz-Markov-Kakutani representation theorem}
In this section, we always assume $X$ is a locally compact Hausdorff space.
Hence we can use the Urysohn lemma in the following way: If a compact subset $K$ and a closed subset $F$ are disjoint, then by applying the Urysohn lemma on a compact neighborhood of $K$, we can find a continuous function $f:X\to[0,1]$ such that $f|_K=1$ and $f|_F=0$.
In particular, there always exists a ``continuous characteristic function'' $\phi\in C_c(X)$ with $\phi|_K=1$.

There are two Riesz-Markov-Kakutani theorems: the first theorem describes the positive elements in $C_c(X)^*$ as Radon measures when the natural colimit topology is assumed, and the second theorem describes $C_c(X)^*$ as the space of finite Radon measures when uniform topology is assumed.

\subsection{The first theorem}
Positivity of linear functional itself implies a rather strong continuity property.
\begin{thm}
Let $X$ be a locally compact Hausdorff space.
If a linear functional on $C_c(X)$ is positive, then it is continuous with respect to the colimit topology.
\end{thm}
\begin{pf}
Let $I$ be a positive linear functional on $C_c(X)$.
We want to show that on every compact subset $K$ of $X$ we have $|I(f)|\les_K\|f\|$ for all $f\in C_K(X)$.
The proof idea comes from the H\"older inequality $|\int_K f\,d\mu|\le\mu(K)\|f\|$.

Choose $\phi\in C_c(X)$ such that $\phi|_K=1$ using the Urysohn lemma.
Since $f=\phi f\le I(\phi)\|f\|$, we have
\[I(f)\le I(\phi\|f\|)=I(\phi)\|f\|.\]
Putting $-f$ instead of $f$, we also get $-I(f)\le I(\phi)\|f\|$.
Therefore, $|I(f)|\le I(\phi)\|f\|$.
\end{pf}





Jordan decomposition: $(C_0(X),\|\cdot\|)^*=(C_c(X),\|\cdot\|)^*\subset(C_c(X),\lim_{\to})^*$
converse?




\chapter{Hmmmm}

\subsection{Convergence in measure}
Since $\{f_n(x)\}_n$ diverges if and only if
\[\exists k>0,\quad\forall n_0>0.\quad\exists n>n_0:\quad |f_n(x)-f(x)|>n^{-1},\]
we have
\begin{align*}
\{x:\{f_n(x)\}_n\text{ diverges}\}&=\bigcup_{k>0}\bigcap_{n_0>0}\bigcup_{n>n_0}\{x:|f_n(x)-f(x)|>n^{-1}\}\\
&=\bigcup_{k>0}\limsup_n\{x:|f_n(x)-f(x)|>n^{-1}\}.
\end{align*}
Since for every $k$
\[\limsup_n\{x:|f_n(x)-f(x)|>k^{-1}\}\subset\limsup_n\{x:|f_n(x)-f(x)|>n^{-1}\},\]
we have
\[\{x:\{f_n(x)\}_n\text{ diverges}\}\subset\limsup_n\{x:|f_n(x)-f(x)|>n^{-1}\}.\]




\begin{thm}
Let $f_n$ be a sequence of measurable functions on a measure space $(X,\mu)$.
If $f_n$ converges to $f$ in measure, then $f_n$ has a subsequence that converges to $f$ $\mu$-a.e.
\end{thm}
\begin{pf}
Since $d_{f_n-f}(1/k)\to0$ as $n\to\infty$, we can extract a subsequence $f_{n_k}$ such that
\[\mu(\{x:|f_{n_k}(x)-f(x)|>k^{-1}\})>2^{-k}.\]
Since
\[\sum_{k=1}^\infty\mu(\{x:|f_{n_k}(x)-f(x)|>k^{-1}\})<\infty,\]
by the Borel-Canteli lemma, we get
\[\mu(\limsup_k\{x:|f_{n_k}(x)-f(x)|>k^{-1}\})=0.\]
Therefore, $f_{n_k}$ converges $\mu$-a.e.
\end{pf}




\end{document}