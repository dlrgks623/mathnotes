\documentclass{../crs}
\usepackage{../../ikany}

\title{Differential Geometry}

\begin{document}
\maketitle
\tableofcontents

\chapter{Manifolds}

\section{Vector bundle}


\section{Differentiable manifold}

\subsection{Manifold and Atlas}
\begin{defn}
A \emph{locally Euclidean space} $M$ of dimension $m$ is a Hausdorff topological space $M$ for which each point $x\in M$ has a neighborhood $U$ homeomorphic to an open subset of $\Bbb{R}^d$.
\end{defn}
\begin{defn}
A \emph{manifold} is a locally Euclidean space satisfying the one of following equivalent conditions: second countability, blabla%
\end{defn}

\begin{defn}
A \emph{chart} or a \emph{coordinate system} for a locally Euclidean space is a map $\varphi$ is a homeomorphism from an open set $U\subset M$ to an open subset of $\Bbb{R}^d$.
A chart is often written by a pair $(U,\varphi)$.
\end{defn}

\begin{defn}
An \emph{atlas} $\mathcal{F}$ is a collection $\mathcal{F}=\{(U_\alpha,\varphi_\alpha)\mid\alpha\in A\}$ of charts on $M$ such that $\bigcup_{\alpha\in A} U_\alpha=M$.
\end{defn}


\begin{defn}
A \emph{differentiable maifold} is a manifold on which a differentiable structure is equipped.
\end{defn}
The definition of differentiable structure will be given in the next subsection.
Actually, a differentiable structure can be defined for a locally Euclidean space.



\subsection{Definition of Differentiable Structure}


\begin{defn}
An atlas $\mathcal{F}$ is called \emph{differentiable} if any two charts $\varphi_\alpha,\varphi_\beta\in\mathcal{F}$ is \emph{compatible}: each \emph{transition function} $\tau_{\alpha\beta}\colon\varphi_\alpha(U_\alpha\cap U_\beta)\to\varphi_\beta(U_\alpha\cap U_\beta)$ which is defined by $\tau_{\alpha\beta}=\varphi_\beta\circ\varphi_\alpha^{-1}$ is differentiable.
\end{defn}
It is called a \emph{gluing condition}.

\begin{defn}
For two differentiable atlases $\mathcal{F},\mathcal{F}'$, the two atlases are \emph{equivalent} if $\mathcal{F}\cup\mathcal{F}'$ is also differentiable.
\end{defn}

\begin{defn}
An differentiable atlas $\mathcal{F}$ is called \emph{maximal} if the following holds:
if a chart $(U,\varphi)$ is compatible to all charts in $\mathcal{F}$, then $(U,\varphi)\in\mathcal{F}$.
\end{defn}

\begin{defn}
A \emph{differentiable structure} on $M$ is a maximal differentiable atlas.
\end{defn}

To differentiate a function on a flexible manofold, first we should define the differentiability of a function.
A differentiable structure, which is usually defined by a maximal differentiable atlas, is roughly a collection of differentiable functions on $M$.
When the charts is already equipped on $M$, it is natural to define a function $f\colon M\to\Bbb{R}$ differentiable if the functions $f\circ\varphi^{-1}\colon\Bbb{R}^d\to\Bbb{R}$ is differentiable.

The gluing condition makes the differentiable function for a chart is also differentiable for any charts because $f\circ\varphi_\alpha^{-1}=(f\circ\varphi_\beta^{-1})\circ(\varphi_\beta\circ\varphi_\alpha^{-1})
=(f\circ\varphi_\beta^{-1})\circ\tau_{\alpha\beta}$.
If a function $f$ is differentiable on an atlas $\mathcal{F}$, then $f$ is also differentiable on any atlases which is equivalent to $\mathcal{F}$ by the definition of the equivalence relation for differential atlases.
We can construct the equivalence classes respected to this equivalence relation.

Therefore, we want to define a differentiable structure as a one of the equivalence classes.
However the differentiable structure is frequently defined as a maximal atlas for the convenience since each equivalence class is determined by a unique maximal atlas.


\begin{ex}
While the circle $S^1$ has a unique smooth structure, $S^7$ has 28 smooth structures.
The number of smooth structures on $S^4$ is still unknown.
\end{ex}

\begin{defn}
A continuous function $f\colon M\to N$ is differentiable if $\psi\circ f\circ\varphi^{-1}$ is differentiable for charts $\varphi,\psi$ on $M,N$ respectively.
\end{defn}




\begin{defn}[Partition of unity]
\end{defn}





\section{}

\begin{defn}
For $f\colon M\to\R$ and $(U,\phi)$ a chart,
\[df\left(\pd{x^\mu}\right):=\pd{f\circ\phi^{-1}}{x^\mu}.\]
\end{defn}



\begin{defn}
Let $\gamma\colon I\to M$ be a smooth curve.
Then, $\dot\gamma(t)$ is defined by a tangent vector at $\gamma(t)$ such that
\[\dot\gamma(t):=d\gamma\left(\pd{t}\right).\]
Let $\phi\colon M\to N$ be a smoth map.
Then, $\phi(t)$ can refer to a curve on $N$ such that
\[\phi(t):=\phi(\gamma(t)).\]
Let $f\colon M\to\R$ be a smooth function.
Then, $\dot f(t)$ is defined by a function $\R\to\R$ such that
\[\dot f(t):=\dd{t}f\circ\gamma.\]
\end{defn}

\begin{prop}
Let $\gamma\colon I\to M$ be a smooth curve on a manifold $M$.
The notation $\dot\gamma^\mu$ is not confusing thanks to
\[(\dot\gamma)^\mu=\dot{(\gamma^\mu)}.\]
In other words,
\[dx^\mu(\dot\gamma)=\dd{t}x^\mu\circ\gamma.\]
\end{prop}










\chapter{Riemannian geometry}

\section{Connection}


\subsection{Connection}

\begin{align*}
\nabla_XY&=X^\mu\nabla_\mu(Y^\nu\pd_\nu)\\
&=X^\mu(\nabla_\mu Y^\nu)\pd_\nu+X^\mu Y^\nu(\nabla_\mu\pd_\nu)\\
&=X^\mu\left(\pd{Y^\nu}{x^\mu}\right)\pd_\nu+X^\mu Y^\nu(\Gamma_{\mu\nu}^\lambda\pd_\lambda)\\
&=X^\mu\left(\pd{Y^\nu}{x^\mu}+\Gamma_{\mu\lambda}^\nu Y^\lambda\right)\pd_\nu.
\end{align*}
The covariant derivative $\nabla_XY$ does not depend on derivatives of $X^\mu$.

\[Y^\nu_{,\mu}=\nabla_\mu Y^\nu=\pd{Y^\nu}{x^\mu},\qquad Y^\nu_{;\mu}=(\nabla_\mu Y)^\nu=\pd{Y^\nu}{x^\mu}+\Gamma_{\mu\lambda}^\nu Y^\lambda.\]
\begin{thm}
For Levi-civita connection for $g$,
\[\Gamma^l_{ij}=\frac12(\pd_ig_{jk}+\pd_jg_{ki}-\pd_kg_{ij}).\]
\end{thm}
\begin{pf}
\begin{align*}
(\nabla_ig)_{jk}&=\pd_ig_{jk}-\Gamma^l_{ij}g_{lk}-\Gamma^l_{ik}g_{jl}\\
(\nabla_jg)_{kl}&=\pd_jg_{kl}-\Gamma^l_{jk}g_{li}-\Gamma^l_{ji}g_{kl}\\
(\nabla_kg)_{ij}&=\pd_kg_{ij}-\Gamma^l_{ki}g_{lj}-\Gamma^l_{kj}g_{il}\\
\end{align*}
If $\nabla$ is a Levi-civita connection, then $\nabla g=0$ and $\Gamma_{ij}^k=\Gamma_{ji}^k$.
Thus,
\[\Gamma^l_{ij}g_{kl}=\frac12(\pd_ig_{jk}+\pd_jg_{ki}-\pd_kg_{ij}).\]
\[\Gamma^l_{ij}=\frac12g^{kl}(\pd_ig_{jk}+\pd_jg_{ki}-\pd_kg_{ij}).\]
\end{pf}















\section{}

\begin{thm}
If $c$ is a geodesic curve, then components of $c$ satisfies a second-order differential equation
\[\dd[2]{\gamma^\mu}{t}+\Gamma_{\nu\lambda}^\mu\dd{\gamma^\nu}{t}\dd{\gamma^\lambda}{t}=0.\]
\end{thm}
\begin{pf}
Note
\[0=\nabla_{\dot\gamma}\dot\gamma=\dot\gamma^\mu\nabla_\mu(\dot\gamma^\lambda\pd_\lambda)
=(\dot\gamma^\nu\pd_\nu\dot\gamma^\mu+\dot\gamma^\nu\dot\gamma^\lambda\Gamma_{\nu\lambda}^\mu)\pd_\mu.\]
Since
\[\dot\gamma^\nu\pd_\nu\dot\gamma^\mu=\dot\gamma(\dot\gamma^\mu)=d\dot\gamma^\mu(\dot\gamma)=d\dot\gamma^\mu\circ d\gamma\left(\pd{t}\right)=d\dot\gamma^\mu\left(\pd{t}\right)=\ddot\gamma^\mu,\]
we get a second-order differential equation
\[\dd[2]{\gamma^\mu}{t}+\Gamma_{\nu\lambda}^\mu\dd{\gamma^\nu}{t}\dd{\gamma^\lambda}{t}=0\]
for each $\mu$.
\end{pf}











\end{document}