\documentclass{../crs}
\usepackage{../../ikany}

\title{Analysis 2 : General Topology}

\begin{document}
\maketitle
\tableofcontents

\frontmatter
%!TeX root=general topology.tex


\chapter*{Preface}

% essentially penetrating short definition
% relation to the other fields
%


One way to state the definition for general topology is the abstract study of topologies and topological spaces.
The word topology is used in two different contexts: analytic sense and geometric sense.
When we are talking the stories of doughnuts and coffee mugs, they are in fact involved in topology of geometric sense, which is also referred to as a branch of mathematics that studies constinuous structures of spaces such as manifolds or CW complexes.
In analysis, the topology is mentioned greatly unrelatedly to the doughnuts, but it refers to the minimal structure that is required in order to define concepts of limit and continuity.
More precisely, once a structure called ``topology'' is settled on a set, then we can expand basic analytic theories about limit and continuity.
Normed spaces are the first examples which possess a particularly nice topology.
With the topologies, we can describe formally whether a sequence converges or a function is continuous.
This book is interested in the latter issues as noted in the title of the book.

According to the usage of topologies, similarly as mentioned, there are two large branches of general topology; both contribute to build nice frameworks for the wide regions of mathematics.
One is for algebraic topology and studies the category of convenient spaces in which well-known constructions and computational tools are available, and the other is for abstract analysis.
In general topology focused on analysis, we are more concerned with the implications among individual topologies and special properties of them, rather than the global shapes of topological spaces.
For real analysis or functional analysis, general topology provides with extremely important viewpoints for recognizing the various convergence modes of functional sequences.
An interesting feature of general topology is that the basic topology in analysis is a preliminary of the abstract study of the spaces used in algebraic topology, hence everyone starts to learn it from analysis.

\iffalse
For generalization, we firstly come up with the following question: exactly what aspects of norms could have made one define the notion of convergence?
This leads the concept of neighborhood system, and unltimately, topology.
We may, furthermore, compare different topologies.
For example, the convergence changes even for a same sequence when judged in different topologies so that we can argue in what sense a sequence converges.
We can ask if a convergence implies another convergence.
Also, as a generalization of metric, we want to ask the essential difference between metric and topology.
The uniformity will answer the question.

Also, we are concerned with the properties of a given topology.
Some convergences that occur within various applications such as pointwise convergence of functional sequences with uncountable domain cannot be formulated with neither norms nor metrics.
They are in fact non-metrizable topologies.
\fi




The purpose of this book is to grasp a big picture and learn basic languages in order to establish frameworks for the next study of modern analysis such as harmonic analysis or functional analysis following after calculus topics, in a quite abstract viewpoints.
In particular, we mainly focus on finding admissible answers for the following questions:
\begin{itemize}
\item Why are topologies defined in that way? Is that a suitably optimized definition?
\item What can metric spaces or normed spaces do more than topological spaces?
\item What properties are needed to take and use sequences for describing topologies instead of general nets without any anxiety? 
\item What does the definition of compactness mean? What roles do they do in practical analysis?
\item What are the purposes of introduction of the compactness related concepts such as sequentially compact, $\sigma$-compact, or relatively compact spaces?
\item Why do locally compact Hausdorff spaces so frequently appear?
\item Why is the uniform convergence natural in a continuous function space?
\item What is the hidden meaning of complicated theorems of like Arzela-Ascoli or Stone-Weierstrass?
\end{itemize}
For the first in this book, the basic topological structures including metrics, topologies, and uniformities are introduced in Chapter 1.
Although many texts do not cover uniform spaces, they are greatly useful in studying nonmetrizable topologies.
In Chapter 2, we learn about continuity of functions and maps.
Continuous maps functionally connects two different topological spaces and allow us to compare them.
Homeomorphisms and some connectivity will be also covered.
Chapter 3 is dedicated to the deeper study of convergence of sequences or nets.
In Chapter 4, 5, and 6, we learn compact spaces, separability axioms, and continuous function spaces.

In this book, we are going to assume the reader is already familiar to the theory of normed spaces and elementary foundations of calculus including the epsilon-delta definitions.
For instance, we can require the reader to know what the uniform convergence is and that it can be regarded as just a convergence in the properly defined norm on a space of functions.

This book would not be a good choice for a standard course text relative to the other existing great books, because it is written to be helpful in self-teaching.
It has been tried to put convincing explanations at every newly defined concept and to cram supplementary stories that are not necessary, but they might not be really satisfied.
Nevertheless, I will be very satisfied only if just one of readers could enjoy math with this book.


\mainmatter
%!TeX root=general topology.tex
\chapter{Topological structures}

Firstly we discuss to what extent the definition of analytic notions such as limit and continuity can be extended.
One of main interest in general topology is to make extended version of mathematical calculus on a set without algebraic operations.
However, lay it up in the heart that several properties must be compromised when we try to make generalizations.

Recall that we measured how near the two points are by taking absolute value of algebraic subtraction of two position vectors in normed spaces.
How can we dismiss the subtraction?
Saying only the results, mathematicians succeeded to generalize limit of sequences and continuity of functions, but compromising the theory of differentiation and integration.
Topology is the term for this successful solution.
In other words, for the most part, wonderful statements purely related to limit and continuity were possible to be extended without big flaws even if we forget the vector space operations by introducing the concept of topology, but differentiation and integration could not on the other hand.
 
Topological structure refers to an additional function on a given set or a more complicated mathematical device which solves the problem by being put on a set.
Norm is a typical example of topological structure, and so is ``topology''.


% 1
\section{Metric}

Metric is a generalization of norm and a special example of ``topology''.
For example, every subset of a normed vector space is equipped with a natural metric.
Since general topology might be too abstract for novices to grasp, we will make a bridge from norms to topologies.

Before start, henceforth, readers should be noticed that the following facts about metrics are widely used in applications in analysis, even though, are nicely generalized into more fundamental and structural forms after the introduction of topology: we are not needed to give propositions too much authority to be memorized by force.
Later, by applying the abstract theory of general topology to metrics as examples, we will find that metric provides with a surprisingly appropriate and widely-applicable tool to understand the nature of mathematical analysis.
To give a short answer for the essentiality of metric is ``a countable uniform topology'' in a sense; understanding what it means would be one of primary goals of this chapter.

% 1-1
\subsection{Metric spaces}

Metric was the first successful trial to find an abstract framework for studying limits.
It is defined as a function which assigns a nonnegative real number, which has a meaning of distance, to an unordered pair of two points.
A metric space is just a set endowed with a metric.

\begin{defn}
Let $X$ be a set.
A \emph{metric} is a function $d:X\x X\to\R_{\ge0}$ such that
\begin{cond}
\item $d(x,y)=0$ iff $x=y$, \hfill(nondegeneracy)
\item $d(x,y)=d(y,x)$ for all $x,y\in X$, \hfill(symmetry)
\item $d(x,z)\le d(x,y)+d(y,z)$ for all $x,y,z\in X$. \hfill(triangle inequality)
\end{cond}
A pair $(X,d)$ of a set $X$ and a metric on $X$ is called a \emph{metric space}.
We often write it simply $X$.
\end{defn}

The most familiar metric comes from the standard norm on Euclidean space $\R^d$.
Notice that the third axiom, the triangle inequality, is named after the one for norms.
In this context, we can see metrics as a generalization of norms for spaces not admitting the vector space structure.
When we use an analytic theory on Euclidean spaces or more generally normed spaces, the metric given in the following example is considered as the standard.
Therefore, if not particularly mentioned, then we will implicitly assume this induced metric out of the norm for any subset of a Euclidean space.
Moreover, every subset of normed space is also an example of metric space because the metric function can be always inherited to every subset of a given metric space.

\begin{ex}
A normed space $X$ is a metric space.
Precisely, the norm structure naturally defines a real-valued function $d$ on $X\x X$ defined by $d(x,y):=\|x-y\|$ and it satisfies the axioms of metric.
\end{ex}
\begin{pf}
It is quite easy.
Just recall the axioms of norm and deduce the conclusion for each axiom of metric.
\end{pf}
\begin{ex}
Let $(X,d)$ be a metric space.
Every subset of $X$ has a natural induced metric, just the restriction of original metric $d$.
\end{ex}
\begin{pf}
Obvious.
\end{pf}

In fact, the converse holds; every metric space can be viewed as a subset of a normed space.
This deeper result on the relation between normed spaces and metric spaces is discovered by Kuratowski[].
Since the theorem does not play any important role in the whole book, readers who want to read fast may skip.
To state the theorem, we introduce isometry, a map preserving metrics.

\begin{defn}
Let $X$ and $Y$ be metric spaces.
A map $\phi:X\to Y$ is called an \emph{isometry} if $d(x,y)=d(\phi(x),\phi(y))$ for all $x,y\in X$.
If there is a bijective isometry between $X$ and $Y$, then we say the spaces are \emph{isometric}.
\end{defn}

Every isometry is clearly injective so that it is bijective if and only if it is surjective.
Also, the inverse of bijective isometry is also an isometry.
If two metric spaces are isometric, we can view them as virtually same, in the ``category'' of metric spaces.
The following theorem tells another characterization of metric spaces.

\begin{prop}[Kuratowski embedding]
Every metric space is isometric to a subset of a normed space.
In other words, for every metric space $(X,d)$, there is an isometry $\phi$ from $X$ to a normed space.
\end{prop}
\begin{pf}
Choose any point $p\in X$.
Let $Y$ be the space of bounded real-valued functions on $X$.
It is a normed space with uniform norm.
Define $\phi:X\to Y$ by $\phi(x)(t)=d(x,t)-d(p,t)$.
Note that $\phi(x)$ is bounded with $\|\phi(x)\|=\sup_{t\in X}|d(x,t)-d(p,t)|=d(x,p)$.
Then,
\[\|\phi(x)-\phi(y)\|=\sup_{t\in X}|\phi(x)(t)-\phi(y)(t)|=\sup_{t\in X}|d(x,t)-d(y,t)|=d(x,y).\]
This proves $\phi$ is a isometry.
\end{pf}
\begin{rmk}
The space $Y$ is somtimes denoted by $\ell^\oo(X)$, and it is in fact a Banach space.
In addition, the image of the isometry $\phi$ is in a closed subspace $C_b(X)\subset\ell^\oo(X)$, the space of bounded real-valued continuous functions.
\end{rmk}

We have seen metrics can be seen as the generalization of norms.
However, there are also many examples of metrics that are not involved directly in the norms.
Even if they are far from subsets of a normed space, we can apply our intuition of balls.
The examples below are given without proofs.

\begin{ex}\label{ex:discrete metric}
Let $X$ be a set.
Then, a function $d:X\x X\to\R_{\ge0}$ defined by
\[d(x,y):=\begin{cases}0&,x=y\\1&,x\ne y\end{cases}\]
is a metric on $X$.
This metric is sometimes called \emph{discrete metric} because balls can separate all single points out.
\end{ex}
\begin{ex}\label{ex:subadditive function metric}
Let $d$ be a metric on a set $X$.
Let $f:[0,\oo)\to[0,\oo)$ be a function such that $f^{-1}(0)=\{0\}$.
If $f$ is monotonically increasing and subadditive, then $f\o d$ satisfies the triangle inequality, hence is another metric on $X$.
Note that a function $f$ is called subadditive if
\[f(x+y)\le f(x)+f(y)\]
for all $x,y$ in the domain.
\end{ex}
\begin{ex}
Let $G=(V,E)$ be a connected graph.
Define $d:V\x V\to\Z_{\ge0}\subset\R_{\ge0}$ as the distance of two vertices; the length of shortest path connecting two vertices.
Then, $(V,d)$ is a metric space.
\end{ex}
\begin{ex}
Let $\cP(X)$ be the power set of a finite set $X$.
Define $d:\cP(X)\x\cP(X)\to\Z_{\ge0}\subset\R_{\ge0}$ as the cardinality of the symmetric difference; $d(A,B):=|(A-B)\cup(B-A)|$.
Then $(\cP(X),d)$ is a metric space.
\end{ex}
\begin{ex}
Let $C$ be the set of all compact subsets of $\R^d$.
Recall that a subset of $\R^d$ is compact if and only if it is closed and bounded.
Then, $d:C\x C\to\R_{\ge0}$ defined by
\[d(A,B):=\max\{\sup_{a\in A}\inf_{b\in B}\|a-b\|,\sup_{b\in B}\inf_{a\in A}\|a-b\|\}\]
is a metric on $C$.
It is a little special case of \emph{Hausdorff metric}.
\end{ex}



% 1-2
\subsection{Limits and continuity}

Many freshmen misunderstand the main role of metric for its name.
They recognize metric as something measures a distance and belonging to the study of geoemtry.
We cannot strongly affirm it is false, but I hope to mention that a metric is quite far from geometric structures, and is rather an analytic structure.
Meaning, metric is in fact not interested in measuring a distance between two points; the main function of metric is to make balls.
The balls centered at each point provide a concrete images of ``system of neighborhoods at a point'' in a more intuistic sense.
Metric can be considered as a device to let someone accept the notion of neighborhoods more friendly, which is vital for analysis of limits and continuity.

\begin{defn}
Let $X$ be a metric space.
A set of the form 
\[\{y\in X:d(x,y)<\e\}\]
for $\e>0$ is called a \emph{ball centered at $x$} and denoted by $B(x,\e)$ or $B_\e(x)$.
\end{defn}

The balls are also called open balls in order to distinguish from the closed balls; $\cl{B(x,\e)}=\{y\in X:d(x,y)\le\e\}$.
The terms openness and closedness will be discussed again in the next section.
Now let us reformulate the definitions of limits and continuity with balls, which we actually use in the usual calculus on Euclidean spaces or generally on normed spaces.
Compare the following definitions to what we remember.

\begin{defn}
Let $\{x_n\}_n$ be a sequence of points on a metric space $(X,d)$.
We say that a point $x$ is a \emph{limit} of the sequence or the sequence \emph{converges to $x$} if for arbitrarily small ball of size $\e$ cenetered at $x$, $B(x,\e)$, we can find $n_0$ such that $x_n\in B(x,\e)$ for all $n>n_0$.
If it is satisfied, then we write
\[\lim_{n\to\oo}x_n=x,\]
or simply
\[x_n\to x\]
as $n\to\oo$.
If there is no such limit $x$, then we say the sequence \emph{diverges}.
\end{defn}
\begin{defn}
A function $f:X\to Y$ between metric spaces is called \emph{continuous at $x\in X$} if for any ball $B(f(x),\e)\subset Y$ centered at $f(x)$ there is a ball $B(x,\delta)\subset X$ centered at $x$ such that
\[f(B(x,\delta))\subset B(f(x),\e).\]
The function $f$ is called \emph{continuous} if it is continuous at every point on $X$.
\end{defn}

There are a lot of deeper and valuable propositions and results for limit and continuity, but we postpone to mention them to later because they are generalized to topological spaces.
However, for a while, let us just check that taking either $\e$ or $\delta$ really means taking a ball of the very radius.
For continuity of a function, we can intuitively describe it by saying that no matter how small ball is taken in the codomain, we can take much smaller ball in the domain.
Then, what we want to know would be how the balls centered at each point are distributed because what the set of balls looks like may determine the continuity or convergence.
To make a vivid illustration, let us give an example.

\begin{ex}
Let $X$ be the discrete metric space in Example \ref{ex:discrete metric}.
Every ball centered at a point $x$ with respect to the discrete metric is either a singleton $B(x,\e)=\{x\}$ when $\e\le1$, or the entire space $B(x,\e)=X$ when $\e>1$.
In particular, a sequence $\{x_n\}_n$ converges to $x$ if and only if it is eventually $x$; there is a positive integer $n_0$ such that $x_n=x$ for all $n>n_0$.
\end{ex}
\begin{ex}
Let $X$ and $Y$ be metric spaces.
If $X$ is equipped with the discrete metric in Example \ref{ex:discrete metric}, then every function $f:X\to Y$ is continuous.
\end{ex}
\begin{ex}
An isometry is always continuous.
\end{ex}

The set of balls at each point plays an important role in determining properties for limits and continuity.
Intuitively, the balls indicate the varying degrees of neighborhoods and relative nearness from a point.
Refer to Example \ref{ex:neighborhood filter}.



% 1-3
\subsection{Topological equivalence}
Take note on the fact that the sequence of real numbers defined by $x_n=\frac1n$ diverges in discrete metric.
Like this example, even for the same sequence on a same set, the convergence depends on the attached metric.
However, we cannot conversely say that different metrics always provide different criteria for convergence.
In other words, when we consider a metric as a function that takes a sequence as input and outputs whether a sequence converges or diverges, there may be two different metrics which gives exactly same answer about convergence.
Of course, the continuity of functions has the same issue.
This allows us to think an equivalence relation on the set of metrics, that is, two equivalent metrics give a common criterion for convergence and continuity.
In this situation, it can be paraphrased into that the two metrics induce exactly same topology.
Then, we want to classfy the metrics by their topologies.

\subsubsection{Topological equivalence of metrics}
Some definitions are given as follows.

\begin{defn}
Let $d_1$ and $d_2$ are metric on a set $X$.
The two metrics are called \emph{topologically equivalent} if the sets of open balls at each point are mutually nested;
for any $x\in X$ and for arbitrary $\e>0$, we can find $\delta_1>0$ and $\delta_2>0$ such that
\[B_1(x,\delta_1)\subset B_2(x,\e)\quad\text{and}\quad B_2(x,\delta_2)\subset B_1(x,\e),\]
where the notations $B_1$ and $B_2$ refer to balls defined with the metrics $d_1$ and $d_2$ respectively.
\end{defn}

The word ``topologically'' is frequently omitted.
This definition looks quite strange, but is directly related to the way how a metric gives rise to a topology, which we have not defined yet.
Intuitively, we can say the neighborhood systems of balls from each metric ``refine'' each other.
There are various characterizations of equivalence among metrics.
Especially the first proposition states that we can recover an equivalence class of metrics when it is known that which sequence converges.

\begin{prop}\label{prop:sequential convergence data}
Let $d_1$ and $d_2$ are metrics on a set $X$.
They are equivalent if and only if they share the same sequential convergence data; a sequence converges in $d_1$ if and only if it converges in $d_2$.
\end{prop}
\begin{pf}
It is easily deduced by applying the following lemma twice.
\end{pf}

\begin{lem}
Let $d_1$ and $d_2$ are metrics on a set $X$.
The followings are equivalent:
\begin{cond}
\item For any $x\in X$ and for arbitrary $\e>0$, there is $\delta>0$ such that
\[B_1(x,\delta)\subset B_2(x,\e),\]
\item If a sequence converges to a point $x$ in $d_1$, then it also converges to $x$ in $d_2$.
\end{cond}
\end{lem}
\begin{pf}
(1)$\Rightarrow$(2)
Let $\{x_n\}_n$ be a sequence in $X$ that converges to $x$ in $d_1$.
By the assumption, for an arbitrary ball $B_2(x,\e)=\{y:d_2(x,y)<\e\}$ centered at $x$, there is $\delta>0$ such that
\[B_1(x,\delta)\subset B_2(x,\e),\]
where $B_1(x,\delta)=\{y:d_1(x,y)<\delta\}$.
Since $\{x_n\}_n$ converges to $x$ in $d_1$, there is an integer $n_0$ such that
\[n>n_0\impl x_n\in B_1(x,\delta).\]
Combining them, we obtain an integer $n_0$ such that
\[n>n_0\impl x_n\in B_2(x,\e).\]
It means $\{x_n\}$ converges to $x$ in the metric $d_2$.

(2)$\Rightarrow$(1)
We prove it by contradiction.
Assume that for some point $x\in X$ we can find $\e_0>0$ such that there is no $\delta>0$ satisfying $B_1(x,\delta)\subset B_2(x,\e_0)$.
In other words, at the point $x$, the difference set $B_1(x,\delta)\setminus B_2(x,\e_0)$ is not empty for every $\delta>0$.
Thus, we can choose $x_n$ to be a point such that
\[x_n\in B_1\left(x,\tfrac1n\right)\setminus B_2(x,\e_0)\]
for each positive integer $n$ by putting $\delta=\frac1n$.

We claim $\{x_n\}_n$ converges to $x$ in $d_1$ but not in $d_2$.
For $\e>0$, if we let $n_0=\ceil{\frac1\e}$ so that we have $\frac1{n_0}\le\e$, then
\[n>n_0\impl x_n\in B_1\left(x,\tfrac1n\right)\subset B_1(x,\e).\]
So $\{x_n\}_n$ converges to $x$ in $d_1$.
However in $d_2$, for $\e=\e_0$, we can find such $n_0$ like $d_1$ since
\[x_n\notin B_2(x,\e_0)\]
for every $n$.
Therefore, $\{x_n\}$ does not converges to $x$ in $d_2$.
\end{pf}
\begin{prop}
Let $d_1$ and $d_2$ are metric on a set $X$.
They are equivalent if and only if the two identity functions $I:(X,d_1)\to(X,d_2)$ and $I:(X,d_2)\to(X,d_1)$ are continuous.
\end{prop}
\begin{pf}
The continuity of $I:(X,d_1)\to(X,d_2)$ is equivalent to the existence of $\delta$ such that $B_1(x,\delta)\subset B_2(x,\e)$.
The opposite part is also true vice versa.
\end{pf}

\begin{rmk}
Generally, there exist two different topologies that have same sequential convergence data.
For example, a sequence in an uncountable set with cocountable topology converges to a point if and only if it is eventually at the point, which is same with discrete topology.
This means the informations of sequence convergence are not sufficient to uniquely characterize a topology.
Instead, convergence data of generalized sequences also called nets, recover the whole topology.
For topologies having a property called the first countability, it is enough to consider only usual sequences in spite of nets.
What we did in this subsection is not useless because topology induced from metric is a typical example of first countable topologies.
These kinds of problems will be profoundly treated in Chapter 3.
\end{rmk}
\begin{rmk}
One can ask some results for the equivalence of metrics characterized by a same set of continuous functions.
However, they are generally difficult problems: is it possible to recover the base space from a continuous function space or a path space?
\end{rmk}


The following two theorems give sufficient conditions for equivalence.
The first theorem is well used to compare norms on a vector space in particular, and the second theorem is going to be used in the next subsection.

\begin{thm}\label{thm:equivalent metrics by inequalities}
Let $d_1$ and $d_2$ are metric on a set $X$.
If for each point $x$ there exist two constants $C_1$ and $C_2$ which may depend on $x$ such that
\[d_2(x,y)\le C_1d_1(x,y)\quad\text{and}\quad d_1(x,y)\le C_2d_2(x,y)\]
for all $y$ in $X$, then $d_1$ and $d_2$ are equivalent.
\end{thm}
\begin{pf}
Since $d_1(x,y)<\e/C_1$ implies $d_2(x,y)<\e$ and $d_2(x,y)<\e/C_2$ implies $d_1(x,y)<\e$, we have
\[B_1\left(x,\frac\e{C_1}\right)\subset B_2(x,\e),\quad B_2\left(x,\frac\e{C_2}\right)\subset B_1(x,\e).\]
By letting $\delta_1=\e/C_1$ and $\delta_2=\e/C_2$, we can see the two metrics are equivalent.
\end{pf}
This theorem can be used only for half:  $d_2\le Cd_1$ for some $C>0$ if and only if convergence of a sequence in $d_1$ implies the convergence in $d_2$.

\begin{thm}\label{thm:equivalent metrics by subadditive function}
Let $d$ be a metric on a set $X$ and let $f$ be a monotonically increasing subadditive real function on $\R_{\ge0}$ such that $f^{-1}(0)=\{0\}$ so that $f\o d$ is a metric.
If $f$ is continuous at 0 in addition, then $f\o d$ is equivalent to $d$.
\end{thm}
\begin{pf}
We have seen that $f\o d$ is a metric in Example \ref{ex:subadditive function metric}.
Firstly, for any ball $B(x,\e)=\{y:d(x,y)<\e\}$, we have a smaller ball
\[B_f(x,f(\e))\subset B(x,\e),\]
where $B_f(x,f(\e))=\{y:f(d(x,y))<f(\e)\}$, since $f(d(x,y))<f(\e)$ implies $d(x,y)<\e$.

The second inclusion requires the continuity of $f$.
Take an arbitrary ball $B_f(x,\e)$.
Since $f$ is continuous at 0, we can find $\delta>0$ such that
\[d(x,y)<\delta\impl f(d(x,y))<\e,\]
which implies $B(x,\delta)\subset B_f(x,\e)$.
\end{pf}

\subsubsection{Topological equivalence of norms}

Topological equivalence of two metrics is quite abstract.
Typical examples of equivalent metrics are given when we consider norms.

It would be natural to apply the concept of topological equivalence to norms.
The idea is same; if two norms gives rise to a same topology, or equivalently, topologically equivalent metrics, then we call them equivalent.
However, the checking procedure becomes rather simple; the converse of Theorem \ref{thm:equivalent metrics by inequalities} holds for norms.
It is because metrics on a vector space induced from norms has the property called translation invariance.
The following thereom is often taken as the definition of norm equivalence.

\begin{thm}
Let $\|\cdot\|_1$ and $\|\cdot\|_2$ be two norms on a vector space $V$.
They induce the equivalent metrics if and only if there are constants $C_1$ and $C_2$ such that
\[\|x\|_2\le C_1\|x\|_1\quad\text{and}\quad\|x\|_1\le C_2\|x\|_2\]
for all $x\in V$.
\end{thm}
\begin{pf}
($\Leftarrow$) It is a corollary of Theorem \ref{thm:equivalent metrics by inequalities}.

($\Rightarrow$)
For any ball $B_2(0,\e)$, there is a smaller ball $B_1(0,\delta)$ such that $B_1(0,\delta)\subset B_2(0,\e)$ by the definition of equivalence of metrics.
It means we have
\[\|x\|_1<\delta\impl\|x\|_2<\e\]
for all $x\in V$.
If we let $C_1:=\e/\delta$, then it is equivalent to
\[C_1\|x\|_1<\e\impl\|x\|_2<\e.\]
If there is a vector $x\in V$ such that $\|x\|_2>C_1\|x\|_1$, then we can lead a contradiction by taking $\e$ between $\|x\|_2$ and $C_1\|x\|_1$.
Therefore, $\|x\|_2\le C_1\|x\|_1$ for all $x\in V$.
The other inequality is also shown by the same way.
\end{pf}

Especially, when we work on a vector space with finite dimension such as a Euclidean space $\R^d$, the situation gets better dramatically.

\begin{thm}\label{thm:equivalent norms on finite dimension}
On a finite dimensional vector space over a complete field such as $\R$ and $\C$, all norms are equivalent.
\end{thm}
\begin{pf}
Let $\F$ be the complete field $\R$ or $\C$.
Both have the absolute value function that makes the vector space complete.
Then, a finite dimensional vector space is isomorphic to $\F^d$ for some $d$.
Fix a basis $\{e_i\}_{i=1}^d$ on $\F^d$ and let $x=\sum_{i=1}^dx_ie_i$ denote an arbitrary element of $\F_d$.
We will prove all norms are equivalent to the standard norm:
\[\|x\|_2=\norm{\sum_{i=1}^dx_ie_i}_2:=\left(\sum_{i=1}^d|x_i|^2\right)^{\frac12}.\]
With this standard norm, we can use any theorems we learn in elementary analysis.
For example, we are allowed to use the Bolzano-Weierstrass theorem.
We usd the subscript 2 for the standard norm since the norm is frequently called $\ell^2$ norm.


Take a norm $\|\cdot\|$ on $\F^d$.
One direction is easy: if we let $C_2:=\sqrt{d}\cdot\max_i\|e_i\|$, then
\begin{align*}
\|x\|=\norm{\sum_{i=1}^dx_ie_i}&\le\sum_{i=1}^d|x_i|\|e_i\|\\
&\le\max_i\|e_i\|\sum_{i=1}^d|x_i|\\
&\le\max_i\|e_i\|\left(\sum_{i=1}^d1^2\right)^{\frac12}\left(\sum_{i=1}^d|x_i|^2\right)^{\frac12}=C_2\|x\|_2.
\end{align*}
For your information, this inequality show that the function $(\F^d,\|\cdot\|_2)\to\R:x\mapsto\|x\|$ is continuous.

There are many proofs for the other direction.
We give a proof using sequences.
Suppose there is no constant $C$ such that the inequality $\|x\|_2\le C\|x\|$ holds.
In other words, for every positive integer $n$, we can find $x_n\in\F^d\setminus\{0\}$ such that
\[\|x_n\|_2>n\|x_n\|.\]
Normalize $x_n$ with respect to $\|\cdot\|_2$ to define a new sequence $y_n:=\frac{x_n}{\|x_n\|_2}$.
Then, we have
\[\|y_n\|_2=1\quad\text{and}\quad\|y_n\|<\tfrac1n.\]
Since the set $\{x:\|x\|_2=1\}$ is bounded in the standard norm, the Bolzano-Weierstrass theorem implies the existence of a subsequence $\{y_{n_k}\}_k$ of $\{y_n\}$ that converges to a point $y$ in $\|\cdot\|_2$.
For this $y$, we have $\|y\|_2=1$ while $\|y\|=0$, which is a contradiction to the axiom of norm.

More rigorously, we get $\|y\|_2=1$ and $\|y\|=0$ by taking limit $k\to\oo$ on the inequalities
\[\bigl|\|y_{n_k}\|_2-\|y\|_2\bigr|\le\|y_{n_k}-y\|_2\]
and
\[\bigl|\|y_{n_k}\|-\|y\|\bigr|\le\|y_{n_k}-y\|\le C_2\|y_{n_k}-y\|_2.\]
This proves that $\|\cdot\|$ is equivalent to the standard norm.
\end{pf}

\begin{rmk}
The equivalence of norms is due to the locally compactness and the completeness.
In fact, locally compactness is a way to characterize finite dimensional spaces.
Hence we may also apply the Heine-Borel theorem or the extreme value theorem instead of the Bolzano-Weierstrass theorem, which are exactly equivalent statements for compactness.
Notice a closed ball is compact in such spaces, following the relation diagram:
\begin{rd}[column sep=huge]
Bounded \ds{r}{fin. dim.} & Totally bounded \ds{r}{complete} & Compact.
\end{rd}
This result is also important in functional analysis.
\end{rmk}

\begin{ex}
Let $\|\cdot\|_1$ and $\|\cdot\|_2$ be norms on $\R^2$ defined as
\[\|(x,y)\|_1:=|x|+|y|,\quad\|(x,y)\|_2:=\sqrt{|x|^2+|y|^2}.\]
Then, since we have inequalities
\[\|(x,y)\|_2\le\|(x,y)\|_1\le\sqrt2\|(x,y)\|_2\]
for all $(x,y)\in\R^2$, the two norms are equivalent.
\end{ex}



% 1-4
\subsection{Family of pseudometrics}

Our goal in this subsection is to describe a topology generated by several metrics, and, in general, by several ``pseudometrics''.
This idea provides a useful method to construct a metric or topology, which can be applied to a quite wide range of applications.

\subsubsection{Finite family of metrics}
At first, let us look into combination of metrics.
Specifically, in a conventional way, metrics are summed to make another metric out of olds since sum of two metrics also satisfies the all axioms of metric.
They are summed because convergence of a sequence in the resulted metric is equivalent to convergence in the summands.
See Proposition \ref{prop:summed metric convergence}.
However, here we give a slightly more general construction using norms restricted onto the closed orthant $(\R_{\ge0})^d$, of which summation becomes just a special case.

\begin{prop}
Let $\{d_i\}_{i=1}^d$ be a finite family of metrics on a set $X$.
Let $\|\cdot\|$ be a norm on $\R^d$.
Then, $d(x,y):=\|(d_1(x,y),\cdots,d_d(x,y))\|$ is another metric on $X$.
\end{prop}
\begin{pf}
Obvious.
\end{pf}
\begin{rmk}
Although it is possible to figure out conditions for $f:[0,\oo)^2\to[0,\oo)$ to have $f(d_1,d_2)$ be a metric, we just compromised it for simplicity and usefulness.
Also, if we use norms, then the same method can be extended to the case of norms.
\end{rmk}

Furthermore, the newly defined metric is unique up to equivalence.
We prove only for a pair of two metrics, but it is easy to check by mathematical induction that any finite family of metrics can be combined to make new metrics, which are essentially equivalent.
In fact, it can be checked to be obviously true without long proof like any other corollaries if we introduce bases of topology.

\begin{prop}
Let $d_1$, $d_2$, $d'_1$, and $d'_2$ be metrics on a set.
Let $\|\cdot\|$ and $\|\cdot\|'$ be norms on $\R^2$.
If $d_1$, $d_2$, and $\|\cdot\|$ are equivalent to $d'_1$, $d'_2$, and $\|\cdot\|'$ repsectively, then $\|(d_1,d_2)\|$ and $\|(d'_1,d'_2)\|'$ are equivalent metrics.
\end{prop}
\begin{pf}
Let $\{y:\|(d'_1(x,y),d'_2(x,y))\|'<\e\}$ be an arbitrary ball centered at a point $x$ taken by the metric $\|(d'_1,d'_2)\|'$.
By Theorem \ref{thm:equivalent norms on finite dimension}, there is a constant $C,C'>0$ such that $C'\|\cdot\|'\le\|\cdot\|_\oo\le C\|\cdot\|$, where we denote $\|(x,y)\|_\oo=\max\{|x|,|y|\}$.

By the equivalence, we can find $\delta_1>0$ and $\delta_2>0$ such that
\[B_1(x,\delta_1)\subset B_{1'}(x,C'\e)\quad\text{and}\quad B_2(x,\delta_2)\subset B_{2'}(x,C'\e),\]
where $B_1,B_{1'},B_2$, and $B_{2'}$ denotes the ball with respect to metrics $d_1,d'_1,d_2$, and $d'_2$ respectively.
With $\delta_1$ and $\delta_2$, define $\delta:=\min\{\delta_1,\delta_2\}/C$.
Then, the ball of radius $\delta$ in the metric $\|(d_1,d_2)\|$ is contained in the ball of radius $\e$ in the metric $\|(d'_1,d'_2)\|'$:
\[\{y:\|(d_1(x,y),d_2(x,y))\|<\delta\}\subset\{y:\|(d'_1(x,y),d'_2(x,y))\|'<\e\}.\]
The opposite part is shown in the same way symmetrically.
\end{pf}

\begin{ex}
Let $d_1$ and $d_2$ be metrics.
Then
\[d_1(x,y)+d_2(x,y)\quad\text{and}\quad\max\{d_1(x,y),d_2(x,y)\}\]
are equivalent metrics.
\end{ex}
\begin{ex}
If $d_1$ and $d_2$ are equivalent metrics, then $d_1+d_2$ is also equivalent to $d_1$ and $d_2$.
\end{ex}

From now, when we need to write a combined metric of a family of metrics, we will just adopt the sum $d_1+d_2$.
Another characterization of the summed metric is given as follows; see Proposition \ref{prop:sequential convergence data}.

\begin{prop}\label{prop:summed metric convergence}
Let $d_1$ and $d_2$ be metrics on a set $X$.
A sequecne $\{x_n\}_n$ converges to $x$ in $d_1+d_2$ if and only if it converges to $x$ in both $d_1$ and $d_2$.
\end{prop}
\begin{pf}
($\Rightarrow$)
Let $\{x_n\}_n$ be a sequence that converges to $x$ in $d_1+d_2$.
For $\e>0$, we have an positive integet $n_0$ such that
\[n>n_0\impl d_1(x_n,x)+d_2(x_n,x)<\e.\]
With this $n_0$, by the nonnegativity of metric functions, we get $d_1(x_n,x)<\e$ and $d_2(x_n,x)<\e$ for $n>n_0$.

($\Leftarrow$)
Suppose a sequence $\{x_n\}_n$ is convergent with respect to both $d_1$ and $d_2$.
By the Hausdorffness of metrics, the sequence converges to a common point $x$ in both metircs.
For $\e>0$, we may find positive integers $n_1$ and $n_2$ such that $n>n_1$ and $n>n_2$ imply $d_1(x_n,x)<\frac\e2$ and $d_2(x_n,x)<\frac\e2$ respectively.
If we define $n_0:=\max\{n_1,n_2\}$, then
\[n>n_0\impl d_1(x_n,x)+d_2(x_n,x)<\e.\qedhere\]
\end{pf}


\subsubsection{Countable family of metrics}
Above this, there is also a method for combining not only finite family of metrics, but also countable family of metrics.
Since the sum of countably many numbers may diverges, we cannot sum the metrics directly.
The strategy used here is to ``bound'' the metrics.
We call a metric bounded when the image of metric is bounded.

\begin{prop}
Every metric possesses an equivalent bounded metric.
\end{prop}
\begin{pf}
Let $d$ be a metric on a set.
Let $f$ be a bounded, monotonically increasing, and subadditive function on $\R_{\ge0}$ that is continuous at 0 and satisfies $f^{-1}(0)=\{0\}$.
The mostly used examples are
\[f(x)=\frac x{1+x}\quad\text{and}\quad f(x)=\min\{x,1\}.\]
Then, $f\o d$ is a bounded metric equivalent to $d$ by Theorem \ref{thm:equivalent metrics by subadditive function}.
\end{pf}

\begin{defn}
Let $d$ be a metric on a set $X$.
A \emph{standard bounded metric} means either metric
\[\min\{d,1\}\quad\text{or}\quad\frac d{d+1},\]
and we will denote it by $\hat d$.
\end{defn}

The supremum of the standard bounded metric is 1.
Every metric can be bounded above by not only 1 but also an arbitrary constant, keeping the topological equivalence, just by giving the constant as a coefficient to $\hat d$.
Following propositions are the reason why we bound the metric.

\begin{prop}
Let $\{d_i\}_{i\in\N}$ be a countable family of metrics on a set $X$.
Then a function $d:X^2\to\R_{\ge0}$ defined by
\[d(x,y):=\sum_{i\in\N}\,2^{-i}\hat d_i(x,y)\]
is a metric.
Furthermore, a sequence $\{x_n\}_n$ converges in $d$ if and only if it converges in every $d_i$.
\end{prop}
\begin{pf}
The function is well-defined by the monotone convergence theorem.
The only nontrivial axiom is the triangle inequality.
Consider the triangle inequality of truncated sum of metrics
\[\sum_{i=1}^k2^{-i}\hat d_i(x,z)\le\sum_{i=1}^k2^{-i}\hat d_i(x,y)+\sum_{i=1}^k2^{-i}\hat d_i(y,z).\]
By taking limit $k\to\oo$, we obtain the triangle inequality for $d$, hence a metric.
Let us show the rest part.

($\Rightarrow$)
We have an ineuqlaity $d_i\le2^i\hat d$ for each $i$, so convergence in $d$ implies the convergence in each $\hat d_i$.
See Theorem \ref{thm:equivalent metrics by inequalities}.
The equivalence of $\hat d_i$ and $d_i$ implies the desired result.

($\Leftarrow$)
Suppose a sequence $\{x_n\}_n$ converges to a point $x$ in $d_i$ for every index $i$.
Take an arbitrary small ball $B(x,\e)=\{y:d(x,y)<\e\}$ with metric $d$.
By the assumption, we can find $n_i$ for each $i$ satisfying
\[n>n_i\impl\hat d_i(x_n,x)<\frac\e2.\]
Define $k:=\lceil1-\log_2\e\rceil$ so that we have $2^{-k}\le\frac\e2$.
With this $k$, define
\[n_0:=\max_{1\le i\le k}n_i.\]
If $n>n_0$, then
\begin{align*}
d(x_n,x)&=\sum_{i=1}^k2^{-i}\hat d_i(x_n,x)+\sum_{i=k+1}^\oo2^{-i}\hat d_i(x_n,x)\\
&<\sum_{i=1}^k2^{-i}\frac\e2+\sum_{i=k+1}^\oo2^{-i}\\
&<\frac\e2+2^{-k}\le\e,
\end{align*}
so $x_n$ converges to $x$ in the metric $d$.
\end{pf}
\begin{prop}
Let $\{d_i\}_{i\in\N}$ be a countable family of metrics on a set $X$.
Then a function $d:X^2\to\R_{\ge0}$ defined by
\[d(x,y):=\sup_{i\in\N}\,i^{-1}\hat d_i(x,y)\]
is a metric.
Furthermore, a sequence $\{x_n\}_n$ converges in $d$ if and only if it converges in every $d_i$.
\end{prop}
\begin{pf}
The function is well-defined by the least upper bound property of real numbers.
The triangle inequality and the direction ($\Rightarrow$) have the same proof with the previous one.

($\Leftarrow$)
Suppose a sequence $\{x_n\}_n$ converges to a point $x$ in each $d_i$, and take an arbitrary small ball $B(x,\e)=\{y:d(x,y)<\e\}$ with metric $d$.
By the assumption, we can find $n_i$ for each $i$ satisfying
\[n>n_i\impl\hat d_i(x_n,x)<\e.\]
Define $k:=\lceil\frac1\e\rceil$ so that we have $k^{-1}\le\e$.
With this $k$, define
\[n_0:=\max_{1\le i\le k}n_i.\]
If $n>n_0$, then
\[i^{-1}\hat d_i(x,y)\le\hat d_i(x,y)<\e\quad\text{for}\quad i\le k\]
and
\[i^{-1}\hat d_i(x,y)\le i^{-1}<k^{-1}\le\e\quad\text{for}\quad i>k\]
imply $d(x_n,x)<\e$, which means that $x_n$ converges to $x$ in the metric $d$.
\end{pf}

From the two propositions and Proposition \ref{prop:sequential convergence data}, we get a corollary:
\begin{cor}
Let $\{d_i\}_{i\in\N}$ be a countable family of metrics on a set $X$.
Then, two metrics
\[d(x,y):=\sum_{i\in\N}\,2^{-i}d_i(x,y)\quad\text{and}\quad d(x,y):=\sup_{i\in\N}\,i^{-1}d_i(x,y)\]
are equivalent.
\end{cor}

The sequences $\{2^{-i}\}_i$ and $\{i^{-1}\}_i$ in the above propositions can be replaced into any positive real sequences $\{a_i\}_{i=1}^\oo$ such that
\[\sum_{i\in\N}a_i<\oo\quad\text{and}\quad\lim_{i\to\oo}a_i=0,\]
respectively.
\begin{rmk}
A metric
\[d'(x,y)=\sup_{i\in\N}\,d_i(x,y)\]
is not used because the convergence in this metric is a stronger condition than the convergence with respect to each metric $d_i$.
In other words, this metric generates a finer topology than the topology generated by subbase of balls.
For example, the topology on $\R^\N$ generated by this metric defined with projection pseudometrics is exactly what we often call the box topology.
\end{rmk}

How about an uncountable family of metrics?
This question will be answered in later sections.


\subsubsection{Pseudometrics}
It is often required to consider combining a family of pseudometrics, which has not been defined yet.
To motivate and introduce pseudometrics, consider an example problem.
Let $X\x Y$ be a cartesian product of two metric spaces $(X,d_X)$ and $(Y,d_Y)$.
We may ask what metric is chosen in the most natural way, and a possible answer will be as follows:
\[(x_n,y_n)\to(x,y)\iff x_n\to x\quad\text{and}\quad y_n\to y\]
as $n\to\oo$.
We wish to recognize this as the sum of two different convergences: one is $x_n\to x$, and the other is $y_n\to y$.
So then try to define two metric functions $d_X$ and $d_Y$ on $X\x Y$ such that
\[\rho_X((x,y),(x',y'))=d_X(x,x')\quad\text{and}\quad\rho_Y((x,y),(x',y'))=d_Y(y,y').\]
We wish that each function would satisfies axioms of metrics, but they fail on the identity of indiscernibles: $\rho_X((x,y),(x',y'))=0$ does not imply $(x,y)=(x',y')$.
The thing is, we can still define convergence or continuity with them; they also give rise to topologies, which lacks some good features including the Hausdorffness.
The definition of pseudometric comes from missing the nondegeneracy condition.

\begin{defn}
A function $\rho:X\x X\to\R_{\ge0}$ is called a \emph{pseudometric} if
\begin{cond}
\item $\rho(x,x)=0$ for all $x\in X$,
\item $\rho(x,y)=\rho(y,x)$ for all $x,y\in X$, \hfill(symmetry)
\item $\rho(x,z)\le \rho(x,y)+\rho(y,z)$ for all $x,y,z\in X$. \hfill(triangle inequality)
\end{cond}
\end{defn}
Compare this with the definition of metric.

Every statement and concept in 1.3.1, topological equivalence of metrics, except Proposition \ref{prop:sequential convergence data} is extended to the pseudometrics.
More precisely, for examples, we have the following propositions:



%% separates points

In other words, we can say that a topology generated by countable family of metrics is metrizable.

We give the first example of a topology which cannot be given by a metric.

\begin{ex}
sequence space pointwise convergence.
\end{ex}



\subsection*{Exercises}
Determine true or false and give a reason briefly:
\begin{enumerate}
\item Every nonempty set can be endowed with a metric.
\item The squared sum of two metrics is a metric.
\end{enumerate}

\subsection*{Problems}
\begin{prb}
Show that there is a metric $d$ on $\R$ such that a sequence $\{x_n\}_n$ defined by $x_n=x+\frac1n$ is convergent with respect to $d$ if and only if $x\ne0$.
\end{prb}
\begin{prb}
Let $\{x_n\}_n$ be a convergent sequence in a metric space $X$.
Show that for each point $p\in X$ there is $M>0$ such that $d(x_n,p)<M$ for all $n$.
\end{prb}
\begin{prb}
Let $d$ and $d'$ be metrics on a set $X$.
Suppose that a sequence $\{x_n\}_n$ in $X$ converges to $x$ in $d$ and converges to $x'$ in $d'$.
Show that $x=x'$.
\end{prb}
\begin{prb}
Let $d$ a metric on a set $X$.
Show that a function $d_p$ defined by $d_p(x,y):=d(x,y)^p$ is a metric equivalent to $d$ for every $p>0$.
\end{prb}
\begin{prb}
Find the range of the function $f(x)=\|x\|_p/\|x\|_q$ defined for $x\in\R^d$, where $\|x\|_p^p:=\sum_{i=1}^d|x_i|^p$ for $x=(x_1,\cdots,x_d)$.
\end{prb}

























% 2
\section{Topology}
We define topology and introduce some supplementary notions.




% 2-1
\subsection{Filters}
Suppose we want to find a proper way to define limit and convergence.
Recall how we define convergence of a sequence of real numbers: we say a sequence $(x_n)_{n\in\N}$ converges to a number $x$ if for each $\e>0$ there is $n_0(\e)\in\N$ such that $|x-x_n|<\e$ whenever $n>n_0$.
Simply, $x_n$ is close to $x$ if $n$ is close to the infinity.
Observe the two necessary structures to make this possible; the ``system of neighborhoods'' at each point $x$, and the total order on the index set $\N$ the set of natural numbers.
Without the order structure, we would not be able to formulate the intuition of the direction toward which a sequence is converging.
Even though the order on $\N$ is totally defined so that we can compare every pair of two elements, but it can be generalized to the case of partial orders.
%% poset can be order-embedded in a power set(complete atomic boolean lattice)
\begin{defn}
A subset $\cD$ of a poset is called \emph{(upward) directed} if for every $a,b\in\cD$ there is $c\in\cD$ such that $a\le c$ and $b\le c$.
Similarly, $\cD$ is called \emph{downward directed} if for every $a,b\in\cD$ there is $c\in\cD$ such that $c\le a$ and $c\le b$.
\end{defn}

The directedness of a partially ordered set is an essential notion to define limit.

Let $X$ be a set and $x\in X$.
Then, the power set $\cP(X)$ is a poset with inclusion relation.
The filter bases are defined abstractly:
\begin{defn}
A \emph{filter base} is a nonempty and downward directed subset of a poset.
\end{defn}
And concretely:
\begin{defn}
A collection $\cB_x$ of subsets of $X$ is called a \emph{filter base at $x$} if every element of $\cB_x$ contains $x$ and it forms a nonempty downward directed subset; every $U\in\cB_x$ contains $x$, and for all $U_1,U_2\in\cB_x$ there is $U\in\cB_x$ such that $U\subset U_1\cap U_2$.
\end{defn}

Among filters, we can give a relation structure as follows.

\begin{defn}
Let $\cB_x,\cB'_x$ be filter bases at $x$.
We say $\cB'_x$ is \emph{finer than} $\cB_x$, or a \emph{refinement} of $\cB_x$ if for every $U\in\cB_x$ there is $U'\in\cB'_x$ such that $U'\subset U$.
\end{defn}

As synonyms, all the following expressions tell the same situation.
\begin{cond}
\item $\cB'_x$ is \emph{finer than} $\cB_x$,
\item $\cB'_x$ is \emph{stronger than} $\cB_x$,
\item $\cB_x$ is \emph{coarser than} $\cB'_x$,
\item $\cB_x$ is \emph{weaker than} $\cB'_x$.
\end{cond}
The relation is a preorder so that we can consider the equivalence classes on which the natural partial order can be defined.

\begin{prop}
The refinement relation between filter bases is a preorder, and each equivalence class contains a unique maximal element.
\end{prop}
\begin{pf}
To show a relation is a preorder, we need to check transitivity.
Suppose $\cB''_x$ is finer than $\cB'_x$ and $\cB'_x$ is finer than $\cB_x$.
For any $U\in\cB_x$, there is $U'\in\cB'_x$ such that $U'\subset U$, and there is also $U''\in\cB''_x$ such that $U''\subset U'$.
Since $U''\subset U$, we can conclude $\cB''_x$ is finer than $\cB_x$.

We can say two filter bases are equivalent if they are both finer than each other.
Consider an equivalence class of filter bases and just denote it by $A$.
Then, $\bigcup_{\cB_x\in A}\cB_x$ is also contained in $A$ since it is equivalent to an arbitrary filter base $\cB_x$ in $A$.
It is also easy to check that this is maximal.
\end{pf}

Now we define filters.

\begin{defn}
A \emph{filter at $x$} is the maximal element of an equivalence class of filter bases at $x$.
\end{defn}

In other words, filters have one-to-one correspondence to the equivalence classes of filter bases.
A filter is identified to an equivalence class of filter bases.
They can be also characterized by three axioms.

\begin{thm}
A collection $\cF_x$ of subsets of $X$ is a filter at $x$ if and only if every element contains $x$ and it is closed under supersets and finite intersections;
\begin{cond}
\item $x\in U$ for $U\in\cF_x$,
\item if $U\subset V$ and $U\in\cF_x$, then $V\in\cF_x$,
\item if $U,V\in\cF_x$, then $U\cap V\in\cF_x$.
\end{cond}
\end{thm}
\begin{pf}
%%%
\end{pf}

Many references use the above theorem as the definition of filter because it is useful for someone who wants to check whether a given family is a filter.

\begin{thm}
A filter $\cF'_x$ is finer than another filter $\cF_x$ if and only if $\cF'_x\supset\cF_x$.
\end{thm}
\begin{pf}
%%%
\end{pf}

The following examples will be helpful to catch the intuition.

\begin{ex}\label{ex:neighborhood filter}
Let $x$ be a point in a metric space.
The set of all open balls cenetered at $x$ is a filter base at $x$.
The set of all open balls containing $x$ is aslo a filter base and they are equivalent.
A filter equivalent to these filter bases are called \emph{neighborhood filter at $x$}.
\end{ex}
\begin{ex}
Let $S$ be a subset of a set.
The set of all subsets containing $S$ is a filter at $x$ for every $x\in S$.
If $S=\{x\}$, then it is called a \emph{principal filter} at $x$.
\end{ex}
\begin{ex}
The set of all subsets of $\N$ whose complement is finite is a filter, but it is not a filter at a point.
However, it is intuitively a filter at infinity.
\end{ex}













% 2-2
\subsection{Topologies}
%%%
Before defining topology, recall that it plays the most important role in the definition of continuous functions to deal with neighborhoods of a point.
We want a structure to give a notion of neighborhoods of a point such as metrics, in other words, we want to generalize metric in a suitable way.
There is a conventional definition of topology: topology is defined as a subset of the power set of underlying space satisfying some axioms, and it is said to consist of open sets so that a topology indicates that which subsets are open or not.
However, this definition is so abstract that it might allow first-readers to lose its intuitions.
Thereby, we attempt to take another way.
Before introducing topology, we shall define a topological basis.
Topological bases are often used to describe a particular topology as bases of vector spaces do.
The main definition of topology will follow.

Let $X$ be a set.
\begin{defn}
A collection $\cB$ of subsets of $X$ is called a \emph{topological base} or simply a \emph{base on $X$} if
\[\{U:x\in U\in\cB\}\]
is a filter base at $x$ for every $x\in X$.
\end{defn}

A topological base is a kind of global version of filter base.

\begin{defn}
Let $\cB$ be a topological base on $X$ and $x\in X$.
A filter base at $x$ is called a \emph{local base} at $x$ if it is equivalent to the filter base $\{U:x\in U\in\cB\}$.
If a local base is a filter at $x$, then it is called \emph{neighborhood filter} of $x$.
\end{defn}
All the followings are synonyns:
\begin{cond}
\item local base
\item neighborhood system
\item fundamental system of neighborhoods
\item complete system of neighborhoods
\item filter base of neighborhood filter
\end{cond}

As we have done in the previous section, we can settle the refinement order on the set of topological bases.

\begin{defn}
Let $\cB,\cB'$ be topological bases on $X$.
We say $\cB$ is \emph{coarser} or \emph{weaker} than $\cB'$, and $\cB'$ is \emph{finer}, \emph{stronger} than $\cB$, or a \emph{refinement} of $\cB$ if every local base $\cB'_x$ is finer than $\cB_x$ at every point $x\in X$.
\end{defn}
\begin{prop}
The refinement relation between topological bases is a preorder, and each equivalence class contains a unique maximal element.
\end{prop}
\begin{pf}
%%%
\end{pf}

A topology is defined to be the maximal element, which means in fact an equivalence class of topological bases.

\begin{defn}
A \emph{topology on $X$} is the maximal element of an equivalence class of topological bases on $X$.
\end{defn}

There is also a criterion for topology.

\begin{thm}\label{thm:def_of_top}
A collection $\cT$ of subsets of $X$ is a topology on $X$ if and only if
\begin{cond}
\item $\varnothing,X\in\mathcal{T}$,
\item if $\{U_\alpha\}_{\alpha\in\mathcal{A}}\subset\mathcal{T}$, then $\bigcup_{\alpha\in\mathcal{A}}U_\alpha\in\mathcal{T}$,
\item if $U_1,U_2\in\mathcal{T}$, then $U_1\cap U_2\in\mathcal{T}$.
\end{cond}
\end{thm}
\begin{pf}
%%%
\end{pf}
Theorem \ref{thm:def_of_top} is usually used as a definition of topology because it allows us to check without difficulty whether a collection of subsets is a topology.


Since all topological structures are made to generalize the standard metric of Euclidean space, so drawing balls for representing base elements is always helpful in the whole story of general topology.






% 2-3
\subsection{Bases and subbases}

\begin{defn}
Let $\cB$ and $\cT$ be a base and a topology on a set $X$.
If $\cT$ is the coarsest topology containing $\cB$, then we say the topology $\cT$ is \emph{generated by} $\cB$.
\end{defn}
\begin{thm}
Let $\cB$ and $\cT$ be a base and a topology on a set $X$.
The followings are equivalent:
\begin{cond}
\item $\cB$ generates $\cT$,
\item $\cB$ and $\cT$ are equivaent bases,
\item $\cT$ is the set of all arbitrary unions of elements of $\cB$.
\end{cond}
\end{thm}

\begin{defn}
Let $\cS\subset\cP(X)$.
If a topology $\cT$ is the coarsest topology containing $\cS$, then we say $\cS$ is called a \emph{subbase} of $\cT$.
\end{defn}
\begin{prop}
Let $\cS\subset\cP(X)$.
The set of finite intersections of elements of $\cS$ is a basis.
\end{prop}

Here is the metric space example.
\begin{ex}
Let $X$ be a metric space.
A set of all balls $\cB=\{B(x,\e):x\in X,\,\e>0\}$ is a base on $X$ because for every point $x\in B(x_1,\e_1)\cap B(x_2,\e_2)$, we have $x\in B(x,\e)\subset B(x_1,\e_1)\cap B(x_2,\e_2)$ where $\e=\min\{\e_1-d(x,x_1),\,\e_2-d(x,x_2)\}$.
\end{ex}

In metric spaces, of course, there can exist infinitely many bases, but they are hardly considered except $\mathcal{B}$.
Sometimes in the context of metric spaces, the term neighborhood or basis are used to say $\mathcal{B}$.
As we have seen, balls in metric spaces are the main concept to state $\e$-$\delta$ argument.
This example would show a basis is fundamental language to describe the nature of limits in metric spaces.










% 2-4
\subsection{Open sets and neighborhoods}
def:nbhd and neighborhood filter
convergence and limit

% 2-5
\subsection{Closed sets and limit points}
 closure
 dense set,
% 2-6
\subsection{Interior and closure}




























% 3
\section{Uniformity}

% 3-1
\subsection{Uniform spaces}
uniformness of metric


% 3-2
\subsection{Entourages}


Uniform spaces are generalization of metric spaces.
The uniform structure is required to define uniform continuity, uniform convergence, completeness, etc.
Although the definition of uniform structure is not so easy at first, they have enormous advantage to learn.
For example, they are extremely useful in functional analysis since every compatible topology on algebraic structures such as topological group and topological vector space must admit a natural uniform structure.
Hence, we can use completeness or something uniform without unnecessary concerns.

\begin{defn}[Uniform space]
A \emph{uniform space} is a set $X$ equipped with a filter of binary relations $\cU\subset\cP(X^2)$ such that for every $E\in\cU$,
\begin{cond}
\item reflexivity: $(x,x)\in E$ for all $x\in X$,
\item triangle inequality: $\exists E'\in\cU:\,E'\circ E'\subset E$,
\item symmetry: $E^{-1}\in\cU$,
\end{cond}
where $\Delta_X=\{(x,x):x\in X\}$ and
\[E\circ F=\{(x,z):(x,y)\in E,(y,z)\in F\},\quad E^{-1}=\{(y,x):(x,y)\in E\}.\]
The collection $\cU$ is called a \emph{uniformity}, and a relation $E\in\cU$ is called an \emph{entourage}.
If $(x,y)\in E$, then we say $x$ and $y$ are $E$-close.
\end{defn}
\begin{defn}
Let $(X,\cU)$ be a uniform space.
Let $\tau$ be a set containing all $U\subset X$ such that for every $x\in U$ there is an entourage $E$ with $E_x\subset U$.
Then $\tau$ defines a topology on $X$, which is called \emph{uniform topology}, or \emph{induced topology}.
\end{defn}
\begin{defn}
A uniform space is called \emph{Hausdorff} if there is an entourage $E$ such that $x\in E$ and $y\notin E$ for every pair of distinct points $x,y\in X$.
This is equivalent for the induced topology to be Hausdorff.
\end{defn}

Note that the axioms for the definition of uniform spaces bear a similarity with the one of metric spaces.
For one exception, the Hausdorffness implies the nondegeneracy.
A uniform space is defined by the collection of relations that embody the concept of nearness.
Unlike neighborhoods in general topological space, an entourage measures the nearness not pointwisely(locally) but uniformly(globally).
We have the following hierarchy:
\[\text{topological space} \supset \text{uniform space} \supset \text{metric space}.\]
\begin{ex}
Let $G$ be a topological group.
Let $U$ be an open neighborhood of the identity $e$.
Define
\[E_U:=\{(g,h):gh^{-1}\in U\}.\]
Then, the set of $E_U$ forms a uniformity.
The difficult part is the triangle inequality, which can be shown from the continuity of group operation.
\end{ex}




% 3-3
\subsection{Pseudometrics}

%defn
%%% one pseudometric can generate a topology so that we can use limit and continuity
%%% but not unique limit
%% example : finite product space


% family of pseudometrics
%%% separates points
% sum of pseudometrics is a pseudometric

Metric can be regarded as the ``countably'' uniform structure in some sense. 
In other texts, for this reason, one frequently introduces metric instead of uniformity in order to avoid superfluously complicated and less intuitive notions of uniform structures, when only doing elementary analysis not requiring uncountable local bases.



One of the mostly used way of characterizing uniformity is to induce the fundamental system of entourages from a family of pseudometrics.
The manner is simple: just take all pseudoballs as the fundamental system of entourages.
\begin{defn}
Let 
\end{defn}
The proof of the following theorem is based on Bourbaki's text (General topology part2, chapter 9).

\begin{thm}
Every uniformity is induced by a family of pesudometrics.
\end{thm}
\begin{pf}

\end{pf}

















% CHAPTER 2
\chapter{Continuity}

% SECTION 2-1
\section{Continuous functions}

\subsection{Various continuity}
continuity, Cauchy continuity, uniform continuity, 

Lipschitz continuity
\begin{ex}
An isometry between metric spaces is Lipschitz continuous with costant 1.
\end{ex}

\subsection{Sequential continuity}




\section{Continuous maps}

\subsection{Mono and epi}

\subsection{Subspaces and quotient spaces}

\subsection{Product space}

\subsection{Homeomorphisms}
continuous bijection
open map
how to show two spaces are not homeomorphic
- topological property: connected, compact


\section{Connectedness}

\subsection{Connected spaces}
component
\subsection{Path connected spaces}

\subsection{Locally connected spaces}

\subsection{Homotopy}













\chapter{Convergence}

\section{Nets}

product of two directed sets
projection is monotone final
uniformity is itself an upward directed set by reverse inclusion, like $\R_{\ge0}$.
cofinality and subsequence

eventuality filter, three definitions of subnets

\section{Sequences}
sequential spaces, first countable

\section{Completeness}
completion















\chapter{Compactness}


\begin{defn}
Let $X$ be a topological space.
A \emph{cover} of a subset $A\subset X$ is a collection $\{U_\alpha\}_{\alpha\in\cA}$ of subsets of $X$ such that $A\subset\bigcup_{\alpha\in\cA}U_\alpha$.
If $U_\alpha$ are all open, then it is called \emph{open cover}.
\end{defn}
\begin{defn}
Let $X$ be a topological space.
A subset $K\subset X$ is called \emph{compact} if every open cover of $K$ has a finite subcover.
\end{defn}
\begin{prop}
Let $X$ be a topological space with a basis $\mathcal{B}$.
A subset $K\subset X$ is compact if and only if every cover of the form $\{B_x\in\mathcal{B}\}_{x\in K}$ has a finite subcover.
\end{prop}
\begin{rmk}
Let $\cP$ be a property of a fuction $f\colon X\to Y$, such as continuity
If we say $f$ has $\cP$ at a point $x$, then it would implies that $x$ has a neighborhood $U$ such that 
\end{rmk}

\subsection{Properties of compactness}
\begin{thm}
Let $X$ and $Y$ be topological spaces.
For a continuous map $f\colon X\to Y$, the image $f(K)$ is compact for compact $K\subset X$.
\end{thm}

\begin{rmk}
This is why the term ``compact space'' is widely used.
\end{rmk}

\begin{cor}[The extreme value theorem]
A continuous function on a closed interval has a global maximum and
\end{cor}

Heine-Cantor,



\subsection{Characterizations of compactness}

\begin{rd}[column sep=large,row sep=large]
net compact \ar{d} & sequentially compact \ar{d} & \\
compact \ar{u}\ar{r} & countably compact \ar{r}\lds{l}{\parbox{10em}{\centering metrizable\\\emph{or} 2nd countable}}\rds{u}{sequential} & limit point compact \lds{l}{$T_1$}
\end{rd}






\section{Relative compactness}

\begin{prop}[?????]
Let $X$ be a locally compact Hausdorff space.
For a subset $A$ of $X$, the followings are all equivalent:
\begin{cond}
\item a uniformity on $X$ makes $A$ have compact completion,
\item the set $A$ has compact closure in $X$,
\item every uniformity on $X$ makes $A$ have compact completion.
\end{cond}
\end{prop}





\begin{defn}
A uniform space is called \emph{relatively compact} or \emph{precompact} if its completion is compact.
\end{defn}

The relative compactness is in particular useful in uniform spaces.
We might be able to say the familiar definition of relative compactness has come from the following proposition:
\begin{prop}
A subset of a complete uniform space, like Banach or Fr\'echet space, is relatively compact if and only if its closure is compact.
\end{prop}
\begin{pf}
In a complete uniform space, the completion can be charaterized as closure.
\end{pf}

The following two definitions are helpful when we need to check relative compactness.
\begin{defn}
A Cauchy space is called \emph{relatively Cauchy compact} if every net has a Cauchy subnet.
\footnote{There are three well-known different definitions: Willard subnet, Kelly subnet, and Aarnes-Anden{\ae}s subnet. However, since the existence of each subnet associated to a common eventuality filter is equivalent, there will be no conflicts among them. In this note, we define a subnet as the monotone cofinal function between the index sets. See Eric schechter's book [].}
\end{defn}
\begin{defn}
A uniform space is called \emph{totally bounded} if for every entourage $E$ there is a finite cover $\{U_i\}_i$ with $U_i\x U_i\subset E$ for each index $i$.
\end{defn}

\begin{thm}
Let $X$ be a uniform space.
The followings are all equivalent:
\begin{cond}
\item $X$ is relatively compact;
\item $X$ is relatively Cauchy compact;
\item $X$ is totally bounded.
\end{cond}
\end{thm}
\begin{pf}
Let $\tld X$ be the completion of $X$ and $\cU$ be the uniquely extended uniformity of $\tld X$ from $X$.

(1)$\impl$(2).
Every net in $X$ has a subnet that converges in $\tld X$.
The subnet is Cauchy.

(1)$\impl$(3).
Let $E\in\cU$.
Since $\{U:U\text{ is open in }\tld X,\ U\x U\subset E\}$ is an open cover of $\tld X$, there is a finite subcover $\{U_i\}_i$ of $\tld X$.
Then, $\{U_i\cap X\}_i$ is a finite cover of $X$ satisfying $(U_i\cap X)\x(U_i\cap X)\subset E$.

(2)$\impl$(1).
If we show $\tld X$ is also relatively Cauchy compact, we are done.
That is because of the fact that compactness and net compactness are always equivalent in all topological spaces.

Let $\tld x:\fA\to\tld X$ be a net in $\tld X$.
Take an approximating net $x:\fA\x\cU\to X$ in $X$ such that
\[(x_{(\alpha,E)},\tld x_\alpha)\in E\]
for all $\alpha\in\fA$ and all entourages $E\in\cU$.
Recall that $(\alpha,E)\succ(\alpha',E')$ if and only if $\alpha\succ\alpha'$ and $E\subset E'$.
By the assumption that $X$ is relatively Cauchy compact, there is a Cauchy subnet
\[xh:\fB\to{h}\fA\x\cU\to X\]
of $x$.
Then,
\[\tld x\alpha h:\fB\to{h}\fA\x\cU\to{\alpha}\fA\to\tld X\]
is a subnet of $\tld x$ since both $h$ and the projection $\alpha$ are monotone and cofinal so that so is their composition.
We claim that $\tld x\alpha h$ is Cauchy.

In order to show this, take an arbitrary entourage $E$.
We may assume $E$ is symmetric.
By the Cauchyness of $xh$, we can find $\beta_0\in\fB$ such that
\[\beta,\beta'\succ\beta_0\impl(x_{h(\beta)},x_{h(\beta')})\in E.\]
Moreover, we may assume $h(\beta_0)=(\alpha_0,E_0)$ satisfies $(\alpha_0,E_0)\succ(\alpha_0,E)$ so that
\[(x_{h(\beta_0)},\tld x_{\alpha h(\beta_0)})=(x_{(\alpha_0,E_0)},\tld x_{\alpha_0})\in E_0\subset E.\]
Then, for $\beta,\beta'\succ\beta_0$, the following three relations
\[(x_{h(\beta)},\tld x_{\alpha h(\beta)})\in E_0\subset E,\quad(x_{h(\beta)},x_{h(\beta')})\in E,\quad(x_{h(\beta')},\tld x_{\alpha h(\beta')})\in E_0\subset E\]
implies $(\tld x_{\alpha h(\beta)},\tld x_{\alpha h(\beta')})\in E^3$.
Therefore, $\tld x\alpha h:\fB\to\tld X$ is Cauchy.

(3)$\impl$(1).
Clearly, $\tld X$ is totally bounded.
A complete totally bounded space is compact.
\end{pf}

In the last of the proof above, we used the famous result of complete totally bounded spaces.
This can be easily proved with diagonal subsequence extracting argument under the metric space condition, but we need the subtle application of the axiom of choice when we require only uniformness of the space.

\begin{thm}
A complete totally bounded uniform space is compact.
\end{thm}
\begin{pf}
Omitted.
Refer to any advanced general topology text or my other expository essay: totally bounded uniform spaces.
\end{pf}


Empirically, compactness checking problems seem to be fallen into two cases: one is to show both completeness and relative compactness, and the other is to apply Tychonoff's theorem.
Followings are some examples.

\begin{ex}[The Heine-Borel theorem]
A subset of a Euclidean space is compact if and only if it is closed and bounded.
\end{ex}
\begin{ex}[The Blaschke selection theorem]
Let $H(X)$ be the metric space of nonempty compact subsets of a metric space $X$ with Hausdorff metric.
If $X$ is compact, then $H(X)$ is compact.
\end{ex}
\begin{ex}[The Banach-Alaoglu theorem]
Let $X$ be a locally convex space and $X^*$ the continuous dual of $X$ with weak$^*$ topology.
If $B\subset X^*$ satisfies $\sup_{f\in B}|f(x)|<\infty$ for each $x\in X$, then $B$ is compact.
\end{ex}
\begin{pf}
We can embed $X^*$ into $\F^X$.
\end{pf}















\chapter{Separation axioms}

\section{Separation axioms}
\section{Metrization theorems}









\chapter{Function spaces}




\section{Compact-open topology}

% 6-1-1
\subsection{Definition}

\begin{defn}
Let $X$ and $Y$ be topological spaces.
The \emph{continuous functions space} $C(X,Y)$ is the set of continuous functions from $X$ to $Y$.
If $Y=\mathbb{R}$ or $\mathbb{C}$, then the continuous function space is denoted by $C(X)$.
\end{defn}

% 6-1-2
\subsection{Compact convergence}
topology of compact convergence
metrizability and hemicompact
topology of uniform convergence
uniform structure of pointwise convergence


$C(X,Y)$ has the compact-open topology
When $Y$ is uniform, $C(X,Y)$ has the topology of compact convergence
When $X$ is compact and $Y$ is uniform, $C(X,Y)$ has the topology of uniform convergence

When $X$ is locally compact Hausdorff, and the exponential space is endowed with compact-open topology


%%%%%%%%%%%%%%%%%%%%%%%%%%%%%
In considering the continuous function space, $Y$ will be assumed to be a metric space because of its usefulness in most applications.
Then, there are two useful topologies on $C(X,Y)$.
Since there is a difficulty to deal with open sets or basis directly in a function space, the convergence will be a reliable alternative to describe the topologies.
Before giving definition of the topologies, define pseudometrics $\rho_K$ on $C(X,Y)$ by
\[\rho_K(f,g)=\sup_{x\in K}d(f(x),g(x))\]
for $K\subset X$ compact.

\begin{defn}
Let $X$ and $Y$ be topological spaces.
The \emph{topology of pointwise convergence} on $C(X,Y)$ is a subspace topology inherited from the product topology on $Y^X$.
\end{defn}

\begin{prop}
Let $X$ be a topological space and $Y$ be a metric space.
The topology of pointwise convergence on $C(X,Y)$ is generated by pseudometrics $\rho_{\{x\}}$, namely all $\{g:d(f(x),g(x))<\e\}$ for $f\in C(X,Y)$, $\e>0$, and $x\in X$.
\end{prop}

\begin{defn}
Let $X$ be a topological space and $Y$ be a metric space.
The \emph{topology of compact convergence} on $C(X,Y)$ is a topology generated by pseudometrics $\rho_K$, namely all $\{g:\rho_K(f,g)<\e\}$ for $f\in C(X,Y)$, $\e>0$, and compact $K\subset X$.
\end{defn}

\begin{prop}
Let $C(X,Y)$ be a continuous function space for a topological space $X$ and a metric space $Y$.
A functional sequence in $C(X,Y)$ converges in the topology of compact convergence if and only if the functional sequence converges compactly.
\end{prop}

\begin{thm}
Let $X$ be a topological space and $Y$ be a metric space.
If $X$ is hemicompact, in other words, $X$ has a sequence of compact subsets $\{K_n\}_{n\in\N}$ such that every compact subset of $X$ is contained in $K_n$ for some $n\in\N$, then the topology of compact convergence on $C(X,Y)$ is metrizable.
\end{thm}

\begin{proof}
bounding and merging pseudometrics
\end{proof}



% 6-1-3
\subsection{Exponentiability}
locally compact Hausdorff spaces
exponential space






$\frac\e3$ argument





% SECTION 6-2
\section{Rings of continuous functions}



\subsection{$C(X), C_0(X), C_b(X)$}



\section{Important theorems on function space}


\subsection{The Arzela-Ascoli theorem}

The Arzela-Ascoli theorem is a main technique to verify compactness of a subspace of continuous function space.
The theorem requires the notion of equicontinuity, which lifts pointwise compactness up onto compactness in topology of compact convergence.

\begin{defn}
Let $X$ be a topological space and $Y$ be a metric space.
A subset $\mathcal{F}\subset C(X,Y)$ is called \emph{(pointwise or locally) equicontinuous} if for every $\e>0$ and each $x_0\in X$, there is an open neighborhood $U$ of $x_0$ such that $x\in U\impl d(f(x),f(x_0))<\e$ for all $f\in\cF$.
\end{defn}
Compare with the following definition:
\begin{defn}
Let $X$ be a metric space and $Y$ be a metric space.
A subset $\mathcal{F}\subset C(X,Y)$ is called \emph{uniformly equicontinuous} if for every $\e>0$ there exists $\delta>0$ such that $d(x,y)<\delta\impl d(f(x),f(y))<\e$ for all $f\in\cF$.
\end{defn}
The uniform equicontinuity is what the Rudin's book says it just equicontinuous.

\begin{thm}[Arzela-Ascoli, conventional version]
Let $X$ be a compact space.
For $\{f_n\}_{n\in\N}\subset C(X)$, if it is equicontinuous and pointwisely bounded, then there is a subsequence that uniformly converges.
\end{thm}

Let $\cT_p$ be the topology of pointwise convergence and $\cT_c$ be the topology of compact convergence.

\begin{thm}[Arzela-Ascoli, metrized version]
Let $X$ be a hemicompact space and $Y$ be a metric space.
If $\cF\subset C(X,Y)$ is equicontinuous and relatively compact in $\cT_p$, then it is relatively compact in $\cT_c$.
\end{thm}
\begin{pf}
Let $\{f_n\}_{n\in\N}$ be a sequence in $\cF$ and $K\subset X$ be a compact.

By equicontinuity, for each $k\in\N$ a finite open cover $\{U_s\}_{s\in S_k}$ with a finite set $S_k\subset K$ can be taken such that $x\in U_s\,\impl\,d(f(x),f(s))<\frac1k$ for all $f\in\cF$.
By the pointwise relative compactness, we can extract a subsequence $\{f_m\}_{m\in\N}$ of $\{f_n\}_n$ such that $\{f_m(s)\}_m$ is Cauchy for each $s\in\bigcup_{k\in\N}S_k$ by the diagonal argument.

For every $\e>0$, let $k=\ceil{\e^{-1}}$ so that $\frac1k\le\e$.
Let $m_{0,s}$ be an index such that $m,m'>m_{0,s}\impl d(f_m(s),f_{m'}(s))<\e$, and define $m_0=\max\{m_{0,s}:s\in S_k\}$.
Then, for arbitrary $x\in K$, we obtain $m,m'>m_0\impl$
\[d(f_m(x),f_{m'}(x))\le d(f_m(x),f_m(s))+d(f_m(s),f_{m'}(s))+d(f_{m'}(s),f_{m'}(x))<3\e\]
by taking $s\in S_k$ such that $x\in U_s$.
Thus, $\{f_m\}_m$ is a subsequence of $\{f_n\}_n$ that is uniformly Cauchy on $K$.
\end{pf}

If $C(X,Y)$ is not metrizable, or if $Y$ is uniform but not metrizable, the subsequence extracting procedure is no more available.

\begin{thm}[Arzela-Ascoli, generalized version]
Let $X$ be a topological space and $Y$ be a uniform space.
If $\cF\subset C(X,Y)$ is equicontinuous and relatively compact in $\cT_p$, then it is relatively compact in $\cT_c$.
\end{thm}
\begin{pf}[1]
Every net in $\cF$ has a subnet that is pointwisely Cauchy.
Then we are done if we prove every pointwisely Cauchy net in $\cF$ is in fact uniformly Cauchy on each compact set $K\subset X$.

Let $f:\fA\to\cF$ be a pointwisely Cauchy net.
Let $E$ be an arbitrary entourage in $Y$.
By equicontinuity, every point $s\in X$ has a neighborhood $U_s$ such that
\[x\in U_s\impl(f(x),f(s))\in E\]
for all $f\in\cF$.
We can find a finite set $S\subset K$ such that $\{U_s\}_{s\in S}$ is a cover of $X$.
(In here, we do not need to extract a subnet because the net is already pointwisely Cauchy.)
Let $\alpha_{0,s}$ be an index such that
\[\alpha,\alpha'\succ\alpha_{0,s}\impl(f_\alpha(s),f_{\alpha'}(s))\in E,\]
and define $\alpha_0=\max\{\alpha_{0,s}\}_{s\in S}$.
Then, for arbitrary $x\in K$ there is $s\in S$ such that $x\in U_s$, so we get
\[\alpha,\alpha'\succ\alpha_0\impl(f_\alpha(x),f_{\alpha'}(x))=(f_\alpha(x),f_\alpha(s))\o(f_\alpha(s),f_{\alpha'}(s))\o(f_{\alpha'}(s),f_{\alpha'}(x))\in E^3.\]
Thus, $f$ is uniformly Cauchy on $K$.
\end{pf}

\begin{proof}[Proof 2]

\end{proof}

\begin{proof}[Proof 3]
We are going to prove the two topologies coincide in $\cF$.
Note that subspace topologies $\cT_p\vert_\cF$ and $\cT_c\vert_\cF$ are also generated by pseudometrics $\rho_{\{x\}}(f,g)=d(f(x),g(x))$ and $\rho_K(f,g)=\sup\{d(f(x),g(x)):x\in K\}$ defined on $\cF$ respectively.
Since $\cT_p\vert_\cF\subset\cT_c\vert_\cF$ clearly, it is enough to show the converse.

Take a subbasis element $B_K(f_0,\e)\cap\cF$ of $\cT_c\vert_\cF$ with $f_0\in\cF$.
By equicontinuity of $\cF$, there is a finite cover $\{U_s\}_{s\in S}$ of a compact subset $K$ such that $x\in U_s\,\impl\,d(f(x),f(s))<\e$ for all $f\in\cF$.

If $f\in\cF$ satisfies $\rho_{\{s\}}(f,f_0)<\e$ for all $s\in S$, then we get
\[\rho_K(f,f_0)\le\sup_{x\in K}d(f(x),f(s))+d(f(s),f_0(s))+\sup_{x\in K}d(f_0(s),f_0(x))<\e\]
by taking $s\in S$ such that $x\in U_s$.
This means
\[\bigcap_{s\in S}B_{\{s\}}(f_0,\e)\subset B_K(f_0,3\e).\]
It implies that the subbasis element contains an open set in $\cT_p\vert_\cF$.
Therefore, $\cT_c\vert_\cF\subset\cT_p\vert_\cF$.
\end{proof}

converse of Arzela-Ascoli

\subsection{The Stone-Weierstrass theorem}








\end{document}












Limiting inequality
dependency diagram
double limit table










