\documentclass[11pt]{amsart}
\usepackage{../../ikany}
\usepackage[margin=1.5in]{geometry}

\title{Vlasov-Poisson system}
\author{Ikhan Choi}

\def\tint{{\textstyle\int}}
\def\loc{\mathrm{loc}}

\begin{document}
\maketitle
\tableofcontents
\section*{Acknowledgement}


\section{Vlasov-Poisson system}
Consider a Cauchy problem of the \emph{Valsov-Poisson system}:
\begin{pde*}
&f_t+v\cdot\del_xf+\gamma E\cdot\del_vf=0,\:(t,x,v)\in\R_t^+\x\R_x^3\x\R_v^3,\\
&E(t,x)=-\del_x\Phi,\\
&\Phi(t,x)=(-\Delta_x)^{-1}\rho,\\
&\rho(t,x)=\tint f\,dv,\\
&f(0,x,v)=f_0(x,v),
\end{pde*}
where $\gamma=\pm1$.
For example, we have \emph{repulsive problem} $\gamma=+1$ for electrons in plasma theory and \emph{attractive problems} $\gamma=-1$ for galactic dynamics.
($\rho$ denotes mass density.)

Results in 1.1 and 1.2 provide basic ingredients that will be used in the whole article.
On the other hand, results in 1.3 and 1.4 cannot be used in any local existence proof because they assume the existence of solutions, but they help understand the fundamental nature of solutions of the Vlasov-Poisson system and are used in the proof of global existence.

We use the asymptotic notation
\[g(t)\les h(t)\iff\exists\,c=c(f_0),\quad g(t)\le c\,h(t)\]
and
\[g(t)\simeq h(t)\iff\exists\,c,\quad g(t)=c\,h(t).\]




\subsection{Poisson equation}
For the boundaryless problem in which the potential function vanishes at infinity, we have
\[\Phi=\tfrac1{4\pi|x|}*\rho,\]
so
\[E=-\del_x\Phi=-\del_x(\tfrac1{4\pi|x|}*\rho)=\frac{x}{4\pi|x|^3}*\rho,\]
or it can be rewritten as
\[E(t,x)=\frac1{4\pi}\int\frac{(x-y)\rho(t,y)}{|x-y|^3}\,dy.\]

The nonlinearity of the system is originated from the force field $E$, so its estimates play the most important role in investigation of the nonlinear system.
Since it is given by the solution of the Poisson equation, estimates of the Riesz potential is directly connected to estimates of the force field.


\begin{lem}[Estimates of Riesz potential]
Let $\rho\in C_c^1(\R^d)$.
\begin{cond}
\item(Field estimate)
\[\|\tfrac1{|x|^{d-1}}*\rho\|_\infty\les\|\rho\|_\infty^{1-1/d}\|\rho\|_1^{1/d}\]
\item(Field derivative estimate)
For $\log^+(x):=\max\{0,\log x\}$,
\[\|\del(\tfrac1{|x|^{d-1}}*\rho)\|_\infty\les1+\|\rho\|_\infty\log^+\|\del\rho\|_\infty+\|\rho\|_1.\]
\end{cond}
\end{lem}

\begin{pfs}
\item
Let $0\le\frac1p<\frac\alpha d<\frac1q\le1$.
Since $(d-\alpha)p<d<(d-\alpha)q$,
\begin{align*}
|\tfrac1{|x|^{d-\alpha}}*\rho|
&=\int_{|x-y|<R}\frac{\rho(y)}{|x-y|^{d-\alpha}}\,dy+\int_{|x-y|\ge R}\frac{\rho(y)}{|x-y|^{d-\alpha}}\,dy\\
&\le\|\rho\|_{p'}(\int_{|y|<R}\frac{dy}{|y|^{(d-\alpha)p}})^{1/p}+\|\rho\|_{q'}(\int_{|y|\ge R}\frac{dy}{|y|^{(d-\alpha)q}})^{1/q}\\
&\simeq\|\rho\|_{p'}(\int_0^Rr^{d-1-(d-\alpha)p}\,dr)^{1/p}+\|\rho\|_{q'}(\int_R^\infty r^{d-1-(d-\alpha)q}\,dr)^{1/q}\\
&\simeq\|\rho\|_{p'}R^{\frac dp-d+\alpha}+\|\rho\|_{q'}R^{\frac dq-d+\alpha}.
\end{align*}
By choosing $R$ such that $\|\rho\|_{p'}R^{\frac dp-d+\alpha}=\|\rho\|_{q'}R^{\frac dq-d+\alpha}$, we get
\[\|\tfrac1{|x|^{d-\alpha}}*\rho\|_\infty\les\|\rho\|_{p'}^{\frac{1-\frac\alpha d-\frac1q}{\frac1p-\frac1q}}\|\rho\|_{q'}^{\frac{\frac1p-1+\frac\alpha d}{\frac1p-\frac1q}},\]
so the inequality
\[\|\tfrac1{|x|^{d-\alpha}}*\rho\|_\infty^{\frac1q-\frac1p}\les\|\rho\|_p^{\frac1q-\frac\alpha d}\|\rho\|_q^{\frac\alpha d-\frac1p}\]
is obtained by interchaning $p$ and $q$ with their conjugates.
The desired result gets $p=\infty$, $\alpha=1$, and $q=1$.

\item
Let $0<R_a\le R_b$ be constants which will be determined later.
Divide the region radially
\begin{align*}
|\del(\tfrac1{|x|^{d-\alpha}}*\rho)|\les\del\int_{|x-y|<R_a}+\del\int_{R_a\le|x-y|<R_b}+\del\int_{R_b\le|x-y|}.
\end{align*}
For the first integral,
\begin{align*}
\int_{|y|<R_a}\frac{\del\rho(x-y)}{|y|^{d-1}}\,dy
&\le\|\del\rho\|_\infty\int_{|y|<R_a}\frac1{|y|^{d-1}}\,dy\\
&\simeq\|\del\rho\|_\infty\int_0^{R_a}1\,dr
=R_a\|\del\rho\|_\infty.
\end{align*}
For the second integral,
\begin{align*}
\int_{R_a\le|x-y|<R_b}\frac{\rho(y)}{|x-y|^d}\,dy
&\le\|\rho\|_\infty\int_{R_a\le|x-y|<R_b}\frac1{|x-y|^d}\,dy\\
&\simeq\|\rho\|_\infty\int_{R_a}^{R_b}\frac1r\,dr
=(\log\tfrac{R_b}{R_a})\|\rho\|_\infty.
\end{align*}
For the third integral,
\[\int_{R_b\le|x-y|}\frac{\rho(y)}{|x-y|^d}\,dy\le R_b^{-d}\|\rho\|_1.\]
Thus,
\[|\del(\tfrac1{|x|^{d-\alpha}}*\rho)|\les R_a\|\del\rho\|_\infty+(\log\tfrac{R_b}{R_a})\|\rho\|_\infty+R_b^{-d}\|\rho\|_1.\]

Assuming $\rho$ is nonzero so that $\|\del\rho\|_\infty>0$, let $R_a=\min\{1,\|\del\rho\|_\infty^{-1}\}$ and $R_b=1$.
Since
\[\log\tfrac1{R_a}\le\log^+\|\del\rho\|_\infty\quad\text{and}\quad R_a\les\|\del\rho\|_\infty,\]
we have
\[\|\del(\tfrac1{|x|^{d-1}}*\rho)\|_\infty\les1+\|\rho\|_\infty\log^+\|\del\rho\|_\infty+\|\rho\|_1.\qedhere\]
\end{pfs}

\subsection{Characteristics and volume preservation}

The Vlasov-Poisson equation is quite hyperbolic so that the method of characteristic curves are extremely useful.
Without explicit representation formula, solutions given by the characteristics make appropriate estimates possible.

\begin{lem}
Let $f\in C^1([0,T],C_c^1(\R^6))$ be a solution of the Vlasov-Poisson system.
\begin{cond}
\item Fix $t,x,v$. The following ordinary equation with variable $s$ has a solution in $C^1([0,T],\R^6)$:
\begin{gather*}
\dot X(s;t,x,v)=V(s;t,x,v),\quad\dot V(s;t,x,v)=\gamma E(t,X(s;t,x,v)),\\
X(t;t,x,v)=x,\qquad V(t;t,x,v)=v.
\end{gather*}
We call them \emph{characteristics}.
\item Fix $t,x,v$. Then, $f(s,X(s;t,x,v),V(s;t,x,v))=\const$.
\item Fix $t,s\in[0,T]$ and let $y=X(s;t,x,v)$ and $w=V(s;t,x,v)$. Then, the Jacobian of the coordinates transformation $(x,v)\mapsto(y,w)$ is 1.
\end{cond}
\end{lem}
\begin{pfs}
\item
Note that we have
\[\rho\in C^1([0,T];C_c^1(\R^6)),\quad\Phi\in C^1([0,T];C^{2,\alpha}(\R^6))\]
so that
\[E\in C^1([0,T];C^{1,\alpha}(\R^6))\]
by the H\"older regularity of the Poisson equation.
Since a map
\[(x,v)\mapsto(v,\gamma E(t,x))\]
is globally Lipschitz with respect to $(x,v)$ for each $t$, we can apply the Picard Lindel\"of theorem.

\item
Differentiate and use the chain rule to get
\begin{align*}
\dd{s}&f(s,y,w)\\
&=\pd_tf(s,y,w)+\dot X(s;s,y,w)\cdot\del_xf(s,y,w)+\dot V(s;s,y,w)\cdot\del_vf(s,y,w)\\
&=\pd_tf(s,y,w)+w\cdot\del_xf(s,y,w)+\gamma E(s,y)\cdot\del_vf(s,y,w)=0,
\end{align*}
where we denote $y=X(s;t,x,v)$ and $w=V(s;t,x,v)$.

\item
Fix $t,x,v$ and let $J(s)=\pd{(y,w)}{(x,v)}$ be the Jacobi matrix.
We want to show
\[\det J(s)=\const\]
becuase when $s=t$ the Jacobian is
\[\det J(0)=\det\pd{(x,v)}{(x,v)}=1.\]
Since
\[J^{-1}(s)\dd{s}J(s)=\pd{(x,v)}{(y,x)}\dd{s}\pd{(y,w)}{(x,v)}=\pd{(\dot y,\dot w)}{(y,w)}=\mat{0&1\\\gamma\pd{E}{y}(s,y)&0},\]
the Jacobi formula deduces that
\[\dd{s}\det J(s)=\det(s)\tr\left(J^{-1}(s)\dd{s}J(s)\right)=0.\qedhere\]

\end{pfs}

\begin{cor}
Let $f\in C^1([0,T],C_c^1(\R^6))$ be a solution of the Vlasov-Poisson system.
Then, for any measurable function $\beta:\R\to\R$ such that $\iint\beta\o f_0(x,v)\,dv\,dx<\infty$, we have
\[\iint\beta\o f(t,x,v)\,dv\,dx=\const.\]
In particular,
\[\|f(t)\|_p=\const\]
for all $1\le p\le\infty$.
\end{cor}
\begin{pf}
Fix $t,s[0,T]$ and denote $y=X(s;t,x,v)$ and $w=V(s;t,x,v)$.
Then,
\begin{align*}
\iint\beta\o f(t,x,v)\,dv\,dx
&=\iint\beta\o f(s,X(s;t,x,v),V(s;t,x,v))\,dv\,dx\\
&=\iint\beta\o f(s,y,w)\,dw\,dy
\end{align*}
for $s\le T$.
\end{pf}
\begin{rmk}
Note that this result can be obtained in the approximation scheme, which will be suggested in the next section.
\end{rmk}

To sum up our weapons obtained in 1.1 and 1.2,
\begin{cor}
If a function $f\in C^1([0,T],C_c^1(\R^6))$ satisfies
\[\iint f(t,x,v)\,dv\,dx=\const,\]
and if we let
\[\rho(t,x)=\int f(t,x,v)\,dv,\quad E(t,x)=\frac1{4\pi}\int\frac{(x-y)\rho(t,y)}{|x-y|^3}\,dy,\]
then
\begin{cond}
\item $\|\rho(t)\|_1=\const$,
\item $\|E(t)\|_\infty\les\|\rho(t)\|_\infty^{2/3}$,
\item $\|\del E(t)\|_\infty\les1+\|\rho\|_\infty\log^+\|\del\rho\|_\infty$.
\end{cond}
\end{cor}



\subsection{Conservative laws}

\begin{lem}
Let $f\in C^1([0,T],C_c^1(\R^6))$ be a solution of the Vlasov-Poisson system.
\begin{cond}
\item(Continuity equation)
\[\rho_t+\del_x\cdot j=0,\qquad\text{where}\quad j=\int vf\,dv.\]
\item(Energy conservation)
\[\iint|v|^2f\,dv\,dx+\gamma\int|E|^2\,dx=\const.\]
\end{cond}
\end{lem}
\begin{pfs}
\item
Integrate with respect to $v$ to get
\begin{align*}
0&=\int f_t\,dv+\int v\cdot\del_xf\,dv\\
&=\rho_t+\del_x\cdot\int vf\,dv\\
&=\rho_t+\del_x\cdot j.
\end{align*}
\item
Multiply $|v|^2$ and integrate with respect to $v$ and $x$ to get
\begin{align*}
\dd{t}\iint|v|^2f\,dv\,dx
&=\iint|v|^2f_t\,dv\,dx=-\iint|v|^2\gamma E\cdot\del_vf\,dv\,dx\\
&=\iint2v\cdot\gamma Ef\,dv\,dx=-2\gamma\int\del_x\Phi\cdot j\,dx\\
&=2\gamma\int\Phi\del_x\cdot j\,dx=2\gamma\int\Phi\Delta_x\Phi_t\,dx\\
&=-\dd{t}\gamma\int|E|^2\,dx.
\end{align*}
Thus
\[\iint|v|^2f\,dv\,dx+\gamma\int|E|^2\,dx=\const.\qedhere\]
\end{pfs}



\subsection{Moment propagation}

We have a bound of kinetic energy even for $\gamma=-1$.

\begin{lem}[$L^{5/3}$ estimate of $\rho$]
Let $f\in C^1([0,T],C_c^1(\R^6))$ be a solution of the Vlasov-Poisson system.
For $t\in[0,T]$,
\begin{cond}
\item $\|\rho(t)\|_{L_x^{5/3}}\les\iint|v|^2f\,dv\,dx$.
\item $\iint|v|^2f\,dv\,dx\les1$.
\end{cond}
\end{lem}
\begin{pfs}
\item
Note
\begin{align*}
\rho(t,x)=\int f(t,x,v)\,dv
&\le\int_{|v|<R}f\,dv+\frac1{R^2}\int_{|v|\ge R}|v|^2f\,dv\\
&\les R^3+ R^{-2}\int|v|^2f\,dv.
\end{align*}
Set $R^3=R^{-2}\int|v|^2f\,dv$ to get
\[\rho(t,x)^{5/3}\les\int|v|^2f\,dv.\]

\item
It is trivial for $\gamma=+1$.
Suppose $\gamma=-1$.
By the Hardy-Littlewood-Sobolev inequality,
\[\frac1p+\frac\alpha d=\frac1q\]
for $p=2$, $d=3$, and $\alpha=1$ implies $q=6/5$, hence the bound of $\|E(t)\|_2$
\[\|E(t)\|_2\simeq\|\frac1{|x|^{d-\alpha}}*_x\rho(t,x)\|_{L_x^2}\les\|\rho(t)\|_{6/5}.\]
So, interpolation with H\"older's inequality gives
\[\|E(t)\|_2\les\|\rho\|_1^{7/12}\|\rho\|_{5/3}^{5/12}\simeq\|\rho\|_{5/3}^{5/12}.\]
Thus (1) gives
\[\iint|v|^2f\,dv\,dx=c+\|E(t)\|_2^2\les c+(\iint|v|^2f\,dv\,dx)^{1/2},\]
so the kinetic energy $\iint f\,dv\,dx$ is bounded.
As a corollary, $\|\rho(t)\|_{5/3}$ is also bounded.\qedhere
\end{pfs}







\section{Local existence}

\subsection{Approximate solution}


\begin{defn}
We define an (global) \emph{approximate solution} as a sequence of functions $f_n\in C^1(\R^+,C_c^1(\R^6))$ such that
\begin{pde*}
&\pd_tf_{n+1}+v\cdot\del_xf_{n+1}+\gamma E_n\cdot\del_vf_{n+1}=0,\\
&E_n(t,x)=-\del_x\Phi_n,\\
&\Phi_n(t,x)=(-\Delta_x)^{-1}\rho_n,\\
&\rho_n(t,x)=\tint f_n\,dv,\\
&f_{n+1}(0,x,v)=f_0(x,v).
\end{pde*}
This definition is made in order to let the force field $E$ constant when solving $f_{n+1}$.
\end{defn}


\begin{prop}
An approximate solution exists.
\end{prop}
\begin{pf}
Let $f_0(t,x,v)=f_0(x,v)$.
Notice that $f_0$ is clearly in $C^1(\R^+;C_c^1(\R^6))$.
Assume $f_n\in C^1(\R^+;C_c^1(\R^6))$ satisfies the approximate system.
We want to show that there is $f_{n+1}$ that satisfies the approximate system and $f_{n+1}\in C^1(\R^+;C_c^1(\R^6))$.

We have
\[\rho_n\in C^1(\R^+;C_c^1(\R^6)),\quad\Phi_n\in C^1(\R^+;C^{2,\alpha}(\R^6)),\ \text{and}\ E_n\in C^1(\R^+;C^{1,\alpha}(\R^6))\]
by the H\"older regularity of the Poisson equation.
Since a map $(x,v)\mapsto(v,\gamma E_n(t,x))$ is globally Lipschitz with respect to $(x,v)$ for each $t$, the classical Picard iteration uniquely solves the characteristic equation
\begin{pde*}
\dot X_{n+1}(s;t,x,v)&=V_{n+1}(s,t,x,v)\\
\dot V_{n+1}(s;t,x,v)&=\gamma E_n(s,X_{n+1}(s;t,x,v))
\end{pde*}
with condition $(X_{n+1}(t;t,x,v),V_{n+1}(t;t,x,v))=(x,v)$ and proves the uniqueness and regularity of the solution $s\mapsto(X_{n+1},V_{n+1})(s;t,x,v)\in C^1(\R^+,\R^6)$.

Define
\[f_{n+1}(t,x,v):=f_0(X_{n+1}(0;t,x,v),V_{n+1}(0;t,x,v)).\]
Then, we can show that
\begin{align*}
f_{n+1}(s,X_{n+1}(s;t,x,v),V_{n+1}(s;t,x,v))&\\
=f_0(X_{n+1}(0;t,x,v),V_{n+1}(0;t,x,v))&=\const
\end{align*}
and that $f_{n+1}$ satisfies the approximate system by the chain rule
\begin{align*}
0&=\left.\dd{s}f_{n+1}(s,X_{n+1}(s;t,x,v),V_{n+1}(s;t,x,v))\right|_{s=t}\\
&=\pd_tf_{n+1}(t,x,v)+\dot X_{n+1}(t;t,x,v)\cdot\del_xf_{n+1}(t,x,v)\\
&\hspace{7.5em}+\dot V_{n+1}(t;t,x,v)\cdot\del_vf_{n+1}(t,x,v)\\
&=\pd_tf_{n+1}(t,x,v)+v\cdot\del_xf_{n+1}(t,x,v)+\gamma E_n(t,x)\cdot\del_vf_{n+1}(t,x,v).
\end{align*}
Also, $f_{n+1}$ has compact support for each $t$ since the characteristic does not blow up; finally we have $f_{n+1}\in C^1(\R^+,C_c^1(\R^6))$.
\end{pf}
\begin{rmk}
Although the approximate solution is unique when given the initial term $f_0(t,x,v)=f_0(x,v)$, we do not care of the uniqueness, but only the existence.
\end{rmk}



\subsection{Local A priori estimates}

Firstly, the volume preserving property still holds for our approximate system, so we have
\[\|\rho_n(t)\|_1=\const,\quad\|f_n(t)\|_p=\const.\]
Next, we prove local-time bounds on fields $E_n$.
Introduce the following quantity.
\begin{defn}
Define the \emph{velocity support} or \emph{maximal velocity}
\[Q_n(t)=\sup\{\,|v|:f_n(s,x,v)\ne0,\ s\in[0,t]\,\}\]
\end{defn}

\begin{lem}
There is a constant $T=T(f_0)$ such that
\begin{cond}
\item
for $t\le T$
\[\|\rho_n(t)\|_\infty+\|E_n(t)\|_\infty+Q_n(t)\les1\]
indendent on $n$.
In addition, the support of $X_n$ is also uniformly bounded in $t\le T$.
\item
for $t\le T$
\[\|\del_x\rho_n(t)\|_\infty+\|\del_xE_n(t)\|_\infty\les1\]
independent on $n$.
\end{cond}
\end{lem}
\begin{pfs}
\item
Since
\[\|\rho_n(t)\|_\infty\le Q_n^3(t)\|f_0\|_\infty\les Q_n^3(t),\]
a rough estimate for $\|E\|_\infty$ gives
\[\|E_n(t)\|_\infty\le\|\rho_n(t)\|_\infty^{2/3}\|\rho_n(t)\|_1^{1/3}\les Q_n^2(t).\]
Let $c=c(f_0)$ be a constant such that $\|E_n(t)\|\le cQ_n^2(t)$.
We claim that
\[Q_n(t)\le\frac{Q_0}{1-cQ_0t}\]
for all $n$.
Easily checked for $n=0$; $Q_0(t)\equiv Q_0\le\frac{Q_0}{1-cQ_0t}$.

Assume $Q_n(t)\le\frac{Q_0}{1-cQ_0t}$.
Then,
\begin{align*}
|V_{n+1}(t;0,x,v)|
&\le|v|+\int_0^t|E_n(s;0,x,v)|\,ds\\
&\le Q_0+c\int_0^tQ_n^2(s)\,ds
\end{align*}
implies
\begin{align*}
Q_{n+1}(t)
&\le Q_0+c\int_0^tQ_n^2(s)\,ds\\
&\le Q_0+c\int_0^t\left(\frac{Q_0}{1-cQ_0s}\right)^2ds
=\frac{Q_0}{1-cQ_0t}.
\end{align*}
By induction, $Q_n(t)\le\frac{Q_0}{1-cQ_0t}\les1$ for all $n$ and $t\in[0,T]$, where $T<(cQ_0)^{-1}$.

For the position support, we can bound it because
\[|X_n(t;0,x,v)|\le|x|+\int_0^t|V_n(s;0,x,v)|\,ds\le|x|+TQ_n(t)\les1.\]
\item
%%%%%%%%
\qedhere
\end{pfs}



\subsection{Convergence of approximate solution}
\begin{lem}
There is $T=T(f_0)$ such that
\begin{cond}
\item
for $t\le T$ and $n\ge1$,
\[\|f_{n+1}(t)-f_n(t)\|_\infty\les\int_0^t\|E_n(s)-E_{n-1}(s)\|_\infty\,ds.\]
\item
for $t\le T$ and $n\ge1$,
\[\|E_n(s)-E_{n-1}(s)\|_\infty\les\|f_n(s)-f_{n-1}(s)\|_\infty.\]
\item $f_n$ converges to a function $f$ uniformly in $C([0,T]\x\R^6)$.
\item $(X_n,V_n)$ converges uniformly in $C([0,T]\x\R^6)$, and its limit $(X,V)$ satisfies the characteristic equation
\[\dot X=V,\quad\dot V=\gamma E,\]
where
\[E(t,x)=\frac1{4\pi}\iint\frac{(x-y)f(t,x,v)}{|x-y|^3}\,dv\,dx.\]
\end{cond}
\end{lem}
\begin{pfs}
\item
The $C^1$ regularity of $f_0$ gives
\begin{align*}
|f&_{n+1}(t,x,v)-f_n(t,x,v)|\\
&=|f_0(X_{n+1}(0;t,x,v),V_{n+1}(0;t,x,v))-f_0(X_n(0;t,x,v),V_n(0;t,x,v))|\\
&\les|X_{n+1}(0;t,x,v)-X_n(0;t,x,v)|+|V_{n+1}(0;t,x,v)-V_n(0;t,x,v)|.
\end{align*}
Take $T$ such that
\[\|\del E_n(t)\|_\infty\les1\quad\text{and}\quad T<\min\{\|\del_xE_n(t)\|_\infty^{-1},1\}\]
for all $n$ and $t\le T$.
Then, because
\begin{align*}
X_n(s;t,x,v)&=x-\int_s^tV_n(s';t,x,v)\,ds',\\
V_n(s;t,x,v)&=v-\int_s^tE_{n-1}(s',X_n(s;t,x,v))\,ds',
\end{align*}
we have
\begin{align*}
|V&_{n+1}(s;t,x,v)-V_n(s;t,x,v)|\\
&\le\int_0^t|E_n(s,X_{n+1}(s;t,x,v))-E_{n-1}(s,X_n(s;t,x,v))|\,ds\\
&\le\int_0^t|E_n(s,X_{n+1})-E_n(s,X_n)|+|E_n(s,X_n)-E_{n-1}(s,X_n)|\,ds\\
&\le T\sup_{s\in[0,t]}\|\del_xE_n(s)\|_\infty|X_{n+1}(s)-X_n(s)|+\int_0^t\|E_n(s)-E_{n-1}(s)\|_\infty\,ds
\end{align*}
and
\begin{align*}
|X_{n+1}(s;t,x,v)-X_n(s;t,x,v)|
&\le\int_0^t|V_{n+1}(s;t,x,v)-V_n(s;t,x,v)|\,ds\\
&\le T\sup_{s\in[0,t]}|V_{n+1}(s)-V_n(s)|
\end{align*}
for $s\in[0,t]$.
Thus, for $t\le T$ we get
\begin{align*}
&\sup_{s\in[0,t]}(|X_{n+1}(s;t,x,v)-X_n(s;t,x,v)|+|V_{n+1}(s;t,x,v)-V_n(s;t,x,v)|)\\
&\qquad\les\int_0^t\|E_n(s)-E_{n-1}(s)\|_\infty\,ds.\qquad\cdots\cdots(\dagger)
\end{align*}
by combining the above two inequalities.

\item
Notice that
\[\|E_n(t)-E_{n-1}(t)\|_\infty\les\|\rho_n(t)-\rho_{n-1}(t)\|_1^{1/3}\|\rho_n(t)-\rho_{n-1}(t)\|_\infty^{2/3}.\]
For $L^\infty$-norm,
\begin{align*}
\|\rho_n(t)-\rho_{n-1}(t)\|_\infty
&\le\max\{Q_n^3(t),Q_{n-1}^3(t)\}\|f_n(t)-f_{n-1}(t)\|_\infty\\
&\les\|f_n(t)-f_{n-1}(t)\|_\infty.
\end{align*}
For $L^1$-norm, since the support of $f_n,f_{n-1}$ is bounded in both directions $x,v$ in finite time,
\[\|\rho_n(t)-\rho_{n-1}(t)\|_1\le\|f_n(t)-f_{n-1}(t)\|_1\les\|f_n(t)-f_{n-1}(t)\|_\infty\]
for $t\le T$, where $T<\infty$ arbitrary.

\item
Let $T$ be the constant taken in (1).
From (1) and (2), there is a constant $c=c(f_0)$ such that for $t<T$,
\[\|f_{n+1}(t)-f_n(t)\|_\infty\le c\int_0^t\|f_n(s)-f_{n-1}(s)\|\,ds.\]
We can easily get with induction
\[\|f_{n+1}(t)-f_n(t)\|\infty\le\frac{(ct)^n}{n!}\sup_{s\in[0,T]}\|f_1(s)-f_0(s)\|_\infty\les\frac{(ct)^n}{n!}.\]
Therefore,
\[\sum_{n=0}^\infty\|f_{n+1}(t)-f_n(t)\|_\infty\simeq e^{ct}\le e^{cT}<\infty\]
implies $f_n$ uniformly converges.

\item
The convergence is clear by the inequality $(\dagger)$ and the results in (2), (3).\qedhere
\end{pfs}

\begin{prop}
Let $f(t):=\lim_{n\to\infty}f_n(t)$.
Then, $f\in C^1([0,T],C_c^1(\R^6))$ and $f$ is a solution of the Vlasov-Poisson system.
\end{prop}
\begin{pf}
Let $X(s;t,x,v)$ and $V(s;t,x,v)$ be the limits of $X_n$ and $V_n$.
Notice that
\begin{align*}
f(t,x,v)=\lim_{n\to\infty}f_n(t,x,v)&=\lim_{n\to\infty}f_0(X_n(0;t,x,v),V_n(0;t,x,v))\\
&=f_0(X(0;t,x,v),V(0;t,x,v)).
\end{align*}
By the chain rule, we can check it solves the system.
\end{pf}


\subsection{Uniqueness}










\section{Global existence}


\subsection{Prolongation criterion}

\begin{prop}
If $Q(t)$ is bounded, then the solution $f$ of the Vlasov-Poisson system is continued globally to the entire $\R^+$.
\end{prop}
\begin{pf}
Suppose $f\in C^1([0,T_{max}),C_c^1(\R^6))$ for $T_{max}<\infty$ is the maximal solution.
\[<\text{To be written...}>\]

It means that the length of time interval for existence has in fact a lower bound $T>0$ that depends only on $Q(T_{max})$.
Apply the local existence result by setting $t=T_{max}-\frac12T$ as a new initial point.
Then, we can have a solution $f\in C^1([0,T_{\max}+\frac12T),C_c^1(\R^6))$, which contradicts to the maximality of $T_{\max}$.
Therefore, the solution $f$ prolonged forever.
\end{pf}


\begin{thm*}[Schaeffer, 1991]
Let $f_0\in C_c^1(\R^6)$ and $f_0\ge0$.
Then, the Cauchy problem for the VP system has a unique $C_c^1$ global solution.
\end{thm*}





\subsection{Lower bound on relative position vectors}
Our goal is to obtain a priori estimate like
\[\|E(t)\|_\infty\les Q(t)^a\qquad\text{for some }a<1.\]
Since the force field $E$ measures the maximal rate of changes in velocity, the estimate can be read very roughly as
\[Q'(t)\les Q(t)^a,\]
which leads its polynomial growth.
So we need to bound $E$.

Fix a time of existence $t$ and a point $(t,\hat x,\hat v)$ and let
\[\hat X(s):=X(s;t,\hat x,\hat v),\qquad\hat V(s):=V(s;t,\hat x,\hat v).\]
Decompose $[t-\Delta,t]\x\R_x^3\x\R_v^3$ as
\begin{align*}
U&=\left\{\,(s,x,v):\ |v-\hat V(t)|\ge P,\quad|y-\hat X(s)|\ge r\,\right\},\\
B&=\left\{\,(s,x,v):\ |v-\hat V(t)|\ge P,\quad|v|\ge P\,\right\}\setminus U,\\
G&=\left\{\,(s,x,v):\ |v-\hat V(t)|<P\quad\text{or}\quad|v|<P\,\right\}.
\end{align*}
(We can let $U\mapsto U\cap\{|v|\ge P\}$ to make the decomposition disjoint.)
Later we choose
\[P=Q^{4/11},\quad r=R\max\{|v|^{-3},\,|v-\hat V(t)|^{-3}\},\quad R=Q^{16/33}(\log^+Q)^{1/2}.\]
Also, later we choose $\Delta\cdot\sup_{s\le t}\|E(s)\|_\infty<\frac P4$.

Reading the proof, letting $y=X(s;t,x,v)$ and $w=V(s;t,x,v)$ be functions of time variable $s$, trace carefully the following four quantities:
\[|x-\hat X(t)|,\ |y-\hat X(s)|,\ |v-\hat V(t)|,\ |w-\hat V(s)|.\]
The following observation suggests a lower bound of relative position.
\begin{prop}
Fix $x,v$.
Let $P>0$ and $0<\Delta<t$ be constants such that
\[\Delta\cdot\sup_{s\le t}\|E(s)\|_\infty<\frac P4.\]
If $v$ satisfies $|v-\hat V(t)|\ge P$, then there is $s_0\in[t-\Delta,t]$ such that
\[|y-\hat X(s)|\ge\frac14|v-\hat V(t)||s-s_0|\]
for all $s\in[t-\Delta,t]$.
\end{prop}
\begin{pf}
Since $\Delta\|E(s)\|_\infty<\frac P4$, we have
\[|v-w|<\frac P4\quad\text{and}\quad|\hat V(t)-\hat V(s)|<\frac P4.\]
The condition $|v-\hat V(t)|\ge P$ implies
\[\frac12|v-\hat V(t)|\le|v-\hat V(t)|-\frac P4-\frac P4<|w-\hat V(s)|.\]

Let $Z(s):=y-\hat X(s)$ be the relative position vector.
Then,
\begin{align*}
Z'(s)&=w-\hat V(s),\\
Z''(s)&=\gamma[E(s,y,w)-E(s,\hat X(s),\hat V(s))].
\end{align*}
Let $s_0\in[t-\Delta,t]$ minimize $s\mapsto|Z(s)|$ and expand $Z$ as
\[Z(s)=Z(s_0)+Z'(s_0)(s-s_0)+\frac{Z''(\sigma)}2(s-s_0)^2\]
for some $\sigma$ between $s$ and $s_0$.
Then,
\[|Z(s_0)+Z'(s_0)(s-s_0)|\ge|Z'(s_0)(s-s_0)|\ge\frac12|v-\hat V(t)||s-s_0|\]
and
\begin{align*}
|\frac{Z''(\sigma)}2(s-s_0)^2|
&\le\|E(t)\|_\infty(s-s_0)^2
\le\|E(t)\|_\infty\Delta|s-s_0|\\
&\le\frac P4|s-s_0|
\le\frac14|v-\hat V(t)||s-s_0|
\end{align*}
proves
\[|y-\hat X(s)|=|Z(s)|\ge\frac14|v-\hat V(t)||s-s_0|.\qedhere\]
\end{pf}

We introduce time averaging to use the above lower bound.
\begin{prop}
Fix $x,v$.
Let $P>0$ and $0<\Delta<t$ be constants such that
\[\Delta\cdot\sup_{s\le t}\|E(s)\|_\infty<\frac P4.\]
If $v$ satisfies $|v-\hat V(t)|\ge P$, then
\[\int_{t-\Delta}^t\frac1{|y-\hat X(s)|^2}\chi_A(s)\,ds\les\frac{r^{-1}}{|v-\hat V(t)|},\]
where $A=\{s:|y-\hat X(s)|\ge r\}$.
\end{prop}
\begin{pf}
Since $|y-\hat X(s)|\ge\frac14|v-\hat V(t)||s-s_0|$,
\begin{align*}
\int_{t-\Delta}^t\frac1{|y-\hat X(s)|^2}\chi_A(s)\,ds
&\le16\int_{t-\Delta}^t\frac1{|v-\hat V(t)|^2|s-s_0|^2}\chi_A(s)\,ds\\
&\le32\int_r^\infty\frac1{|v-\hat V(t)|^3|s-s_0|^2}\,d(|v-\hat V(t)||s-s_0|)\\
&=32\,\frac{r^{-1}}{|v-\hat V(t)|}.\qedhere
\end{align*}
\end{pf}


\subsection{Divide and conquer}
\subsubsection{Ugly set estimate}

Therefore, if we let $r^{-1}\simeq\min\{|v|^3,|v-\hat V(t)|^3\}$, then
\[\int_{t-\Delta}^t\frac1{|y-\hat X(s)|^2}\chi_A(s)\,ds\les|v|^2\]
so that we have
\[\iiint_U\frac{f(s,y,w)}{|y-\hat X(s)|^2}\,dw\,dy\,ds\les R^{-1}\int|v|^2f(t,x,v)\,dv\,dx\les R^{-1}\]
when
\[U\subset\{\,(s,x,v):\ |v-\hat V(t)|\ge P,\quad|y-\hat X(s)|\ge R\max\{|v|^{-3},|v-\hat V(t)|^{-3}\}\,\}.\]

\subsubsection{Bad set estimate}
Consider $U^c$.
We need to control the union of two regions
\[|y-\hat X(s)|<R|v|^{-3}\quad\text{and}\quad|y-\hat X(s)|<R|v-\hat V(t)|^{-3}.\]
Without any conditions, the integration of fundamental solution with respect to $y$ gives
\[\int_{|y-\hat X(s)|<r}\frac1{|y-\hat X(s)|^2}\,dy\simeq r.\]
\begin{clm}
If $|v|\ge P$ and $|v-\hat V(t)|\ge P$, then
\[\int_{U^c}\frac1{|y-\hat X(s)|^2}\,dy\les\max\{|w|^{-3},|w-\hat V(s)|^{-3}\}\]
for $s\in[t-\Delta,t]$.
\end{clm}
\begin{pf}
It follows from
\[|w|\simeq|v|,\quad|w-\hat V(s)|\simeq|v-\hat V(t)|\]
for $|v|\ge P$ and $|v-\hat V(t)|\ge P$.
\end{pf}

\subsubsection{Good set estimate}

\subsection{Polynomial decay}
\begin{lem}
Along the time of existence we have
\[\|E(t)\|_{L_x^\infty}\les Q(t)^{4/3}.\]
\end{lem}
\begin{pf}
Note that we have
\[\|E\|_\infty\les\|\rho\|_\infty^{4/9}\|\rho\|_{5/3}^{5/9}.\]
Since the velocity support of $f$ is bounded by finite $Q(t)$,
\[\rho(t,x)=\int_{|v|<Q(t)}f(t,x,v)\,dv\les Q(t)^3\|f_0(x)\|_{L_v^\infty}\les Q(t)^3,\]
so
\[\|E(t)\|_{L_x^\infty}\les\|\rho(t)\|_{L_x^\infty}^{4/9}\les Q(t)^{4/3}.\qedhere\]
\end{pf}



\end{document}