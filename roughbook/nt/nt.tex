\documentclass[11pt]{article}
\usepackage{../../ikany}
\usepackage[margin=1.5in]{geometry}

\let\realsection\section
\renewcommand\section{\newpage\realsection}

\begin{document}
\tableofcontents



\section{Elliptic curves}
\subsection{Reduction of Weierstrass equations}
In this subsection, we want to investigate the important constants of elliptic curves such as $c_4$, $c_6$, $\Delta$, $j$ by calculating equations with hands.

\textbf{Step 1.}
The Riemann-Roch theorem proves that every curve of genus 1 with a specified base point can be described by the first kind of Weierstrass equation.
Explicitly, the first form of Weierstrass equation is
\[y^2+a_1xy+a_3y=x^3+a_2x^2+a_4x+a_6.\tag{1}\]

\textbf{Step 2.}
\emph{Elimination of $xy$ and $y$}.
Factorize the left hand side
\begin{align*}
y\,(y+a_1x+a_3)=x^3+a_2x^2+a_4x+a_6.
\end{align*}
By translation
\begin{align*}
\boxed{x\mapsto x,\qquad y\mapsto y-\tfrac12(a_1x+a_3)}
\end{align*}
we have
\begin{align*}
y^2-(\tfrac12(a_1x+a_3))^2&=x^3+a_2x^2+a_4x+a_6,\\
y^2&=x^3+(\tfrac14a_1^2+a_2)x^2+(\tfrac12a_1a_3+a_4)x+(\tfrac14a_3^2+a_6),\\
y^2&=x^3+\frac14(a_1^2+4a_2)x^2+\frac12(a_1a_2+2a_4)x+\frac14(a_3^2+4a_6).
\end{align*}
Introduce new coefficients $b$ to write it as
\begin{align*}
y^2=x^3+\frac14b_2x^2+\frac12b_4x+\frac14b_6.
\end{align*}
By scaling
\begin{align*}
\boxed{x\mapsto x,\qquad y\mapsto \tfrac12y}
\end{align*}
we get
\[y^2=4x^3+b_2x^2+2b_4x+b_6.\tag{2}\]

\textbf{Step 3.}
\emph{Elimination of $x^2$}.
By translation
\begin{align*}
\boxed{x\mapsto x-\tfrac1{12}b_2}
\end{align*}
we have
\begin{align*}
y^2=4\left(x^3-3\cdot\frac1{12}b_2x^2+3\cdot\frac1{12^2}b_2^2x-\frac1{12^3}b_2^3\right)&\\
+b_2\left(x^2-2\cdot\frac1{12}b_2x+\frac1{12^2}b_2^2\right)&\\
+2b_4\left(x-\frac1{12}b_2\right)&\\
+b_6,&
\end{align*}
so
\begin{align*}
y^2&=4x^3+\left(4\cdot3\cdot\frac1{12^2}b_2^2-2\cdot\frac1{12}b_2^2+2b_4\right)x+\left(-4\cdot\frac1{12^3}b_2^3+\frac1{12^2}b_2^3-2\cdot\frac1{12}b_2b_4+b_6\right)\\
&=4x^3+\frac1{12}\left(-b_2^2+24b_4\right)x+\frac1{216}\left(b_2^3-36b_2b_4+216b_6\right).
\end{align*}
Write it as
\begin{align*}
y^2=4x^3-\frac1{12}c_4x-\frac1{216}c_6.
\end{align*}
We want to match the coefficients of $y^2$ and $x^3$ but also want the coefficients of $c_4x$ and $c_6$ to be integers.
Iterative scaling implies
\begin{align*}
x\mapsto\tfrac16x:&\qquad 216y^2=4x^3-3c_4x-c_6\\
y\mapsto\tfrac1{36}y:&\qquad y^2=24x^3-18c_4x-6c_6\\
x\mapsto\tfrac16x:&\qquad 9y^2=x^3-27c_4x-54c_6\\
y\mapsto\tfrac13y:&\qquad y^2=x^3-27c_4x-54c_6.
\end{align*}
Thus, we get the famous third form of Weierstrass equation:
\[y^2=x^3-27c_4x-54c_6.\tag{3}\]



\begin{thm}
Let
\[E:y^2=x^3-Ax-B.\]
TFAE:
\begin{cond}
\item A point $(x,y)$ is a singular point of $E$.
\item $y=0$ and $x$ is a double root of $x^3-Ax-B$.
\item $\Delta=0$.
\end{cond}
\end{thm}
\begin{pf}
(1)$\Rightarrow$(2)
$\pd_yf=0$ implies $y=0$. $f=\pd_xf=0$ implies $x$ is a double root of $x^3-Ax-B$.
$A$ determines whether $x$ is either cusp of node.
\end{pf}












\section{Algebraic integer}
\subsection{Quadratic integer}
\begin{thm}
Every quadratic field is of the form $\Q(\sqrt d)$ for a square-free $d$.
\end{thm}
\begin{thm}
Let $d$ be a square-free.
\[\cO_{\Q(\sqrt d)}=\begin{cases}\Z[\sqrt d]&,d\equiv2,3\pmod4\\\Z[\tfrac{1+\sqrt d}2]&,d\equiv1\pmod4\end{cases}\]
\[\Delta_{\Q(\sqrt d)}=\begin{cases}4d&,d\equiv2,3\pmod4\\ d&,d\equiv1\pmod4\end{cases}\]
\end{thm}
\begin{ex}
\[\Delta_{\Q(i)}=-4,\quad\Delta_{\Q(\sqrt2)}=8,\quad\Delta_{\Q(\gamma)}=5,\quad\Delta_{\Q(\omega)}=-3\]
where $\gamma:=\frac{1+\sqrt5}2$ and $\omega=\zeta_3$.
\end{ex}
\begin{thm}
Let $\theta^3=hk^2$ for $h,k$ square-free's.
\[\cO_{\Q(\theta)}=\begin{cases}\bigmath\Z+\theta\Z+\frac{\theta^2}k\Z&,m\not\equiv\pm1\pmod9\\
\bigmath\Z+\theta\Z+\frac{\theta^2\pm\theta k+k^2}{3k}\Z&,m\equiv\pm1\pmod9\end{cases}\]
\end{thm}
\begin{cor}
If $\theta^3$ is a square free integer, then
\[\cO_{\Q(\theta)}=\Z[\theta].\]
\end{cor}

\subsection{Integral basis}
\begin{thm}
Let $\alpha\in K$.
$Tr_K(\alpha)\in\Z$ if $\alpha\in\cO_K$.
$N_K(\alpha)\in\Z$ if and only if $\alpha\in\cO_K$.
\end{thm}
\begin{thm}
Let $\{\omega_1,\cdots,\omega_n\}$ be a basis of $K$ over $\Q$.
If $\Delta(\omega_1,\cdots,\omega_n)$ is square-free, then $\{\omega_1,\cdots,\omega_n\}$ is an integral basis.
\end{thm}
\begin{thm}
Let $\{\omega_1,\cdots,\omega_n\}$ be a basis of $K$ over $\Q$ consisting of algebraic integers.
If $p^2\mid\Delta$ and it is not an integral basis, then there is a nonzero algebraic integer of the form
\[\frac1p\sum_{i=1}^n\lambda_i\omega_i.\]
\end{thm}

\subsection{Fractional ideals}
\begin{thm}
Every fractional ideal of $K$ is a free $\Z$-module with rank $[K:\Q]$.
\end{thm}
\begin{pf}
This theorem holds because $K/\Q$ is separable and $\Z$ is a PID.

\end{pf}


\clearpage
\subsection{Frobenius element}
\begin{defn}
Let $L/K$ be abelian.
Let $\fp$ be a prime in $\cO_K$ and $\fq$ be a prime in $\cO_L$ over $\fp$.
The \emph{decomposition group} $D_{\fq|\fp}$ is a subgroup of $\Gal(L/K)$ whose element fixes the prime $\fq$.
Since $L/K$ is Galois, the followings do not depend on the choice of $\fq$ over $\fp$.

By definition, $D_{\fq|\fp}$ acts on the set $\cO_L/\fq$ and fixes $\cO_K$.
\end{defn}
\begin{lem}
The following sequence of abelian groups is exact:
\begin{es} 0 \> I_{\fq|\fp} \> D_{\fq|\fp} \> \Gal(k(\fq)/k(\fp)) \> 0, \end{es}
where $k(\fq):=\cO_L/\fq$ and $k(\fp):=\cO_K/\fp$ are residue fields.
\end{lem}
\begin{pf}
We first show 
\end{pf}
The Frobenius element is defined as an element of $D_{\fq|\fp}/I_{\fq|\fp}\cong\Gal(k(\fq)/k(\fp))$, which is a cyclic group.
\begin{defn}
The Frobenius element is defined by $\phi_{\fq|\fp}\in\Gal(L/K)$ such that $\phi_{\fq|\fp}(\fq)=\fq$ and
\[\phi_{\fq|\fp}(x)\equiv x^{|\cO_K/\fp|}\pmod{\fq}\quad\text{for}\ x\in\cO_L.\]
It gives a generator of the cyclic group $D_{\fq|\fp}/I_{\fq|\fp}$.
\end{defn}
\begin{rmk}
Fermat's little theorem states $\phi_{\fq|\fp}=\id_{\cO_K/\fp}$, i.e.
\[\phi_{\fP|\fp}(x)\equiv x\pmod{\fp}\quad\text{for}\ x\in\cO_K,\]
which means $\phi_{\fP|\fp}$ fixes the field $\cO_K/\fp$ so that $\phi_{\fP|\fp}\in\Gal(k(\fP)/k(\fp))$.
\end{rmk}


\subsection{Quadratic Dirichlet character}
Let $D$ be a quadratic discriminant.
For $\zeta_D=e^{\frac{2\pi i}D}$, it is known that the cyclotomic field $\Q(\zeta_D)$ is the smallest cyclotomic extension of the quadratic field $\Q(\sqrt D)$.
Let $K=\Q(\sqrt{D})$ and $L=\Q(\zeta_D)$.
\begin{cd}[column sep = 0pt]
\sigma_p \mp{d}{\cdot|_K} &\in& D_{\fq|p} &\le& \Gal(L/\Q) \ep{d}{\cdot|_K} &\cong& (\Z/|D|\Z)^\x \ep{d}{\chi_K=\left(\frac D\cdot\right)} & \\
\sigma_p|_K&\in& D_{\fp|p} &\le& \Gal(K/\Q) &\cong& \<\pm1\>.
\end{cd}

For $p\nmid D$ so that $p$ is unramified, let $\sigma_p:=(\zeta_D\mapsto\zeta_D^p)\in\Gal(L/\Q)$.
Then, what is $\sigma_p|_K$ in $\Gal(K/\Q)$?
In other words, which is true: $\sigma_p(\sqrt D)=\pm\sqrt D$?

Notice that $\sigma$ satisfies the condition to be the Frobenius element: $\sigma_pI_{\fq|p}=\phi_{\fq|p}$.
Therefore, $\phi_{\fp|p}=\sigma_p|_K$ is also a Frobenius element.
There are only two cases:
\begin{cond}
\item If $f=|D(\fp/p)|=1$, then $\sigma_p|_K$ is the identity, so $\chi_K(p)=1$
\item If $f=|D(\fp/p)|=2$, then $\sigma_p|_K$ is not trivial, so $\chi_K(p)=-1$
\end{cond}

Artin reciprocity: $(\Z/D\Z)^\x$ is extended to $I_K^S$.







\section{Diophantine equations}

\subsection{Quadratic equation on a plane}
Ellipse is reduced by finitely many computations.


Especially for hyperbola, here is a strategy to use infinite descent.
\begin{cond}
\item Let midpoint to be origin.
\item Find the subgroup of $\SL_2(\Z)$ preserving the image of hyperbola(which would be isomorphic to $\Z$).
\item Find an impossible region.
\item Assume a solution and reduce it to the either impossible region or the ground solution.
\end{cond}
\begin{ex}[Pell's equation]
Consider\[x^2-2y^2=1.\]
A generator of hyperbola generating group is $g=\begin{pmatrix}3&4\\2&3\end{pmatrix}$.
It has a ground solution $(1,0)$ and impossible region $1<x<3$.
If $(a,b)$ is a solution with $a>0$, then we can find $n$ such that $g^n(a,b)$ is in the region $[1,3)$.
The possible case is $g^n(a,b)=(1,0)$.
\end{ex}
\begin{ex}[IMO 1988, the last problem]
Consider a family of equations\[x^2+y^2-kxy-k=0.\]
By the vieta jumping, a generator is $g:(a,b)\mapsto(b,kb-a)$.
It has an impossible region $xy<0$ : $x^2+y^2-kxy-k\ge x^2+y^2>0$.
If $(a,b)$ is a solution with $a>b$, then we can find $n$ such that $g^n(a,b)$ is in the region $xy\le0$.
Only possible case is $g^n(a,b)=(\sqrt k,0)$ or $g^n(a,b)=(0,-\sqrt k)$.
In ohter words, the equation has a solution iff $k$ is a perfect square.
\end{ex}


\clearpage


\subsection{The Mordell equations}
(The reciprocity laws let us learn not only which prime splits, but also which prime factors a given polynomial has.)
\[y^2=x^3+k\]
There are two strategies for the Mordell equations:
\begin{itemize}
\item $x^2-2x+4$ has a prime factor of the form $4k+3$
\item $x^3=N(y-a)$ for some $a$.
\end{itemize}
First case: $k=7,-5,-6,45,6,46,-24,-3,-9,-12$.
\begin{ex} Solve $y^2=x^3+7$. \end{ex}
\begin{pf}
Taking mod 8, $x$ is odd and $y$ is even.
Consider
\[y^2+1=(x+2)(x^2-2x+4).\]
Since
\[x^2-2x+4=(x-1)^2+3,\]
there is a prime $p\equiv3\pmod4$ that divides the right hand side.
Taking mod $p$, we have
\[y^2\equiv-1\pmod p,\]
which is impossible.
Therefore, the equation has no solutions.
\end{pf}
\begin{ex} Solve $y^2=x^3-2$. \end{ex}
\begin{pf}
Taking mod 8, $x$ and $y$ are odd.
Consider a ring of algebraic integers $\Z[\sqrt{-2}]$.
We have
\[N(y-\sqrt{-2})=(y-\sqrt{-2})(y+\sqrt{-2})=x^3.\]
For a common divisor $\delta$ of $y\pm\sqrt{-2}$, we have 
\[N(\delta)\mid N((y-\sqrt{-2})-(y+\sqrt{-2}))=N(2\sqrt{-2})=|(2\sqrt{-2})(-2\sqrt{-2})|=8.\]
On the other hand,
\[N(\delta)\mid x^3 \equiv1\pmod2,\]
so $N(\delta)=1$ and $\delta$ is a unit.
Thus, $y\pm\sqrt{-2}$ are relatively prime.
Since the units in $\Z[\sqrt{-2}]$ are $\pm1$, which are cubes, $y\pm\sqrt{-2}$ are cubics in $\Z[\sqrt{-2}]$.

Let
\[y+\sqrt{-2}=(a+b\sqrt{-2})^3=a(a^2-6b^2)+b(3a^2-2b^2)\sqrt{-2},\]
so that $1=b(3a^2-2b^2)$.
We can conclude $b=\pm1$.
The possible solutions are $(x,y)=(3,\pm5)$, which are in fact solutions.
\end{pf}



\section{The local-global principle}
\subsection{The local fields}
Let $f\in\Z[x]$.
\[\text{\emph{Does $f=0$ have a solution in $\Z$?}}\]
\[\text{\emph{Does $f=0$ have a solution in $\Z/(p^n)$ for all $n$?}}\]
\[\text{\emph{Does $f=0$ have a solution in $\Z_p$?}}\]
In the first place, here is the algebraic definition.
\begin{defn}
Let $p\in\Z$ be a prime.
The ring of the $p$-adic integers $\Z_p$ is defined by the inverse limit:
\begin{es}
\Z_p:={\bigmath\lim_{\substack{\longleftarrow\\n\in\N}}}\F_{p^n}  \>  \cdots  \>  \Z/(p^3)  \>  \Z/(p^2)  \>  \F_p.
\end{es}
\end{defn}
\begin{defn}
$\Q_p=\Frac\Z_p$.
\end{defn}
Secondly, here is the analytic definition.
\begin{defn}
Let $p\in\Z$ be a prime.
Define a absolute value $|\cdot|_p$ on $\Q$ by $|p^ma|_p=\frac1{p^m}$.
The local field $\Q_p$ is defined by the completion of $\Q$ with respect to $|\cdot|_p$.
\end{defn}
\begin{defn}
$\Z_p:=\{x\in\Q_p:|x|_p\le1\}$.
\end{defn}

\begin{ex}
Observe
\begin{align*}
3^{-1}&\equiv2_5\pmod5\\
&\equiv32_5\pmod{5^2}\\
&\equiv132_5\pmod{5^3}\\
&\equiv1313132_5\pmod5^7\cdots.
\end{align*}
Therefore, we can write
\[3^{-1}=\overline{13}2_5=2+3p+p^2+3p^3+p^4+\cdots\]
for $p=5$.
Since there is no negative power of 5, $3^{-1}$ is a $p$-adic integer for $p=5$.
\end{ex}
\begin{ex}

\begin{align*}
7&\equiv1_3^2\pmod3\\
&\equiv111_3^2\pmod{3^3}\\
&\equiv20111_3^2\pmod{3^5}\\
&\equiv120020111_3^2\pmod{3^9}\cdots.
\end{align*}
Therefore, we can write
\[\sqrt7=\cdots120020111_3=1+p+p^2+2p^4+2p^7+p^8+\cdots\]
for $p=3$.
Since there is no negative power of 3, $\sqrt7$ is a $p$-adic integer for $p=3$.
\end{ex}


There are some pathological and interesting phenomena in local fields.
Actually note that the values of $|\cdot|_p$ are totally disconnected.
\begin{thm}
The absolute value $|\cdot|_p$ is nonarchimedean: it satisfies $|x+y|_p\le\max\{|x|_p,|y|_p\}$.
\end{thm}
\begin{pf}
Trivial.
\end{pf}

\begin{thm}
Every triangle in $\Q_p$ is isosceles.
\end{thm}

\begin{thm}
$\Z_p$ is a discrete valuation ring: it is local PID.
\end{thm}
\begin{pf}
asdf
\end{pf}

\begin{thm}
$\Z_p$ is open and compact.
Hence $\Q_p$ is locally compact Hausdorff.
\end{thm}
\begin{pf}
$\Z_p$ is open clearly.
Let us show limit point compactness.
Let $A\subset\Z_p$ be infinite.
Since $\Z_p$ is a finite union of cosets $p\Z_p$, there is $\alpha_0$ such that $A\cap(\alpha_0+p\Z_p)$ is infinite.
Inductively, since
\[\alpha_n+p^{n+1}\Z_p=\bigcup_{1\le x<p}(\alpha_n+xp^{n+1}+p^{n+2}\Z_p),\]
we can choose $\alpha_{n+1}$ such that $\alpha_n\equiv\alpha_{n+1}\pmod{p^{n+1}}$ and $A\cap(\alpha_{n+1}+p^{n+2}\Z_p)$ is infinite.
The sequence $\{\alpha_n\}$ is Cauchy, and the limit is clearly in $\Z_p$.
\end{pf}



\subsection{Hensel's lemma}

\begin{thm}[Hensel's lemma]
Let $f\in\Z_p[x]$.
If $f$ has a simple solution in $\F_p$, then $f$ has a solution in $\Z_p$.
\end{thm}
\begin{pf}
asdf
\end{pf}

\begin{rmk}
Hensel's lemma says: for $X$ a scheme over $\Z_p$, $X$ is smooth iff $X(\Z_p)\twoheadrightarrow X(\F_p)$....???
\end{rmk}

\begin{ex}
$f(x)=x^p-x$ is factorized linearly in $\Z_p[x]$.
\end{ex}

\subsection{Sums of two squares}


\begin{thm}[Euler]
A positive integer $m$ can be written as a sum of two squares if and only if $v_p(m)$ is even for all primes $p\equiv3\pmod4$.
\end{thm}
\begin{lem}
Let $p$ be a prime with $p\equiv1\pmod4$.
Every $p$-adic integer is a sum of two squares of $p$-adic integers.
\end{lem}







\section{Dedekind domain}

\begin{rd}
Notherian: & ED \ar{r} & PID \ar{d}\ar{r} & UFD \ar{d} & \\
Non-noetherian: & & B\'ezout \ar{r} & GCD \ar{r} & Euclid's lemma.
\end{rd}

\begin{prop}
Let $A$ be a Dedekind domain.
Then, $A$ is a PID if and only if Euclid's lemma holds.
\end{prop}

If $R$ satisfies the \emph{ascending chain condition for principal ideals}, then $R$ is a PID iff $R$ is a B\'ezout domain, and $R$ is a UFD iff Euclid's lemma holds in $R$.

Every valuation ring is a B\'ezout domain.

\end{document}


