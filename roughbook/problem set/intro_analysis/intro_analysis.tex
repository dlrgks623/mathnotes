\documentclass{article}

\usepackage{amsmath,amssymb,amsfonts}
\usepackage[margin=1.5in]{geometry}

\def\R{\mathbb{R}}
\def\cF{\mathcal{F}}
\def\cB{\mathcal{B}}
\def\e{\varepsilon}

% 
%
%
\begin{document}

\subsection*{Compact sets}
\begin{enumerate}
\item Let $X\subset\R^d$. Show that if $X$ is bounded then every sequence in $X$ has a convergent subsequence. (Bolzano-Weierstrass)
\item Let $X\subset\R^d$. Show that if every sequence in $X$ has a convergent subsequence, then $X$ is closed and bounded.
\item Let $X\subset\R^d$ be compact. Suppose an infinite set $\mathcal{C}\subset\mathcal{P}(X)$ only contains closed subsets of $X$. Show that if $\bigcap_{C\in A}C$ is nonempty for all finite subset $A\subset\mathcal{C}$, then $\bigcap_{C\in\mathcal{C}}C$ is nonempty.
\end{enumerate}

\subsection*{Continuous functions}
\begin{enumerate}
\item Let $X$ be a set. Let $f_n:X\to\R$ be a sequence of functions. Show that $f_n$ converges to $f:X\to\R$ uniformly if and only if $\lim_{n\to\infty}\sup_{x\in X}|f_n(x)-f(x)|=0$.
\item Let $X\subset\R^d$. Let $f_n:X\to\R$ be a sequence of continuous functions. Show that if $f_n$ converges to $f:X\to\R$ uniformly, then $f$ is also continuous. (In other words, the set of real-valued continuous functions $C(X)$ is always closed under the topology of uniform convergence.)
\item Let $X\subset\R^d$ be compact. Show that is $f:X\to\R$ is continuous then it is uniformly continuous.
\item Let $f_n:[a,b]\to\R$ be a sequence of continuous functions. Show that if $f_n\to f$ pointwisely and $f'_n\to g$ uniformly, then $g=f'$.
\end{enumerate}

\clearpage

\subsection*{Measures}
Let $X$ be a set and $\cF$ be a $\sigma$-algebra on $X$.
A \emph{measure} on $\cF$ is a function $\mu:\cF\to[0,\infty]$ such that
\begin{itemize}
\item $\mu(\varnothing)=0$,
\item $\mu(\bigcup_{i=1}^\infty E_i)=\sum_{i=1}^\infty\mu(E_i)$ for a sequence of disjoint sets $E_i\in\cF$. (countable-additivity)
\end{itemize}
We call an element in $\cF$ \emph{measurable} (when we are known $\cF$).
\begin{enumerate}
\item Show that if $E_i$ is a monotonically increasing sequence of measurable subsets, then $\mu(\bigcup_{i=1}^\infty E_i)=\lim_{i\to\infty}\mu(E_i)$. (Continuity from below)
\item Show that if $E_i$ is a monotonically decreasing sequence of measurable subsets, then $\mu(\bigcap_{i=1}^\infty E_i)=\lim_{i\to\infty}\mu(E_i)$ when given $\mu(E_1)<\infty$. (Continuity from above)
\item Show that there is no measure $\mu$ defined on the entire power set $\mathcal{P}(\R)$ such that $\mu([a,b])=b-a$ and $\mu(x+E)=\mu(E)$ for $x\in\R,\,E\subset\R$. (Hint: Define an equivalence relation on $\R$ such that $x\sim y$ iff $x-y\in\mathbb{Q}$. Take $N\subset[0,1)$ such that $N$ contains precisely one member of each equivalence class. Show $1\le\sum_{r\in\mathbb{Q}\cap[0,1)}\mu(N)\le3$ to lead a contradiction.)
\end{enumerate}

\subsection*{Measurable functions}
Let $X$ be a set.
A $\sigma$-algebra $\cF$ on $X$ is also called a \emph{measurable structure} and $X$ with $\cF$ is called a \emph{measurable space}.
A function $f:X\to Y$ between measurable spaces is called \emph{measurable} if the measurability of $E\subset Y$ implies the measurability of $f^{-1}(E)$.

On $\R$, the smallest $\sigma$-algebra containing open sets is called \emph{Borel $\sigma$-algebra} and its elements are called \emph{Borel sets}.
We will denote it by $\cB(\R)$.
For a function $f:X\to\R$ where $X$ is a measurable space, we call $f$ just measurable if $f^{-1}(E)$ is measurable for all Borel sets $E$.

\begin{enumerate}
\item Let $X$ be a measurable space. Show that if $f,g:X\to\R$ is measurable, then $f+g$, $|f|$, $f^2$, and $fg$ are all measurable.
\item Let $X$ be a measurable space and $f_n$ be a sequence of bounded measurable functions. Show that $g=\sup_nf_n$ and $h=\limsup_nf_n$ are measurable.
\end{enumerate}


\clearpage

Week 3

\subsection*{Simple functions}
Let $(X,\cF)$ be a measurable space.
A \emph{characteristic function} or \emph{indicator function} of a measurable set $E$ is a function $\chi_E:X\to\R$ defined by
\[\chi_E(x)=\begin{cases}1&,x\in E\\0&,x\notin E\end{cases}.\]
A finite sum of characteristic functions is called \emph{simple function}, and it is a slight generalization of step functions used in the Riemann integral.

\begin{enumerate}
\item Show that a subset $E$ is measurable iff its characteristic function $\chi_E$ is measurable.
\item Let $f:X\to[0,\infty]$ be a measurable function. Construct a monotonically increasing sequence of simple functions $\phi_n$ such that $\phi_n\to f$ pointwise. (Hint: $E_n^k=f^{-1}((k^{-1}2^{-n},(k+1)2^{-n}])$, $F_n=f^{-1}((2^n,\infty])$.)
\item Show that $\phi_n\to f$ uniformly if $f$ is bounded.
\end{enumerate}

\subsection*{Almost everywhere convergence}
Let $f_n:X\to\R$ be a sequence of measurable functions.
We say $f_n$ converges to $f$ $\mu$-\emph{almost everwhere} if $E=\{x:f_n(x)\text{ does not converges to }f(x)\}$ satisfies $\mu(E)=0$.

\begin{enumerate}
\item Let $f_n:X\to\R$ be a sequence of measurable functions. Let $\mu$ be a complete measure(in other words, $\mu(E)=0$ implies all subsets of $E$ are measurable). Show that if $f_n\to f$ $\mu$-a.e., then $f$ is measurable.
\item (Optional!) Prove the Egorov's theorem: Let $X$ be a probability space, and let $f_n:X\to\R$ be a sequence of measurable functions. If $f_n\to f$ a.e., then for every $\e>0$ there is a measurable subset $E\subset X$ such that $\mu(E)>1-\e$ and $f_n\to f$ uniformly on $E$.
\end{enumerate}





\end{document}