\documentclass[11pt]{article}
\usepackage{../ikany}
\usepackage[margin=1.5in]{geometry}

\let\realsection\section
\renewcommand\section{\newpage\realsection}

\def\tint{{\textstyle\int}}

\begin{document}
\tableofcontents

\section{Sturm-Liouville theory}
\subsection{Self-adjointness}
Let $I=[a,b]$ and
\begin{gather*}
L=-\frac1{w(x)}\left[\dd{x}\left(p(x)\dd{x}\right)+q(x)\right],\\
0\le p(x)\in C^\infty(I),\quad q(x)\in C^\infty(I),\quad 0<w(x)\in C^\infty(I).
\end{gather*}
We expect $L$ to be self-adjoint.
In this regard, our interest is ellimination of the difference term
\[\<f,Lg\>-\<Lf,g\>=p(f'g-fg')|_a^b.\]


\begin{center}
\renewcommand{\arraystretch}{2.5}
\begin{tabular}{l|l|c|l}
\hline
Name & Operator & Domain & B.C. \\[5pt]
\hline
Helmholtz & $\displaystyle L=-\dd[2]{x}$ & $[a,b]$ & Periodic \\
Helmholtz & $\displaystyle L=-\dd[2]{x}$ & $[a,b]$ & Separated Robin \\[5pt]
\hline
Legendre & $\displaystyle L=-\dd{x}\left((1-x^2)\dd{x}\right)$ & $[-1,1]$ & None \\
A. Legendre & $\displaystyle L=-\left[\dd{x}\left((1-x^2)\dd{x}\right)-\frac{m^2}{1-x^2}\right]$ & $[-1,1]$ & Dirichlet \\
Hermite & $\displaystyle L=-e^{x^2}\left[\dd{x}\left(e^{-x^2}\dd{x}\right)\right]$ & $(-\infty,\infty)$ & Polynomial growth\\
Laguerre
\\[5pt]
\hline
\end{tabular}
\end{center}


\subsection{Regular Sturm-Liouville problem}
We mean regular Sturm-Liouville problems by the case that $p$ does not vanish on the boundary of $I$.
View the Sturm-Liouville operator $L$ as a non-densely defined operator on the space $C^\infty(I)$ with inner product $\<f,g\>=\int_Ifgw$ with domain
\[D(L)=\{\,u\in C^\infty(I):\alpha_1u(a)+\alpha_2u'(a)=0,\ \beta_1u(b)+\beta_2u'(b)=0\,\},\]
the subspace for the \emph{separated} Robin boundary condition.
\begin{prop}
The operator $L:D(L)\to C^\infty(I)$ is self-adjoint when $C^\infty(I)$ has the inner product $\<f,g\>=\int_Ifgw$.
\end{prop}
We are interested in the eigenvalue problem of $L:D(L)\to C^\infty(I)$ on $V$.
Fortunately, if we choose a constant $z\in\C\setminus\R$, then $(L-z)^{-1}:C^\infty(I)\to D(L)$ is well-defined.
\begin{prop}
If $z$ is not an eigenvalue of $L$, then $L-z: D(L)\to C^\infty(I)$ is bijective.
\end{prop}
\begin{pf}
The injectivity follows from the definition of eigenvalues.
We may assume that $L$ is injective by translation $q\mapsto q-\lambda$.

Suppose $f\in C^\infty(I)$.
The surjectivity is equivalent to the existence of a second order inhomogeneous boundary problem:
\begin{gather*}
pu''+p'u'+qu=-wf,\\
a_0u(0)+b_0u'(0)=0,\quad a_1u(1)+b_1u'(1)=0.
\end{gather*}
Let $u_0$, $u_1$ be the unique solutions of the corresponding homogeneous equation with initial conditions
\[a_0u(0)+b_0u'(0)=0,\qquad a_1u(1)+b_1u'(1)=0.\]
Then we can define $L^{-1}:C^\infty([0,1])\to D(L)$ by
\[L^{-1}f(x):=-u_0(x)\int_x^1\frac{u_1}{W[u_0,u_1]}\frac fpw-u_1(x)\int_0^x\frac{u_0}{W[u_0,u_1]}\frac fpw,\]
where $W[u_0,u_1]:=u_0u_1'-u_1u_0'$ denotes the Wronskian.
This formula is derived from variation of parameters: compute $c_0$ and $c_1$ from the fact that
\[\begin{pmatrix}0\\-\frac fpw\end{pmatrix}=\begin{pmatrix}u_0&u_1\\u_0'&u_1'\end{pmatrix}\begin{pmatrix}c_0'\\c_1'\end{pmatrix}\impl L(c_0u_0+c_1u_1)=f.\]
Then, we can easily check that
\[L^{-1}Lu=u\]
for $u\in D(L)$, which implies $L$ is surjective.
\end{pf}


\subsection{Legendre's equation}
The Legendre equation is
\[(1-x^2)u''-2xu'+l(l+1)u=0,\quad\text{ on }[-1,1].\]
The Sturm-Liouville operator is
\[L=-\dd{x}\left((1-x^2)\dd{x}\right).\]
Since $p(\pm1)=0$, the operator $L:C^\infty([-1,1])\to C^\infty([-1,1])$ is self-adjoint on the whole domain.


Its eigenvalues and corresponding eigenspaces are
\begin{center}\renewcommand{\arraystretch}{1.2}
\begin{tabular}{c|c|l}
\hline
    & Eigenvalue & Eigenbasis \\
$l$ & $l(l+1)$   & \\
\hline
0   & 0          & $P_0(x)=1$ \\
1   & 2          & $P_1(x)=x$ \\
2   & 6          & $P_2(x)=\frac32x^2-\frac12$ \\
3   & 12         & $P_3(x)=\frac52x^3-\frac32x$ \\
4   & 20         & $P_4(x)=\frac{35}8x^4-\frac{15}4x^2+\frac38$\\
\hline
\end{tabular}
\end{center}
If we admit
\[Q_0(x)=\frac12\log\frac{1+x}{1-x},\quad Q_1(x)=1-\frac12x\log\frac{1+x}{1-x},\ \cdots\ \in L^2(-1,1)\setminus C^\infty([-1,1])\]
as eigenvectors of $L$, then the self-adjointness fails on the extended domain.
For example,
\begin{align*}
\<Q_0,Lf\>-\<LQ_0,f\>
&=\left.p(x)\bigl(Q_0'(x)f(x)-Q_0(x)f'(x)\bigr)\right|_{-1}^1\\
&=f(1)-f(-1)
\end{align*}
does not vanish in general even for $f\in C^\infty([-1,1])$.

\subsection{Bessel's equation}
The Bessel equation is
\[x^2u''+xu'+(k^2x^2-\nu^2)u=0,\quad\text{ on }(0,\infty).\]
The Sturm-Liouville operator is
\[-\frac1x\left[\dd{x}\left(x\dd{x}\right)-\nu^2\frac1x\right].\]









\section{Kinetic theory}

\subsection{Vlasov-Poisson equation}
Consider a Cauchy problem of the \emph{Valsov-Poisson system}:
\begin{pde*}
&f_t+v\cdot\del_xf+\gamma E\cdot\del_vf=0,\:(t,x,v)\in\R_t^+\x\R_x^3\x\R_v^3,\\
&E(t,x)=\del_x\Delta_x^{-1}\rho,\\
&\rho(t,x)=\tint f\,dv,\\
&f(0,x,v)=f_0(x,v).
\end{pde*}
For boundaryless problem, we have
\[E=\del_x(-\tfrac1{4\pi|x|}*\rho)=\frac{x}{4\pi|x|^3}*\rho.\]

\begin{thm}[Schaeffer, 1991]
Let $f_0\in C_{0,x,v}^1$ and $f_0\ge0$.
Then, the Cauchy problem for the VP system has a unique $C^1$ global solution.
\end{thm}
\begin{defn}
For a local solution $f$,
\[Q(t):=1+\sup\{|v|:f(s,x,v)\ne0\text{ for some }s\in[0,t],\,x\in\R_x^3\}.\]
\end{defn}

Electric field can be controlled by density $\rho$, henceforth by $Q$.
\begin{lem}
For $1\le q<\frac N{N-2}=3<p\le\infty$,
\[\|E(t,x)\|_{L_x^\infty}\les\|\rho(t,x)\|_{L_x^p}^{\frac{\frac2N-1+\frac1q}{\frac1q-\frac1p}}\|\rho(t,x)\|_{L_x^q}^{\frac{1-\frac1p-\frac2N}{\frac1q-\frac1p}}.\]
Furthermore,
\[\|E(t,x)\|_{L_x^\infty}\les Q(t)^\e\]
for any $0<\e\le6$.
\end{lem}
\begin{pf}
Fix time $t$.
For $2p<N=3<2q$,
\begin{align*}
4\pi|E(x)|
&=|\tfrac1{|x|^2}*\rho(x)|\\
&\le\int_{|x-y|<R}\frac{\rho(y)}{|x-y|^2}\,dy+\int_{|x-y|\ge R}\frac{\rho(y)}{|x-y|^2}\,dy\\
&\le\|\rho\|_{p'}(\int_{|y|<R}\frac{dy}{|y|^{2p}})^{1/p}+\|\rho\|_{q'}(\int_{|y|\ge R}\frac{dy}{|y|^{2q}})^{1/q}\\
&\simeq\|\rho\|_{p'}(\int_0^Rr^{N-1-2p},dr)^{1/p}+\|\rho\|_{q'}(\int_R^\infty r^{N-1-2q},dr)^{1/q}\\
&\simeq\|\rho\|_{p'}R^{\frac Np-2}+\|\rho\|_{q'}R^{\frac Nq-2}.
\end{align*}
By choosing $R$ such that $\|\rho\|_{p'}R^{\frac Np-2}=\|\rho\|_{q'}R^{\frac Nq-2}$, we get
\[\|E(t,x)\|_{L_x^\infty}\les\|\rho(t,x)\|_{L_x^{p'}}^{\frac{\frac2N-\frac1q}{\frac1p-\frac1q}}\|\rho(t,x)\|_{L_x^{q'}}^{\frac{\frac1p-\frac2N}{\frac1p-\frac1q}}.\]

If $p=\infty$ and $N=3$, then $1\le q<3$ implies
\[\|E\|_\infty^3\les\|\rho\|_\infty^{3-q}\|\rho\|_q^q.\]
For example, if we let $(p,q)=(\infty,1)$ or $(\infty,5/3)$, then
\[\|E\|_\infty\les\|\rho\|_\infty^{2/3}\|\rho\|_1^{1/3},\qquad
\|E\|_\infty\les\|\rho\|_\infty^{4/9}\|\rho\|_{5/3}^{5/9}.\]
To control $E$ with $Q(t)$,
\[\rho(t,x)=\int_{|v|<Q(t)}f(t,x,v)\,dv\les Q(t)^3\|f_0(x,v)\|_{L_v^\infty},\]
so by letting $\e=3(3-q)$,
\[\|E(t,x)\|_{L_x^\infty}\les Q(t)^\e\]
for any $0<\e\le6$.
($L^p$-norm of $\rho$ for $p<\infty$ well propagates by the measure preserving property.)
\end{pf}


For a characteristic curve $s\mapsto(X(s;t,x,v),V(s;t,x,v))$, we have
\[f(t,x,v)=f(s,X(s;t,x,v),V(s;t,x,v))\]
and
\begin{align*}
&\iint f(s,x,v)\f(x,v)\,dv\,dx\\
&\qquad=\iint f(s,X(s;t,x,v),V(s;t,x,v))\f(X(s;t,x,v),V(s;t,x,v))\,dv\,dx\\
&\qquad=\iint f(t,x,v)\f(X(s;t,x,v),V(s;t,x,v))\,dv\,dx.
\end{align*}

\subsubsection{Schaeffer's global existence proof}

Fix a characteristic $(X^*(s),V^*(s))=(X(s;t^*,x^*,v^*),V(s;t^*,x^*,v^*))$.
For $t\in[0,t^*]$,
\begin{align*}
\int_t^{t^*}|E(s,X^*(s))|\,ds
&\les\int_t^{t^*}\iint\frac{f(s,x,v)}{|x-X^*(s)|^2}\,dv\,dx\,ds\\
&=\int_t^{t^*}\iint\frac{f(t,x,v)}{|X(s;t,x,v)-X^*(s)|^2}\,dv\,dx\,ds.
\end{align*}

\begin{align*}
|v-V(t;t^*,x,v)|
&=|V(t^*;t^*,x,v)-V(t;t^*,x,v)|\\
&\le\int_t^{t^*}|E(s,X(s,x,v))|\,ds
\les(t^*-t)Q^{4/3}.
\end{align*}




\subsection{Velocity averaging lemmas}
The velocity averaging lemma is used to get regularity of averaged quantity when boundary condition is not given.
\begin{thm}[Velocity averaging]
Let $L$ be a free transport operator $\pd_t+v\cdot\del_x$ on $\R_t\x\R_x^n\x\R_v^n$.
Then,
\[\|\int u\f\,dv\|_{H_{t,x}^{1/2}}\les_\f\|u\|_{L_{t,x,v}^2}^{1/2}\|Lu\|_{L_{t,x,v}^2}^{1/2}\]
for $\f\in C_c^\infty(\R_v^n)$,
\end{thm}
\begin{pf}
Let $m(t,x)=\int u\f\,dv$.
By Fourier transform with respect to $t$ and $x$, we have
\[\hat u(\tau,\xi,v)=\frac1i\,\frac{\hat{Lu}(\tau,\xi,v)}{\tau+v\cdot\xi}\]
and
\[\hat m(\tau,\xi)=\int\hat u(\tau,\xi,v)\f(v)\,dv.\]
Fixing $\tau,\xi$, decompose the integral and use H\"older's inequality to get
\begin{align*}
|\hat m(\tau,\xi)|
&\le\int_{|\tau+v\cdot\xi|<\alpha}|\hat u\f|\,dv+\int_{|\tau+v\cdot\xi|\ge\alpha}\frac{|\hat{Lu}\f|}{|\tau+v\cdot\xi|}\,dv\\
&\le\|\hat u\|_{L_v^2}^{1/2}\ (\int_{|\tau+v\cdot\xi|<\alpha}|\f|^2\,dv)^{1/2}+\|\hat{Lu}\|_{L_v^2}^{1/2}\ (\int_{|\tau+v\cdot\xi|\ge\alpha}\frac{|\f|^2}{|\tau+v\cdot\xi|^2}\,dv)^{1/2},
\end{align*}
where $\alpha>0$ is an arbitrary constant that will be determined later.
Let
\[I_s(\tau,\xi,\alpha):=\int_{|\tau+v\cdot\xi|<\alpha}|\f|^2\,dv,\qquad
I_n(\tau,\xi,\alpha):=\int_{|\tau+v\cdot\xi|\ge\alpha}\frac{|\f|^2}{|\tau+v\cdot\xi|}\,dv.\]
We are going to estimate the integrals as
\[I_s\les\frac{\alpha}{\sqrt{\tau^2+|\xi|^2}},\qquad
I_n\les\frac1{\alpha\sqrt{\tau^2+|\xi|^2}}.\]

Define coordinates $(v_1,v_2)$ on $\R_v$ as follows:
\[v_1:=\frac{\tau+v\cdot\xi}{|\xi|}\ \in\R\ ,\qquad v_2:=v-\frac{v\cdot\xi}{|\xi|^2}\xi\ \in\ker(\xi^T)\cong\R^{n-1}.\]
Note that
\[|v|^2=(v_1-\frac\tau{|\xi|})^2+|v_2|^2\qquad\text{and}\qquad\int\,dv=\iint\,dv_2\,dv_1.\]

For the first integral, suppose that $\f$ is supported on a ball $|v|\le R$.
If $\frac{|\tau|-\alpha}{|\xi|}>R$, then the region of integration vanishes so that $I_s=0$.
If $|\tau|\le\alpha+R|\xi|$, then
\begin{align*}
I_s&\les\int_{|v_1|<\frac\alpha{|\xi|}}\int_{|v_2|^2\le R^2-(v_1-\frac\tau{|\xi|})^2}\,dv_2\,dv_1\\
&\les\int_{|v_1|<\frac\alpha{|\xi|},\ |v_1|\le R}\int_{|v_2|\le R}\,dv_2\,dv_1\\
&\les\min\{\frac{2\alpha}{|\xi|},R\}\cdot R^{n-1}\\
&\les\frac1{\sqrt{1+(\frac{|\xi|}\alpha)^2}}\\
&\les\frac\alpha{\sqrt{\tau^2+|\xi|^2}}.
\end{align*}

For the second integral, suppose that $\f$ is supported on $|v|<C$ so that $|v_1-\frac\tau{|\xi|}|,|v_2|<C$.
Then,
\begin{align*}
I_n&\les\int_{|v_1|\ge\frac\alpha{|\xi|},\ |v_1-\frac\tau{|\xi|}|<C}\int_{|v_2|<C}\frac1{v_1^2|\xi|^2}\,dv_2\,dv_1\\
&\simeq\int_{|v_1|\ge\frac\alpha{|\xi|},\ |v_1-\frac\tau{|\xi|}|<C}\frac{dv_1}{v_1^2|\xi|^2}.
\end{align*}
If $|\xi|\ges|\tau|$, then
\[I_n\les\int_{|v_1|\ge\frac\alpha{|\xi|}}\frac{dv_1}{v_1^2|\xi|^2}\simeq\frac1{\alpha|\xi|}\les\frac1{\alpha\sqrt{\tau^2+|\xi|^2}}.\]
If $|\xi|\ll|\tau|$ such that at least $|\tau|>C|\xi|$, then
\begin{align*}
I_n&\les\int_{\max\{\frac\alpha{|\xi|},\frac{|\tau|}{|\xi|}-C\}\le v_1<\frac{|\tau|}{|\xi|}+C}\frac{dv_1}{v_1^2|\xi|^2}\\
&\simeq\frac1{|\xi|^2}(\frac1{\max\{\frac\alpha{|\xi|},\frac{|\tau|}{|\xi|}-C\}}-\frac1{\frac{|\tau|}{|\xi|}+C})\\
&\les\frac1{\alpha\sqrt{\tau^2+|\xi|^2}}.\quad(\text{This is not easy..!})
\end{align*}

To sum up, we have
\[|\hat m(\tau,\xi)|\les\frac1{(\tau^2+|\xi|^2)^{1/4}}(\sqrt\alpha\cdot\|\hat u\|_{L_v^2}^{1/2}+\frac1{\sqrt\alpha}\cdot\|\hat{Lu}\|_{L_v^2}^{1/2}).\]
Letting $\alpha=\sqrt{\|\hat{Lu}\|_{L_v^2}/\|\hat u\|_{L_v^2}}$ and squaring,
\[(\tau^2+|\xi|^2)^{1/2}|\hat m(\tau,\xi)|^2\les\|\hat u\|_{L_v^2}^{1/2}\|\hat{Lu}\|_{L_v^2}^{1/2}.\]
Therefore, the integration on $\R_\tau\x\R_\xi^n$ and Plancheral's theorem gives
\[\|m\|_{H_{t,x}^{1/2}}\les_\f\|u\|_{L_{t,x,v}^2}^{1/2}\|Lu\|_{L_{t,x,v}^2}^{1/2}.\]
\end{pf}

\begin{rmk}
We can obtain $L^p$ result by applying
\[\min\{\frac1x,\frac1y\}\le\frac2{(x^p+y^p)^{1/p}}\le\max\{\frac1x,\frac1y\}\]
on the first integral, and
\[x\gg1\impl\frac1{\max\{a,x-c\}}-\frac1{x+c}\les\frac1{a(x^p+1)^{1/p}}\]
on the second integral.
\end{rmk}

\begin{cor}
Let $\cF$ be a family of functions on $\R_t\x\R_x^n\x\R_v^n$.
If $\cF$ and $L\cF$ are bounded in $L_{t,x,v}^2$, then $\int\cF\f\,dv$ is bounded in $H_{t,x}^{1/2}$.
\end{cor}

\begin{thm}
Let $\cF$ be a family of functions on $I_t\x\R_x^n\x\R_v^n$.
If $\cF$ is weakly relatively compact and $L\cF$ is bounded in $L_{t,x,v}^1$, then $\int\cF\f\,dv$ is relatively compact in $L_{t,x}^1$.
\end{thm}







\section{Peetre's theorem}


\begin{lem}
Suppose a linear operator $L:C_c^\infty(M)\to C_c^\infty(M)$ satisfies
\[\supp(Lu)\subset\supp(u)\quad\text{for}\quad u\in C_c^\infty(X).\]
For each point $x\in M$, there is a bounded neighborhood $U$ together with a nonnegative integer $m$ such that 
\[\|Lu\|_{C^0}\les\|u\|_{C^m}\]
for $u\in C_c^\infty(U\setminus\{x\})$.
\end{lem}
\begin{pf}
Suppose not.
There is a point $x$ at which the inequality fails; for every bounded neighborhood $U$ and for every nonnegative $m$, we can find $u\in C_c^\infty(U\setminus\{x\})$ such that
\[\|Lu\|_{C^0}\ge C\|u\|_{C^m},\]
for arbitrarily large $C$.
We want to construct a function $u\in C_c^\infty(U)$ such that $Lu$ has a singularity at $x$.

(Induction step)
Take a bounded neighborhood $U_m$ of $x$ such that
\[U_m\subset U\setminus\bigcup_{i=0}^{m-1}\cl U_i.\]
There is $u_m\in C_c^\infty(U_m\setminus\{x\})$ such that
\[\|Lu_m\|_{C^0}>4^m\|u_m\|_{C^m}.\]

Note that
\[\supp(u_i)\cap\supp(u_j)=\varnothing\quad\text{for}\quad i\ne j.\]
Define
\[u:=\sum_{i\ge0}2^{-i}\frac{u_i}{\|u_i\|_{C^i}}.\]
We have that $u\in C_c^\infty(U)$ since the series converges in the inductive topology of the LF space $C_c^\infty(U)$: it converges absolutely with respect to the seminorms $\|\cdot\|_{C^m}$ for all $m$:
\begin{align*}
\sum_{i\ge0}\|2^{-i}\frac{u_i}{\|u_i\|_{C^i}}\|_{C^m}
&=\sum_{0\le i<m}2^{-i}\frac{\|u_i\|_{C^m}}{\|u_i\|_{C^i}}+\sum_{i\ge m}2^{-i}\frac{\|u_i\|_{C^m}}{\|u_i\|_{C^i}}\\
&\le\sum_{0\le i<m}2^{-i}\frac{\|u_i\|_{C^m}}{\|u_i\|_{C^i}}+\sum_{i\ge m}2^{-i}\\
&<\infty.
\end{align*}
Also, since the supports of each term are disjoint and $L$ is locally defined, we have
\[Lu=\sum_{i\ge0}2^{-i}\frac{Lu_i}{\|u_i\|_{C^i}}.\]
Thus,
\[\|Lu\|_{C^0}=\sup_{i\ge0}2^{-i}\frac{\|Lu_i\|_{C^0}}{\|u_i\|_{C^i}}>\sup_{i\ge0}2^{-i}\cdot4^i=\infty,\]
which leads a contradiction.

\end{pf}



\section{Characteristic curve}
Algorithm:
\begin{cond}
\item Establish the associated vector field by substituting $u\mapsto y$.
\item Find the integral curve.
\item Eliminate the auxiliary variables to get an algebraic equation.
\item Verify the computed solution is in fact the real solution.
\end{cond}
\begin{prop}
Suppose that there exists a smooth solution $u:\Omega\to\R_y$ of an initial value problem
\begin{pde*}
u_t+u^2u_x&=0, \: (t,x)\in\Omega\subset\R_{t\ge0}\x\R_x,\\
u(0,x)&=x, \: \text{at}\ x\in\R,
\end{pde*}
and let $M$ be the embedded surface defined by $y=u(t,x)$.

Let $\gamma:I\to\Omega\times\R_y$ be an integral curve of the vector field
\[\pd{t}+y^2\pd{x}\]
such that $\gamma(0)\in M$.
Then, $\gamma(\theta)\in M$ for all $\theta\in I$.
\end{prop}
\begin{pf}
We may assume $\gamma$ is maximal.
Define $\tilde\gamma:\tilde I\to M$ as the maximal integral curve of the vector field
\[\tilde X=\pd{t}+u^2\pd{x}\in\Gamma(TM)\]
such that $\tilde\gamma(0)=\gamma(0)$.
Since $X$ and $\tilde X$ coincide on $M$, the curve $\tilde\gamma$ is also an integral curve of $X$ with $\tilde\gamma(0)=\gamma(0)$.
By the uniqueness of the integral curve, we get $\tilde I\subset I$ and $\gamma(\theta)=\tilde\gamma(\theta)$ for all $\theta\in\tilde I$.

Since $M$ is closed in $E$, the open interval $\tilde I=\gamma^{-1}(M)$ is closed in $I$, hence $\tilde I=I$ by the connectedness of $I$.
\end{pf}
\begin{defn}
The projection of the integral curve $\gamma$ onto $\Omega$ is called a \emph{characteristic}.
\end{defn}
This proposition implies that we might be able to describe the points on the surface $M$ explicitly by finding the integral curves of the vector field $X$.
Once we find a necessary condition of the form of algebraic equation, we can demostrate the computed hypothetical solution by explicitly checking if it satisfies the original PDE.

Since $X$ does not depend on $u$, we can solve the ODE: let $\gamma(\theta)=(t(\theta),x(\theta),y(\theta))$ be the integral curve of $X$ such that $\gamma(0)=(0,\xi,\xi)$.
Then, the system of ODEs
\begin{alignat*}{2}
\dd{t}{\theta}&=1,   &\qquad t(0)&=0,\\
\dd{x}{\theta}&=y(\theta)^2, & x(0)&=\xi,\\
\dd{y}{\theta}&=0,   & y(0)&=\xi
\end{alignat*}
is solved as
\[t(\theta)=\theta,\qquad y(\theta)=\xi,\qquad x(\theta)=\xi^2\theta+\xi.\]
Therefore,
\[u(t,x)=\frac{-1+\sqrt{1+4tx}}{2t}.\]
From this formula, we would be able to determine the suitable domain $\Omega$ as
\[\Omega=\{\,(t,x):tx>-\tfrac14\,\}.\]


\subsection{Wave equation}

\begin{align*}
&u_{tt}-c^2u_{xx}=0 \quad\text{for}\quad t,x>0, \\
&u(0,x)=g(x),\qquad u(0,x)=h(x),\qquad u_x(t,0)=\alpha(t).
\end{align*}

Define $v:=u_t-cu_x$.
Then we have
\begin{pde*}
v_t+cv_x &= 0 \: t,x>0,\\
v(0,x) &= h(x)-cg'(x). \:
\end{pde*}
By method of characteristic,
\[v(t,x)=h(x-ct)-cg'(x-ct).\]

Then, we can solve two system
\begin{pde*}
u_t-cu_x &= v, \: x>ct>0,\\
u(0,x) &= g(x), \:
\end{pde*}
and
\begin{pde*}
u_t-cu_x &= v, \: ct>x>0,\\
u_x(t,0) &= \alpha(t), \:
\end{pde*}

For the first system, introducing parameter $\xi>0$,
\begin{gather*}
\dd{t}{\theta}=1,\qquad\dd{x}{\theta}=-c,\qquad\dd{y}{\theta}=-v(t,x),\\
t(0)=0,\qquad x(0)=\xi,\qquad y(0)=g(\xi)
\end{gather*}
is solved as
\[t(\theta)=\theta,\qquad x(\theta)=-c\theta+\xi,\qquad y(\theta)=g(\xi)+\int_0^\theta-v(\theta',\xi-c\theta')\,d\theta',\]
hence for $x>ct>0$,

\begin{align*}
u(t,x)&=g(\xi)-\int_0^\theta v(s,\xi-cs)\,ds\\
&=g(x+ct)\\
&=\frac{3g(x+ct)-g(x-ct)}2-\int_0^th(x+c(t-2s))\,ds
\end{align*}



\clearpage
\subsection{Burgers' equation}

Consider the inviscid Burgers' equation
\[u_t+uu_x=0.\]
\begin{cond}
\item Suppose $u(0,x)=\tanh(x)$. For what values of $t>0$ does the solution of the quasi-linear PDE remain smooth and single valued? Given an approximation sketch of the characteristics in the $tx$-plane.
\item Suppose $u(0,x)=-\tanh(x)$. For what values of $t>0$ does the solution of the quasilinear PDE remain smooth and single valued? Given an approximation sketch of the characteristics in the $tx$-plane.
\item Suppose
\[u(0,x)=\begin{cases}0,&x<0\\x,&0\le x<1,\\1,&1\le x\end{cases}.\]
Sketch the characteristics. Solve the Cauchy problem. Hint: solve the problem in each region separately and ``paste'' the solution together.
\end{cond}


\section{Weak convergences}



\section{Fixed point methods}

\subsection{Picard-Lindel\"of theorem}
\begin{thm}
Consider the following initial value problem:
\[x'=f(t,x),\qquad x(0)=x_0,\]
where
\[f(-,x)\in C([0,T]),\qquad f(t,-)\in\Lip(\cl{B(x_0,R)}).\]
If $T$ and $R$ satisfy
\begin{cond}
\item $TK<1$,
\item $TM\le R$,
\item $|f(t,x)|\le M$ on $[0,T]\times\cl{B(x_0,R)}$,
\end{cond}
then the equation has a unique solution on $[-T,T]$.
\end{thm}
\begin{pf}
Define $\f:C([0,T],\cl{B(x_0,R)})\to C([0,T],\cl{B(x_0,R)})$ as:
\[\f(x)(t):=x_0+\int_0^tf(s,x(s))\,ds.\]
It is well-defined since
\begin{align*}
|\f(x)(t)-x_0|&\le\int_0^t|f(s,x(s))|\,ds\\
&\le TM\le R.
\end{align*}
It is a contraction since we have
\begin{align*}
|\f(x)(t)-\f(y)(t)|
&\le\int_0^t|f(s,x(s))-f(s,y(s))|\,ds\\
&\le\int_0^tK|x(s)-y(s)|\,ds\\
&\le TK\|x(s)-y(s)\|
\end{align*}
so that
\[\|\f(x)-\f(y)\|\le TK\|x-y\|\]
\end{pf}
The above one looses the Lipschitz condition to local condition.









\section{Statements in functional analysis and general topology}
Function analysis:
\begin{itemize}
\item Suppose a densely defined operator $T$ induces a Hilbert space structure on its domain. If the inclusion is bounded, then $T$ has the bounded inverse. If the inclusion is compact, then $T$ has the compact inverse.
\item A closed subspace of an incomplete inner product space may not have orthogonal complement: setting $L^2$ inner product on $C([0,1])$, define $\phi(f)=\int_0^{\frac12}f$.
\item Every seperable Banach space is linearly isomorphic and homeomorphic. But there are two non-isomorphic Banach spaces.
\item open mapping theorem -> continuous embedding is really an embedding.
\item $D(\Omega)$ is defined by a \emph{countable stict} inductive limit of $D_K(\Omega)$, $K\subset\Omega$ compact. Hence it is not metrizable by the Baire category theorem. (Here strict means that whenever $\alpha<\beta$ the induced topology by $\cT_\beta$ coincides with $\cT_\alpha$)
\item A net $(\phi_d)_d$ in $D(\Omega)$ converges if and only if there is a compact $K$ such that $\phi_d\in D_K(\Omega)$ for all $d$ and $\phi_d$ converges uniformly.
\item Th integration with a locally integrable function is a distribution. This kind of distribution is called \emph{regular}. The nonregular distribution such as $\delta$ is called \emph{singular}.
\item $D'$ is equipped with the weak$^*$ topology.
\item $\pd{x}\colon D'\to D'$ is continuous. They commute (Schwarz theorem holds).
\item $D\to S\to L^p$ are continuous (immersion) but not imply closed subspaces (embedding).
\end{itemize}
General topology:
\begin{itemize}
\item $H\subset\C$ and $H\subset\hat\C$ have distinct Cauchy structures which give a same topology. In addition, the latter is precompact while the former is not.
\end{itemize}






\section{Ultrafilter}
\begin{defn}
An \emph{ultrafilter} is a synonym for maximal filter.
If we sat $\cU$ is an \emph{ultrafilter on a set} $A$, then it means $\cU$ is a maximal filter as a directed subset of $\cP(A)$.
\end{defn}
existence of ultrafilter.
\begin{thm}
Let $\cU$ be an ultrafilter on a set $A$ and $X$ be a compact space.
For a function $f:A\to X$, the limit $\cU$-$\lim f$ always exists.
\end{thm}

\begin{thm}
Let $X=\prod_{\alpha\in\cA}X_\alpha$ be a product space of compact spaces $X_\alpha$.
A net $f:\cD\to X$ has a convergent subnet.
\end{thm}
\begin{pf}[1]
Use Tychonoff.
Compactness and net compactness are equivalent.
\end{pf}
\begin{pf}[2]
It is a proof without Tychonoff.
Let $\cU$ be a ultrafilter on a set $\cD$ contatining all $\uparrow d$.
Define a directed set $\cE=\{(d,U)\in\cD\x\cU:d\in U\}$ as $(d,U)\succ(d',U')$ for $U\subset U'$.
Let $f:\cE\to X$ be a subnet of $f:\cD\to X$ defined by $f_{(d,U)}=f_d$.

By the previous theorem, $\cU$-$\lim\pi_\alpha f_d\in X_\alpha$ exsits for each $\alpha$.
Define $f\in X$ such that $\pi_\alpha f=\cU$-$\lim\pi_\alpha f_d$.
Let $G=\prod_\alpha G_\alpha\subset X$ be any open neighborhood of $f$.
Then, $\pi_\alpha f\in G_\alpha$ and we have $G_\alpha=X_\alpha$ except finite.
For $\alpha$, we can take $U_\alpha:=\{d:\pi_\alpha f_d\in G_\alpha\}\in\cU$ by definition of convergence with ultrafilter
Since $U_\alpha=\cD$ except finites, we can take an upper bound $U_0\in\cU$ of $\{U_\alpha\}_\alpha$.
Then, by taking any $d_0\in U_0$, we have $f_{(d,U)}\in G$ for every $(d,U)\succ(d_0,U_0)$.
This means $f=\lim_\cE f_{(d,U)}$, so we can say $\lim_\cE f_{(d,U)}$ exists.
\end{pf}







\section{Analysis problems}

Problems I made:
\begin{enumerate}
\item Let $f$ be $C^2$ with $f''(c)\ne0$. Defined a function $\xi$ such that $f(x)-f(c)=f'(\xi(x))(x-c)$ with $|\xi-c|\le|x-c|$, show that $\xi'(c)=1/2$.
\item Let $f$ be a $C^2$ function such that $f(0)=f(1)=0$. Show that $\|f\|\le\frac18\|f''\|$.
\item Show that a measurable subset of $\R$ with positive measure contains an arbitrarily long subsequence of an arithmetic progression.
\item Show that there is no continuous bijection from $[0,1]^2\setminus\{p\}$ to $[0,1]^2$. %나도 못풂
\end{enumerate}
\bigskip

Elementary analysis
\begin{cond}
\item Show that for a nonnegative sequence $a_n$ if $\sum a_n$ diverges then $\sum\frac{a_n}{1+a_n}$ also diverges.
\item Show that if both limits of a function and its derivative exist at infinity then the former is 0.
\item Show that every real sequence has a monotonic subsequence that converges to the limit superior of the supersequence.
\item Show that if a decreasing nonnegative sequence $a_n$ converges to 0 and satisfies $S_n\le1+na_n$ then $S_n$ is bounded by 1.
\item Show that the set of local minima of a convex function is connected.
\item Show that a smooth function such that for each $x$ there is $n$ having the $n$th derivative vanish is a polynomial.
\item Show that if a continuously differentiable $f$ satisfies $f(x)\ne0$ for $f'(x)=0$, then in a bounded set there are only finite points at which $f$ vanishes.
\item Let a function $f$ be differentiable. For $a<a'<b<b'$ show that there exist $a<c<b$ and $a'<c'<b'$ such that $f(b)-f(a)=f'(c)(b-a)$ and $f(b')-f(a')=f'(c')(b'-a')$.
\item Show that if $xf'(x)$ is bounded and $x^{-1}\int_0^xf\to L$ then $f(x)\to L$ as $x\to\infty$.
\item Show that if a sequence of real functions $f_n\colon[0,1]\to[0,1]$ satisfies $|f(x)-f(y)|\le|x-y|$ whenever $|x-y|\ge\frac1n$, then the sequence has a uniformly convergent subsequence.
\item (Flett)
\item Let $f$ be a differentiable function with $f(0)=0$. Show that there is $c\in(0,1)$ such that $cf(c)=(1-c)f'(c)$.
\item Find the value of $\lim_{n\to\infty}\frac1n\left(\sum_{k=1}^n\frac1nf\left(\frac kn\right)-\int_0^1f(x)\,dx\right)$.
\item Let $f:\R\to\R$ be continuous. Show that $f(x)=c$ cannot have exactly two solutions for every constant $c\in\R$.
\item Show that a continuous function that takes on no value more than twice takes on some value exactly once.
\item Let $f$ be a function that has the intermediate value property. Show that if the preimage of every singleton is closed, then f is continuous.
\end{cond}

Complex analysis
\begin{cond}
\item Show that if a holomorphic function has positive real parts on the open unit disk then $|f'(0)|<2\Re f(0)$.
\item Show that if at least one coefficient in the power series of a holomorphic function at each point is 0 then the function is a polynomial.
\item Show that if a holomorphic function on a domain containing the closed unit disk is injective on the unit circle then so is on the disk.
\item Show that for a holomorphic function $f$ and every $z_0$ in the domain there are $z_1\ne z_2$ such that $\frac{f(z_1)-f(z_2)}{z_1-z_2}=f'(z_0)$.
\item For two linearly independent entire functions, show that one cannot dominate the other.
\item Show that uniform limit of injective holomorphic function is either constant or injective.
\item Suppose the set of points in a domain $U\subset\C$ at which a sequence of bounded holomorphic functions $(f_n)$ converges has a limit point. Show that $(f_n)$ compactly converges.
\end{cond}
\clearpage

\begin{prb}
The following series diverges: \[\sum_{n=1}^\infty\frac1{n^{1+|\sin n|}}.\]
\end{prb}
\begin{sol}
Let $A_k:=[1,2^k]\cap\{x:|\sin x|<\frac1k\}$.
Divide the unit circle $\R/2\pi\Z$ by $7k$ uniform arcs.
There are at least $2^k/7k$ integers that are not exceed $2^k$ and are in a same arc.
Let $S$ be the integers and $x_0$ be the smallest element.
Since, $|x-x_0|\pmod{2\pi}<\frac{2\pi}{7k}$ for $x\in S$,
\[|\sin(x-x_0)|<|x-x_0|\pmod{2\pi}<\frac{2\pi}{7k}<\frac1k.\]
Also, $1\le x-x_0\le x\le2^k$, $x-x_0\in A_k$.
\[|A_k|\ge\frac{2^k}{7k}.\]
Therefore,
\begin{align*}
\sum_{n=1}^\infty\frac1{n^{1+|\sin n|}}
&\ge\sum_{n\in A_N}\frac1{n^{1+|\sin n|}}\\
&\ge\sum_{k=1}^N(|A_k|-|A_{k-1}|)\frac1{2^{k+1}}\\
&=\sum_{k=1}^N\frac{|A_k|}{2^{k+1}}-\sum_{k=1}^{N-1}\frac{|A_k|}{2^{k+2}}\\
&=\frac{|A_N|}{2^{N+1}}+\sum_{k=1}^{N-1}\frac{|A_k|}{2^{k+2}}\\
&>\sum_{k=1}^N\frac{2^k}{2^{k+2}}\frac1{7k}\\
&=\frac1{28}\sum_{k=1}^N\frac1k\\
&\to\infty.
\end{align*}
\end{sol}

\clearpage
\begin{prb}
If $|xf'(x)|\le M$ and $\frac1x\int_0^xf(y)\,dy\to L$, then $f(x)\to L$ as $x\to\infty$.
\end{prb}
\begin{sol}
Since
\begin{align*}
\abs{f(x)-\frac{F(x)-F(a)}{x-a}}
&\le\frac1{x-a}\int_a^x|f(x)-f(y)|\,dy\\
&=\frac1{x-a}\int_a^x(x-y)|f'(c)|\,dy\\
&\le\frac M{x-a}\int_a^x\frac{x-y}c\,dy\\
&\le M\frac{x-a}a
\end{align*}
by the mean value theorem and 
\[f(x)-L=\left[f(x)-\frac{F(x)-F(a)}{x-a}\right]+\frac x{x-a}\left[\frac{F(x)}x-L\right]+\frac a{x-a}\left[\frac{F(a)}a-L\right],\]
we have for any $\e>0$
\[\limsup_{x\to\infty}|f(x)-L|\le\e\]
where $a$ is defined by $\frac{x-a}a=\frac\e M$.
\end{sol}

\clearpage
\begin{prb}
Let $f_n:I\to I$ be a sequence of real functions that satisfies $|f_n(x)-f_n(y)|\le|x-y|$ whenever $|x-y|\ge\frac1n$, where $I=[0,1]$.
Then, it has a uniformly convergent subsequence.
\end{prb}
\begin{sol}
By the Bolzano-Weierstrass theorem and the diagonal argument for subsequence extraction, we may assume that $f_n$ converges to a function $f:\Q\cap I\to I$ pointwisely.

\Step[1]
For $n\ge4$, we claim
\begin{align}|x-y|\le\frac1n\impl|f_n(x)-f_n(y)|\le\frac5n.\end{align}
Fix $x\in I$ and take $z\in I$ such that $|x-z|=\frac2n$ so that
\[|f_n(x)-f_n(z)|\le|x-z|=\frac2n.\]
If $y$ satisfies $|x-y|\le\frac1n$, then we have $|y-z|\ge|x-z|-|x-y|\ge\frac1n$, so we get
\[|f_n(y)-f_n(z)|\le|y-z|\le|y-x|+|x-z|\le\frac3n.\]
Combining these two inequalities proves what we want.

\Step[2]
For $\e>0$ and $N:=\ceil{\frac{15}\e}$ we claim
\begin{align}|x-y|\le\frac1N\quad\text{and}\quad n>N\impl|f_n(x)-f_n(y)|\le\frac\e3\end{align}
when $N\ge4$.
It is allowed for $|x-y|$ to have the following two cases:
\[|x-y|\le\frac1n\quad\text{or}\quad\frac1n<|x-y|\le\frac1N.\]
For the former case, by the inequality (1) we have
\[|f_n(x)-f_n(y)|\le\frac5n<\frac5N\le\frac\e3.\]
For the latter case, by the assumption at the beginning of the problem, we have
\[|f_n(x)-f_n(y)|\le|x-y|\le\frac1N\le\frac\e{15}.\]
Hence the claim is proved.

\Step[3]
We will prove $f$ is uniformly continuous.
For $\e>0$, take $\delta:=\frac1N$, where $N:=\ceil{\frac{15}\e}$.
We will show
\[|x-y|<\delta\impl|f(x)-f(y)|<\e\]
for $x,y\in\Q\cap I$ and $N\ge4$.
Fix rational numbers $x$ and $y$ in $I$ which satisfy $|x-y|<\delta$.
Since $f_n(x)$ and $f_n(y)$ converges to $f(x)$ and $f(y)$ respectively, we may take an integer $n_x$ and $n_y$, such that
\begin{align}n>n_x\impl |f_n(x)-f(x)|<\frac\e3\end{align}
and
\begin{align}n>n_y\impl |f_n(y)-f(y)|<\frac\e3.\end{align}
Choose an integer $n$ such that $n>\max\{n_x,n_y,N\}$.
Then, combining (3), (2), and (4), we obtain
\begin{align*}
|f(x)-f(y)|&\le|f(x)-f_n(x)|+|f_n(x)-f_n(y)|+|f_n(y)-f(y)|\\
&<\frac\e3+\frac\e3+\frac\e3=\e.
\end{align*}

Since $f$ is continuous on a dense subset $\Q\cap I$, it has a unique continuous extension on the whole $I$.
Let it denoted by the same notation $f$.

\Step[4]
Finally, we are going to show $f_n\to f$ uniformly.
For $\e>0$, let $N:=\ceil{\frac{15}\e}$.
The uniform continuity of $f$ allows to have $\delta>0$ such that
\begin{align}|x-y|<\delta\impl|f(x)-f(y)|<\frac23\e.\end{align}
Take a rational $r\in I$, depending on $x\in I$, such that $|x-r|<\min\{\frac1N,\delta\}$.
Then, by (2) and (5), given $n>N\ge4$, we have an inequality
\begin{align*}
|f_n(x)-f(x)|&\le|f_n(x)-f_n(r)|+|f_n(r)-f(r)|+|f(r)-f(x)|\\
&<\frac\e3+|f_n(r)-f(r)|+\frac23\e
\end{align*}
for any $x\in I$.
By limiting $n\to\oo$, we obtain
\[\lim_{n\to\oo}|f_n(x)-f(x)|<\e.\]
Since $\e$ and $x$ are arbitrary, we can deduce the uniform convergence of $f_n$ as $n\to\oo$.
\end{sol}


\clearpage
\begin{prb}
A measurable subset of $\R$ with positive measure contains an arbitrarily long subsequence of an arithmetic progression. (made by me!)
\end{prb}
\begin{sol}
Let $E\subset\R$ be measurable with $\mu(E)>0$. We may assume $E$ is bounded so that we have $E\subset I$ for a closed bounded interval since $\R$ is $\sigma$-compact.
Let $n$ be a positive integer arbitrarily taken. Then, we can find $N$ such that $\sum_{k=1}^N\frac1k>(n-1)\frac{\mu(I)}{\mu(E)}$.

Assume that every point $x$ in $E$ is contained in at most $n-1$ sets among
\[E,\ \frac12E,\ \frac13E,\ \cdots,\ \frac1NE.\]
In other words, it is equivalent to:
\[\bigcap_{k\in A}\frac1kE=\mt\]
for any subset $A\subset\{1,\cdots,N\}$ with $|A|\ge n$.
Define
\[E_A:=\bigcap_{k\in A}\frac1kE\cap\bigcap_{k'\in A}\left(\frac1{k'}E\right)^c\]
for $A\subset\{1,\cdots,N\}$.
Then, $\mu(E_A)=0$ for $|A|\ge n$.

Note that we have
\[\mu(\tfrac1kE)=\sum_{k\in A}\mu(E_A)=\sum_{\substack{k\in A\\|A|<n}}\mu(E_A).\]
Summing up, we get
\[\sum_{k=1}^N\mu(\tfrac1kE)=\sum_{k=1}^N\sum_{\substack{k\in A\\|A|<n}}\mu(E_A)=\sum_{|A|<n}|A|\mu(E_A)\]
by double counting,
and since $E_A$ are dijoint, we have
\[\sum_{|A|<n}|A|\mu(E_A)=(n-1)\sum_{0<|A|<n}\mu(E_A)\le(n-1)\mu(I),\]
hence a contradiction to
\[\sum_{k=1}^N\mu(\tfrac1kE)>(n-1)\mu(I).\]
Therefore, we may find an element $x$ that belongs to $\frac1kE$ for $k\in A$, where $A\subset\{1,\cdots,N\}$ with $|A|=n$.
Then, $ax\in E$ for all $a\in A\subset\Z$.
\end{sol}


\section{Physics problem}
\subsection{Resonance}
Let $m,b,k,A,\omega_d$ be positive real constants.
Consider an underdamped oscillator with sinusoidal diving force described as
\[mx''+bx'+kx=A\sin\omega_dt,\quad x(0)=x_0,\ x'(0)=0.\]
There are some observations:
\begin{cond}
\item The underdamping condition means $b^2-4mk<0$ so that the roots of characteristic equation are imaginary.
\item The positivity of $m,b$ implies the real part of solution that will be denoted by $-\beta=-\frac b{2m}$ is negative; it shows exponential decay of solutions.
\item Introducing the natural frequency $\omega_n=\sqrt{k/m}$, we can rewrite the equation as
\[x''+2\zeta\omega_n x'+\omega_n^2x=A\sin\omega t.\]
\item The complementary solution is computed as
\[x_c(t)=x_0\ e^{-\beta t}\cos\sqrt{\beta^2-\omega_n^2}t,\]
and it can be verified that this solution is asymptotically stable, i.e.
\[\lim_{t\to\infty}x_c(t)=0.\]
\item The condition $\beta>\omega_n$ is equivalent to that the oscillator is underdamped.
\item Let $m,k$ be fixed. Then, the solution $x_c$ decays most fastly when $b$ satisfied $b^2=4mk$, equivalently, $\beta=\omega_n$.
\item When $\omega_d=\omega_n$ such that the amplitude of particular solution diverges.
\end{cond}



\end{document}

최고차항의 계수가 1인 사차함수 f(x)는 다음 조건을 만족시킨다.
[각각의 양의 실수 s에 대하여, |f(t)|≤g(t)를 만족시키는 실수 t가 두 개 존재하고 그 차가 s가 되게 하는 이차 이하의 다항함수 g(x)가 유일하게 존재한다.]
또한 위의 조건에서 s=1,3에 대응하는 함수 g(x)를 각각 g₁(x), g₃(x)라 할 때, 방정식 g₁(x)+g₃(x-1/2)=0을 만족시키는 실수 x가 단 한 개 존재한다. 함수 f(x)가 최솟값 -p/q을 가질 때, p+q를 구하시오. (단, p와 q는 서로 소인 자연수이다.)

ans: 43

\iffalse
\begin{defn}
A quantum field is defined as an operator valued tempered distribution satisfying:
\end{defn}

이차양자화: 임베드 L(H) -> L(F(H))
생성소멸: f를 left tensor prod하는 작용소로 바꾸기
		H의 원소를 L(F(H))


일차양자화에서 R이 L(H)로 바뀐 것처럼,
이차양자화도   R이 L(H)로 바뀐 것.


근데 다체론에서 H 대신 F를 쓰고 싶은데, L(H) 에서 L(F(H))로 확장시키는 방법이 있어서 L(F)로 바꿔도 됨.


양자장은 H에 작용해서 입자를 만들거나 없애는 작용소이며, 물리량과 조화롭게 연관됨

Wightman quantum field theory:
quantum field = operator valued tempered distribution
F carries a unitary representation of restricted orthochronous Poincare group

이 양자장조차 미분가능해 어머

RQFT satisfies: invariance of c, Poincare covariance


time-translation covariant quantum field에 대해서 hamiltonian과의 bracket이 미분과 같다고 식을 세우면 이게 양자장의 방정식.




H는 해공간
L(H)는 미분연산자공간

물리과 3학년 버전 파동함수: H = C valued L2 space		L(H) = C coeff pdd valued fields (물리량)
디랙입자에 대한 고전스피너: H = DA irrep valued L2 space	L(H) = DA coeff pdd valued fields?? (물리량)

M이 국소적으로 민코프스키 메트릭을 가지면 그 접공간은 STA irrep
spinor로서의 DA irrep은 four vector가 아냐 (이 STA irrep과 다른 거야, 접공간이 아냐)


디랙방정식의 해 in H 를 classical Dirac field, free Dirac spinor라고 함

STA = 실수 16차원    DA = 복소 16차원


creation operator 하나로 된 field에 vaccum state를 넣으면 state vector가 나온다.
\fi
