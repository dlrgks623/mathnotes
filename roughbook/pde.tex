\documentclass[11pt]{article}
\usepackage{../ikany}
\usepackage[margin=1.5in]{geometry}

\let\realsection\section
\renewcommand\section{\newpage\realsection}

\begin{document}
\tableofcontents



\section{Kinetic theory}

The velocity averaging lemma is used to get regularity of averaged quantity when boundary condition is not given.
\begin{thm}[Velocity averaging]
Let $T$ be a free transport operator $\pd_t+v\cdot\del_x$ on $\R_t\x\R_x^n\x\R_v^n$.
Then,
\[\|\int u\f\,dv\|_{H_{t,x}^{1/2}}\les_\f\|u\|_{L_{t,x,v}^2}^{1/2}\|Tu\|_{L_{t,x,v}^2}^{1/2}\]
for $\f\in C_c^\infty(\R_v^n)$,
\end{thm}
\begin{pf}
Let $m(t,x)=\int u\f\,dv$.
By Fourier transform with respect to $t$ and $x$, we have
\[\hat u(\tau,\xi,v)=i\,\frac{\hat{Tu}(\tau,\xi,v)}{\tau+v\cdot\xi}\]
and
\[\hat m(\tau,\xi)=\int\hat u(\tau,\xi,v)\f(v)\,dv.\]
Fixing $\tau,\xi$, decompose the integral and use H\"older's inequality to get
\begin{align*}
|\hat m(\tau,\xi)|
&\le\int_{|\tau+v\cdot\xi|<\alpha}|\hat u\f|\,dv+\int_{|\tau+v\cdot\xi|\ge\alpha}\frac{|\hat{Tu}\f|}{|\tau+v\cdot\xi|}\,dv\\
&\le\|\hat u\|_{L_v^2}^{1/2}\ (\int_{|\tau+v\cdot\xi|<\alpha}|\f|^2\,dv)^{1/2}+\|\hat{Tu}\|_{L_v^2}^{1/2}\ (\int_{|\tau+v\cdot\xi|\ge\alpha}\frac{|\f|^2}{|\tau+v\cdot\xi|^2}\,dv)^{1/2}
\end{align*}
We are going to estimate the integrals as
\[\int_{|\tau+v\cdot\xi|<\alpha}|\f|^2\,dv\les\frac{\alpha}{\sqrt{\tau^2+|\xi|^2}},\qquad
\int_{|\tau+v\cdot\xi|\ge\alpha}\frac{|\f|^2}{|\tau+v\cdot\xi|}\,dv\les\frac1{\alpha\sqrt{\tau^2+|\xi|^2}}.\]
We may assume that $\sqrt{\tau^2+|\xi|^2}\gg1$, that is, it is enough to show them for $\sqrt{\tau^2+|\xi|^2}\ge C$ with arbitrarily taken constant $C$, because the case that $\sqrt{\tau^2+|\xi|^2}\les1$ easily proves the inequality.

Define coordinates $(v_1,v_2)$ on $\R_v$ as follows:
\[v_1:=\frac{\tau+v\cdot\xi}{|\xi|}\ \in\R\ ,\qquad v_2:=v-\frac{v\cdot\xi}{|\xi|^2}\xi\ \in\ker(\xi^T)\cong\R^{n-1}.\]
Note that
\[|v|^2=(v-\frac\tau{|\xi|})^2+|v_2|^2\qquad\text{and}\qquad\int\,dv=\iint\,dv_2\,dv_1.\]

For the first integral, suppose that $\f$ is supported on a ball $|v|<R$.
Then,
\begin{align*}
\int_{|\tau+v\cdot\xi|<\alpha}|\f|^2\,dv
&\les\int_{|v_1|<\frac\alpha{|\xi|}}\int_{|v_2|^2\le R^2-(v_1-\frac\tau{|\xi|})^2}\,dv_2\,dv_1\\
&\les(R^2-\frac{\tau^2}{|\xi|^2})^{\frac{n-1}2}\cdot\frac{2\alpha}{|\xi|},
\end{align*}
where we value the term $(R^2-\frac{\tau^2}{|\xi|^2})$ as 0 when $R^2<\frac{\tau^2}{|\xi|^2}$.
Since
\[(R^2-\frac{\tau^2}{|\xi|^2})^{\frac{n-1}2}\frac{2\alpha}{|\xi|}\cdot\sqrt{\tau^2+|\xi|^2}\les\begin{cases}0,&|\tau|\gg1\\C,&|\xi|\gg1\end{cases},\]
we have
\[\int_{|\tau+v\cdot\xi|<\alpha}|\f|^2\,dv\les\frac{\alpha}{\sqrt{\tau^2+|\xi|^2}}.\]

For the second integral, suppose that $\f$ is supported on $|v|<C$ so that $|v_1-\frac\tau{|\xi|}|,|v_2|<C$.
Then,
\begin{align*}
\int_{|\tau+v\cdot\xi|\ge\alpha}\frac{|\f|^2}{|\tau+v\cdot\xi|}\,dv
&\les\int_{|v_1|<\frac\alpha{|\xi|},\ |v_1-\frac\tau{|\xi|}|<C}\int_{|v_2|<C}\frac1{v_1^2|\xi|^2}\,dv_2\,dv_1\\
&\simeq\int_{|v_1|<\frac\alpha{|\xi|},\ |v_1-\frac\tau{|\xi|}|<C}\frac{dv_1}{v_1^2|\xi|^2}.
\end{align*}
If $|\xi|\ges|\tau|$, then
\begin{align*}
\int_{|v_1|<\frac\alpha{|\xi|},\ |v_1-\frac\tau{|\xi|}|<C}\frac{dv_1}{v_1^2|\xi|^2}
&\les\int_{|v_1|<\frac\alpha{|\xi|}}\frac{dv_1}{v_1^2|\xi|^2}\\
&\simeq\frac1{\alpha|\xi|}\les\frac1{\alpha\sqrt{\tau^2+|\xi|^2}}.
\end{align*}
If $|\xi|\ll|\tau|$ such that at least $|\tau|>C|\xi|$, then
\begin{align*}
\int_{|v_1|<\frac\alpha{|\xi|},\ |v_1-\frac\tau{|\xi|}|<C}\frac{dv_1}{v_1^2|\xi|^2}
&\les\int_{|v_1-\frac\tau{|\xi|}|<C}\frac{dv_1}{v_1^2|\xi|^2}\\
&\simeq\frac1{|\xi|^2}(\frac1{\frac\tau{|\xi|}-C}-\frac1{\frac\tau{|\xi|}+C})\\
&=\frac{2C}{\tau^2-C^2|\xi|^2}\ll\frac1{\sqrt{\tau^2+|\xi|^2}},
\end{align*}
hence
\[\int_{|\tau+v\cdot\xi|\ge\alpha}\frac{|\f|^2}{|\tau+v\cdot\xi|}\,dv\les\frac1{\alpha\sqrt{\tau^2+|\xi|^2}}.\]

To sum up, we have
\[|\hat m(\tau,\xi)|\les\frac1{(\tau^2+|\xi|^2)^{1/4}}(\sqrt\alpha\cdot\|\hat u\|_{L_v^2}^{1/2}+\frac1{\sqrt\alpha}\cdot\|\hat{Tu}\|_{L_v^2}^{1/2}).\]
Letting $\alpha=\sqrt{\|\hat{Tu}\|_{L_v^2}/\|\hat u\|_{L_v^2}}$ and squaring,
\[(\tau^2+|\xi|^2)^{1/2}|\hat m(\tau,\xi)|^2\les\|\hat u\|_{L_v^2}^{1/2}\|\hat{Tu}\|_{L_v^2}^{1/2}.\]
Therefore, the integration on $\R_\tau\x\R_\xi^n$ and Plancheral's theorem gives
\[\|m\|_{H_{t,x}^{1/2}}\les_\f\|u\|_{L_{t,x,v}^2}^{1/2}\|Tu\|_{L_{t,x,v}^2}^{1/2}.\]
\end{pf}

\begin{cor}
Let $\cF$ be a family of functions on $\R_t\x\R_x^n\x\R_v^n$.
If $\cF$ and $T\cF$ are bounded in $L_{t,x,v}^2$, then $\int\cF\f\,dv$ is bounded in $H_{t,x}^{1/2}$.
\end{cor}

\begin{thm}
Let $\cF$ be a family of functions on $I_t\x\R_x^n\x\R_v^n$.
If $\cF$ is weakly relatively compact and $T\cF$ is bounded in $L_{t,x,v}^1$, then $\int\cF\f\,dv$ is relatively compact in $L_{t,x}^1$.
\end{thm}







\section{Peetre's theorem}


\begin{lem}
Suppose a linear operator $L:C_c^\infty(M)\to C_c^\infty(M)$ satisfies
\[\supp(Lu)\subset\supp(u)\quad\text{for}\quad u\in C_c^\infty(X).\]
For each point $x\in M$, there is a bounded neighborhood $U$ together with a nonnegative integer $m$ such that 
\[\|Lu\|_{C^0}\les\|u\|_{C^m}\]
for $u\in C_c^\infty(U\setminus\{x\})$.
\end{lem}
\begin{pf}
Suppose not.
There is a point $x$ at which the inequality fails; for every bounded neighborhood $U$ and for every nonnegative $m$, we can find $u\in C_c^\infty(U\setminus\{x\})$ such that
\[\|Lu\|_{C^0}\ge C\|u\|_{C^m},\]
for arbitrarily large $C$.
We want to construct a function $u\in C_c^\infty(U)$ such that $Lu$ has a singularity at $x$.

(Induction step)
Take a bounded neighborhood $U_m$ of $x$ such that
\[U_m\subset U\setminus\bigcup_{i=0}^{m-1}\cl U_i.\]
There is $u_m\in C_c^\infty(U_m\setminus\{x\})$ such that
\[\|Lu_m\|_{C^0}>4^m\|u_m\|_{C^m}.\]

Note that
\[\supp(u_i)\cap\supp(u_j)=\varnothing\quad\text{for}\quad i\ne j.\]
Define
\[u:=\sum_{i\ge0}2^{-i}\frac{u_i}{\|u_i\|_{C^i}}.\]
We have that $u\in C_c^\infty(U)$ since the series converges in the inductive topology of the LF space $C_c^\infty(U)$: it converges absolutely with respect to the seminorms $\|\cdot\|_{C^m}$ for all $m$:
\begin{align*}
\sum_{i\ge0}\|2^{-i}\frac{u_i}{\|u_i\|_{C^i}}\|_{C^m}
&=\sum_{0\le i<m}2^{-i}\frac{\|u_i\|_{C^m}}{\|u_i\|_{C^i}}+\sum_{i\ge m}2^{-i}\frac{\|u_i\|_{C^m}}{\|u_i\|_{C^i}}\\
&\le\sum_{0\le i<m}2^{-i}\frac{\|u_i\|_{C^m}}{\|u_i\|_{C^i}}+\sum_{i\ge m}2^{-i}\\
&<\infty.
\end{align*}
Also, since the supports of each term are disjoint and $L$ is locally defined, we have
\[Lu=\sum_{i\ge0}2^{-i}\frac{Lu_i}{\|u_i\|_{C^i}}.\]
Thus,
\[\|Lu\|_{C^0}=\sup_{i\ge0}2^{-i}\frac{\|Lu_i\|_{C^0}}{\|u_i\|_{C^i}}>\sup_{i\ge0}2^{-i}\cdot4^i=\infty,\]
which leads a contradiction.

\end{pf}



\section{Characteristic curve}
Algorithm:
\begin{cond}
\item Establish the associated vector field by substituting $u\mapsto y$.
\item Find the integral curve.
\item Eliminate the auxiliary variables to get an algebraic equation.
\item Verify the computed solution is in fact the real solution.
\end{cond}
\begin{prop}
Suppose that there exists a smooth solution $u:\Omega\to\R_y$ of an initial value problem
\begin{pde*}
u_t+u^2u_x&=0, \: (t,x)\in\Omega\subset\R_{t\ge0}\x\R_x,\\
u(0,x)&=x, \: \text{at}\ x\in\R,
\end{pde*}
and let $M$ be the embedded surface defined by $y=u(t,x)$.

Let $\gamma:I\to\Omega\times\R_y$ be an integral curve of the vector field
\[X=\pd{t}+y^2\pd{x}\]
such that $\gamma(0)\in M$.
Then, $\gamma(\theta)\in M$ for all $\theta\in I$.
\end{prop}
\begin{pf}
We may assume $\gamma$ is maximal.
Define $\tilde\gamma:\tilde I\to M$ as the maximal integral curve of the vector field
\[\tilde X=\pd{t}+u^2\pd{x}\in\Gamma(TM)\]
such that $\tilde\gamma(0)=\gamma(0)$.
Since $X$ and $\tilde X$ coincide on $M$, the curve $\tilde\gamma$ is also an integral curve of $X$ with $\tilde\gamma(0)=\gamma(0)$.
By the uniqueness of the integral curve, we get $\tilde I\subset I$ and $\gamma(\theta)=\tilde\gamma(\theta)$ for all $\theta\in\tilde I$.

Since $M$ is closed in $E$, the open interval $\tilde I=\gamma^{-1}(M)$ is closed in $I$, hence $\tilde I=I$ by the connectedness of $I$.
\end{pf}
\begin{defn}
The projection of the integral curve $\gamma$ onto $\Omega$ is called a \emph{characteristic}.
\end{defn}
This proposition implies that we might be able to describe the points on the surface $M$ explicitly by finding the integral curves of the vector field $X$.
Once we find a necessary condition of the form of algebraic equation, we can demostrate the computed hypothetical solution by explicitly checking if it satisfies the original PDE.

Since $X$ does not depend on $u$, we can solve the ODE: let $\gamma(\theta)=(t(\theta),x(\theta),y(\theta))$ be the integral curve of $X$ such that $\gamma(0)=(0,\xi,\xi)$.
Then, the system of ODEs
\begin{alignat*}{2}
\dd{t}{\theta}&=1,   &\qquad t(0)&=0,\\
\dd{x}{\theta}&=y(\theta)^2, & x(0)&=\xi,\\
\dd{y}{\theta}&=0,   & y(0)&=\xi
\end{alignat*}
is solved as
\[t(\theta)=\theta,\qquad y(\theta)=\xi,\qquad x(\theta)=\xi^2\theta+\xi.\]
Therefore,
\[u(t,x)=\frac{-1+\sqrt{1+4tx}}{2t}.\]
From this formula, we would be able to determine the suitable domain $\Omega$ as
\[\Omega=\{\,(t,x):tx>-\tfrac14\,\}.\]


\subsection{Wave equation}

\begin{align*}
&u_{tt}-c^2u_{xx}=0 \quad\text{for}\quad t,x>0, \\
&u(0,x)=g(x),\qquad u(0,x)=h(x),\qquad u_x(t,0)=\alpha(t).
\end{align*}

Define $v:=u_t-cu_x$.
Then we have
\begin{pde*}
v_t+cv_x &= 0 \: t,x>0,\\
v(0,x) &= h(x)-cg'(x). \:
\end{pde*}
By method of characteristic,
\[v(t,x)=h(x-ct)-cg'(x-ct).\]

Then, we can solve two system
\begin{pde*}
u_t-cu_x &= v, \: x>ct>0,\\
u(0,x) &= g(x), \:
\end{pde*}
and
\begin{pde*}
u_t-cu_x &= v, \: ct>x>0,\\
u_x(t,0) &= \alpha(t), \:
\end{pde*}

For the first system, introducing parameter $\xi>0$,
\begin{gather*}
\dd{t}{\theta}=1,\qquad\dd{x}{\theta}=-c,\qquad\dd{y}{\theta}=-v(t,x),\\
t(0)=0,\qquad x(0)=\xi,\qquad y(0)=g(\xi)
\end{gather*}
is solved as
\[t(\theta)=\theta,\qquad x(\theta)=-c\theta+\xi,\qquad y(\theta)=g(\xi)+\int_0^\theta-v(\theta',\xi-c\theta')\,d\theta',\]
hence for $x>ct>0$,

\begin{align*}
u(t,x)&=g(\xi)-\int_0^\theta v(s,\xi-cs)\,ds\\
&=g(x+ct)\\
&=\frac{3g(x+ct)-g(x-ct)}2-\int_0^th(x+c(t-2s))\,ds
\end{align*}



\clearpage
\subsection{Burgers' equation}

Consider the inviscid Burgers' equation
\[u_t+uu_x=0.\]
\begin{cond}
\item Suppose $u(0,x)=\tanh(x)$. For what values of $t>0$ does the solution of the quasi-linear PDE remain smooth and single valued? Given an approximation sketch of the characteristics in the $tx$-plane.
\item Suppose $u(0,x)=-\tanh(x)$. For what values of $t>0$ does the solution of the quasilinear PDE remain smooth and single valued? Given an approximation sketch of the characteristics in the $tx$-plane.
\item Suppose
\[u(0,x)=\begin{cases}0,&x<0\\x,&0\le x<1,\\1,&1\le x\end{cases}.\]
Sketch the characteristics. Solve the Cauchy problem. Hint: solve the problem in each region separately and ``paste'' the solution together.
\end{cond}













\section{Statements in functional analysis and general topology}
Function analysis:
\begin{itemize}
\item Suppose a densely defined operator $T$ induces a Hilbert space structure on its domain. If the inclusion is bounded, then $T$ has the bounded inverse. If the inclusion is compact, then $T$ has the compact inverse.
\item A closed subspace of an incomplete inner product space may not have orthogonal complement: setting $L^2$ inner product on $C([0,1])$, define $\phi(f)=\int_0^{\frac12}f$.
\item Every seperable Banach space is linearly isomorphic and homeomorphic. But there are two non-isomorphic Banach spaces.
\item open mapping theorem -> continuous embedding is really an embedding.
\item $D(\Omega)$ is defined by a \emph{countable stict} inductive limit of $D_K(\Omega)$, $K\subset\Omega$ compact. Hence it is not metrizable by the Baire category theorem. (Here strict means that whenever $\alpha<\beta$ the induced topology by $\cT_\beta$ coincides with $\cT_\alpha$)
\item A net $(\phi_d)_d$ in $D(\Omega)$ converges if and only if there is a compact $K$ such that $\phi_d\in D_K(\Omega)$ for all $d$ and $\phi_d$ converges uniformly.
\item Th integration with a locally integrable function is a distribution. This kind of distribution is called \emph{regular}. The nonregular distribution such as $\delta$ is called \emph{singular}.
\item $D'$ is equipped with the weak$^*$ topology.
\item $\pd{x}\colon D'\to D'$ is continuous. They commute (Schwarz theorem holds).
\item $D\to S\to L^p$ are continuous (immersion) but not imply closed subspaces (embedding).
\end{itemize}
General topology:
\begin{itemize}
\item $H\subset\C$ and $H\subset\hat\C$ have distinct Cauchy structures which give a same topology. In addition, the latter is precompact while the former is not.
\end{itemize}








\section{Analysis problems}

\begin{prb}
The following series diverges: \[\sum_{n=1}^\infty\frac1{n^{1+|\sin n|}}.\]
\end{prb}
\begin{sol}
Let $A_k:=[1,2^k]\cap\{x:|\sin x|<\frac1k\}$.
Divide the unit circle $\R/2\pi\Z$ by $7k$ uniform arcs.
There are at least $2^k/7k$ integers that are not exceed $2^k$ and are in a same arc.
Let $S$ be the integers and $x_0$ be the smallest element.
Since, $|x-x_0|\pmod{2\pi}<\frac{2\pi}{7k}$ for $x\in S$,
\[|\sin(x-x_0)|<|x-x_0|\pmod{2\pi}<\frac{2\pi}{7k}<\frac1k.\]
Also, $1\le x-x_0\le x\le2^k$, $x-x_0\in A_k$.
\[|A_k|\ge\frac{2^k}{7k}.\]
Therefore,
\begin{align*}
\sum_{n=1}^\infty\frac1{n^{1+|\sin n|}}
&\ge\sum_{n\in A_N}\frac1{n^{1+|\sin n|}}\\
&\ge\sum_{k=1}^N(|A_k|-|A_{k-1}|)\frac1{2^{k+1}}\\
&=\sum_{k=1}^N\frac{|A_k|}{2^{k+1}}-\sum_{k=1}^{N-1}\frac{|A_k|}{2^{k+2}}\\
&=\frac{|A_N|}{2^{N+1}}+\sum_{k=1}^{N-1}\frac{|A_k|}{2^{k+2}}\\
&>\sum_{k=1}^N\frac{2^k}{2^{k+2}}\frac1{7k}\\
&=\frac1{28}\sum_{k=1}^N\frac1k\\
&\to\infty.
\end{align*}
\end{sol}

\clearpage
\begin{prb}
If $|xf'(x)|\le M$ and $\frac1x\int_0^xf(y)\,dy\to L$, then $f(x)\to L$ as $x\to\infty$.
\end{prb}
\begin{sol}
Since
\begin{align*}
\abs{f(x)-\frac{F(x)-F(a)}{x-a}}
&\le\frac1{x-a}\int_a^x|f(x)-f(y)|\,dy\\
&=\frac1{x-a}\int_a^x(x-y)|f'(c)|\,dy\\
&\le\frac M{x-a}\int_a^x\frac{x-y}c\,dy\\
&\le M\frac{x-a}a
\end{align*}
by the mean value theorem and 
\[f(x)-L=\left[f(x)-\frac{F(x)-F(a)}{x-a}\right]+\frac x{x-a}\left[\frac{F(x)}x-L\right]+\frac a{x-a}\left[\frac{F(a)}a-L\right],\]
we have for any $\e>0$
\[\limsup_{x\to\infty}|f(x)-L|\le\e\]
where $a$ is defined by $\frac{x-a}a=\frac\e M$.
\end{sol}

\clearpage
\begin{prb}
Let $f_n:I\to I$ be a sequence of real functions that satisfies $|f_n(x)-f_n(y)|\le|x-y|$ whenever $|x-y|\ge\frac1n$, where $I=[0,1]$.
Then, it has a uniformly convergent subsequence.
\end{prb}
\begin{sol}
By the Bolzano-Weierstrass theorem and the diagonal argument for subsequence extraction, we may assume that $f_n$ converges to a function $f:\Q\cap I\to I$ pointwisely.

\Step[1]
For $n\ge4$, we claim
\begin{align}|x-y|\le\frac1n\impl|f_n(x)-f_n(y)|\le\frac5n.\end{align}
Fix $x\in I$ and take $z\in I$ such that $|x-z|=\frac2n$ so that
\[|f_n(x)-f_n(z)|\le|x-z|=\frac2n.\]
If $y$ satisfies $|x-y|\le\frac1n$, then we have $|y-z|\ge|x-z|-|x-y|\ge\frac1n$, so we get
\[|f_n(y)-f_n(z)|\le|y-z|\le|y-x|+|x-z|\le\frac3n.\]
Combining these two inequalities proves what we want.

\Step[2]
For $\e>0$ and $N:=\ceil{\frac{15}\e}$ we claim
\begin{align}|x-y|\le\frac1N\quad\text{and}\quad n>N\impl|f_n(x)-f_n(y)|\le\frac\e3\end{align}
when $N\ge4$.
It is allowed for $|x-y|$ to have the following two cases:
\[|x-y|\le\frac1n\quad\text{or}\quad\frac1n<|x-y|\le\frac1N.\]
For the former case, by the inequality (1) we have
\[|f_n(x)-f_n(y)|\le\frac5n<\frac5N\le\frac\e3.\]
For the latter case, by the assumption at the beginning of the problem, we have
\[|f_n(x)-f_n(y)|\le|x-y|\le\frac1N\le\frac\e{15}.\]
Hence the claim is proved.

\Step[3]
We will prove $f$ is uniformly continuous.
For $\e>0$, take $\delta:=\frac1N$, where $N:=\ceil{\frac{15}\e}$.
We will show
\[|x-y|<\delta\impl|f(x)-f(y)|<\e\]
for $x,y\in\Q\cap I$ and $N\ge4$.
Fix rational numbers $x$ and $y$ in $I$ which satisfy $|x-y|<\delta$.
Since $f_n(x)$ and $f_n(y)$ converges to $f(x)$ and $f(y)$ respectively, we may take an integer $n_x$ and $n_y$, such that
\begin{align}n>n_x\impl |f_n(x)-f(x)|<\frac\e3\end{align}
and
\begin{align}n>n_y\impl |f_n(y)-f(y)|<\frac\e3.\end{align}
Choose an integer $n$ such that $n>\max\{n_x,n_y,N\}$.
Then, combining (3), (2), and (4), we obtain
\begin{align*}
|f(x)-f(y)|&\le|f(x)-f_n(x)|+|f_n(x)-f_n(y)|+|f_n(y)-f(y)|\\
&<\frac\e3+\frac\e3+\frac\e3=\e.
\end{align*}

Since $f$ is continuous on a dense subset $\Q\cap I$, it has a unique continuous extension on the whole $I$.
Let it denoted by the same notation $f$.

\Step[4]
Finally, we are going to show $f_n\to f$ uniformly.
For $\e>0$, let $N:=\ceil{\frac{15}\e}$.
The uniform continuity of $f$ allows to have $\delta>0$ such that
\begin{align}|x-y|<\delta\impl|f(x)-f(y)|<\frac23\e.\end{align}
Take a rational $r\in I$, depending on $x\in I$, such that $|x-r|<\min\{\frac1N,\delta\}$.
Then, by (2) and (5), given $n>N\ge4$, we have an inequality
\begin{align*}
|f_n(x)-f(x)|&\le|f_n(x)-f_n(r)|+|f_n(r)-f(r)|+|f(r)-f(x)|\\
&<\frac\e3+|f_n(r)-f(r)|+\frac23\e
\end{align*}
for any $x\in I$.
By limiting $n\to\oo$, we obtain
\[\lim_{n\to\oo}|f_n(x)-f(x)|<\e.\]
Since $\e$ and $x$ are arbitrary, we can deduce the uniform convergence of $f_n$ as $n\to\oo$.
\end{sol}


\clearpage
\begin{prb}
A measurable subset of $\R$ with positive measure contains an arbitrarily long subsequence of an arithmetic progression. (made by me!)
\end{prb}
\begin{sol}
Let $E\subset\R$ be measurable with $\mu(E)>0$. We may assume $E$ is bounded so that we have $E\subset I$ for a closed bounded interval since $\R$ is $\sigma$-compact.
Let $n$ be a positive integer arbitrarily taken. Then, we can find $N$ such that $\sum_{k=1}^N\frac1k>(n-1)\frac{\mu(I)}{\mu(E)}$.

Assume that every point $x$ in $E$ is contained in at most $n-1$ sets among
\[E,\ \frac12E,\ \frac13E,\ \cdots,\ \frac1NE.\]
In other words, it is equivalent to:
\[\bigcap_{k\in A}\frac1kE=\mt\]
for any subset $A\subset\{1,\cdots,N\}$ with $|A|\ge n$.
Define
\[E_A:=\bigcap_{k\in A}\frac1kE\cap\bigcap_{k'\in A}\left(\frac1{k'}E\right)^c\]
for $A\subset\{1,\cdots,N\}$.
Then, $\mu(E_A)=0$ for $|A|\ge n$.

Note that we have
\[\mu(\tfrac1kE)=\sum_{k\in A}\mu(E_A)=\sum_{\substack{k\in A\\|A|<n}}\mu(E_A).\]
Summing up, we get
\[\sum_{k=1}^N\mu(\tfrac1kE)=\sum_{k=1}^N\sum_{\substack{k\in A\\|A|<n}}\mu(E_A)=\sum_{|A|<n}|A|\mu(E_A)\]
by double counting,
and since $E_A$ are dijoint, we have
\[\sum_{|A|<n}|A|\mu(E_A)=(n-1)\sum_{0<|A|<n}\mu(E_A)\le(n-1)\mu(I),\]
hence a contradiction to
\[\sum_{k=1}^N\mu(\tfrac1kE)>(n-1)\mu(I).\]
Therefore, we may find an element $x$ that belongs to $\frac1kE$ for $k\in A$, where $A\subset\{1,\cdots,N\}$ with $|A|=n$.
Then, $ax\in E$ for all $a\in A\subset\Z$.

\end{sol}






\end{document}