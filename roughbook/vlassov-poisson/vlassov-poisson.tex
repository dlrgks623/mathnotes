\documentclass[11pt]{amsart}
\usepackage{ikany}
\usepackage[margin=1.5in]{geometry}

\title{Vlassov-Poisson system}
\author{Ikhan Choi}

\def\tint{{\textstyle\int}}
\def\loc{\mathrm{loc}}

\begin{document}
\maketitle
\tableofcontents


\section{Vlasov-Poisson equation}
Consider a Cauchy problem of the \emph{Valsov-Poisson system}:
\begin{pde*}
&f_t+v\cdot\del_xf+\gamma E\cdot\del_vf=0,\:(t,x,v)\in\R_t^+\x\R_x^3\x\R_v^3,\\
&E(t,x)=-\del_x\Phi\\
&\Phi(t,x)=(-\Delta_x)^{-1}\rho,\\
&\rho(t,x)=\tint f\,dv,\\
&f(0,x,v)=f_0(x,v),
\end{pde*}
where $\gamma=\pm1$ denotes the charge of particles we are concerned with.
For example, $\gamma=-1$ for electrons in plasma and $\gamma=+1$ for galaxies.
For the boundaryless problem in which the potential function vanishes at infinity, we have
\[E=-\del_x\Phi=-\del_x(-\tfrac1{4\pi|x|}*\rho)=-\frac{x}{4\pi|x|^3}*\rho\]
for $\gamma=-1$.
($\rho$ denotes mass density.)

\subsection{A priori estimates}

\begin{lem}
\[\|\rho(t)\|_{L_x^{5/3}}\les1.\]
\end{lem}
\begin{pf}
\begin{align*}
\rho(t,x)=\int f(t,x,v)\,dv
&\le\int_{|v|<R}f\,dv+\frac1{R^2}\int_{|v|\ge R}|v|^2f\,dv\\
&\les R^3+ R^{-2}\int|v|^2f\,dv.
\end{align*}
Set $R^3=R^{-2}\int|v|^2f\,dv$ to get
\[\rho(t,x)^{5/3}\les\int|v|^2f\,dv.\]

Take $d=3$, $p=2$, and $\lambda=2$.
Then, by the Hardy-Littlewood-Sobolev inequality,
\[\frac1p+1=\frac1q+\frac\lambda d\]
implies $q=6/5$ and we can bound the $L^2$-norm of the Riesz potential $\|E(t)\|_2$ by interpolation of $\|\rho(t)\|_{6/5}$ and $\|\rho(t)\|_1$:
\[\|E(t)\|_{L_x^2}\simeq\|\frac1{|x|^2}*_x\rho(t,x)\|_{L_x^2}\les\|\rho(t)\|_{6/5}\le\|\rho\|_1^{7/12}\|\rho\|_{5/3}^{5/12}.\]
Thus
\[\|E(t)\|_2\les\|\rho(t)\|_{5/3}^{5/12}\les(\iint|v|^2f\,dv\,dx)^{1/4}.\]
It means $(\iint|v|^2f\,dv\,dx)^{1/2}$ bounds $(\iint|v|^2f\,dv\,dx)$, hence the total kinetic energy of the system remains bounded in any time even if $\gamma=+1$.
As a corollary, $\|\rho\|_{5/3}$ is also bounded.
\end{pf}

\begin{lem}
For $1\le q<\frac N{N-2}=3<p\le\infty$,
\[\|E(t,x)\|_{L_x^\infty}\les\|\rho(t,x)\|_{L_x^p}^{\frac{\frac2N-1+\frac1q}{\frac1q-\frac1p}}\|\rho(t,x)\|_{L_x^q}^{\frac{1-\frac1p-\frac2N}{\frac1q-\frac1p}}.\]
\end{lem}
\begin{pf}
Fix time $t$.
For $2p<N<2q$,
\begin{align*}
4\pi|E(t,x)|
&=|\tfrac1{|x|^2}*_x\rho(t,x)|\\
&\le\int_{|x-y|<R}\frac{\rho(t,y)}{|x-y|^2}\,dy+\int_{|x-y|\ge R}\frac{\rho(t,y)}{|x-y|^2}\,dy\\
&\le\|\rho\|_{p'}(\int_{|y|<R}\frac{dy}{|y|^{2p}})^{1/p}+\|\rho\|_{q'}(\int_{|y|\ge R}\frac{dy}{|y|^{2q}})^{1/q}\\
&\simeq\|\rho\|_{p'}(\int_0^Rr^{N-1-2p}\,dr)^{1/p}+\|\rho\|_{q'}(\int_R^\infty r^{N-1-2q}\,dr)^{1/q}\\
&\simeq\|\rho\|_{p'}R^{\frac Np-2}+\|\rho\|_{q'}R^{\frac Nq-2}.
\end{align*}
By choosing $R$ such that $\|\rho\|_{p'}R^{\frac Np-2}=\|\rho\|_{q'}R^{\frac Nq-2}$, we get
\[\|E(t,x)\|_{L_x^\infty}\les\|\rho(t,x)\|_{L_x^{p'}}^{\frac{\frac2N-\frac1q}{\frac1p-\frac1q}}\|\rho(t,x)\|_{L_x^{q'}}^{\frac{\frac1p-\frac2N}{\frac1p-\frac1q}},\]
hence the inequality by interchaning $p$ and $q$ with their conjugates.
\end{pf}



\clearpage
\subsection{Schaeffer's global existence proof}

\begin{thm*}[Schaeffer, 1991]
Let $f_0\in C_{c,x,v}^1$ and $f_0\ge0$.
Then, the Cauchy problem for the VP system has a unique $C^1$ global solution.
\end{thm*}

\begin{defn}
For a local solution $f$,
\[Q(t):=1+\sup\{|v|:f(s,x,v)\ne0\text{ for some }s\in[0,t],\,x\in\R_x^3\}.\]
\end{defn}

Decompose $[t-\Delta,t]\x\R_x^3\x\R_v^3$ as
\begin{align*}
U&=\left\{\,(s,x,v):\ |v-\hat V(t)|\ge P,\quad|y-\hat X(s)|\ge r\,\right\},\\
B&=\left\{\,(s,x,v):\ |v-\hat V(t)|\ge P,\quad|v|\ge P\,\right\}\setminus U,\\
G&=\left\{\,(s,x,v):\ |v-\hat V(t)|<P\quad\text{or}\quad|v|<P\,\right\}.
\end{align*}
(We can let $U\mapsto U\cap\{|v|\ge P\}$ to make the decomposition disjoint.)
Later we choose
\[P=Q^{4/11},\quad r=R\max\{|v|^{-3},\,|v-\hat V(t)|^{-3}\},\quad R=Q^{16/33}\log^{1/2}Q.\]
Also, later we choose $\Delta\cdot\sup_{s\le t}\|E(s)\|_\infty<\frac P4$.

\subsubsection{Some observations}
Our goal is to obtain a priori estimate like
\[\|E(t)\|_\infty\les Q(t)^a\qquad\text{for some }a<1.\]
Since the force field $E$ measures the maximal rate of changes in velocity, the estimate can be read very roughly as
\[Q'(t)\les Q(t)^a,\]
which lead its polynomial growth.

So we need to bound the Riesz potential $E$.
The following observation suggests a lower bound of relative velocity.
\begin{clm}
Fix $t,x,v$.
If $|v-\hat V(t)|\ge P$, then
\[|y-\hat X(s)|\ge\frac14|v-\hat V(t)||s-s_0|\]
for some $s_0\in[t-\Delta,t]$, where $\Delta\cdot\sup_{s\le t}\|E(s)\|_\infty<\frac P4$.
\end{clm}
\begin{pf}
Since $\Delta\|E(s)\|_\infty<\frac P4$, we have
\[|v-w|<\frac P4\quad\text{and}\quad|\hat V(t)-\hat V(s)|<\frac P4.\]
The condition $|v-\hat V(t)|\ge P$ implies
\[\frac12|v-\hat V(t)|\le|v-\hat V(t)|-\frac P4-\frac P4<|w-\hat V(s)|.\]

Let $Z(s):=y-\hat X(s)$.
Then,
\begin{align*}
Z'(s)&=w-\hat V(s),\\
Z''(s)&=\gamma[E(s,y,w)-E(s,\hat X(s),\hat V(s))].
\end{align*}
Let $s_0\in[t-\Delta,t]$ minimize $s\mapsto|Z(s)|$ and expand $Z$ as
\[Z(s)=Z(s_0)+Z'(s_0)(s-s_0)+\frac{Z''(\sigma)}2(s-s_0)^2\]
for some $\sigma$ between $s$ and $s_0$.
Then,
\[|Z(s_0)+Z'(s_0)(s-s_0)|\ge|Z'(s_0)(s-s_0)|\ge\frac12|v-\hat V(t)||s-s_0|\]
and
\begin{align*}
|\frac{Z''(\sigma)}2(s-s_0)^2|
&\le\|E(t)\|_\infty(s-s_0)^2
\le\|E(t)\|_\infty\Delta|s-s_0|\\
&\le\frac P4|s-s_0|
\le\frac14|v-\hat V(t)||s-s_0|
\end{align*}
proves
\[|y-\hat X(s)|=|Z(s)|\ge\frac14|v-\hat V(t)||s-s_0|.\qedhere\]
\end{pf}

We introduce time averaging to use the above lower bound.
\begin{clm}
Fix $t,x,v$.
If $|v-\hat V(t)|\ge P$, then
\[\int_{t-\Delta}^t\frac1{|y-\hat X(s)|^2}\chi_A(s)\,ds\les\frac{r^{-1}}{|v-\hat V(t)|},\]
where $A=\{s:|y-\hat X(s)|\ge r\}$.
\end{clm}
\begin{pf}
Since $|y-\hat X(s)|\ge\frac14|v-\hat V(t)||s-s_0|$,
\begin{align*}
\int_{t-\Delta}^t\frac1{|y-\hat X(s)|^2}\chi_A(s)\,ds
&\le16\int_{t-\Delta}^t\frac1{|v-\hat V(t)|^2|s-s_0|^2}\chi_A(s)\,ds\\
&\le32\int_r^\infty\frac1{|v-\hat V(t)|^3|s-s_0|^2}\,d(|v-\hat V(t)||s-s_0|)\\
&=32\,\frac{r^{-1}}{|v-\hat V(t)|}.\qedhere
\end{align*}
\end{pf}

\subsubsection{Ugly set}

Therefore, if we let $r^{-1}\simeq\min\{|v|^3,|v-\hat V(t)|^3\}$, then
\[\int_{t-\Delta}^t\frac1{|y-\hat X(s)|^2}\chi_A(s)\,ds\les|v|^2\]
so that we have
\[\iiint_U\frac{f(s,y,w)}{|y-\hat X(s)|^2}\,dw\,dy\,ds\les R^{-1}\int|v|^2f(t,x,v)\,dv\,dx\les R^{-1}\]
when
\[U\subset\{\,(s,x,v):\ |v-\hat V(t)|\ge P,\quad|y-\hat X(s)|\ge R\max\{|v|^{-3},|v-\hat V(t)|^{-3}\}\,\}.\]

\subsubsection{Bad set}
Consider $U^c$.
We need to control the union of two regions
\[|y-\hat X(s)|<R|v|^{-3}\quad\text{and}\quad|y-\hat X(s)|<R|v-\hat V(t)|^{-3}.\]
Without any conditions, the integration of fundamental solution with respect to $y$ gives
\[\int_{|y-\hat X(s)|<r}\frac1{|y-\hat X(s)|^2}\,dy\simeq r.\]
\begin{clm}
If $|v|\ge P$ and $|v-\hat V(t)|\ge P$, then
\[\int_{U^c}\frac1{|y-\hat X(s)|^2}\,dy\les\max\{|w|^{-3},|w-\hat V(s)|^{-3}\}\]
for $s\in[t-\Delta,t]$.
\end{clm}
\begin{pf}
It follows from
\[|w|\simeq|v|,\quad|w-\hat V(s)|\simeq|v-\hat V(t)|\]
for $|v|\ge P$ and $|v-\hat V(t)|\ge P$.
\end{pf}

\subsubsection{Polynomial decay}
\begin{lem}
Along the time of existence we have
\[\|E(t)\|_{L_x^\infty}\les Q(t)^{4/3}.\]
\end{lem}
\begin{pf}
Note that we have
\[\|E\|_\infty\les\|\rho\|_\infty^{4/9}\|\rho\|_{5/3}^{5/9}.\]
Since the velocity support of $f$ is bounded by finite $Q(t)$,
\[\rho(t,x)=\int_{|v|<Q(t)}f(t,x,v)\,dv\les Q(t)^3\|f_0(x)\|_{L_v^\infty}\les Q(t)^3,\]
so
\[\|E(t)\|_{L_x^\infty}\les\|\rho(t)\|_{L_x^\infty}^{4/9}\les Q(t)^{4/3}.\qedhere\]
\end{pf}



\end{document}