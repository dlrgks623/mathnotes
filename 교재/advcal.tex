\documentclass[11pt]{exam}
\usepackage{../ikany}
\usepackage{kotex}

\printanswers
\linespread{1.6}
\renewcommand{\arraystretch}{0.625}

\everymath{\displaystyle}
\renewcommand\partlabel{(\arabic{partno})}
\renewcommand\labelenumi{(\roman{enumi})}
\renewcommand\partshook{\setlength\leftmargin{3em}\setlength\topsep{12pt}\setlength\itemsep{3pt}}
\renewcommand\questionshook{\setlength\listparindent{1em}\setlength\parsep{0pt}}

\unframedsolutions

\newcommand\Hint{\vspace{12pt}\noindent\large\textbf{Hint: }\par\noindent\normalsize}
\newcommand\Solution{\vspace{12pt}\noindent\large\textbf{Solution: }\par\noindent\normalsize}
\newcommand\Note{\vspace{12pt}\noindent\large\textbf{Note: }\par\noindent\normalsize}

\DeclareDocumentEnvironment{sol}{}{\begin{solution}}{\end{solution}}

\iffalse % 예문
\question 문제
\begin{solution}
\Hint
\Solution
\Note
\end{solution}
\fi




\begin{document}

\section{함수}
\begin{questions}

\subsection{다항함수}

% 1
\question 실수 $x$에 대하여 다음 함수 $f$의 최댓값과 최솟값을 구하여라:
\[f(x)=\frac{x^2-x+1}{x^2+x+1}.\]
\begin{sol}
우리가 구하고자 하는 $f$의 범위라는 것은 곧 $f$의 치역을 말한다.
함수 $f$의 치역이란 정의상 $f(x)=k$를 만족하는 정의역의 원소 $x$가 존재하는 $k$들의 집합이다.
그러므로 $(x^2-x+1)=k(x^2+x+1)$를 만족하는 \emph{실수} $x$가 존재하는 $k$를 구하면 된다.
이는 이차방정식의 판별식으로 해의 존재성을 판별하는 문제가 된다.
위의 이차방정식을 $x$에 대해 묶으면 $(k-1)x^2+(k+1)x+(k-1)=0$이고, 판별식 $(k+1)^2-4(k-1)(k-1)=-(3k-1)(k-3)$이 0 이상일 필요충분조건은 $\frac13\le k\le 3$인 것이다.
따라서 답은 최솟값 $\frac13$, 최댓값 3이 된다.

많은 학생들은 최댓값과 최솟값을 구하는 문제에서 기계적으로 미분을 하려는 경향이 있다.
하지만 의외로 미적분학의 복잡한 계산 없이 부등식 문제가 깔끔히 풀리는 경우가 있는데 그 중 한 가지로 \emph{판별식}을 사용하는 방법이 있다.
이 방법은 우선 이 문제처럼 분모분자가 이차식인 유리함수에서 유용하게 사용될 수 있고, 또한 코시 슈바르츠 부등식의 가장 표준적인 증명에 판별식이 사용되기도 한다.
\end{sol}





\subsection{지수함수}

\subsection{미분과 부등식}


\end{questions}
\section{수열}
\begin{questions}
\subsection{극한값}
\question a
\subsection{무한급수}

\subsection{스케일}


\end{questions}


\end{document}