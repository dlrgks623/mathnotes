\documentclass{../exp}
\usepackage{../../ikany}

\def\a{\alpha}
\title{Classical differential geometry}

\begin{document}
\maketitle

\section{Introduction}
\subsection{Two ways to represent curves or surfaces}

\subsection{Coordinates and parametrizations}
\begin{defn}
An \emph{$m$-dimensional parametrization} is a smooth map $U\to\R^n$ such that
\begin{cond}
\item $U\subset\R^m$ is open and connected,
\item $\a$ is one-to-one (optional),
\item $d\a$ is nondegenerate; $\{\pd_i\a\}_{i=1}^m$ is linearly independent.
\end{cond}
\end{defn}
The third condition is important; in language of manifolds, the third condition defines what we call \emph{immersed submanifolds}.
We will see that the second condition is not important at all.

\begin{defn}
A \emph{regular curve} is a subset of $\R^n$ that is the image of a one-dimensional parametrization.
\end{defn}
\begin{defn}
A \emph{regular surface} is a subset of $\R^n$ that is the image of a two-dimensional parametrization.
\end{defn}





\section{Curves in a space}
\subsection{Arc-length parameterization}
\begin{thm}
For every regular curve, there is a parametrization $\a$ such that $\|\a'\|=1$.
\end{thm}
\begin{pf}
Suppose we have a parametrization $\beta:I_t\to\R^d$.
Define $\tau:I_t\to I_s$ such that
\[\tau(t_0):=\int_0^{t_0}\|\beta'(t)\|\,dt.\]
Then, $s$ is a diffeomorphism.
Define $\a:I_s\to\R^d$ by $\a:=\beta\circ\tau^{-1}$.
Then, by the chain rule,
\[\a'=\dd{\a}{s}=\dd{\beta}{t}\dd{\tau^{-1}}{s}=\beta'\left(\dd{\tau}{t}\right)^{-1}=\frac{\beta'}{\|\beta'\|}.\qedhere\]
\end{pf}


\begin{defn}[Frenet-Serret frame]
Let $\a$ be a nondegenerate curve.
We define \emph{tangent unit vector}, \emph{normal unit vector}, \emph{binormal unit vector} by:
\[\bT:=\frac{\a'}{\norm{\a'}},\qquad\bN:=\frac{\bT'}{\|\bT'\|},\qquad\bB:=\bT\x\bN,\]
and \emph{curvature} and \emph{torsion} by:
\[\kappa:=\<\bT',\bN\>,\quad\tau:=-\<\bB',\bN\>.\]
Note that $\kappa$ cannot vanish by definition.
\end{defn}

\begin{thm}[Frenet-Serret formula]
Let $\a$ be a unit speed curve.
\[\begin{pmatrix}\bT'\\\bN'\\\bB'\end{pmatrix}=\begin{pmatrix}0&\kappa&0\\-\kappa&0&\tau\\0&-\tau&0\end{pmatrix}\begin{pmatrix}\bT\\\bN\\\bB\end{pmatrix}.\]
\end{thm}
\begin{pf}
\Step{1}[$\bT',\bB',\bN$ are parallel]
Two vectors $\bT'$ and $\bN$ are parallel by definition.
Since $\<\bT,\bB\>=0$ and $\<\bB,\bB\>=1$ are constant, we have
\[\<\bB',\bT\>=\<\bB,\bT\>'-\<\bB,\bT'\>=0,\qquad\<\bB',\bB\>=\tfrac12\<\bB,\bB\>'=0,\]
which show $\bB'$ and $\bN$ are parallel.
%%%
\end{pf}


\begin{thm}
Let $\a$ be a unit speed curve.
\begin{align*}
\a'&=\bT,\\
\a''&=\kappa\bN,\\
\a'''&=-\kappa^2\bT+\kappa'\bN+\kappa\tau\bB.
\end{align*}
\end{thm}

Skew-symmetricity is due to the fact the differential of an orthogonal matrix forms a skew symmetric matrix.

%%% Strategies !!!
\begin{itemize}
\item Aim for finding the coefficients of a special vector and its derivatives with respect to the Frenet-Serret frame.
\item In particular, differentiate equations of the form
\[\<v^{(k)},\bT\text{ or }\bN\text{ or }\bB\>=\const\]
to get useful information.
\item Heuristically find represent a vector and show what you want directly.
\end{itemize}



\begin{ex}
Let $\a$ be a curve in $\R^3$.
If the normal line of $\a$ always passes through a fixed point, then $\a$ is contained in a circle.
\end{ex}
\begin{pf}
\Step{1}[Formulate conditions]
Reparametrize $\a$ to become a unit speed curve.
By the assumption, there is a constant point $p\in\R^3$ such that the vectors $\a-p$ and $\bN$ are parallel so that we have
\[\<\a-p,\bT\>=0,\qquad\<\a-p,\bB\>=0.\]
Our goal is to show that $\|\a-p\|$ is constant and there is a constant vector $v$ such that $\<\a-p,v\>=0$.

\Step{2}[Collect information]
Differentiate $\<\a-p,\bT\>=0$ to get
\[\<\a-p,\bN\>=-\frac1\kappa.\]
%\[\<\a-p,\bN\>=\frac1\kappa\<\a-p,\bT'\>=\frac1\kappa(\<\a-p,\bT\>'-\<\a',\bT\>)=-\frac1\kappa.\]
Differentiate $\<\a-p,\bB\>=0$ to get
\[\tau=0.\]

\Step{3}[Complete proof]
We can deduce that $\|\a-p\|$ is constant from
\[(\|\a-p\|^2)'=\<\a-p,\a-p\>'=2\<\a-p,\bT\>=0.\]
Also, if we let $v=\bB$, then $\bB$ is constant since
\[v=-\tau\bN=0,\]
and $\<\a-p,v\>=0$
\end{pf}

\begin{ex}[Plane curves]
Let $\a$ be a nondegenerate curve in $\R^3$.
TFAE:
\begin{cond}
\item the curve $\alpha$ lies on a plane,
\item $\tau=0$,
\item the osculating plane constains a fixed point.
\end{cond}
\end{ex}


\begin{ex}[Helices]
Let $\a$ be a nondegenerate curve in $\R^3$.
TFAE:
\begin{cond}
\item the curve $\a$ is a helix,
\item $\tau/\kappa=\const$,
\item normal lines are parallel to a plane.
\end{cond}
\end{ex}

\begin{ex}[Sphere curves]
Let $\a$ be a nondegenerate curve in $\R^3$.
TFAE:
\begin{cond}
\item the curve $\a$ lies on a sphere,
\item $(1/\kappa)^2+((1/\kappa)'/\tau)^2=\const$,
\item $\tau/\kappa=(\kappa'/\tau\kappa^2)'$,
\item normal planes contain a fixed point.
\end{cond}
$^*$ A sphere curve of constant curvature lies in a circle.
\end{ex}

\begin{ex}[Bertrand mates]
Let $\a$ be a nondegenerate curve in $\R^3$.
TFAE:
\begin{cond}
\item the curve $\a$ has a Bertrand mate,
\item there are two constants $\lambda\ne0,\mu$ such that $1/\lambda=\kappa+\mu\tau$.
\end{cond}
$^*$ A curve is a circular helix iff it has more than one Bertrand mates.
\end{ex}



















\section{Surfaces in a space}
\[\nu_x=S(\alpha_x)=\kappa_1\alpha_x\]


\section{Curves on a surface}


\end{document}