\documentclass{../exp}
\usepackage{../../ikany}

\def\a{\alpha}
\title{Classical differential geometry}

\begin{document}
\maketitle

\section{Introduction}
\subsection{Two ways to represent curves or surfaces}

\subsection{Coordinates and parametrizations}
\begin{defn}
A \emph{parametrization} is a smooth map $U\to\R^d$ such that
\begin{cond}
\item $U\subset\R^c$ is open and connected,
\item $\a$ is one-to-one,
\item $d\a$ is nondegenerate; $\{\pd_i\a\}_{i=1}^c$ is linearly independent.
\end{cond}
\end{defn}

\begin{defn}
A \emph{regular curve} is a subset of $\R^d$ that is the image of some parametrization $\a:I\subset\R\to\R^d$.
\end{defn}
\begin{defn}
A \emph{regular surface} is a subset of $\R^d$ that is the image of some parametrization $\a:U\subset\R^2\to\R^d$.
\end{defn}





\section{Curves in a space}
\subsection{Arc-length parameterization}
\begin{thm}
For every regular curve, there is a parametrization $\a$ such that $\|\a'\|=1$.
\end{thm}
\begin{pf}
Suppose we have a parametrization $\beta:I_t\to\R^d$.
Define $\tau:I_t\to I_s$ such that
\[\tau(t_0):=\int_0^{t_0}\|\beta'(t)\|\,dt.\]
Then, $s$ is a diffeomorphism.
Define $\a:I_s\to\R^d$ by $\a:=\beta\circ\tau^{-1}$.
Then, by the chain rule,
\[\a'=\dd{\a}{s}=\dd{\beta}{t}\dd{\tau^{-1}}{s}=\beta'\left(\dd{\tau}{t}\right)^{-1}=\frac{\beta'}{\|\beta'\|}.\qedhere\]
\end{pf}


\begin{defn}[Frenet-Serret frame]
Let $\a$ be a curve such that $\kappa\ne0$.
Define \emph{tangent unit vector}, \emph{normal unit vector}, \emph{binormal unit vector} by:
\[\bT:=\frac{\a'}{\norm{\a'}},\qquad\bN:=\frac{\bT'}{\|\bT'\|},\qquad\bB:=\bT\x\bN.\]
\end{defn}
\begin{defn}
\[\kappa:=\bT'\cdot\bN,\quad\tau:=-\bB'\cdot\bN.\]
\end{defn}

\begin{thm}[Frenet-Serret formula]
Let $\a$ be a unit speed curve.
\[\begin{pmatrix}\bT'\\\bN'\\\bB'\end{pmatrix}=\begin{pmatrix}0&\kappa&0\\-\kappa&0&\tau\\0&-\tau&0\end{pmatrix}\begin{pmatrix}\bT\\\bN\\\bB\end{pmatrix}.\]
\end{thm}
\begin{pf}
The vectors $\bT',\bB',\bN$ are collinear.
\end{pf}


\begin{thm}
Let $\a$ be a unit speed curve.
\begin{align*}
\a'&=\bT\\
\a''&=\kappa\bN\\
\a'''&=-\kappa^2\bT+\kappa'\bN+\kappa\tau\bB
\end{align*}
\end{thm}

Skew-symmetricity is due to the fact the differential of an orthogonal matrix forms a skew symmetric matrix.

\begin{ex}
Let $\a$ be a curve in $\R^3$.
If the normal line always passes through a point, then $\a$ is contained in a circle.
\end{ex}
\begin{pf}
Let $\a$ be a unit speed curve.
By the assumption, there is a constant point $p\in\R^3$ such that the vectors $\a-p$ and $\bN$ are parallel so that we have
\[\<\a-p,\bT\>=0,\qquad\<\a-p,\bB\>=0.\]
Our goal is that $\|\a-p\|$ is constant and there is a constant vector $v$ such that $\<\a-p,v\>=0$.

\[0=\<\a-p,\bT\>'=\<\a',\bT\>+\<\a-p,\kappa\bN\>=1+\kappa\<\a-p,\bN\>.\]

\[0=\<\a-p,\bB\>'=\<\a-p,-\tau\bN\>=-\tau\cdot(-\frac1{\kappa})\]

\begin{align*}
(\|\a-p\|^2)'&=\<\a-p,\a-p\>'\\
&=2\<\a-p,\a'\>\\
&=2\<\a-p,\bT\>\\
&=0
\end{align*}


\begin{align*}
\bB'&=-\tau\bN=0.
\end{align*}
\end{pf}



\section{Surfaces in a space}
\[\nu_x=S(\alpha_x)=\kappa_1\alpha_x\]


\section{Curves on a surface}


\end{document}