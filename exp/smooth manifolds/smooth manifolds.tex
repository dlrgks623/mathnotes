\documentclass{../exp}
\usepackage{../../ikany}

\title{Smooth Manifolds}

\begin{document}
\maketitle
\tableofcontents

\section{Manifolds}
\subsection{Smooth structure}


\subsection{Partition of unity}
We prove the existence of partition of unity and exhaustion function.








\section{Tangent bundle}

\subsection{Definition of tangent space}
We will provide four different definitions of tangent spaces:
\begin{cond}
\item the space of equivalence classes of smooth curves,
\item the space of tangent vectors embedded in an ambient space,
\item the space of derivations on the ring of smooth functions,
\item the dual space of algebraically defined cotangent spaces.
\end{cond}


\subsection{Definition of tangent bundle}
Consider the disjoint union of tangent spaces where the base points run through the whole manifold.
This subsection discusses two ways of giving topology on the disjoint union to make it a vector bundle: one uses the local trivializations and coefficients to make a map to a Euclidean space, and the other inherits the topology of the ambient space.
Both topologies are so sufficiently smooth that we can settle a smooth structure, which are identical.

The latter case is essentially same with the topologize pullback bundle with respect to the embedding $M\to\R^A$ for some $A$.
We need to show it does not depend on the choice of the index set $A$.
(We have checked that there is a natural choice of $A=C^\oo(M)$ with embedding $i:M\to\R^{C^\oo(M)}$ such that $\pi_f(i(x))=f(x)$.)










\section{Categorical aspects}

\subsection{Immersion, submersion, and embedding}


\subsection{Basic differential topology}
We are going to summary the basic tools to investigate the nature of smooth manifolds.
These includes the inverse function theorem, the implicit function theorem, the slice lemma, the constant rank theorem.

\subsection{Category of smooth manifolds}


\subsection{Tangent bundle functor}

\begin{thm}
The tangent bundle functor $T$ preserves finitary products
\end{thm}


\subsection{Pullbacks}
When is the pullback possible?








\section{Sheaf theoretical aspects}
Consider a (commutative unital) ring $A$ such that every residue field is isomorphic to a field $k$.
Familiar examples include any Banach algebras by the Gelfand-Mazur theorem.
The unital condition is attached because we want to treat maximal ideals.
Then, maximal ideals correspond to a nonzero multipicative linear functional to $k$ because the residue field is $k$.
Therefore, the set of maximal ideals can be identified with the set of all nonzero multiplicative linear functionals.

\iffalse
in general, the following implications hold:
mul functional -> max ideal:   the converse holds if every residue field is isomorphic (e.g. Gelfand-Mazur thm)
point -> mul functional:       the converse holds if .... case by case
\fi


\subsection{The ring $C^\oo(M)$}
The following theorem is presented as the problem 1-C in the book of Milnor and Stasheff about characteristic classes.
\begin{thm}
Every ring homomorphism $C^\oo(M)\to\R$ to $M$ is obtained by an evaluation at a point of $M$.
\end{thm}
\begin{pf}
Suppose $\phi:C^\oo(M)\to\R$ is not an evaluation.
Let $h$ be a positive exhaustion function.
Take a compact set $K:=h^{-1}([0,\phi(h)])$.
For every $p\in K$, we can find $f_p\in C^\oo(M)$ such that $\phi(f_p)\ne f_p(p)$ by the assumption.
Summing $(f_p-\phi(f_p))^2$ finitely on $K$ and applying the extreme value theorem, we obtain a function $f\in C^\oo(M)$ such that $f\ge0$, $f|_K>1$, and $\phi(f)=0$.
Then, the function $h+\phi(h)f-\phi(h)$ is in kernel of $\phi$ although it is strictly positive and thereby a unit.
It is a contradiction.
\end{pf}











\end{document}