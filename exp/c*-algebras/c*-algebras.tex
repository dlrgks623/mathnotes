\documentclass{../exp}
\usepackage{../../ikany}

\title{$C^*$-algebras}

\begin{document}
\maketitle
\tableofcontents
\begin{defn*}
In this note, an \emph{algbera} refers to a vector space over $\C$ that has a pseudo-ring structure; always associative but possibly nonunital.
\end{defn*}
\begin{defn*}
An normed $*$-algebra $\cA$ is called \emph{$C^*$-algebra} if
\begin{cond}
\item $\cA$ is Banach,
\item $\cA$ satisfies the $C^*$-identity: $\|x^*x\|=\|x\|^2$.
\end{cond}
\end{defn*}


\begin{thm*}
Every nonunital $C^*$-algebra is a maximal ideal of a unital $C^*$-algebra.
\end{thm*}
\begin{pf}
Let $\cA$ be a nonunital $C^*$-algebra.
It is enough to show the existence of unital $C^*$-algebra $\tld\cA$ such that $\cA$ is a normed $*$-subalgebra of $\tld\cA$ with codimension one.
It is because a subalgebra is a maximal ideal if and only if the quotient can have a natural ring structure that makes a field.

\Step{1}[Construct a unital normed $*$-algebra]
Since $\cA$ is a Banach space, the space of bounded operators $B(\cA)$ is a Banach algebra.
We can recognize $\cA$ as a normed subalgebra of $B(\cA)$ because the left multiplication $(y\mapsto xy)\in B(\cA)$ has the norm
\[\|(y\mapsto xy)\|=\sup_{y\in\cA}\frac{\|xy\|}{\|y\|}\]
that is shown to be equal to $\|x\|$ by putting $y=x^*$ and applying the \emph{$C^*$-identity}.
Define an algebra $\tld\cA$ as the subalgebra:
\[\tld\cA:=\{\,(y\mapsto xy+\lambda y)\in B(\cA):x\in\cA,\,\lambda\in\C\,\}.\]
Since $\tld\cA\cong\cA\oplus\C$ as algebras, let us write the map $y\mapsto xy+\lambda y$ as $(x,\lambda)$.
Then, $\tld\cA$ is a normed $*$-algebra with induced norm and involution
\[\|(x,\lambda)\|=\sup_{y\in\cA}\frac{\|xy+\lambda y\|}{\|y\|},\qquad(x,\lambda)^*=(x^*,\cl{\lambda}).\]
Then, $\cA$ is a normed $*$-subalgebra of $\tld\cA$ because the norm and involution of $\cA$ agree with $\tld\cA$.

\Step{2}[$\tld\cA$ is Banach]
Suppose $(x_n,\lambda_n)$ is Cauchy in $\tld\cA$.
Since $\cA$ is complete so that it is closed in $\tld\cA$, we can induce a norm on the quotient $\tld\cA/\cA$ so that the canonical projection is (uniformly) continuous so that $\lambda_n$ is Cauchy.
Also, the inequality $\|x\|\le\|(x,\lambda)\|+|\lambda|$ shows that $x_n$ is Cauchy in $\cA$.

Since a finite dimensional normed space is always Banach and $\cA$ is Banach, $\lambda_n$ and $x_n$ converge.
Finally, the inequality $\|(x,\lambda)\|\le\|x\|+|\lambda|$ implies that $(x_n,\lambda_n)$ converges.

\Step{3}[$\tld\cA$ is $C^*$]
The $C^*$-identity easily follows from the following inequality:
\begin{align*}
\|(x,\lambda)\|^2&=\sup_{\|y\|=1}\|xy+\lambda y\|^2\\
&=\sup_{\|y\|=1}\|(xy+\lambda y)^*(xy+\lambda y)\|\\
&=\sup_{\|y\|=1}\|y^*((x^*x+\lambda x^*+\cl\lambda x)y+|\lambda|^2y)\|\\
&\le\sup_{\|y\|=1}\|(x^*x+\lambda x^*+\cl\lambda x)y+|\lambda|^2y\|\\
&=\|(x,\lambda)^*(x,\lambda)\|.\qedhere
\end{align*}
\end{pf}


\section{Basics}
\subsection{Continuous functional calculus}
\begin{thm}[Gelfand-Naimark]
For commutative unital $C^*$-algebra $\cA$, the Gelfand transform gives an isometric $^*$-isomorphism $\Gamma:\cA\to C(\sigma(\cA))$.
\end{thm}
\begin{pf}
\Step{1}[$\Gamma$ is a $*$-homomorphism]
We will show $h(x^*)=\cl{h(x)}$ for linear characters $h\in\sigma(\cA)$.
First assume that $x\in\cA$ is self-adjoint.

By the holomorphic functional calculus,
\[e^{itx}=\sum_{n=1}^\infty\frac{(itx)^n}{n!}.\]
Since the involution is continuous,
\[(e^{itx})^*=\sum_{n=1}^\infty\frac{(-itx)^n}{n!}=e^{-itx},\]
so we have $\|e^{itx}\|^2=\|e^{itx}e^{-itx}\|=1$.
Then, the inequality
\[1=\|e^{itx}\|\ge|h(e^{itx})|=|e^{ith(x)}|=e^{-t\Im h(x)}\]
proves $h(x)\in\R$.

For arbitrary $x\in\cA$, if we define self-adjoints
\[\Re x:=\frac{x+x^*}2,\qquad\Im x:=\frac{x-x^*}{2i},\]
then
\[h(x^*)=h(\Re x)-ih(\Im x)=\cl{h(\Re x)}-i\cl{h(\Im x)}=\cl{h(\Re x)+ih(\Im x)}=\cl{h(x)}\]
for all $h\in\sigma(\cA)$.

\Step{2}[$\Gamma$ is isometric]
Note that we have
\[\|\hat x\|=\sup_{h\in\sigma(\cA)}|\hat x(h)|=\sup_{h\in\sigma(\cA)}|h(x)|=r(x).\]
For self adjoint $x\in\cA$, since we have $\|x\|^2=\|x^*x\|=\|x^2\|$, the spectral radius coincides with the norm by the Gelfand formula for spectral radius in Banach algebras:
\[r(x)=\lim_{n\to\infty}\|x^{2^n}\|^{1/2^n}=\|x\|.\]
Hence
\[\|x\|^2=\|x^*x\|=\|\hat{x^*x}\|=\|\hat{x}^*\hat{x}\|=\|\hat{x}\|\]
for arbitrary $x\in\cA$.

\Step{3}[$\Gamma$ is surjective]
The step 1 shows that $\Gamma(\cA)$ is a unital $*$-subalgebra of $C(\sigma(\cA))$, and it separates points by definition.
By the Stone-Weierstrass theorem, $\Gamma(\cA)$ is dense in $C(\sigma(\cA))$.
The step 2 shows that $\Gamma(\cA)$ is complete and hence closed so that $\Gamma(\cA)=C(\sigma(\cA)$.
\end{pf}
\begin{thm}[Gelfan-Naimark]
For commutative $C^*$-algebra $\cA$, the Gelfand transform gives an isometric $^*$-isomorphism $\Gamma:\cA\to C_0(\sigma(\cA))$.
\end{thm}


\subsection{Positive elements}
a


\end{document}