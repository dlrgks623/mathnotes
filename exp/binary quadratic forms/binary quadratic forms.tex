\documentclass{../exp}
\usepackage{../../ikany}

\title{Binary Quadratic Forms}

\begin{document}
\maketitle

\section{Equivalence}


\begin{defn}
Two forms are called \emph{equivalent} if they are in a same oribit with respect to $\GL_2(\Z)$-action.
\end{defn}

\begin{defn}
Two forms are called \emph{properly equivalent} if they are in a same oribit with respect to $\SL_2(\Z)$-action.
\end{defn}

For representation problems, $\GL_2(\Z)$-action is important.
For the correspondence with the theory of qudratic fields, $\SL_2(\Z)$ is rather important.
From now, all equivalence relations are by $\SL_2(\Z)$.

\begin{ex}
Two forms $(a,b,c)$ and $(a,-b,c)$ are equivalent but not properly equivalent.
\end{ex}
The term ``reduced'' means that it is considered as the unique representative of each orbit.
The thing is that the group action may differ in the context.

\section{Definite forms}

\begin{prop}
The $\SL_2(\Z)$-action on the definite forms is not faithful, i.e. the kernel is given by a nontrivial group $\{\pm I\}$.
\end{prop}
\begin{prop}
The $\PSL_2(\Z)$-action on the definite forms is faithful.
\end{prop}
The faithfulness is not important though, so we choose $\Gamma=\SL_2(\Z)$ as the modular group instead of $\PSL_2(\Z)$.

\subsection{Positive definite forms}
\begin{prop}
The set of positive definite forms admits the $\SL_2(\Z)$-action.
\end{prop}
\begin{prop}
The $\SL_2(\Z)$-actions on positive definite forms and negative definite forms are isomorphic.
\end{prop}



\section{Indefinite forms}


\section{Class group}



\end{document}