\documentclass{../exp}
\usepackage{../../ikany}
\title{Galois Theory}

\begin{document}
\maketitle
\tableofcontents








\section{Basic field theory}
\begin{defn}
A \emph{field extension} is a pair of fields $(E,F)$ such that $E>F$, i.e. $F$ belongs to $E$ as a subset.
We often denote it by $E/F$.
\end{defn}
A field $E$ is also called a \emph{field extension} or a\emph{superfield} of another field $F$ if $E>F$.
Notice that we may write $E>F$ even if $E=F$.
Our goal is to understand field extensions.

We review three basic topics on field theory.
One is about vector space structure on field extension.
The degree, defined by the dimension of the vector space, measures the size of field extension.

Next, an important property of field homomorphism that general ring homomorpgisms do not have will be investigated.
This can be one of main reasons why field theory has its own unique features that are quite far from commutative ring theory.

Finally, some languages about simple extensions will be studied.
Because of its simplicity, simple extensions are often used to construct examples or to make computations.

\subsection{Vector space over a subfield}

\begin{thm}
Let $E/F$ be a field extension.
Then, $E$ is a vector space over $F$.
\end{thm}
\begin{pf}
Obvious.
\end{pf}

\begin{defn}
A \emph{degree} of a field extension $E/F$ is the dimension of the vector space $E$ over $F$ and denoted by $[E:F]$.
\end{defn}

\begin{defn}
A field extension is called \emph{finite} if its degree is finite.
\end{defn}

\begin{thm}[Multiplicity of degree]
If $K$ is an intermediate field in a field extension $E/F$, then
\[[E:F]=[E:K][K:F].\]
\end{thm}
\begin{pf}
Boring basis counting.
\end{pf}

\begin{cor}
Finite extension of finite extension is finite.
\end{cor}




\subsection{Field embeddings}

Unlike general rings, field homomorphisms is extremely rigid.
The following theorem deeply related to Schur's lemma in representation theory, and it holds for not only fields but also division rings.

\begin{prop}
A nontrivial field homomorphism is injective.
\end{prop}
\begin{pf}
The kernel should be either entire or zero.
\end{pf}
A nontrivial field homomorphism is called \emph{embedding} or \emph{isomorphism onto a subfield}.
A field isomorphism is just a surjective field embedding.



\subsection{Simple extension}
Simple extension is a field extension by an element.
It is very useful when we consider where specific element goes to through a given field homomorphism.

\begin{defn}
A field extension $E/F$ is called \emph{simple} if there is an element $\alpha\in E$ such that $E$ is the smallest field containing both $\alpha$ and $F$.
In this case, we write $E=F(\alpha)$.
\end{defn}

\begin{lem}
Let $E/F$ be a finite extension.
There is a finite tower of finite simple extensions.
\end{lem}

Although it is hard to find a counterexample, there is a finite extension which is not simple.
We will see in Section 3.



\subsection{Characteristic of fields}

% fixing a field










\section{Algebraic extension}

We do not prove the basic properties of polynomial ring over a field: they satisfy the axioms of ED, PID, and UFD, so every prime ideal is maximal and every irreducible element is prime.

\subsection{Algebraic elements}
Finite simple extensions are the most basic examples of the field extensions that we should become perfectly familiar with them.
An element that generates a finite simple extension field is called algebraic.
For these elements, we can define minimal polynomials and conjugates for algebraic elements.
The minimal polynomial is an essential tool to compute basic information of a given finite simple extension.
Conjugates are for useful when we construct a map between finite simple field extensions.

\subsubsection{Minimal polynomial}
Let us get started from the minimal polynomials.

\begin{defn}[Algebraic element]
Let $E/F$ be a field extension.
An element $\alpha\in E$ is \emph{algebraic over $F$} if the simple extension $F(\alpha)/F$ is finite.
If $\alpha$ is not algebraic over $F$, we call it \emph{transcendental} over $F$.
\end{defn}

We give some equivalent conditions for algebraicity.
\begin{thm}
Let $E/F$ be a field extension and $\alpha\in E$.
TFAE:
\begin{cond}
\item $\alpha$ is algebraic over $F$,
\item there is a nonzero polynomial $f\in F[x]$ such that $f(\alpha)=0$, in other words, an ideal given by the kernel of ring homomorphism
\[\eval_\alpha:F[x]\to F[\alpha]:f(x)\mapsto f(\alpha)\]
is nonempty,
\item $F(\alpha)=F[\alpha]$, i.e. $F[\alpha]$ is a field.
\end{cond}
\end{thm}
\begin{pf}
(1)$\impl$(2).
Since $d=[F(\alpha):F]<\oo$, we can find a linearly dependent finite subset of infinite set $\{1,\alpha,\alpha^2,\cdots\}\subset F(\alpha)$ over $F$.
The coefficients on the linear dependency relation construct the polynomial.

(2)$\impl$(3).
The kernel of $\eval_\alpha$ is a prime ideal because the quotient $F[x]/\ker(\eval_\alpha)\cong\im(\eval_\alpha)=F[\alpha]$ is an integral domain.
It is also maximal since $F[x]$ is a PID(Krull dimension 1).
Therefore, the quotient $F[\alpha]$ is a field.

(3)$\impl$(2).
There is $g\in F[x]$ such that $\alpha^{-1}=g(\alpha)$.
Then, $f\in F[x]$ defined by $f(x)=xg(x)-1$ satisfies $f(\alpha)=0$.

(2)$+$(3)$\impl$(1).
If there is $f\in F[x]$ with $f(\alpha)=0$, then we can show every element $g(\alpha)$ of $F(\alpha)=F[\alpha]$ for some $g\in F[x]$ is represented as a linear combination of $\{1,\alpha,\cdots,\alpha^{\deg f-1}\}$ by the Euclidean algorithm; divide $g$ by $f$.
Therefore, a finite set spans $F(\alpha)$, so the dimension $F(\alpha)$ over $F$ is finite.
\end{pf}

Note that, due to the fact that $F[x]$ is a PID, there exists a one-to-one correspondence:
\begin{rd}
\textbf{M}aximal \textbf{I}deal in $F[x]$ \lds{rr}{generator}&\quad& \textbf{M}onic \textbf{I}rreducible polynomial in $F[x]$. \lds{ll}{principal ideal}
\end{rd}
Since the ideal $\ker(\eval_\alpha)\subset F[x]$ for algebraic $\alpha$ is maximal, the following definition makes sense:

\begin{defn}[Minimal polynomial]
Let $E/F$ be a field extension and $\alpha\in E$ is algebraic.
The unique monic irreducible polynomial $\mu_{\alpha,F}\in F[x]$ satisfying 
\[\mu_{\alpha,F}(\alpha)=0\]
is called the \emph{minimal polynomial of $\alpha$ over $F$}.
\end{defn}

The following theorem says that we can compute the degree of a finite simple extension via finding the minimal polynomial.
\begin{thm}
Let $E/F$ be a field extension and $\alpha\in E$ is algebraic.
Then,
\[F(\alpha)\cong F[x]/(\mu_{\alpha,F}).\]
In particular, $[F(\alpha):F]=\deg\mu_{\alpha,F}$.
\end{thm}
\begin{pf}
The kernel of $\eval_\alpha:F[x]\to F(\alpha)$ is characterized as the principal ideal generated by $\mu_{\alpha,F}$, so we find the isomorphism $F[x]/(\mu_{\alpha,F})\cong F(\alpha)$.

Now we claim the dimension of $F[x]/(f)$ over $F$ is the degree of $f\in F[x]$.
It is enough to show $\{1,x,\cdots,x^{d-1}\}$ is a basis where $d=\deg f$.
We can check this with the Euclidean algorithm.
\end{pf}


\begin{ex}
Let the base field is $\Q$.
The minimal polynomial of $\sqrt2$ is $x^2-2$ since it is monic irreducible and has a root $\sqrt2$.
Similarly, the minimal polynomial of $\frac{-1+\sqrt{-3}}2$ is $x^2+x+1$.
\end{ex}
\begin{ex}
The minimal polynomial $\sqrt2+\sqrt3$ over $\Q$ is $x^4-10x^2+1$.
Therefore, $[\Q(\sqrt2+\sqrt3):\Q]=4$.

On the other hand, recall that we have
\[\left[\Q(\sqrt2,\sqrt3):\Q\right]=\left[\Q(\sqrt2,\sqrt3):\Q(\sqrt2)\right]\cdot\left[\Q(\sqrt2):\Q\right]=2\cdot2=4.\]
Also, $\sqrt2+\sqrt3\in\Q(\sqrt2,\sqrt3)$ implies $\Q(\sqrt2+\sqrt3)<\Q(\sqrt2,\sqrt3)$.
Since the dimensions as vector spaces are equal, we get $\Q(\sqrt2+\sqrt3)=\Q(\sqrt2,\sqrt3)$.

Actually, we have
\[\sqrt2=\frac12\left(\alpha-\frac1\alpha\right)\quad\text{and}\quad\sqrt3=\frac12\left(\alpha+\frac1\alpha\right),\]
where $\alpha=\sqrt2+\sqrt3$.

These kind of \emph{dimension argument} is one of powerful tools to attack field theory.
It will be discovered later that the dimension argument has an analogy with computation of group orders in finite group theory.
\end{ex}

\begin{ex}
The base field is important: we have
\[\mu_{\sqrt2,\Q}(x)=x^2-2,\c{but} \mu_{\sqrt2,\Q(\sqrt2)}=x-\sqrt2.\]
\end{ex}

\begin{ex}
Although $\Q(\sqrt2)=\Q(1+\sqrt2)$, the minimal polynomial of each generator over $\Q$ are $x^2-2$ and $(x-1)^2-2$ respectively.

Polynomials are usually used in order to be provided as a computational tool, so we frequently want to find a suitable minimal polynomial for a given field extension.
However, note that a finite simple extension does not specify only one minimal polynomial as the above example.
It is enough to find a suitable minimal polynomial that is easy to compute.
\end{ex}

\subsubsection{Conjugates}
Now, we begin to define conjugates.

\begin{defn}
Let $E/F$ be a field extension and $\alpha,\beta\in E$ be algebraic over $F$
They are said to be \emph{conjugate over $F$} if they share a common minimal polynomial over $F$.
\end{defn}

In other words, conjugates share the maximal ideal $\ker(\eval)$, hence we get that $F(\alpha)$ and $F(\beta)$ are isomorphic.
For the practical isomorphism map, we have the following great theorem.

\begin{thm}[Conjugation isomorphism]
Let $E/F$ be a field extension and $\alpha,\beta\in E$.
A map
\[\phi:F(\alpha)\to F(\beta):\alpha\mapsto\beta\]
fixing $F$ provides a well-defined field isomomorphism iff they are conjugates over $F$.
\end{thm}
\begin{pf}
($\Leftarrow$)
To prove well-definedness, since the two conditions that $\phi(\alpha)=\beta$ and $\phi$ fixes $F$ determines $\phi$ uniquely, so we just need to show the existence of such $\phi$.

Let $\mu\in F[x]$ be the common minimal polynomial of $\alpha$ and $\beta$ over $F$.
We will show that a map
\[\psi:F(\alpha)\to{\sim}F[x]/(\mu)\to{\sim}F(\beta)\]
is in fact $\phi$.
The intermediate isomorphisms are defined by the quotient of evaluation maps.
Since we have $\psi(\alpha)=\beta$ and $\psi$ fixes $F$ clearly, $\phi$ is well-defined.
The fact that $\phi$ is a field isomorphism is followed by $\psi$.

($\Rightarrow$)
Suppose $\phi$ is an isomorphism fixing $F$.
Then, $\phi$ commutes with a polynomial function with coefficients in $F$.
From
\[\mu_{\alpha,F}(\beta)=\mu_{\alpha,F}(\phi(\alpha))=\phi(\mu_{\alpha,F}(\alpha))=\phi(0)=0,\]
we get $\mu_{\beta,F}\mid\mu_{\alpha,F}$.
The irreducibility of $\mu_{\alpha,F}$ implies $\mu_{\alpha,F}=\mu_{\beta,F}$.
\end{pf}
\begin{cor}
Let $\phi:F\to F$ is a field automorphism.
Then, $\alpha$ and $\phi(\alpha)$ are always conjugates.
\end{cor}

In the following examples, it would be helpful to consider the case that $E=\C$ and $F=\Q$.
\begin{ex}
The base fields are important.
There are two conjugates of $\sqrt2$ over $\Q$: $\pm\sqrt2$.
However, there is only one conjugate of $\sqrt2$ over $\Q(\sqrt2)$ or $\C$: itself.
\end{ex}
\begin{ex}
There are two conjugates of $\omega:=\frac{-1+\sqrt{-3}}2$ over $\Q$: $\omega$ and $\cl\omega$.
It means that there are only two automorphisms on $\Q(\omega)$: one is identity, the other is complex conjugation.
\end{ex}
\begin{ex}
Two different conjugates can define the same isomomorphism.
See Section 3.
\end{ex}
\begin{ex}
The isomorphism does not have to be an automorphism.
There are four conjugates of $\sqrt[4]2$ over $\Q$: $\pm\sqrt[4]2$ and $\pm i\sqrt[4]2$.
However, $\Q(\sqrt[4]2)\ne\Q(i\sqrt[4]2)$ even though they are isomorphic.
See Section 4.
\end{ex}



\subsection{Algebraic extension}

Algebraic extension is a generalization of finite extension, for instance, every finite extension
In Galois theory, which will be studied later, we will not care elements that are not algebriac.
Therefore, it is natural to think of a field extension that only consists of algebraic elements, which is called also algebraic.
The main interests in Galois theory will be restricted to algebraic extensions.
To people who know the category theory, an algebraic extension is just a direct limit of finite simple extensions.

\begin{defn}
A field extension $E/F$ is called \emph{algebraic} if all elements $\alpha\in E$ are algebraic over $F$.
\end{defn}

The easiest example of an algebraic extension is a finite extension.
The relations between finite extensions and algebraic extension are as follows.

\begin{thm}
For finite extensions and algebraic extensions, we have:
\begin{cond}
\item a finite extension is algebraic,
\item a simple algebraic extension is finite.
\end{cond}
\end{thm}
\begin{pf} Easy. \end{pf}


Now, we are going to get some basic criteria for determining or constructing algebraic extensions.
If summarized, we can just say any operations of algebraic extensions are algebraic.
Before that, we introduce a good notion about algebraic extensions, and it is just the set of all algebraic elements in a given field.

In the rest of this subsection, assume that we have fixed a sufficiently large field $L$.
Restricting the ``domain of discourse'' by assuming a large entire field is a greatly helpful idea in order not to be confused with the theory of extensions, which might be able to grow without limits.
Note that the following definition depends on choice of $L$.
\begin{defn}
The set of all algebraic elements in $L$ over $F$ is called \emph{relative algebraic closure of $F$ in $L$} and denoted by $\cl F$.
\end{defn}
\begin{rmk}
The reason of the word ``relative'' is explained later.
Also, honestly, the notation $\cl F$ is not so good that it is often used to represent an ``algebraic closure'', not a relative one.
We, however, proceed with this notation to grasp concepts of algebraic extensions.
\end{rmk}
\begin{prop}
The relative algebraic closure $\cl F$ of $F$ in $L$ is always a field.
\end{prop}
\begin{pf}
An element is algebraic over $F$ if and only if it is contained in a finite extension $E/F$ because $\alpha\in E$ is equivalent to $F(\alpha)<E$.

Let $\alpha,\beta\in L$ be nonzero algebraic elements over a field $F$.
Since $\alpha+\beta$, $\alpha\beta$, and $\alpha^{-1}$ are all in $F(\alpha,\beta)$, which is a finite extension of $F$ with degree $\deg_F(\alpha)\deg_F(\beta)$, the set of algebraic elements over $F$ in $L$ is a field.
\end{pf}

\begin{lem}
Let $E,F<L$ be fields.
Then,
\begin{cond}
\item $F<E$ implies $\cl F<\cl E$,
\item $\cl{(\cl F)}=\cl F$.
\end{cond}
\begin{pf}
%%%
\end{pf}
\end{lem}
\begin{lem}
Given a condition that every algebraic element over $L$ is in $L$, a field extension $E/F$ is algebraic iff $\cl E=\cl F$.
\end{lem}
\begin{pf}
%%%
\end{pf}
\begin{rmk}
The condition of $L$ has a name \emph{algebraiclly closedness}.
The above lemma is useful because we already know a field $L$ such that
\begin{itemize}
\item algebraic elements over $L$ is in $L$,
\item $L$ contains a lot of usual example fields.
\end{itemize}
It is the field of complex numbers, $\C$.

After studying algebraic closures in the next subsection, this lemma will be dealt within more details.
\end{rmk}

By the above lemma, theorems we want to get are given as follows.
They are all obviously proved from the lemmas above, so we give elementary proofs here.

\begin{thm}
Let $K$ be an intermediate field of a field extension $E/F$.
Then, $E/F$ is algebraic iff $E/K$ and $K/F$ are algebraic.
\end{thm}
\begin{pf}
One direction is clear.

Suppose $E/K$ and $K/F$ are algebraic.
Taking $\alpha\in E$, we show that it is algebraic over $F$.
Let $L$ be a field generated by the coefficients of $\mu_{\alpha,K}$ on $F$ so that $\mu_{\alpha,K}\in L[x]$ and $L(\alpha)/L$ is finite.
The extension $L/F$ is also finite since coefficients of $\mu_{\alpha,K}$ are finitely many and algebraic.

Since finite extensions $L(\alpha)/L$ and $L/F$ imply that $L(\alpha)/F$ is finite, a subextension $F(\alpha)/F$ is therefore finite.
\end{pf}

\begin{thm}
A field $E$ is algebraic over $F$ if and only if there is a tower of fields $\{K_n\}_n$ such that $K_n/F$ are all finite and the ascending union is $E$.
\end{thm}
\begin{pf}
%%%
\end{pf}

\begin{thm}
Let $E/F$ and $E/F$ are algebraic extensions in a superfield $L$.
Then, the compositum $EE'/F$ is algebriac.
\end{thm}
\begin{pf}
%%%
\end{pf}


\begin{ex}
For a transcendental number such as $\pi$, an extension $\Q(\alpha)/\Q$ is not algebraic since it contains an element that is not algebraif.
To give another reason, that is because a simple extension is algebraic if and only if it is finite.
\end{ex}
\begin{ex}
Finite extensions are not only the algebraic extensions.
For examples,
\[\Q(\sqrt2,\sqrt[4]2,\sqrt[8]2,\cdots),\c{or}\Q(\sqrt2,\sqrt3,\sqrt5,\sqrt7,\cdots)\]
are infinite algebraic extensions.
\end{ex}








\subsection{Algebraic closures}

Algebraic closure is intuitively a maximal algebraic extension.
It is well described using the notion of algebraically closed fields.
Although the existence will be proved later, we give definitions.

\subsubsection{Algebraically closed fields}
\begin{defn}
A field $F$ is called \emph{algebraically closed} if it has no proper algebraic extension.
\end{defn}

There is another important characterization of algebraically closed fields.

\begin{prop}
A field $F$ is algebraically closed iff every polynomial in $F[x]$ has a root in $F$.
\end{prop}
\begin{pf}
Easy.
\end{pf}

If $f$ has a root $\alpha$, then we can inductively apply this theorem for a new polynomial $f(x)/(x-\alpha)$ of a lower degree to make the complete linear factorization.
In particular, the theorem implies that algebraically closedness is a internal property; it is preserved under isomorphisms.

\begin{defn}
A field $\cl F$ is called an \emph{algebraic closure} of a field $F$ if $\cl F$ is algebraically closed field and $\cl F/F$ is algebraic.
\end{defn}


\begin{prop}
Let $E/F$ be a field extension with algebraically closed field $E$.
Then the set of all algebraic elements in $E$ over $F$ is the only algebraic closure of $F$ contained in $E$.
\end{prop}
\begin{pf}
The set of algebraic elements is algebraically closed.
%%%
\end{pf}

This is a relation with relative algebraic closure: the relative algebraic closure in a algebraically closed field is really an algebraically closure.
The proposition allows us to choose a standard algebraic closure when provided a large superfield like $\C$.
In number theory, it is convenient for all algebraically closed fields to be considered that they are in $\C$. 

\begin{ex}
The set of all complex numbers $\C$ is an algebraically closed field by the fundamental theorem of algebra.
\end{ex}
\begin{ex}
The set of all algebraic numbers (over $\Q$) is an algebraically closed field by the proposition above and is a subfield of $\C$.
\end{ex}

\subsubsection{Uniqueness and existence}
Here is an extremely useful lemma that allows to apply the axiom of choice to field theory.

\begin{thm}[Isomorphism extension theorem]
Let $E/F$ be an algebraic extension.
Let $\phi:F\cong F'$ be a field isomorphism.
Let $\cl F'$ be an algebraic closure of $F'$.
Then, there is an embedding $\tld\phi:E\to\cl F'$ which extends $\phi$.
\begin{cd}
&\cl F' \ar[dash]{dd}\\
E\ar{r}{\tld\phi}\ar[dash]{d}&\quad\\
F\ar{r}{\phi}&F'
\end{cd}
\end{thm}
\begin{pf}
Let $S$ be the set of all field homomorphisms $K\to\cl F'$ which extends $\phi$ and satisfies $K<E$.
The set $S$ is nonempty since $\phi\in S$ and satisfies the chain condition since the increasing union defines the upper bound of chain.
Use the Zorn's lemma on $S$ to obtain a maximal element $\tld\phi:K\to\cl F'$.
We want to show $K=E$.

Suppose $K$ is a proper subfield of $E$ and let $\alpha\in E\setminus K$.
Let $\alpha'\in\cl F'$ be a root of the pushforward polynomial $\phi_*(\mu_{\alpha,F})\in F'[x]$.
Then, we can construct a field homomorphism $K(\alpha)\to\cl F':\alpha\mapsto\alpha'$.
It leads a contradiction to the maximality of $\tld\phi$.
Therefore, $K=E$.
\end{pf}


\begin{thm}[Uniqueness of algebraic closure]
Algebraic closure is unique up to isomorphism.
\end{thm}
\begin{pf}
Suppose there are two algebraic closures $\cl F_1, \cl F_2$ of a field $F$.
By the isomorphism extension theorem, we have a field homomorphism $\phi:\cl F_1\to\cl F_2$ which extends the identitiy map on $F$.
Since the image $\phi(F_1)$ is also algebraically closed and the field extension $F_2/\phi(F_1)$ is algebraic, we must have $\phi(F_1)=F_2$ by the definition of algebraically closedness.
Thus, $\phi$ is surjective so that it is an isomorphism.
\end{pf}

\begin{thm}[Existence of algebraic closure]
Every field has an algebraic closure.
\end{thm}
\begin{pf}
Let $F$ be a field.

\Step[1]{Construct an algebraically closed field containing $F$}
At first we want to construct a field $K_1>F$ such that every $f\in F[x]$ has a root in $K_1$.
This is satisfied by $K_1:=R/\fm$, where $R$ and $\fm$ is defined as follows:
Let $S$ be the set of all irreducibles and nonconstant polynomials in $F[x]$.
Define $R:=F[\{x_f\}_{f\in S}]$.
Let $I$ be an ideal in $R$ generated by $f(x_f)$ as $f$ runs through all $S$.
It has a maximal ideal $\fm\supset I$ in $R$.

Construct a sequence $\{K_n\}_n$ of fields inductively such that every nonconstant $k\in K_n[x]$ has a root in $K_{n+1}$.
Define $K:=\lim_{\to}K_n$ as the inductive limit.
It is in other word just the union of $K_n$ for all $n\in\N$.
Then, $K$ is easily checked to be algebraically closed.

\Step[2]{Construct the algebraic closure of $F$}
Let $\cl F$ be the set of all algebraic elements of $K$ over $F$.
Then, this is an algebraic closure.
\end{pf}
\begin{rmk}
In fact, this $K_1$ is already algebraically closed, but it is hard to prove directly, so we are going to construct another algebraically closed field, $K$.
\end{rmk}






























\section{Separable extension}

The most fields considered in mathematics is separable.

\subection{Separable polynomial}
\begin{defn}
A polynomial $f\in F[x]$ is called \emph{separable} if it is square-free in $\cl F[x]$.
An element $\alpha\in\cl F'$ is called \emph{separable} over $F$ if $\mu_{\alpha,F}$ is separable.
\end{defn}

\begin{defn}
A field extension $E/F$ is called \emph{separable} if all elements in $E$ is separable over $F$.
\end{defn}

separable degree

differentiation



\subsection{Separable closure}



\subsection{Separable degree}

\begin{thm}
The separable degree of a field extension $E/F$ is the number of field homomorphisms $E\to\cl F$ fixing $F$.
It is denoted by $\{E:F\}$.
\end{thm}

\begin{lem}
All roots of an irreducible polynomial has same multiplicity.
\end{lem}
\begin{pf}
%%%
\end{pf}

\begin{thm}
Let $K$ be an intermediate field of a finite extension $E/F$.
Then,
\[\{E:F\}\mid[E:F]\]
\end{thm}
\begin{pf}
%%%
\end{pf}

\begin{thm}
A finite field extension $E/F$ is separable if and only if
\[\{E:F\}=[E:F].\]
\end{thm}
\begin{pf}
%%%
\end{pf}


multiplcation formula





\subsection{Perfect fields}









\section{Normal extension}
\subsection{Automorphism group}

\subsection{Normal extension}









\section{Galois theory}

 * reducible case, irreducible<=>transitivity
 * resolvent polynomial1: discriminant
 * resolvent polynomial2: cubic resolvent
 * double quadratic, reciprocal equation: finding symmetry
 * number of imaginary roots=2n: composition of n transpositions
 * $x^n-\alpha$ :  Jacobson-Velez
 * reduction modulo p (over $\F_p$)

\subsection{Quartic}
In this section, we assume the following setting:
\begin{itemize}
\item $F$ is a perfect field, %유한체는 사실 잘 모르겠음
\item $f$ is an irreducible quartic over $F$,
\item $E$ is the splitting of $f$ over $F$,
\item $G=\Gal(E/F)$,
\item $H=G\cap V_4$. 
\end{itemize}
\begin{thm}
There are only five isomorphic types of transitive subgroups of the symmetric group $S_4$.
\end{thm}
\begin{cor}
$G\cong S_4,\ A_4,\ D_4,\ V_4,\ or\ C_4$.
\end{cor}
\begin{prop}
Two groups $A_4$ and $V_4$ are only transitive normal subgroups of $S_4$.
\end{prop}

Now we define our resolvent polynomial.
\begin{prop}
Let $K$ be the fixed field of $H$.
Then,
\[K=F(\alpha_1\alpha_2+\alpha_3\alpha_4,\ \alpha_1\alpha_3+\alpha_2\alpha_4,\ \alpha_1\alpha_4+\alpha_2\alpha_3).\]
\end{prop}
\begin{defn}
Let $K$ be the fixed field of $H$.
A \emph{resolvent cubic} is a cubic $R_3$ that has $K$ as the splitting field over $F$.
\end{defn}

\begin{thm}
We have
\begin{cond}
\item $G\cong S_4$ if $R_3$ is irreducible and ,
\item $G\cong A_4$ if $R_3$ is irreducible and ,
\item $G\cong D_4$ if $R_3$ has only one root in $K$ and $f$ is irreducible over $K$,
\item $G\cong C_4$ if $R_3$ has only one root in $K$ and $f$ is reducible over $K$,
\item $G\cong V_4$ if $R_3$ splits in $K$.
\end{cond}
\end{thm}
\begin{pf}
There are five possible cases:
\[(G,H)=(S_4,V_4),\ (A_4,V_4),\ (D_4,V_4),\ (V_4,V_4),\ (C_4,C_2).\]
We have
\[[K:F]=|G/H|,\qquad[E:K]=|H|.\]

If $f$ is reducible over $K$, then $\Gal(E/K)$ is no more a transitive subgroup of $S_4$ so that $H\ne V_4$ and $G\cong C_4$.
\end{pf}
\begin{cd}
E \ar{rrrr} & & & & 1 \ds{d}{2} &\\
&&&& C_2 \ds{dl}[swap]{2}\ds{dr}{2}&\\
K=F_{V_4\cap G}\ds{uu}&&& V_4 \ds{dl}[swap]{3}\ds{dr}{2} && C_4 \ds{dl}{2}\\
F \ds{u}\ar{r}& ? & A_4 \ds{dr}[swap]{2} && D_4 \ds{dl}{3} &\\
&&&S_4&&
\end{cd}




\end{document}