\documentclass{../exp}
\usepackage{../../ikany}

\title{Topological structures for analysis}

\begin{document}
\maketitle

\section{Introduction}


\section{Metric spaces}


\section{Topologcial spaces}


\section{Uniform spaces}
\subsection{Metric and topology}
Topology is a minimal structure to define continuity.
The most commonly used topologies in analysis are the ones indueced by metrics.
Topology generalize the notion of neighborhoods and openness in metric spaces.
Unlike topological spaces, in a metric space, neighborhoods are indexed by real numbers.

\begin{defn}[Metric space]
A \emph{metric space} is a set $X$ equipped with a function $d\colon X\x X\to\R_{\ge0}$ such that
\begin{cond}
\item non-degeneracy: $d(x,y)=0$ iff $x=y$,
\item triangle inequality: $d(x,y)+d(y,z)\ge d(x,z)$,
\item symmetry: $d(x,y)=d(y,x)$.
\end{cond}
The function $d$ is called \emph{metric}.
\end{defn}
\begin{defn}
Let $X$ be a set.
A \emph{(topological) basis} on $X$ is a collection $\cB\subset\cP(X)$ such that every \emph{local basis} $\cB_x=\{B\in\cB:x\in B\}$ is a directed set.
\end{defn}
Define a preorder on bases by refinements: $\cB\prec\cB'$ iff
\[\forall B\in\cB_x\impl\exists B'\in\cB'_x:\quad B'\subset B.\]
\begin{prop}
There is only one maximal element in each equivalence class of bases.
\end{prop}
\begin{defn}
A \emph{topology} is the maximal element of an equivalence class of bases.
\end{defn}
\begin{defn}[Topological space]
A \emph{topological space} is a set $X$ with a topology.
\end{defn}

What we want to recognize is that a topology is in fact a representative of bases.
This definition of topology differs from the traditional definition that includes three axioms of open sets, but we can check that a topology satisfies the three axioms and a topology, conversely, is a basis; the definitions characterize same objects.
The traditional definition provides us a good criterion to determine whether a collection of subsets is a topology.

When introducing metric spaces, a question comes up with: Why are they defined so? Why are they important?
The first available answer is intuition because we can imagine a neighborhood as a disk or a ball, which is easy to be visualized.
Next, hisotrical reason can be another answer.
For the last answer, technically, metric spaces are the topological spaces that are ``countably uniform''.
Let us give an example to get this.
\begin{ex}
Let $X$ and $Y$ be metric spaces.
Let $f\colon X\to Y$ be a continuous map.
We say $f$ is \emph{uniformly continuous} if
\[\forall\e>0,\,\exists\delta>0:\quad y\in B_\delta(x)\impl f(y)\in B_\e(f(x)).\]
Note that the definition does not make sense when either $X$ or $Y$ is supposed to be a general topological space because there is no ``$\e$, $\delta$'' in topological space.
The uniform continuity requires that open neighborhoods, even if they are disjoint, should be indexed by a \emph{same directed set} such as $\R$.
By the same reason, in addition, uniform convergence is usually defined when the codomain is a metric space.
\end{ex}


\subsection{Uniform structure via entourages}

Uniform spaces are generalization of metric spaces.
The uniform structure is required to define uniform continuity, uniform convergence, completeness, etc.
Although the definition of uniform structure is not so easy at first, they have enormous advantage to learn.
For example, they are extremely useful in functional analysis since every compatible topology on algebraic structures such as topological group and topological vector space must admit a natural uniform structure.
Hence, we can use completeness or something uniform without unnecessary concerns.

\begin{defn}[Uniform space]
A \emph{uniform space} is a set $X$ equipped with a filter of binary relations $\cU\subset\cP(X^2)$ such that for every $E\in\cU$,
\begin{cond}
\item reflexivity: $(x,x)\in E$ for all $x\in X$,
\item triangle inequality: $\exists E'\in\cU:\,E'\circ E'\subset E$,
\item symmetry: $E^{-1}\in\cU$,
\end{cond}
where $\Delta_X=\{(x,x):x\in X\}$ and
\[E\circ F=\{(x,z):(x,y)\in E,(y,z)\in F\},\quad E^{-1}=\{(y,x):(x,y)\in E\}.\]
The collection $\cU$ is called a \emph{uniformity}, and a relation $E\in\cU$ is called an \emph{entourage}.
If $(x,y)\in E$, then we say $x$ and $y$ are $E$-close.
\end{defn}
\begin{defn}
Let $(X,\cU)$ be a uniform space.
Let $\tau$ be a set containing all $U\subset X$ such that for every $x\in U$ there is an entourage $E$ with $E_x\subset U$.
Then $\tau$ defines a topology on $X$, which is called \emph{uniform topology}, or \emph{induced topology}.
\end{defn}
\begin{defn}
A uniform space is called \emph{Hausdorff} if there is an entourage $E$ such that $x\in E$ and $y\notin E$ for every pair of distinct points $x,y\in X$.
This is equivalent for the induced topology to be Hausdorff.
\end{defn}

Note that the axioms for the definition of uniform spaces bear a similarity with the one of metric spaces.
For one exception, the Hausdorffness implies the nondegeneracy.
A uniform space is defined by the collection of relations that embody the concept of nearness.
Unlike neighborhoods in general topological space, an entourage measures the nearness not pointwisely(locally) but uniformly(globally).
We have the following hierarchy:
\[\text{topological space} \supset \text{uniform space} \supset \text{metric space}.\]
\begin{ex}
Let $G$ be a topological group.
Let $U$ be an open neighborhood of the identity $e$.
Define
\[E_U:=\{(g,h):gh^{-1}\in U\}.\]
Then, the set of $E_U$ forms a uniformity.
The difficult part is the triangle inequality, which can be shown from the continuity of group operation.
\end{ex}

\subsection{Uniform structure via family of pseudometrics}

Uniform spaces can be characterized by several ways: by entourages, by uniform covers, and by a family of pseudometrics. 
Here, pseudometric on a set is a function $X^2\to\R_{\ge0}$ satisfying all conditions of metric with an exception, non-degeneracy.
We are not going to treat uniform covers, but investigate how pseudometrics are related to uniform structures.
The proof of the following theorem is based on Bourbaki's text.%cite
\begin{thm}
A topological space is uniformizable if and only if it is induced by a family of pesudometrics.
\end{thm}
\begin{pf}

\end{pf}


\section{Cauchy spaces}


\end{document}



