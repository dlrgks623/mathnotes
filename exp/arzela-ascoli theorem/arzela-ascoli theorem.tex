\documentclass{../exp}
\usepackage{../../ikany}

\title{The Arzela-Ascoli theorem}

\begin{document}
\maketitle

\section{Introduction}
Compactness gives us a notion of a certian ``finiteness'' of a topological space.
When doing analysis, we often encounter situations where we need to check compactness of a subset.
There are a lot of tools, including the Heine-Borel theorem, the Tychonoff theorem, and the Banach-Alaoglu theorem, that make it possible.
In case of continuous function spaces, we have a great solution, the Arzela-Ascoli theorem.

The purpose of this article is to provide varied viewpoints for the compactness and space of continuous functions via the Arzela-Ascoli theorem, which is my favorite theorem.
By doing so, we can learn deeper and more generalized understanding of the nature of topologies on continuous function spaces.

For preliminaries, we will review on compactness at first.
Next, the definitions and properties of topologies on continuous function spaces are going to be introduced.
After that, various statements with various proofs and some applications of the theorem will follow.






\section{More on compactness}

\subsection{Relative compactness}

\begin{defn}
A uniform space is called \emph{relatively compact} or \emph{precompact} if its completion is compact.
\end{defn}
\begin{prop}
Let $X$ be a complete uniform space.
A subset of $X$ is relatively compact if and only if its closure is compact.
\end{prop}
\begin{defn}
A uniform space is called \emph{relatively net compact} if every net has a Cauchy subnet.
\end{defn}
\begin{defn}
A uniform space is called \emph{totally bounded} if for every entourage $E$ there is a finite cover $\{U_i\}$ with $U_i^2\in E$ for each index $i$.
\end{defn}
\begin{thm}
Let $X$ be a uniform space.
The followings are all equivalent:
\begin{cond}
\item $X$ is relatively compact;
\item $X$ is relatively net compact;
\item $X$ is totally bounded.
\end{cond}
\end{thm}
\begin{pf}
Let $\tld X$ be a completion of $X$.

(1)$\impl$(2).
Every net in $X$ has a subnet that converges in $\tld X$.
The subnet is Cauchy.

(1)$\impl$(3).
Let $E$ be an entourage.
Since $\{U\in\tld X:U^2\in E\}$ is an open cover of $\tld X$, there is a finite subcover $\{U_i\}$ of $\tld X$.
Then, $\{U_i\cap X\}$ is a finite cover of $X$ satisfying $(U_i\cap X)^2\in E$.

(2)$\impl$(1).
Let $\tld x:\cD\to\tld X$ be a net in $\tld X$.
Take a net $x:\cD\x\cU\to X$ such that $x_{(d,E)}$ is $E$-closed to $\tld x_d$, where $\cU$ is the uniformity of $\tld X$.
By the condition, there is a Cuachy subnet $x:\cE\to[h]\cD\x\cU\to X$.
Then, a subnet $\tld x:\cE\to[h]\cD\x\cU\to\cD\to\tld X$ is Cauchy.

In order to show this, take an entourage $E$.
By the Cauchyness, we can find $e_0\in\cE$ such that $h(e_0)=(d_{e_0},E_{e_0})$ satisfies $E_{e_0}\subset E$ and $e,e'>e_0\impl(x_e,x_{e'})\in E$.
Then, we get $(\tld x_e,\tld x_{e'})\in E^3$ because $(\tld x_e,x_e)=(\tld x_{d_e},x_{(d_e,E_e)})\in E_e\subset E_{e_0}\subset E$, $(x_e,x_{e'})\in E$, and $(x_{e'},\tld x_{e'})=(x_{(d_{e'},E_{e'})},\tld x_{d_{e'}})\in E_{e'}\subset E_{e_0}\subset E$.
Therefore, $\tld x:\cE\to\tld X$ is Cauchy.

(3)$\impl$(1).
Clearly, $\tld X$ is totally bounded.


\end{pf}

Empirically, problems related to proof of compactness seem to be fallen into two cases: one is to show completeness and relative compactness, and the other is to apply Tychonoff's theorem.
Followings are some examples.
\begin{ex}[The Heine-Borel theorem]

\end{ex}
\begin{ex}[The Blaschke selection theorem]
Let $H(X)$ be the metric space of nonempty compact subsets of a metric space $X$ with Hausdorff metric.
If $X$ is compact, then $H(X)$ is compact.
\end{ex}

\begin{ex}[The Banach-Alaoglu theorem]
Let $X$ be a locally convex space and $X^*$ the continuous dual of $X$ with weak$^*$ topology.
If $B\subset X^*$ satisfies $\sup_{f\in B}|f(x)|<\infty$ for each $x\in X$, then $B$ is compact.
\end{ex}
\begin{pf}
We can embed $X^*$ into $\F^X$.
\end{pf}

\subsection{Some weeker compactness}
sigma compact and hemicompact







\section{Two topologies on continuous function spaces}
\subsection{Pointwise topology}

\subsection{Compact-open topology}











\section{Variations on the Arzela-Ascoli theorem}

\begin{thm}[Arzela-Ascoli, conventional version]
Let $X$ be a compact space.
For $(f_n)_{n\in\N}\subset C(X)$, if it is equicontinuous and pointwisely bounded, then there is a subsequence that uniformly converges.
\end{thm}

Let $\cT_p$ be the topology of pointwise convergence and $\cT_c$ be the topology of compact convergence.
\begin{thm}[Arzela-Ascoli, metrized version]
Let $X$ be a hemicompact space and $Y$ be a metric space.
If $\cF\subset C(X,Y)$ is equicontinuous and relatively compact in $\cT_p$, then it is relatively compact in $\cT_c$.
\end{thm}
\begin{pf}
Let $\{f_n\}_{n\in\N}$ be a sequence in $\cF$ and $K\subset X$ be a compact.

By equicontinuity, for each $k\in\N$ a finite open cover $\{U_s\}_{s\in S_k}$ with a finite set $S_k\subset K$ can be taken such that $x\in U_s\,\impl\,d(f(x),f(s))<\frac1k$ for all $f\in\cF$.
By the pointwise relative compactness, we can extract a subsequence $\{f_m\}_{m\in\N}$ of $\{f_n\}_n$ such that $\{f_m(s)\}_m$ is Cauchy for each $s\in\bigcup_{k\in\N}S_k$ by the diagonal argument.

For every $\e>0$, let $k=\ceil{\e^{-1}}$ so that $\frac1k\le\e$.
Let $m_{0,s}$ be an index such that $m,m'>m_{0,s}\impl d(f_m(s),f_{m'}(s))<\e$, and define $m_0=\max\{m_{0,s}:s\in S_k\}$.
Then, for arbitrary $x\in K$, we obtain $m,m'>m_0\impl$
\[d(f_m(x),f_{m'}(x))\le d(f_m(x),f_m(s))+d(f_m(s),f_{m'}(s))+d(f_{m'}(s),f_{m'}(x))<3\e\]
by taking $s\in S_k$ such that $x\in U_s$.
Thus, $\{f_m\}_m$ is a subsequence of $\{f_n\}_n$ that is uniformly Cauchy on $K$.
\end{pf}


If $C(X,Y)$ is not metrizable, or even $Y$ is only uniform but not metrizable, the subsequence extracting procedure is not available.

\begin{thm}[Arzela-Ascoli, generalized version]
Let $X$ be a topological space and $Y$ be a uniform space.
If $\cF\subset C(X,Y)$ is equicontinuous and relatively compact in $\cT_p$, then it is relatively compact in $\cT_c$.
\end{thm}
\begin{pf}[1]
Every net in $F$ has a subnet that is pointwisely Cauchy.
We are going to prove this subnet is in fact uniformly Cauchy on a compact set $K$.

Let $(f_d)_{d\in\cD}$ be the pointwisely Cauchy subnet.
Let $E$ be an entourage in $Y$.
By equicontinuity, a finite open cover $\{U_s\}_{s\in S}$ of $K$ can be taken such that $x\in U_s\impl(f(x),f(s))\in E$ for all $f\in\cF$.
(In here, we do not need to extract a subnet because the subnet is already pointwisely convergent.)
Let $d_{0,s}$ be an index such that $d,d'>d_{0,s}\impl(f_d(s),f_{d'}(s))\in E$, and define $d_0=\max\{d_{0,s}:s\in S\}$.
Then, for arbitrary $x\in K$, we get $d,d'>d_0\impl(f_d(x),f_{d'}(x))\in E^3$ by taking $s\in S$ such that $x\in U_s$.
Thus, $(f_d)_d$ is uniformly Cauchy on $K$.
\end{pf}

\begin{proof}[Proof 2]

\end{proof}

\begin{proof}[Proof 3]
We are going to prove the two topologies coincide in $\cF$.
Note that subspace topologies $\cT_p\vert_\cF$ and $\cT_c\vert_\cF$ are also generated by pseudometrics $\rho_{\{x\}}(f,g)=d(f(x),g(x))$ and $\rho_K(f,g)=\sup\{d(f(x),g(x)):x\in K\}$ defined on $\cF$ respectively.
Since $\cT_p\vert_\cF\subset\cT_c\vert_\cF$ clearly, it is enough to show the converse.

Take a subbasis element $B_K(f_0,\e)\cap\cF$ of $\cT_c\vert_\cF$ with $f_0\in\cF$.
By equicontinuity of $\cF$, there is a finite cover $\{U_s\}_{s\in S}$ of a compact subset $K$ such that $x\in U_s\,\impl\,d(f(x),f(s))<\e$ for all $f\in\cF$.

If $f\in\cF$ satisfies $\rho_{\{s\}}(f,f_0)<\e$ for all $s\in S$, then we get
\[\rho_K(f,f_0)\le\sup_{x\in K}d(f(x),f(s))+d(f(s),f_0(s))+\sup_{x\in K}d(f_0(s),f_0(x))<\e\]
by taking $s\in S$ such that $x\in U_s$.
This means
\[\bigcap_{s\in S}B_{\{s\}}(f_0,\e)\subset B_K(f_0,3\e).\]
It implies that the subbasis element contains an open set in $\cT_p\vert_\cF$.
Therefore, $\cT_c\vert_\cF\subset\cT_p\vert_\cF$.
\end{proof}


\section{Applications}

\end{document}

발표!!
1.topology, metric, norm, pseudometric, seminorm
  (countability/merging)
2.compactness in top, cau, unif, met spaces
  (cpt, net cpt, tot bd, seq cpt)
3.topologies on continuous function space
4.arzela-ascoli 증명
5.루딘과의 비교
6.쓰임새




