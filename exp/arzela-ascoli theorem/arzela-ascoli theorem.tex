\documentclass{../exp}
\usepackage{../../ikany}

\title{The Arzela-Ascoli Theorem for Uniform Spaces}

\begin{document}
\maketitle

\section{Introduction}
Compactness gives us a notion of a certian ``finiteness'' of a topological space.
When doing analysis, we often encounter situations where we need to check compactness of a subset.
There are a lot of tools, including the Heine-Borel theorem, the Tychonoff theorem, and the Banach-Alaoglu theorem, that make it possible.
In case of continuous function spaces, we have a great solution, the Arzela-Ascoli theorem.

The purpose of this article is to provide varied viewpoints for the compactness and space of continuous functions via the Arzela-Ascoli theorem, which is one of my favorite theorems.
By doing so, we can learn deeper and more generalized understanding of the nature of topologies on continuous function spaces.

For preliminaries, we will rediscover the relative compactness for uniform spaces at first.
Next, the definitions and properties of topologies on continuous function spaces are going to be introduced.
After that, various statements with various proofs and some applications of the theorem will follow.




\section{Relative compactness}

\begin{defn}
A uniform space is called \emph{relatively compact} or \emph{precompact} if its completion is compact.
\end{defn}

The relative compactness is in particular useful in uniform spaces.
We might be able to say the familiar definition of relative compactness has come from the following proposition:
\begin{prop}
A subset of a complete uniform space, like Banach or Fr\'echet space, is relatively compact if and only if its closure is compact.
\end{prop}
\begin{pf}
In a complete uniform space, the completion can be charaterized as closure.
\end{pf}

The following two definitions are helpful when we need to check relative compactness.
\begin{defn}
A uniform space is called \emph{relatively net compact} if every net has a Cauchy subnet.
\footnote{There are three well-known different definitions: Willard subnet, Kelly subnet, and Aarnes-Anden{\ae}s subnet. However, since the existence of each subnet associated to a common eventuality filter is equivalent, there will be no conflicts among them. In this note, we define a subnet as the monotone cofinal function between the index sets. See Eric schechter's book [].}
\footnote{The relatively net compactness can defined not only for uniform spaces but also Cauchy spaces, more generally.}
\end{defn}
\begin{defn}
A uniform space is called \emph{totally bounded} if for every entourage $E$ there is a finite cover $\{U_i\}_i$ with $U_i\x U_i\in E$ for each index $i$.
\footnote{Uniformity is the minimal structure to define totally bounded spaces.}
\end{defn}

\begin{thm}
Let $X$ be a uniform space.
The followings are all equivalent:
\begin{cond}
\item $X$ is relatively compact;
\item $X$ is relatively net compact;
\item $X$ is totally bounded.
\end{cond}
\end{thm}
\begin{pf}
Let $\tld X$ be the completion of $X$ and $\cU$ be the uniquely extended uniformity of $\tld X$ from $X$.

(1)$\impl$(2).
Every net in $X$ has a subnet that converges in $\tld X$.
The subnet is Cauchy.

(1)$\impl$(3).
Let $E$ be an entourage.
Since $\{U\in\tld X:U\x U\in E\}$ is an open cover of $\tld X$, there is a finite subcover $\{U_i\}_i$ of $\tld X$.
Then, $\{U_i\cap X\}_i$ is a finite cover of $X$ satisfying $(U_i\cap X)\x(U_i\cap X)\in E$.

(2)$\impl$(1).
By the fact that compactness and net compactness are always equivalent in all topological spaces, we are done if we show every net in $\tld X$ has a Cauchy subnet.
Let $\tld x:\fA\to\tld X$ be a net in $\tld X$.
Take an approximating net $x:\fA\x\cU\to X$ in $X$ such that
\[(x_{(\alpha,E)},\tld x_\alpha)\in E\]
for all $\alpha\in\fA$ and all entourages $E\in\cU$.
Recall that $(\alpha,E)\succ(\alpha',E')$ if and only if $\alpha\succ\alpha'$ and $E\subset E'$.
By the assumption that $X$ is relatively net compact, there is a Cauchy subnet
\[xh:\fB\to{h}\fA\x\cU\to X\]
of $x$.
Then,
\[\tld x\alpha h:\fB\to{h}\fA\x\cU\to{\alpha}\fA\to\tld X\]
is a subnet of $\tld x$ since both $h$ and the projection $\alpha$ are monotone and cofinal so that so is their composition.
We claim that $\tld x\alpha h$ is Cauchy.

In order to show this, take an arbitrary entourage $E$.
By the Cauchyness of $xh$, we can find $\beta_0\in\fB$ such that
\[\beta,\beta'\succ\beta_0\impl(x_{h(\beta)},x_{h(\beta')})\in E.\]
Moreover, we may assume $h(\beta_0)=(\alpha_0,E_0)$ satisfies $(\alpha_0,E_0)\succ(\alpha_0,E)$ so that
\[(x_{h(\beta_0)},\tld x_{\alpha h(\beta_0)})=(x_{(\alpha_0,E_0)},\tld x_{\alpha_0})\in E_0\subset E.\]
Then, for $\beta,\beta'\succ\beta_0$, the following three relations
\[(x_{h(\beta)},\tld x_{\alpha h(\beta)})\in E_0\subset E,\quad(x_{h(\beta)},x_{h(\beta')})\in E,\quad(x_{h(\beta')},\tld x_{\alpha h(\beta')})\in E_0\subset E\]
implies $(\tld x_{\alpha h(\beta)},\tld x_{\alpha h(\beta')})\in E^3$.
Therefore, $\tld x\alpha h:\fB\to\tld X$ is Cauchy.

(3)$\impl$(1).
Clearly, $\tld X$ is totally bounded.
A complete totally bounded space is compact.
\end{pf}

In the last of the proof above, we used the famous result of complete totally bounded spaces.
This can be easily proved with diagonal subsequence extracting argument under the metric space condition, but we need the subtle application of the axiom of choice when we require only uniformness of the space.

\begin{thm}
A complete totally bounded uniform space is compact.
\end{thm}
\begin{pf}[1]
Let $X$ be a complete totally bounded uniform space.
Let $\cU$ and $\cT$ be the uniformity and topology of $X$.
Let $x:\fA\to X$ be a net in $X$.

\Step[1]{Applying Zorn's lemma}
Define a subset $\fA'\subset\fA\x\cT\x\cU$ by
\[(\alpha,U,E)\in\fA'\iff U^2\in E,\ x^{-1}(U)\text{ is cofinal},\ x_\alpha\in U.\]
Define a subset $Z\subset\cP(\fA')$ by
\[\fB\in Z\iff \pi_\cT(\fB)\subset\cT\text{ is a totally ordered set},\]
where $\pi_\cT:\fA\x\cT\x\cU\to\cT$ is the canonical projection.
We apply Zorn's lemma on $Z$.

First, we claim $Z\ne\mt$.
For any entourage $E\in\cU$, using totally boundedness, we can find a finite open cover $\{U_i\}_i$ of $\tld X$ such that $U_i^2\in E$.
Since $\bigcup_ix^{-1}(U_i)=x^{-1}(\tld X)=\fA$, we can find an open set $U$ such that $U^2\in E$ and $x^{-1}(U)$ is a cofinal subset of $\fA$.
If we choose any $\alpha\in x^{-1}(U)$, then the singleton $\{(\alpha,U,E)\}$ is in $Z$.
In particular, we get an additional result $\pi_\cU(Z)=\cU$ from this.

The upper bound of chain is obtained by union.
Therefore, there is a maximal element in $Z$.
Let it denoted by $\fM$.

\Step[2]{Verification of Cauchy subnet}
We need to show the three conditions: Cauchyness, directedness of $\fM$, and monotone cofinality of $\fM\to\fA$.

By the maximality, $\pi_\cU(\fM)$ must contains every entourage.
The Cauchyness follows from this easily.
The directedness is from the linearity of $\pi_\cT(\fM)$.
The cofinality is shown by the cofinality of $x^{-1}(U)$.
\end{pf}
\begin{pf}[2]
This proof is by DL Frank, Columbia university, 1965 [].

\end{pf}


Empirically, compactness checking problems seem to be fallen into two cases: one is to show both completeness and relative compactness, and the other is to apply Tychonoff's theorem.
Followings are some examples.

\begin{ex}[The Heine-Borel theorem]
A subset of a Euclidean space is compact if and only if it is closed and bounded.
\end{ex}
\begin{ex}[The Blaschke selection theorem]
Let $H(X)$ be the metric space of nonempty compact subsets of a metric space $X$ with Hausdorff metric.
If $X$ is compact, then $H(X)$ is compact.
\end{ex}
\begin{ex}[The Banach-Alaoglu theorem]
Let $X$ be a locally convex space and $X^*$ the continuous dual of $X$ with weak$^*$ topology.
If $B\subset X^*$ satisfies $\sup_{f\in B}|f(x)|<\infty$ for each $x\in X$, then $B$ is compact.
\end{ex}
\begin{pf}
We can embed $X^*$ into $\F^X$.
\end{pf}








\section{Topologies on continuous function spaces}


\iffalse

$C(X,Y)$ has the compact-open topology
When $Y$ is uniform, $C(X,Y)$ has the topology of compact convergence
When $X$ is compact and $Y$ is uniform, $C(X,Y)$ has the topology of uniform convergence


When $X$ is locally compact Hausdorff, and the exponential space is endowed with compact-open topology

\fi


\section{Variations on the Arzela-Ascoli theorem}

\begin{thm}[Arzela-Ascoli, conventional version]
Let $X$ be a compact space.
For $\{f_n\}_{n\in\N}\subset C(X)$, if it is equicontinuous and pointwisely bounded, then there is a subsequence that uniformly converges.
\end{thm}

Let $\cT_p$ be the topology of pointwise convergence and $\cT_c$ be the topology of compact convergence.

\begin{thm}[Arzela-Ascoli, metrized version]
Let $X$ be a hemicompact space and $Y$ be a metric space.
If $\cF\subset C(X,Y)$ is equicontinuous and relatively compact in $\cT_p$, then it is relatively compact in $\cT_c$.
\end{thm}
\begin{pf}
Let $\{f_n\}_{n\in\N}$ be a sequence in $\cF$ and $K\subset X$ be a compact.

By equicontinuity, for each $k\in\N$ a finite open cover $\{U_s\}_{s\in S_k}$ with a finite set $S_k\subset K$ can be taken such that $x\in U_s\,\impl\,d(f(x),f(s))<\frac1k$ for all $f\in\cF$.
By the pointwise relative compactness, we can extract a subsequence $\{f_m\}_{m\in\N}$ of $\{f_n\}_n$ such that $\{f_m(s)\}_m$ is Cauchy for each $s\in\bigcup_{k\in\N}S_k$ by the diagonal argument.

For every $\e>0$, let $k=\ceil{\e^{-1}}$ so that $\frac1k\le\e$.
Let $m_{0,s}$ be an index such that $m,m'>m_{0,s}\impl d(f_m(s),f_{m'}(s))<\e$, and define $m_0=\max\{m_{0,s}:s\in S_k\}$.
Then, for arbitrary $x\in K$, we obtain $m,m'>m_0\impl$
\[d(f_m(x),f_{m'}(x))\le d(f_m(x),f_m(s))+d(f_m(s),f_{m'}(s))+d(f_{m'}(s),f_{m'}(x))<3\e\]
by taking $s\in S_k$ such that $x\in U_s$.
Thus, $\{f_m\}_m$ is a subsequence of $\{f_n\}_n$ that is uniformly Cauchy on $K$.
\end{pf}

If $C(X,Y)$ is not metrizable, or if $Y$ is uniform but not metrizable, the subsequence extracting procedure is no more available.

\begin{thm}[Arzela-Ascoli, generalized version]
Let $X$ be a topological space and $Y$ be a uniform space.
If $\cF\subset C(X,Y)$ is equicontinuous and relatively compact in $\cT_p$, then it is relatively compact in $\cT_c$.
\end{thm}
\begin{pf}[1]
Every net in $\cF$ has a subnet that is pointwisely Cauchy.
Then we are done if we prove every pointwisely Cauchy net in $\cF$ is in fact uniformly Cauchy on each compact set $K\subset X$.

Let $f:\fA\to\cF$ be a pointwisely Cauchy net.
Let $E$ be an arbitrary entourage in $Y$.
By equicontinuity, every point $s\in X$ has a neighborhood $U_s$ such that
\[x\in U_s\impl(f(x),f(s))\in E\]
for all $f\in\cF$.
We can find a finite set $S\subset K$ such that $\{U_s\}_{s\in S}$ is a cover of $X$.
(In here, we do not need to extract a subnet because the net is already pointwisely Cauchy.)
Let $\alpha_{0,s}$ be an index such that
\[\alpha,\alpha'\succ\alpha_{0,s}\impl(f_\alpha(s),f_{\alpha'}(s))\in E,\]
and define $\alpha_0=\max\{\alpha_{0,s}\}_{s\in S}$.
Then, for arbitrary $x\in K$ there is $s\in S$ such that $x\in U_s$, so we get
\[\alpha,\alpha'\succ\alpha_0\impl(f_\alpha(x),f_{\alpha'}(x))=(f_\alpha(x),f_\alpha(s))\o(f_\alpha(s),f_{\alpha'}(s))\o(f_{\alpha'}(s),f_{\alpha'}(x))\in E^3.\]
Thus, $f$ is uniformly Cauchy on $K$.
\end{pf}

\begin{proof}[Proof 2]

\end{proof}

\begin{proof}[Proof 3]
We are going to prove the two topologies coincide in $\cF$.
Note that subspace topologies $\cT_p\vert_\cF$ and $\cT_c\vert_\cF$ are also generated by pseudometrics $\rho_{\{x\}}(f,g)=d(f(x),g(x))$ and $\rho_K(f,g)=\sup\{d(f(x),g(x)):x\in K\}$ defined on $\cF$ respectively.
Since $\cT_p\vert_\cF\subset\cT_c\vert_\cF$ clearly, it is enough to show the converse.

Take a subbasis element $B_K(f_0,\e)\cap\cF$ of $\cT_c\vert_\cF$ with $f_0\in\cF$.
By equicontinuity of $\cF$, there is a finite cover $\{U_s\}_{s\in S}$ of a compact subset $K$ such that $x\in U_s\,\impl\,d(f(x),f(s))<\e$ for all $f\in\cF$.

If $f\in\cF$ satisfies $\rho_{\{s\}}(f,f_0)<\e$ for all $s\in S$, then we get
\[\rho_K(f,f_0)\le\sup_{x\in K}d(f(x),f(s))+d(f(s),f_0(s))+\sup_{x\in K}d(f_0(s),f_0(x))<\e\]
by taking $s\in S$ such that $x\in U_s$.
This means
\[\bigcap_{s\in S}B_{\{s\}}(f_0,\e)\subset B_K(f_0,3\e).\]
It implies that the subbasis element contains an open set in $\cT_p\vert_\cF$.
Therefore, $\cT_c\vert_\cF\subset\cT_p\vert_\cF$.
\end{proof}


\section{Applications}

\end{document}

references



