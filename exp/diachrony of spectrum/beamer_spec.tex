\documentclass{beamer}
\usepackage{../../ikany}

\title{Diachrony of spectrum}
\author{Ikhan Choi}

\begin{document}
\maketitle

\section{Introduction}
\begin{frame}
\begin{defn}
	Let $R$ be a commutative ring.
	The \emph{spectrum} of $R$ is the set of prime ideals of $R$.
\end{defn}
\pause Question: Why is it defined like this?
\end{frame}


\section{Hydrogen atom}
\begin{frame}
\frametitle{Hydrogen spectral series}
\end{frame}
\begin{frame}
\frametitle{Hilbert spaces in quantum mechanics}
\end{frame}
\begin{frame}
\frametitle{Rydberg's formula : Bohr model}
\end{frame}
\begin{frame}
\frametitle{Rydberg's formula : Schr\"odinger equation}
\end{frame}


\section{Spectral theory of elliptic equations}
\begin{frame}
\frametitle{Separation of variables}
\end{frame}
\begin{frame}
\frametitle{Spectral theorem of normal matrices}
\end{frame}
\begin{frame}
\frametitle{Spectral theorem of compact operators}
\end{frame}
\begin{frame}
\frametitle{Spectral theorem of elliptic operators}
\end{frame}


\section{Gelfand theory}
\begin{frame}
\frametitle{Banach algebras and $C^*$-algebras}
\end{frame}
\begin{frame}
\frametitle{Example 1 : Bounded operators}
\end{frame}
\begin{frame}
\frametitle{Example 2 : Continuous functions}
\end{frame}
\begin{frame}
\frametitle{Spectra, multiplicative homomorphisms, maximal ideals}
\end{frame}
\begin{frame}
\frametitle{Gelfand-Naimark theorem}
\end{frame}


\section{Algebraic geometry}
\begin{frame}
\frametitle{Algebraic variety}
\end{frame}
\begin{frame}
\frametitle{Coordinate ring}
\end{frame}
\begin{frame}
\frametitle{Maximal ideal is a point}
\end{frame}
\begin{frame}
\frametitle{Problem of unified codomains}
\end{frame}
\begin{frame}
\frametitle{Functoriality}
\end{frame}

\end{document}