\documentclass{beamer}
\usepackage{../../ikany}

\title{Diachrony of spectrum}
\author{Ikhan Choi}

\begin{document}
\maketitle

\section{Introduction}
\begin{frame}
  \begin{defn}
    Let $R$ be a commutative ring.
    The \emph{spectrum} of $R$ is the set of prime ideals of $R$.
  \end{defn}
  \pause Question: Why is it defined like this?
\end{frame}


\section{Hydrogen atom}
\begin{frame}
\frametitle{Hydrogen spectral series}
\end{frame}

\begin{frame}
\frametitle{Hilbert spaces in quantum mechanics}
\end{frame}

\begin{frame}
\frametitle{Rydberg's formula : Bohr model}
  Bohr's postulates:\pause
  \begin{itemize}[<+->]
    \item The electrons are on certian stable orbits.
    \item The stationary orbits are computed by the old quantization assumption for angular momenta:
    \[mvr=n\hbar.\]
    \item An electron absorbs or emits light frequency $f$ when they jump from an orbit to another, satisfying
    \[\Delta E=hf.\]
  \end{itemize}
\end{frame}

\begin{frame}
\frametitle{Rydberg's formula : Bohr model}
From the three relations
\[mvr=n\hbar,\quad\frac{mv^2}r=k\frac{e^2}{r^2},\quad E=K+V=\frac12mv^2-k\frac{e^2}r,\]
\pause we deduce
\[E=-\frac{k^2e^4m}{2\hbar^2}\frac1{n^2}.\]
\end{frame}

\begin{frame}
\frametitle{Rydberg's formula : Schr\"odinger equation}
\end{frame}


\section{Spectral theory of elliptic equations}
\begin{frame}
\frametitle{Separation of variables}
\end{frame}
\begin{frame}
\frametitle{Spectral theorem of normal matrices}
\end{frame}
\begin{frame}
\frametitle{Spectral theorem of compact operators}
\end{frame}
\begin{frame}
\frametitle{Spectral theorem of elliptic operators}
\end{frame}


\section{Gelfand theory}
\begin{frame}
\frametitle{Banach algebras and $C^*$-algebras}
\end{frame}
\begin{frame}
\frametitle{Example 1 : Bounded operators}
\end{frame}
\begin{frame}
\frametitle{Example 2 : Continuous functions}
\end{frame}
\begin{frame}
\frametitle{Spectra, multiplicative homomorphisms, maximal ideals}
\end{frame}
\begin{frame}
\frametitle{Gelfand-Naimark theorem}
\end{frame}


\section{Algebraic geometry}
\begin{frame}
\frametitle{Algebraic variety}
\end{frame}
\begin{frame}
\frametitle{Coordinate ring}
\end{frame}
\begin{frame}
\frametitle{Maximal ideal is a point}
\end{frame}
\begin{frame}
\frametitle{Problem of unified codomains}
\end{frame}
\begin{frame}
\frametitle{Functoriality}
\end{frame}

\end{document}