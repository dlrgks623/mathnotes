\documentclass{../exp}
\usepackage{../../ikany}

\title{Vector calculus}

\begin{document}
\maketitle
\tableofcontents

In this notes, we use some conventions:
\begin{cond}
\item Every function is sufficiently differentiable,
\item $I$ is either an open interval containing 0 or the closed interval $[0,1]$.
\end{cond}

Let $M$ denote a certain nice subset of the space $\R^3$.
The set $M$ is our own world; we hope to expand the calculus theory on the set $M$. 
For example, in this note, we may think $M$ as one of the following:
\begin{cond}
\item $\R^3$,
\item an open subset of $\R^3$,
\item $S^2=\{(x,y,z)\in\R^3:x^2+y^2+z^2=1\}$.
\end{cond}
These are examples of a special class of spaces what we call ``manifold''.
From Section 3, we restrict the situation for the case $M=\R^3$.

Note that $M$ and $\R^3$ are different even when $M=\R^3$ ``as a set'' because we cannot say they are equal as vector spaces: $M$ does not have an intrinsic vector space structure like vector addition and scalar multiplication, while $\R^3$ has.
In other words, we are assumed to be basically prohibited to apply vector addition or scalar multipication among ``points'' in $M$, while they can be done among ``vectors'' in $\R^3$, if nothing is mentioned.
Thus, readers may think we are having declared that we will not add or multiply by scalars anything if the notation $M$ is used instead of $\R^3$.

In vector calculus, there are extremely various way to formulate and construct the theory.
Every course and note just chose one of them according to their own purposes and levels of students.
It is same for me.







\section{Vector fields}
Our objective is to understand what are vector fields.


\subsection{Tangent vectors of curves}
A curve on $M$ can be defined by a function $\gamma:I\to M$.
When we differentiate $\gamma$, the vector space structure of $\R^3$ that contains $M$ is used.
So we will temporarily allow points in $M$ to be summed and multiplied by scalars.
This is why 
\begin{defn}
Let $\gamma:I\to M$ be a curve such that $\gamma(t)=p$.
Consider $\gamma$ as a curve on $\R^3$ using the inclusion $M\subset\R^3$.
Then, the \emph{tangent vector} of $\gamma$ at $t$ or at $p$ is a vector in $\R^3$ defined by
\[\gamma'(t):=\lim_{h\to0}\frac{\gamma(t+h)-\gamma(t)}{h}.\]
\end{defn}
The definition looks so familiar, but never let your guard down.
We must note that $\gamma(t)\in M$ but $\gamma'(t)\in\R^3$.
Generally, $\gamma'$ might be outside $M$, and tangent vectors are ``vector''.

Points are not vectors; in some special cases we describe points with vectors, but it is just for special cases.


\subsection{Tangent vectors at a point}

Vectors usually used in Newtonian mechanics or linear algebra do not care the starting point.
However, when we consider vector fields, we must care the points at which vectors are attched.
In particular, we often consider the tangent vectors attached at a point.
The term ``vector field'' can be seen as an abbreviation of ``tangent vector field''.
From the definition of tangent vectors of curves, we want to forget the curves in order to consider tangent vectors at a point.
There are the definitions.

\begin{defn}
A vector $v$ in $\R^3$ is called a \emph{tangent vector at a point} $p\in M$ if there is a curve $\gamma:I\to M$ such that
\[\gamma(0)=p\quad\text{and}\quad\gamma'(0)=v.\]
\end{defn}

\begin{ex}
Let $M=S^2$, $p=(1,0,0)\in M$, and $v=(1,0,0)\in\R^3$.
Then, $v$ is not a tangent vector at $p$ since there is no curve $\gamma:I\to S^2$ such that $\gamma(0)=(1,0,0)$ and $\gamma'=(1,0,0)$.
\end{ex}


\begin{defn}
Let $p\in M$ be a point.
The \emph{tangent space at} $p$ is the set of tangent vectors at $p$ and denoted by $T_pM$.
\end{defn}
Here are some examples.
\begin{ex}
If $M$ is open in $\R^3$, then $T_pM=\R^3$ for every $p\in M$.
\end{ex}
\begin{ex}
For $p=(1,0,0)\in S^2$, we have $T_pS^2=\{(x,y,z)\in\R^3:x=0\}$.
\end{ex}


\subsection{Tangent vectors as derivations}

\begin{defn}[Derivation]
Let 
\end{defn}
\begin{defn}
Let $f:M\to\R$ and $X_p\in T_pM$.
We can view $X_p$ as a function 
$X_p:C^\oo(\R^3)\to\R$ as follows:
\[X_pf:=\evalat{\dd{t}f\o\gamma}{t=0}\]
for some a curve $\gamma:I\to M$ satisfying $\gamma'(0)=X_p$.
\end{defn}
\begin{prop}
The function $X_p:C^\oo(\R^3)\to\R$ is well defined: the $X_pf$ is independent of choice of the curve $\gamma$.
\end{prop}

partial derivatives
\[\pd{x}:=\gamma'(0)\quad\text{for}\quad \gamma(t)=(t,0,0).\]

\subsection{Vector fields}























\section{Differential forms}
\subsection{Total derivative of functions}
A scalar field is just a function.
\[df(X):=Xf\]






















\section{Inner product}
\subsection{Musical isomorphisms}
\subsection{Hodge dual}
















\section{Stokes' theorem}





\section{Coordiates transformation}
\subsection{Total derivative of maps}
map -> curve to curve, pushforward operator
vector spaces -> frechet, gateaux



\end{document}