\documentclass{../exp}
\usepackage{../../ikany}

\title{Interchanging Limits, Derivatives, and Integrals}

\begin{document}
\maketitle

\section{Limit and derivative}

$f_n$ pointwisely, $Df_n$ uniformly








\section{Limit and integral}
We want to find a criterion for 
This question asks the convergence
\[f_n\to f\quad\text{in}\quad L^1.\]

\begin{thm}[Lebesgue dominated convergence theorem]
Let $\{f_\alpha\}_\alpha$ be a net of measurable functions $(X,\mu)\to\R$.
Define a maximal function
\[Mf(x)=\sup_\alpha|f_\alpha(x)|.\]

If $\|Mf\|_1<\oo$, then
\[\lim_\alpha|f_\alpha(x)-f(x)|=0\quad a.e.\impl\lim_\alpha\|f_\alpha-f\|_{L^1}=0.\]
\end{thm}

continuity application








\section{Derivative and integral}

\begin{thm}[Scheffe]
Let $\{f_n\}_n$ be a sequence of nonnegative functions in $L^1$.
Suppose it converges to $f$ pointwisely.
Then,
\[\lim_{n\to\oo}\|f_n\|_1=\|f\|_1\iff\lim_{n\to\oo}\|f_n-f\|_1=0.\]
\end{thm}

Define the Newton quotient as
\[D_hf(t,x):=\frac{f(t+h,x)-f(t,x)}h\]
for $h\ne0$.
We mainly recognize $D_h$ as an operator that maps $f(0,x)$ to a function of $x$.
Then, we can say that the partial derivative $\pd_tf(0,x)$ is well-defined a.e. $x$ if and only if
\[\lim_{h\to0}D_hf(0,x)=\pd_tf(0,x)\qquad\text{a.e. }x.\]

We may ask about conditions for the following to hold:
\[\lim_{h\to0}D_hf(0,x)=\pd_tf(0,x)\qquad\text{in }L_x^1(X).\]
This question naturally arise because it implies the commutability
\[\dd{t}\int f(t,x)\,dx=\int\pd{t}f(t,x)\,dx\]
at $t=0$.
As necessary conditions to formalize the statement, we must basically assume that $f(t,x)\in L_x^1$ for $|t|<\e$, and $\pd_tf(0,x)\in L_x^1$.
Above this, if we give a stronger condition $\esssup_{|t|<\e}|\pd_tf(t,x)|\in L_x^1$ than $\pd_tf(0,x)\in L_x^1$, then the $L_x^1$ convergence is obtained.
\begin{thm}[Leibniz rule]
Let $f:(-\e,\e)\times X\to\R$ be a curve of integrable functions such that $f(t,x)$ is absolutely continuous in $t$ for a.e. $x$.
If
\[\esssup_{|t|<\e}|\pd_tf(t,x)|\]
is in $L_x^1$, then
\[\lim_{h\to0}D_hf(0,x)=\pd_tf(0,x)\qquad\text{in }L_x^1(X).\]
\end{thm}
\begin{pf}
Our strategy is 
Define a maximal function
\[Mf(x):=\sup_{|h|<\e}|D_hf(0,x)|.\]
Since
\[D_hf(0,x)=\frac{f(h,x)-f(0,x)}h=\frac1h\int_0^h\pd_tf(t,x)\,dt,\]
we have
\begin{align*}
|D_hf(0,x)|&\le\frac1h\int_0^h|\pd_tf(t,x)|\,dt\\
&\le\esssup_{|t|\le|h|}|\pd_tf(t,x)|\left[\frac1h\int_0^h\,dt\right]\\
&\le\esssup_{|t|<\e}|\pd_tf(t,x)|.
\end{align*}
Since the right hand side is constant with respect to $h$, we can deduce that $Mf$ is in $L_x^1$.
The pointwise 
Applying the Lebesgue dominated convergence theorem, we get the desired result...???
\end{pf}
\begin{rmk}
If $f$ is assumed to be differentiable everywhere on $t\in(-\e,\e)$, then we may use the mean value theorem to prove the theorem instead of the differentiation theorem:
we directly get
\[|D_hf(0,x)|\le\sup_{|t|<\e}|\pd_tf(t,x)|.\]
If $f$ is assumed to be continuously differentiable on $t\in(-\e,\e)$, then 
\end{rmk}


$F$ is absolutely continuous,
\[\pd_tF=f\iff F(x,t)=\int_c^tf(x,s)\,dx.\]
Then, for
\[T_hf(x,0):=\frac1h\int_0^hf(x,s)\,ds,\]

For $\|f\|_{L_x^1L_t^\oo}=\|\sup_t|f(x,t)|\|_{L_x^1}<\oo$
we have
\begin{align*}
|T_hf(x,0)|&\le\frac1h\int_0^h|f(x,s)|\,ds\\
&\le\left[\frac1h\int_0^h\,ds\right]\cdot\sup_t|f(x,t)|\\
&=\sup_t|f(x,t)|.
\end{align*}
Thus,
\[Mf(x,0)=\sup_h|T_hf(x,0)|\le\sup_t|f(x,t)|\in L_x^1.\]

Since $f(x,0)\in L_x^1$, by the Lebesgue differentiation theorem, we get
\[\lim_{h\to0}T_hf(x,0)=f(x,0)\]
for a.e. $x$.











\end{document}