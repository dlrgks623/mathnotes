\documentclass{../exp}
\usepackage{../../ikany}

\def\a{\alpha}

\title{Classical differential geometry}
\setcounter{tocdepth}{3}

\begin{document}
\maketitle
\tableofcontents



%   좌표
%    매개화 좌표
%   미분 
%    <<바깥의 공변미분 사용>>
%    매개화를 미분하기 -> 곡선의 접벡터(프라임/디디티), 곡면의 접벡터(아래첨자/라운드)
%                     곡선을 통한 접공간 정의
%    접벡터로 미분하기 -> 스칼라장과 벡터장, 곡선비의존, 곡선미분, 기저벡터미분
%           1은 x지수, 2는 지수
%           접벡터로 미분해도 전부 R3의 공변미분으로 미분하는 것
%           4장 전까지는 접벡터를 곡면 위의 미분작용소로 인식하지 말기!!!
%   접공간
%    안쪽기저와 바깥기저
%    내적
%    외적
%
%         선형독립    벡터들이 한점에서 주어졌을 때 ->
%         선형독립    벡터장이 근방에서 주어졌을 때 -> 일반적으론 2차원에서만
%         선형독립 가환벡터장이 근방에서 주어졌을 때 -> n차원 다돼
%         선형독립    직교벡터들이 한점에서 주어졌을 때
%         선형독립    직교벡터장이 근방에서 주어졌을 때
%         선형독립 직교가환벡터장이 근방에서 주어졌을 때
%             -> 곡률의 선, 점근곡선, 측지좌표

% Local theory
% -이론
%   측정량의 계산식
%    기본적으로 -> 기저 벡터를 미분하면 어떻게 되느냐
%    곡선:
%         기저깔기: 티엔비, 오소노멀
%         프레네세레: 카파, 타우
%    곡면:
%         기저깔기: 접벡터 두개+법선벡터, 제1기본형식
%         접벡터 미분: 크리스토펠
%         법벡터 미분: 법선곡률, 제2기본형식(모양연산, 바인갈텐), 주곡률, 평균곡률, 가우스곡률
%           곡률텐서
%           additional topic 좌표의존성과 바깥의존성

% -계산
%   평면곡선, 구면곡선, 헬릭스, 버트런드
%   그래프, 회전체, 룰드곡면(접선), 프리이미지
%   뿔, 데벨로퍼블, 엘립소이드, 파라볼로이드, 쌍곡면, 헬리코이드; 가우스곡률 부호

% Intrinsic geometry of surface
% 평행수송, 공변미분, 측지선, 측지꼬임/곡률, 측지완비/호프리노프, 지수사상, 야코비장, 아다마르

% Global theory
%  곡선: 등주부등식, (네 꼭짓점, 펜첼/페리-밀너), 볼록성/오발
%  곡면: 최소곡면, (컴팩트곡면분류, 가우스-보네), 임베딩문제,


\section*{Acknowledgement}
This note is written for teaching during the undergraduate tutoring program in 2019 fall semester.
Main resources I refered are books by Manfredo P. do Carmo \cite{}, and Richard S. Millman and George D. Parker \cite{}.

\clearpage
\section{Introduction}
\subsection{Parametrizations and coordinates}
For each text on classical differential geometry, the definitions frequently vary.
In this note, we define as follows.
\begin{defn}
An \emph{$m$-dimensional parametrization} is a smooth map $\a:U\to\R^n$ such that
\begin{cond}
\item $U\subset\R^m$ is open,
\item $\a$ is one-to-one (optional),
\item the Fr\'echet derivative $d\a:U\times\R^m\to\R^n$ is injective everywhere.
\end{cond}
The Euclidean space $\R^n$ is called the \emph{ambient space}.
\end{defn}

The first condition is necessary to avoid differentiating at points that are not in the interior of domain.
Of course, it is possible to generalize the definition of differentiation on boundary points, but we will not introduce the notion for simplicity.

For the second condition, although it is written to be optional, we will always require the injectivity of $\a$ in this note.
If not, two distinct ordered tuple of real numbers may represent the same point.
Namely, this second condition allows to use the inverse map $\a^{-1}:\im\a\to U$.
To describe a geometric object that cannot be covered by a single injective parametrization, such as a circle or a sphere, we can admit several parametrizations.

The third condition is the most important one.
This condition is paraphrased as follows: the set of partial derivatives $\{\pd_i\a(x)\}_{i=1}^m\subset\R^n$ is linearly independent at every point $x\in U$.
Differential geoemtry do not consider parametrizations that fail this.
This condition is necessary for providing with appropriate and well-defined linear approximation of curves or surfaces.
If it is not staisfied, every definition including tangent spaces in differential geometry can suffer.

\begin{defn}
A subset $M\subset\R^n$ is called a \emph{regular curve} (resp. \emph{regular surface}) if there exists a one-dimensional (resp. two-dimensional) parametrization whose image is exactly $M$.
\end{defn}

All curves and surfaces in this note are assumed to be regular.
We also just often say that $\a$ is a regular curve (resp. regular surface) for \emph{a particular parametrization $\a$}.
However, note that a curve or surface admits infinitely many parametrizations.
We can solve many geometry or physics problems very easily by choosing an appropriate parametrization.
Related to the choice of parametrizations, the following issues are always importantly considered when developing a theory of differential geometry:
\begin{itemize}
\item Well-definedness of a structure with respect to the dependency on parametrizations(coordinates).
\item Existence of a parametrization(coordinates) that has nice properties we want.
\end{itemize}

\begin{defn}
Let $M\subset\R^n$ be a regular curve or a regular surface.
The inverse $\f:M\to U$ of a parametrization is called a \emph{coordinate map}.
\end{defn}

Coordinates and parametrizations have equivalent information except that the direction is opposite (only if parametrization satisfies the injectivity).
We mostly take a parametrization for a curve while coordinates are more usefully taken in more-than-one-dimensional geometry such as a surface, or the timespace.
We use the term \emph{reparametrization} to refer to nothing but a choice of another parametrization for the same curve or surface.
As said, the choice of coordinate(parametrization) is important in differential geometry.

\begin{ex}
Let $\a:\R\to\R^3$ be a map given by
\[\a(t)=(\cos t,\sin t, t).\]
Since $d\a|_t(1)=\a'(t)=(-\sin t,\cos t,1)$ is always nonzero so that $d\a$ is injective everywhere, $\a$ is a parametrization of the regular curve
\[\{\,(x,y,z)\in\R^3:x=\cos z,\ y=\sin z\,\}.\]
Notice that it is enough to check $\a'(t)\ne0$ for a curve parametrization $\a$ to show the injectivity of $d\a$.
This curve is called a circular helix.
\end{ex}
\begin{ex}
Let $\a:\R\to\R^3$ be a map given by
\[\a(t)=(t^3,t^6,t^9).\]
Since $\a'(t)=(3t^2,6t^5,9t^8)$ is zero when $t=0$, \emph{it would be better to avoid calling $\a$ a parametrization}.
Instead, the restrictions $\a_+:(0,\infty)\to\R^3$ and $\a_-:(-\infty,0)\to\R^3$ satisfy the axioms of parametrization at the beginning.

However, by reparametrization, we can show the image of $\a$ is a regular curve, that is, we can find a parametrization that shares the image with $\a$, even though \emph{we sometimes say that $\a$ is not a regular curve} according to the fact $\a'$ can vanish.
Consider $\beta:\R\to\R^3$ defined by
\[\beta(t)=(t,t^2,t^3).\]
This map has the same image $\im\a=\im\beta$, but $\beta'(t)=(1,2t,3t^2)\ne0$ for all $t\in\R$.
\end{ex}
\begin{ex}
Let $S^1$ be the unit circle in $\R^2$, precisely
\[S^1:=\{\,(x,y)\in\R^2:x^2+y^2=1\,\}.\]
It cannot be covered by a single parametrization, so we can consider two different parametrizations $\a:(0,2\pi)\to\R^2$ and $\beta:(\pi,3\pi)\to\R^2$ for $S^1$:
\[\alpha(t)=(\cos t,\sin t),\qquad\beta(t)=(\cos t,\sin t).\]
Then, we have $S^1=\im\a\cup\im\beta$.
If we want to investigate the geometry of $S^1$ near the point $(1,0)$, we can choose $\beta$ rather than $\a$ because $(1,0)\notin\im\a$.
\end{ex}
\begin{ex}
Let $f:\R^2\to\R$ be a smooth function and $\a:\R^2\to\R^3$ be a map given by
\[\a(x,y)=(x,y,f(x,y)).\]
Then, $\a$ is a two-dimensional parametrization because
\begin{align*}
d\a|_{(x,y)}(1,0)&=\pd{\a}{x}(x,y)=\Bigl(1,0,\pd{f}{x}(x,y)\Bigr),\\
d\a|_{(x,y)}(0,1)&=\pd{\a}{y}(x,y)=\Bigl(0,1,\pd{f}{y}(x,y)\Bigr)
\end{align*}
are linearly independent for every $(x,y)\in\R^2$.
A parametrization of this form is called a \emph{Monge patch}.
Notice that it is enough to check that the two partial derivatives $\pd_x\a$ and $\pd_y\a$ are linearly independent for a surface parametrization $\a$.

Let $S=\im\a$ be the regular surface determined by $\a$, and let $p$ be a point on the surface $S$ so that we have $p=(x,y,f(x,y))$.
Associated with $\a$, a coordinate map $\f:S\to\R^2$ can be defined as
\[\f(p)=(x,y).\]
This map $\f$ consists of two real-valued functions on $S$,
\[x:S\to\R:p\mapsto x,\qquad y:S\to\R:p\mapsto y.\]
In this regard, we often write the coordinates $\f$ as $(x,y)$.
\end{ex}
\begin{ex}
Let
\[S=\{\,(x,y)\in\R^2:x>0\text{ or }y\ne0\,\}.\]
The set $S$ is a regular surface.
Consider two different coordinates
\begin{gather*}
(x,y):S\to S:(x,y)\mapsto(x,y),\\
(r,\theta):S\to(0,\infty)\times(-\pi,\pi):(x,y)\mapsto\left(\sqrt{x^2+y^2},\ 2\tan^{-1}\frac y{\sqrt{x^2+y^2}+x}\right),
\end{gather*}
where $\tan^{-1}(t):=\int_0^t\frac{ds}{1+s^2}$.
They are the inverses of parametrizations $\a:S\to\R^2$ and $\beta:(0,\infty)\times(-\pi,\pi)\to\R^2$ defined by
\[\a(x,y)=(x,y),\qquad\beta(r,\theta)=(r\cos\theta,r\sin\theta).\]
The coordinate maps $(x,y)$ and $(r,\theta)$ are called \emph{Cartesian coordinates} and \emph{polar coordinates} respectively.
\end{ex}

 


\subsection{Differentiation}
Differentiation in differential geometry can be understood in many different viewpoints.
We, here, review the two kinds of main usages of differentiation: differentiation of parametrizations, and differentiation by directional vectors.
Do not forget that all differentiations in this note will be done thanks to the structure of the ambient space $\R^n$.

\subsubsection{Differentiation of parametrizations}
We introduce the notion of tangent spaces, geometrically the spaces of vectors that starts from each base point, by differentiation of parametrization.
In this note we define tangent spaces in several equivalent ways:
\begin{defn}
Let $M$ be a regular curve or a regular surface with an $m$-dimensional parametrization $\a:U\to M\subset\R^n$.
Let $p\in M$ be a point and $x=\a^{-1}(p)\in U$ be the coordinates of $p$.
The \emph{tangent space} of $M$ at $p$, denoted by $T_pM$, can be defined as either one of the followings:
\begin{cond}
\item the span of the linearly independent set of vectors $\{\pd_i\a(x)\}_{i=1}^m\subset\R^n$,
\item the image of the Fr\'echet derivative $d\a|_x:\R^m\to\R^n$,
\item the set of vectors $v\in\R^n$ such that there exists a curve $\gamma:I\to M$ satisfying $\gamma(0)=p$ and $\gamma'(0)=v$.
\end{cond}
We can check the definitions are independent on the parametrization $\a$, and that the tangent space $T_pM$ is an $m$-dimensional linear subspace of $\R^n$.
\end{defn}
\begin{rmk}
We can show the three conditions are equivalent, but the proof will not be given; what is more important is to understand the role and meaning of tangent spaces because there is no agreed standard definition of tangent spaces in the level of this note.
There exist a lot more neat but difficult characterizations for tangent spaces we will not cover.
\end{rmk}
\begin{rmk}
We can easily check that $T_p\R^3=\R^3$ for any $p\in\R^3$.
The notation $T_p\R^3$ will be used to emphasize that a vector in $\R^3$ is geometrically recognized to cast from the point $p$.
Since $T_p\R^3=\R^3=T_q\R^3$ for every pair of points $p,q\in\R^3$, summation and inner product of a vector in $T_p\R^3$ and a vector in $T_q\R^3$ make sense.
This identification of tangent spaces are allowed \emph{only for the case of linear spaces} such as $\R^3$.
(In fact, the identification $T_p\R^3=\R^3$ is \emph{natural} in categorical language.)
\end{rmk}
\begin{rmk}
One way to view tangent spaces is to see them as domains and codomains of Fr\'echet derivatives.
For open sets $U\subset\R^m$ and $V\subset\R^n$, the Fr\'echet derivative of a smooth map $f:U\to V$ at $x\in U$ is a linear transformation $df|_x:T_xU\to T_{f(x)}V$.
Since $T_xU=\R^m$ and $T_{f(x)}V=\R^n$, the original definition on Euclidean spaces agrees with it.
In this reason, the Fr\'echet derivative $df$ is also called a \emph{tangent map}, \emph{pushforward}, or \emph{differential} in differential geoemtry.
\end{rmk}

% Examples

\subsubsection{Differentiation by tangent vectors}

%%%%
% 벡터장과 스칼라장 정의

% 대입 기호
%  미분 d\a|_p
%  for a field s, s|_p
%  접벡터 X|_p
%  ex \pd_i\a는 접벡터
%   따라서 \pd_i\a|_p = \pd_i\a(\a^{-1}(p))

\begin{ex}
Let $\a$ be an $m$-dimensional parametrization.
Then, $\pd_i\a(x)$ is always a tanget vector at each point $p=\a(x)$, and $\pd_i\a$ becomes a vector field.
It means we may write
\[\pd_i\a|_p=\pd_i\a(x)=\pd_i\a(\a^{-1}(p)).\]
\end{ex}


% 벡터에 의한 방향미분
%  공변미분 아니면 \pd_v 기호만 사용할 것
%  \pd_x:=\pd_{\a_x}로 사용할 것
%  \del_v는 사용하지 말 것

\begin{ex}
Let $\a$ be an $m$-dimensional parametrization and $s$ be either a function or a vector field on $\a$.
Then, we can define the directional derivative $\pd_{\pd_i\a}s$ of $s$ with respect to the direction $\pd_i\a$.
Abusing notation
\[\pd_is:=\pd_{\pd_i\a}s\]
will be mostly used when we are given a fixed parametrization without confusion.
Since directional derivatives do not depend on the choice of curves $\gamma$ but the value of $\gamma(0)=p$ and $\gamma'(0)=\pd_i\a$, we can compute the directional derivative as
\[\pd_is=\pd_i(s\o\a)\] by taking $\gamma(t)=\a(\a^{-1}(p)+te_i)$, where $e_i$ is the $i$th standard basis vector in $\R^m$.
\end{ex}

\begin{ex}
Let $\a:\R^2\to\R^3$ be a regular surface given by
\[\a(x,y)=\left(\frac{2x}{1+x^2+y^2},\,\frac{2y}{1+x^2+y^2},\,1-\frac2{1+x^2+y^2}\right).\]
This map gives a (partial) parametrization for the sphere $S^2$, and is called the \emph{stereographic projection}.
Let $f:S^2\to\R$ be the height function of $\a$ defined by
\[f(p):=z\]
for $p=(x,y,z)$.
Then, the directional derivative is
\[f_x=\pd_xf=\pd{(f\o\a)}{x}=\pd{x}\left(1-\frac2{1+x^2+y^2}\right)=\frac{4x}{(1+x^2+y^2)^2}.\]
Note that $f_x\ne\pd{x}z=0$.
\end{ex}



% 아래첨자 기호
%  x,y: 매개화 및 필드의 x,y방향 "편미분"
%   for a field s,
%   s_x=\pd_xs:=\pd_{\a_x}s   (s\o\a)_x\o\a^{-1}=\pd_x(s\o\a)\o\a^{-1}
%  i,j 및 숫자: 섹션(=필드)의 성분 인덱스, 메트릭 및 크리스토펠 "인덱스"
%          매개화 편미분, 즉 벡터장에도 쓰임


% notations

% \a' \a_x \pd_i\a \pd_v\a \a_v
% f' f_x \pd_if \pd_vf f_v
% X^i X^1

\subsection{Linear algebra on tangent spaces}


%%%% 내적 외적 삼중곱
% 벡터의 좌표표현 안쪽 기저와 바깥 기저









\section{Local theory of curves}

\subsection{Theory}

\subsubsection{Reparametrization}

We introduce the arc-length reparametrization.
It is a general choice for the local study of curves.
\begin{defn}
A parametrization $\a$ of a regular curve is called a \emph{unit speed curve} or an \emph{arc-length parametrization} when it satisfies $\|\a'\|=1$.
\end{defn}
\begin{thm}
Every regular curve may be assumed to have unit speed.
Precisely, for every regular curve, there is a parametrization $\a$ such that $\|\a'\|=1$.
\end{thm}
\begin{pf}
By the definition of regular curves, we can take a parametrization $\beta:I_t\to\R^d$ for a given regular curve.
We will construct an arc-length parametrization from $\beta$.

Define $\tau:I_t\to I_s$ such that
\[\tau(t):=\int_0^t\|\beta'(s)\|\,ds.\]
Since $\tau$ is smooth and $\tau'>0$ everywhere so that $\tau$ is strictly increasing, the inverse $\tau^{-1}:I_s\to I_t$ is smooth by the inverse function theorem.
Define $\a:I_s\to\R^d$ by $\a:=\beta\circ\tau^{-1}$.
Then, by the chain rule,
\[\a'=\dd{\a}{s}=\dd{\beta}{t}\dd{\tau^{-1}}{s}=\beta'\left(\dd{\tau}{t}\right)^{-1}=\frac{\beta'}{\|\beta'\|}.\qedhere\]
\end{pf}




\subsubsection{Frenet-Serret frame}
The Frenet-Serret frame is defined for nondegenerate regular curves.
It provides with a useful orthonormal basis of $T_p\R^3\supset T_pC$ for points $p$ on a regular curve $C$.
\begin{defn}
We call a curve parametrized as $\a:I\to\R^3$ is \emph{nondegenerate} if the normalized tangent vector $\a'/\|\a'\|$ is never locally constant everywhere.
In other words, $\a$ is nowhere straight.
\end{defn}

\begin{defn}[Frenet-Serret frame]
Let $\a$ be a nondegenerate curve.
The \emph{tangent unit vector}, \emph{normal unit vector}, \emph{binormal unit vector} are $T_p\R^3$-valued vector fields on $\a$ defined by:
\[\rT(t):=\frac{\a'(t)}{\|\a'(t)\|},\qquad\rN(t):=\frac{\rT'(t)}{\|\rT'(t)\|},\qquad\rB(t):=\rT(t)\times\rN(t).\]
The set of vector fields $\{\rT,\rN,\rB\}$, which is called \emph{Frenet-Serret frame}, forms an orthonormal basis of $T_p\R^3$ at each point $p$ on $\a$.
The Frenet-Serret frame is uniquely determined up to sign as $\a$ changes.
\end{defn}



\subsubsection{Differentiation of Frenet-Serret frame}

We study the derivatives of the Frenet-Serret frame and their coordinate representations.
In the coordinate representations on the Frenet-Serret frame, important geometric measurements such as curvatrue and torsion come out as coefficients.

\begin{defn}
Let $\a$ be a nondegenerate curve.
The \emph{curvature} and \emph{torsion} are scalar fields on $\a$ defined by:
\[\kappa(t):=\frac{\<\rT'(t),\rN(t)\>}{\|\a'\|},\quad\tau(t):=-\frac{\<\rB'(t),\rN(t)\>}{\|\a'\|}.\]
Note that $\kappa>0$ cannot vanish by definition of nondegenerate curve.
This definition is independent on $\a$.
\end{defn}

\begin{thm}[Frenet-Serret formula]
Let $\a$ be a nondegenerate curve.
Then,
\[\begin{pmatrix}\rT'\\\rN'\\\rB'\end{pmatrix}=\|\a'\|\begin{pmatrix}0&\kappa&0\\-\kappa&0&\tau\\0&-\tau&0\end{pmatrix}\begin{pmatrix}\rT\\\rN\\\rB\end{pmatrix}.\]
\end{thm}
\begin{pf}
Note that $\{\rT,\rN,\rB\}$ is an orthonormal basis.
We first show the first and third rows, and the second row later.

\Step{1}[Show that $\rT',\rB',\rN$ are parallel]
Two vectors $\rT'$ and $\rN$ are parallel by definition of $\rN$.
Since $\<\rT,\rB\>=0$ and $\<\rB,\rB\>=1$ are constant, we have
\[\<\rB',\rT\>=\<\rB,\rT\>'-\<\rB,\rT'\>=0,\qquad\<\rB',\rB\>=\tfrac12\<\rB,\rB\>'=0,\]
which show $\rB'$ and $\rN$ are parallel.
By the definition of $\kappa$ and $\tau$, we get
\[\rT'=\|\a'\|\kappa\rN,\qquad\rB'=-\|\a'\|\tau\rN.\]

\Step{2}[Describe $\rN'$]
Since
\begin{align*}
\<\rN',\rT\>&=-\<\rN,\rT'\>=-\|\a'\|\kappa,\\
\<\rN',\rN\>&=\tfrac12\<\rN,\rN\>'=0,\\
\<\rN',\rB\>&=-\<\rN,\rB'\>=\|\a'\|\tau,
\end{align*}
we have
\[\rN'=\|\a'\|(-\kappa\rT+\tau\rB).\qedhere\]
\end{pf}
\begin{rmk}
Let $\rX(t)$ be the curve of orthogonal matrices $(\rT(t),\rN(t),\rB(t))^T$.
Then, the Frenet-Serret formula reads
\[\rX'(t)=A(t)\rX(t)\]
for a matrix curve $A(t)$ that is completely determined by $\kappa(t)$ and $\tau(t)$.
This is a typical form of an ODE system, so we can apply the Picard-Lindel\"of theorem to get the following proposition: if we know $\kappa(t)$ and $\tau(t)$ for all time $t$, and if $\rT(0)$ and $\rN(0)$ are given so that an initial condition
\[\rX(0)=(\rT(0),\,\rN(0),\,\rT(0)\times\rN(0))\]
is established, then the solution $\rX(t)$ exists and uniquely determined in a short time range.
Furthermore, if $\a(0)$ is given in addition, the integration
\[\a(t)=\a(0)+\int_0^t\rT(s)\,ds\]
provides a complete formula for unit speed parametrization $\a$.
\end{rmk}
\begin{rmk}
Skew-symmetry in the Frenet-Serret formula is not by chance.
Let $\rX(t)=(\rT(t),\rN(t),\rB(t))^T$ and write $\rX'(t)=A(t)\rX(t)$ as we did in the above remark.
Since $\rX(t+h)=R_t(h)\rX(t)$ for a family of special orthogonal matrices $\{R_t(h)\}_h$ with $R_t(0)=I$, we can describe $A(t)$ as 
\[A(t)=\left.\dd{R_t}{h}\right\rvert_{h=0}.\]
By differentiating the relation $R_t^T(h)R_t(h)=I$ with respect to $h$, we get to know that $A(t)$ is skew-symmetric for all $t$.
In other words, the tangent space $T_I\SO(3)$ forms a skew symmetric matrix.
\end{rmk}



\subsubsection{Formulas for computation}

The following proposition gives the most effective and shortest way to compute $\kappa$ and $\tau$ in general case.
If we try to find $\kappa$ by differentiating $\rT$, then we must encounter the normalizing term of the form $\sqrt{(-)^2+(-)^2+(-)^2}^{-1}$, and it is painful when time is limited.
The Frenet-Serret frame is useful in proofs of interesting propositions, but not a good choice for practical computation.
So, the direct computation from derivatives of parametrization is highly recommended, instead of differentiating $\rT$.
\begin{prop}
Let $\a$ be a nondegenerate curve.
\[\kappa=\frac{\|\a'\x\a''\|}{\|\a'\|^3},\qquad\tau=\frac{\a'\x\a''\cdot\a'''}{\|\a'\x\a''\|}.\]
\end{prop}
\begin{pf}
If we let $s=\|\a'\|$, then
\begin{align*}
\a'&=s\rT,\\
\a''&=s'\rT+s^2\kappa\rN,\\
\a'''&=(s''-s^3\kappa^2)\rT+(3ss'\kappa+s^2\kappa')\rN+(s^3\kappa\tau)\rB.
\end{align*}
Now the formulas are easily derived.
\end{pf}




\subsection{Problems}
We are interested in regular curves, not a particular parametrization.
By the Theorem 2.1, we may always assume that a parametrization $\a$ has unit speed.
Let $\a$ be a nondegenerate unit speed space curve, and let $\{\rT,\rN,\rB\}$ be the Frenet-Serret frame for $\a$.
Consider a diagram as follows:
\begin{cd}
\<\a,\rT\>=\ ?\ar{r}\ar{d} & \<\a,\rN\>=\ ? \ar{l}\ar{d}\ar{r} & \<\a,\rB\>=\ ? \ar{l}\ar{d} \\
\<\a',\rT\>=1 & \<\a',\rN\>=0 &\<\a',\rB\>=0.
\end{cd}
Here the arrows indicate which term we are able to get by differentiation.
For example, if we know a condition
\[\<\a(t),\rT(t)\>=f(t),\]
then we can obtain
\[\<\a(t),\rN(t)\>=\frac{f'(t)-1}{\kappa(t)}\]
by direct differentiation since we have known $\<\a',\rT\>$ but not $\<\a,\rN\>$.
Further, we get
\[\<\a(t),\rB(t)\>=\frac{\left(\frac{f'(t)-1}{\kappa(t)}\right)'+\kappa(t)f(t)}{\tau(t)}\]
since we have known $\<\a,\rT\>$ and $\<\a',\rN\>$ but not $\<\a,\rB\>$.
Thus, $\<\a,\rT\>=f$ implies
\[\a(t)=f(t)\cdot\rT+\frac{f'(t)-1}{\kappa(t)}\cdot\rN+\frac{\left(\frac{f'(t)-1}{\kappa(t)}\right)'+\kappa(t)f(t)}{\tau(t)}\cdot\rB,\]
when given $\tau(t)\ne0$.

We suggest a strategy for space curve problems:
\begin{itemize}
\item Build and differentiate equations of the following form:
\[\<\ \text{(interesting vector)},\ \text{(Frenet-Serret basis)}\ \>\ =\ \text{(some function)}.\]
\item Aim for finding the coefficients of the position vector in the Frenet-Serret frame, and obtain relations of $\kappa$ and $\tau$ by comparing with assumptions.
\item Heuristically find a constant vector and show what you want directly.
\end{itemize}
Here we give example solutions of several selected problems.
Always $\a$ denotes a reparametrized unit speed nondegenerate curve in $\R^3$.



\begin{prb}
A curve whose normal lines always pass through a fixed point lies in a circle.
\end{prb}
\begin{sol}
\Step{1}[Formulate conditions]
By the assumption, there is a constant point $p\in\R^3$ such that the vectors $\a-p$ and $\rN$ are parallel so that we have
\[\<\a-p,\rT\>=0,\qquad\<\a-p,\rB\>=0.\]
Our goal is to show that $\|\a-p\|$ is constant and there is a constant vector $v$ such that $\<\a-p,v\>=0$.

\Step{2}[Collect information]
Differentiate $\<\a-p,\rT\>=0$ to get
\[\<\a-p,\rN\>=-\frac1\kappa.\]
Differentiate $\<\a-p,\rB\>=0$ to get
\[\tau=0.\]

\Step{3}[Complete proof]
We can deduce that $\|\a-p\|$ is constant from
\[(\|\a-p\|^2)'=\<\a-p,\a-p\>'=2\<\a-p,\rT\>=0.\]
Also, if we heuristically define a vector $v:=\rB$, then $v$ is constant since
\[v'=-\tau\rN=0,\]
and clearly $\<\a-p,v\>=0$
\end{sol}

\begin{prb}
A spherical curve of constant curvature lies in a circle.
\end{prb}
\begin{sol}
\Step{1}[Formulate conditions]
The condition that $\a$ lies on a sphere can be given as follows: for a constant point $p\in\R^3$,
\[\|\a-p\|=\const.\]
Also we have
\[\kappa=\const.\]

\Step{2}[Collect information]
Differentiate $\|\a-p\|^2=\const$ to get
\[\<\a-p,\rT\>=0.\]
Differentiate $\<\a-p,\rT\>=0$ to get
\[\<\a-p,\rN\>=-\frac1\kappa.\]
Differentiate $\<\a-p,\rN\>=-1/\kappa=\const$ to get
\[\tau\<\a-p,\rB\>=0.\]

There are two ways to show that $\tau=0$.

\emph{Method 1}:
Assume that there is $t$ such that $\tau(t)\ne0$.
By the continuity of $\tau$, we can deduce that $\tau$ is locally nonvanishing.
In other words, we have $\<\a-p,\rB\>=0$ on an open interval containing $t$.
Differentiate $\<\a-p,\rB\>=0$ at $t$ to get $\<\a-p,\rN\>=0$ near $t$, which is a contradiction.
Therefore, $\tau=0$ everywhere.

\emph{Method 2}:
Since $\<\a-p,\rB\>$ is continuous and
\[\<\a-p,\rB\>=\pm\sqrt{\|\a-p\|^2-\<\a-p,\rT\>^2-\<\a-p,\rN\>^2}=\pm\const,\]
we get $\<\a-p,\rB\>=\const$.
Differentiate to get $\tau\<\a-p,\rN\>=0$.
Finally we can deduce $\tau=0$ since $\<\a-p,\rN\>\ne0$.

\Step{3}[Complete proof]
The zero torsion implies that the curve lies on a plane.
A planar curve in a sphere is a circle.
\end{sol}

\begin{prb}
A curve such that $\tau/\kappa=(\kappa'/\tau\kappa^2)'$ lies on a sphere.
\end{prb}
\begin{sol}
\Step{1}[Find the center heuristically]
If we assume that $\a$ is on a sphere so that we have $\|\a-p\|=r$ for constants $p\in\R^3$ and $\r>0$, then by the routine differentiations give
\[\<\a-p,\rT\>=0,\qquad\<\a-p,\rN\>=-\frac1\kappa,\qquad\<\a-p,\rB\>=-\left(\frac1\kappa\right)'\frac1\tau,\]
that is,
\[\a-p=-\frac1\kappa\rN-\left(\frac1\kappa\right)'\frac1\tau\rB.\]

\Step{2}[Complete proof]
Let us get started the proof.
Define
\[p:=\a+\frac1\kappa\rN+\left(\frac1\kappa\right)'\frac1\tau\rB.\]
We can show that it is constant by differentiation.
Also we can show that
\[\<\a-p,\a-p\>\]
is constant by differentiation.
So we are done.
\end{sol}

\begin{prb}
A curve with more than one Bertrand mates is a circular helix.
\end{prb}
\begin{sol}
\Step{1}[Formulate conditions]
Let $\beta$ be a Bertrand mate of $\a$ so that we have
\[\beta=\a+\lambda\rN,\qquad\rN_\beta=\pm\rN,\]
where $\lambda$ is a function not vanishing somewhere and $\{\rT_\beta,\rN_\beta,\rB_\beta\}$ denotes the Frenet-Serret frame of $\beta$.
We can reformulate the conditions as follows:
\begin{cd}[cells={text width=80pt, align=center}]
\<\beta-\a,\rT\>=0 \ar{r}\ar{d}& \<\beta-\a,\rN\>=\lambda \ar{r}\ar{l}\ar{d}& \<\beta-\a,\rB\>=0 \ar{l}\ar{d} \\
\<\rT_\beta,\rT\>=? \ar{r}\ar{d}& \<\rT_\beta,\rN\>=0 \ar{r}\ar{l}\ar{d}& \<\rT_\beta,\rB\>=? \ar{l}\ar{d} \\
\<\rN_\beta,\rT\>=0 \ar{r}& \<\rN_\beta,\rN\>=\pm1 \ar{r}\ar{l}& \<\rN_\beta,\rB\>=0 \ar{l}.
\end{cd}
Note that $\beta$ is not unit speed.

\Step{2}[Collect information]
Differentiate $\<\beta-\a,\rN\>=\lambda$ to get
\[\lambda=\const\ne0.\]
Differentiate $\<\beta-\a,\rT\>=0$ and $\<\beta-\a,\rB\>=0$ to get
\[\<\rT_\beta,\rT\>=\frac{1-\lambda\kappa}{\|\beta'\|},\qquad\<\rT_\beta,\rB\>=\frac{\lambda\tau}{\|\beta'\|}.\]
Differentiate $\<\rT_\beta,\rT\>$ and $\<\rT_\beta,\rB\>$ to get
\[\frac{1-\lambda\kappa}{\|\beta'\|}=\const,\qquad\frac{\lambda\tau}{\|\beta'\|}=\const.\]
Thus, there exists a constant $\mu$ such that
\[1-\lambda\kappa=\mu\lambda\tau\]
if $\a$ is not planar so that $\tau\ne0$.

We have shown that the torsion is either always zero or never zero at every point: $\lambda\tau/\|\beta'\|=\const$.
The problem can be solved by dividing the cases, but in this solution we give only for the case that $\a$ is not planar; the other hand is not difficult.

\Step{3}[Complete proof]
If
\[\beta_1=\a+\lambda_1\rN,\qquad\beta_2=\a+\lambda_2\rN\]
are different Bertrand mates of $\a$ with $\lambda_1\ne\lambda_2$, then $(\kappa,\tau)$ solves a two-dimensional linear system
\begin{pde*}
\kappa+\mu_1\tau&=\lambda_1^{-1},\\
\kappa+\mu_2\tau&=\lambda_2^{-1}.
\end{pde*}
It is nonsingular since $\mu_1=\mu_2$ implies $\lambda_1=\lambda_2$, which means we can represent $\kappa$ and $\tau$ in terms of constants $\lambda_1,\lambda_2,\mu_1,$ and $\mu_2$.
Therefore, $\kappa$ and $\tau$ are constant.
\end{sol}

Here is a well-prepared problem set for exercises.

\begin{prb}[Plane curves]
Let $\a$ be a nondegenerate curve in $\R^3$.
TFAE:
\begin{cond}
\item the curve $\a$ lies on a plane,
\item $\tau=0$,
\item the osculating plane constains a fixed point.
\end{cond}
\end{prb}

\begin{prb}[Helices]
Let $\a$ be a nondegenerate curve in $\R^3$.
TFAE:
\begin{cond}
\item the curve $\a$ is a helix,
\item $\tau/\kappa=\const$,
\item normal lines are parallel to a plane.
\end{cond}
\end{prb}

\begin{prb}[Sphere curves]
Let $\a$ be a nondegenerate curve in $\R^3$.
TFAE:
\begin{cond}
\item the curve $\a$ lies on a sphere,
\item $(1/\kappa)^2+((1/\kappa)'/\tau)^2=\const$,
\item $\tau/\kappa=(\kappa'/\tau\kappa^2)'$,
\item normal planes contain a fixed point.
\end{cond}
\end{prb}

\begin{prb}[Bertrand mates]
Let $\a$ be a nondegenerate curve in $\R^3$.
TFAE:
\begin{cond}
\item the curve $\a$ has a Bertrand mate,
\item there are two constants $\lambda\ne0,\mu$ such that $1/\lambda=\kappa+\mu\tau$.
\end{cond}
\end{prb}












\section{Local theory of surfaces}
\subsection{Theory}

\subsubsection{Reparametrization}

%         선형독립    벡터들이 한점에서 주어졌을 때 ->
%         선형독립    벡터장이 근방에서 주어졌을 때 -> 일반적으론 2차원에서만
%         선형독립 가환벡터장이 근방에서 주어졌을 때 -> n차원 다돼
%         선형독립    직교벡터들이 한점에서 주어졌을 때
%         선형독립    직교벡터장이 근방에서 주어졌을 때
%         선형독립 직교가환벡터장이 근방에서 주어졌을 때
% preimage theorem

\begin{thm}
Let $S$ be a regular surface.
Let $v,w$ be linearly independent vectors in $T_pS$ for a point $p\in S$.
Then, $S$ admits a parametrization $\a$ such that $\a_x|_p=v$ and $\a_y|_p=w$.
\end{thm}
\begin{thm}
Let $S$ be a regular surface.
Let $v,w$ be linearly independent vector fields on $S$.
Then, $S$ admits a parametrization $\a$ such that $\a_x|_p$ and $\a_y|_p$ are parallel to $v|_p,w|_p$ respectively for each $p\in S$.
\end{thm}
\begin{thm}
Let $S$ be a regular surface.
Let $v,w$ be linearly independent vector fields on $S$.
If $\pd_v\pd_wf=\pd_w\pd_vf$ for all smooth functions $f$, then $S$ admits a parametrization $\a$ such that $\a_x|_p=v|_p$ and $\a_y|=w|_p$ for each $p\in S$.
\end{thm}


\subsubsection{Gauss map}

\begin{defn}
Let $\a$ be a regular surface.
The \emph{Gauss map} or \emph{normal unit vector} $\nu:U\to\R^3$ is a $T_p\R^3$-valued vector field on $\a$ defined by:
\[\nu(x,y):=\frac{\a_x\times \a_y}{\|\a_x\times \a_y\|}(x,y).\]
The set of vector fields $\{\a_x,\a_y,\nu\}$ form a basis of $T_p\R^3$ at each point $p$ on $\a$.
The Gauss map is uniquely determined up to sign as $\a$ changes.
\end{defn}


\subsubsection{Differentiation of tangent vectors}
\begin{defn}
Let $\a$ be a regular surface.
The \emph{Christoffel symbols} refer to eight scalar functions $\{\Gamma_{ij}^k\}_{i,j,k=1}^2$ on $\a$ defined by
\begin{align*}
\pd_x\a_x=\a_{xx}&=:\Gamma_{11}^1\a_x+\Gamma_{11}^2\a_y+L\nu,\\
\pd_x\a_y=\a_{xy}&=:\Gamma_{12}^1\a_x+\Gamma_{12}^2\a_y+M\nu=\\
\pd_y\a_x=\a_{yx}&=:\Gamma_{21}^1\a_x+\Gamma_{21}^2\a_y+M\nu,\\
\pd_y\a_y=\a_{yy}&=:\Gamma_{22}^1\a_x+\Gamma_{22}^2\a_y+N\nu.
\end{align*}
The functions $L,M,$ and $N$ are not Christoffel symbols, and will be defined again later.
The Christoffel symbols \emph{do depend} on $\a$.
\end{defn}

The Christoffel symbols are deeply connected to the inner product structure of tangent spaces.
Let $S$ be a regular surface.
The inner product on $T_pS$ induced from the standard inner product of $\R^3$ can be represented not only as a matrix
\[\begin{pmatrix}1&0&0\\0&1&0\\0&0&1\end{pmatrix}\]
in the basis $\{(1,0,0),(0,1,0),(0,0,1)\}\subset\R^3$, but also as a matrix
\[\begin{pmatrix}\<\a_x,\a_x\>&\<\a_x,\a_y\>\\\<\a_y,\a_x\>&\<\a_y,\a_y\>\end{pmatrix}\]
in the basis $\{\a_x,\a_y\}\subset T_pS$.

\begin{defn}
Let $\a$ be a regular surface.
\[E:=\<\a_x,\a_x\>,\qquad F:=\<\a_x,\a_y\>,\qquad G:=\<\a_y,\a_y\>.\]
The following notations are also widely used in both geometry and physics:
\[g_{11}:=\<\a_x,\a_x\>,\qquad g_{12}=g_{21}:=\<\a_x,\a_y\>,\qquad g_{22}:=\<\a_y,\a_y\>.\]
\end{defn}


\subsubsection{Differentiation of normal vector}
\begin{defn}
Let $\a$ be a regular curve on a regular surface $S$.
The \emph{normal curvature} of $\a$ on $S$ is
\[\kappa_n(t):=\<\a''(t),\nu(t)\>\]
\end{defn}



% second fundamental form
% shape operator, weingarten map
% principal curvature
% mean curvature, gaussian curvature

% curvature tensor?

\subsubsection{Formulas for computations}


Weingarten equations:
\begin{align*}
\nu_x&=\frac{FM-GL}{EG-F^2}\a_x+\frac{FL-EM}{EG-F^2}\a_y,\\
\nu_x&=\frac{FN-GM}{EG-F^2}\a_x+\frac{FM-EN}{EG-F^2}\a_y.
\end{align*}


\end{document}