\documentclass{../exp}
\usepackage{../../ikany}

\def\a{\alpha}

\title{Classical differential geometry}
\setcounter{tocdepth}{3}

\begin{document}
\maketitle
\tableofcontents



%   좌표
%    매개화 좌표
%   미분 
%    <<바깥의 공변미분 사용>>
%    매개화를 미분하기 -> 곡선의 접벡터(프라임/디디티), 곡면의 접벡터(아래첨자/라운드)
%                     곡선을 통한 접공간 정의
%    접벡터로 미분하기 -> 스칼라장과 벡터장, 곡선비의존, 곡선미분, 기저벡터미분
%   접공간
%    안쪽기저와 바깥기저
%    내적
%    외적
%
%         선형독립    벡터들이 한점에서 주어졌을 때 ->
%         선형독립    벡터장이 근방에서 주어졌을 때 -> 일반적으론 2차원에서만
%         선형독립 가환벡터장이 근방에서 주어졌을 때 -> n차원 다돼
%         선형독립    직교벡터들이 한점에서 주어졌을 때
%         선형독립    직교벡터장이 근방에서 주어졌을 때
%         선형독립 직교가환벡터장이 근방에서 주어졌을 때
%             -> 곡률의 선, 점근곡선, 측지좌표

% Local theory
% -이론
%   측정량의 계산식
%    기본적으로 -> 기저 벡터를 미분하면 어떻게 되느냐
%    곡선:
%         기저깔기: 티엔비, 오소노멀
%         프레네세레: 카파, 타우
%    곡면:
%         기저깔기: 접벡터 두개+법선벡터, 제1기본형식
%         접벡터 미분: 크리스토펠
%         법벡터 미분: 법선곡률, 제2기본형식(모양연산, 바인갈텐), 주곡률, 평균곡률, 가우스곡률
%           곡률텐서
%           additional topic 좌표의존성과 바깥의존성

% -계산
%   평면곡선, 구면곡선, 헬릭스, 버트런드, 이볼루트
%   프리이미지, 그래프, 접선쓸기, 회전체
%   뿔, 엘립소이드, 파라볼로이드, 쌍곡면, 헬리코이드; 곡률 부호
%    룰드 곡면, 데벨로퍼블

% Intrinsic geometry of surface
% 평행수송, 공변미분, 측지선, 측지꼬임/곡률, 측지완비/호프리노프, 지수사상, 야코비장, 아다마르

% Global theory
%  곡선: 등주부등식, (네 꼭짓점, 펜첼/페리-밀너), 볼록성/오발
%  곡면: 최소곡면, (컴팩트곡면분류, 가우스-보네), 임베딩문제,


\section*{Acknowledgement}
This note is written for teaching during the undergraduate tutoring program in 2019 fall semester.
Main resources I refered are books by Manfredo P. do Carmo \cite{}, and Richard S. Millman and George D. Parker \cite{}.

\clearpage
\section{Introduction}
\subsection{Parametrizations and coordinates}
For each text on classical differential geometry, the definitions frequently vary.
In this note, we define as follows.
\begin{defn}
An \emph{$m$-dimensional parametrization} is a smooth map $\a:U\to\R^n$ such that
\begin{cond}
\item $U\subset\R^m$ is open,
\item $\a$ is one-to-one (optional),
\item the Fr\'echet derivative $d\a:U\times\R^m\to\R^n$ is injective everywhere.
\end{cond}
The Euclidean space $\R^n$ is called the \emph{ambient space}.
\end{defn}
\begin{rmk}
Although it is written that the second condition is optional, we will always require the injectivity of $\a$ in this note.
If not, two distinct ordered tuple of real numbers may represent the same point.
To describe a geometric object that cannot be covered by a single injective parametrization, such as a circle or a sphere, we can admit several parametrizations.
\end{rmk}
\begin{rmk}
The third condition is important.
This condition is paraphrased as follows: the set of partial derivatives $\{\pd_i\a(p)\}_{i=1}^m\subset\R^n$ is linearly independent at every point $p\in U$.
Differential geoemtry do not consider parametrizations that fail this; for example, $t=0$ should be excluded from the domain of a curve parametrization $t\mapsto(t^2,t^3,t^4)$.
\end{rmk}
\begin{defn}
A subset $M\subset\R^n$ is called a \emph{regular curve} (resp. \emph{regular surface}) if there exists a one-dimensional (resp. two-dimensional) parametrization whose image is exactly $M$.

We just often say that $\a$ is a regular curve (resp. regular surface) for a parametrization $\a$.
\end{defn}
\begin{defn}
Let $M\subset\R^n$ be a regular curve or a regular surface.
The inverse $\f:M\to U$ of a parametrization is called a \emph{coordinate map}.
\end{defn}

Reparametrization is just a choice of another parametrization for the same curve or surface.
The choice of coordinate(parametrization) is extremely important in differential geometry.

% Examples


\subsection{Differentiation}
Differentiation in differential geometry can be understood in many different viewpoints.
We, here, review the two kinds of main usages of differentiation: differentiation of parametrizations, and differentiation by directional vectors.
Do not forget that all differentiations in this note will be done thanks to the structure of the ambient space $\R^n$.

\subsubsection{Differentiation of parametrizations}
We introduce the notion of tangent spaces, geometrically the spaces of vectors that starts from each base point, by differentiation of parametrization.
Before that, let us make sure the notations for differentiation.
The precise definition of differentiation is skipped.

\begin{notn}
Let $\a:I\to\R^n$ be a regular curve.
Its tangent vector is denoted by
\[\a'=\dot\a=\dd{\a}{t}:I\to\R^n.\]
Let $\a:U\to\R^n$ be a regular surface.
Its tangent vectors are denoted by
\[\a_x=\pd_x\a=\pd{\a}{x},\ \a_y=\pd_y\a=\pd{\a}{y}:U\to\R^n.\]
\end{notn}

Now we define tangent spaces in several equivlent ways:
\begin{defn}
Let $M$ be a regular curve or a regular surface with parametrization $\a:U\to M\subset\R^n$.
Let $p\in M$ be a point.

The \emph{tangent space} of $M$ at $p$, denoted by $T_pM$, can be defined as either one of the followings:
\begin{cond}
\item the span of linearly independent set of vectors $\{\pd_i\a\}_{i=1}^m\subset\R^n$,
\item the image of the Fr\'echet derivative $d\a_p:\R^m\to\R^n$.
This definition is independent on the parametrization $\a$,
\item the set of vectors $v\in\R^n$ such that there exists a curve $\gamma:I\to M$ satisfying $\gamma(0)=p$ and $\gamma'(0)=v$.
\end{cond}
\end{defn}
\begin{rmk}
We can show the three conditions are equivalent, but the proof will not be given; what is more important is to understand the role and meaning of tangent spaces.
There exist a lot more neat characterizations for tangent spaces we will not cover.
\end{rmk}
\begin{rmk}
We can easily check that $T_p\R^3=\R^3$ for any $p\in\R^3$.
The notation $T_p\R^3$ will be used to emphasize that a vector in $\R^3$ is geometrically recognized to cast from the point $p$.
\end{rmk}

% Examples

\subsubsection{Differentiation by tangent vectors}

%%%%
















\section{Local theory of curves}

\subsection{Theory}

\subsubsection{Frenet-Serret frame}
The Frenet-Serret frame is defined for nondegenerate regular curves.
It provides with a useful orthonormal basis of $T_p\R^3\supset T_pC$ for points $p$ on a regular curve $C$.
\begin{defn}
We call a curve parametrized as $\a:I\to\R^3$ is \emph{nondegenerate} if the normalized tangent vector $\a'/\|\a'\|$ is never locally constant everywhere.
In other words, $\a$ is nowhere straight.
\end{defn}

\begin{defn}[Frenet-Serret frame]
Let $\a$ be a nondegenerate curve.
The \emph{tangent unit vector}, \emph{normal unit vector}, \emph{binormal unit vector} are $T_p\R^3$-valued vector fields on $\a$ defined by:
\[\rT:=\frac{\a'}{\norm{\a'}},\qquad\rN:=\frac{\rT'}{\|\rT'\|},\qquad\rB:=\rT\times\rN.\]
The set of vector fields $\{\rT,\rN,\rB\}$, which is called \emph{Frenet-Serret frame}, form an orthonormal basis of $T_p\R^3$ at each point $p$ on $\a$.
Note that the Frenet-Serret frame is uniquely determined up to sign as parametrizations vary.
\end{defn}


\subsubsection{Reparametrization}
The Frenet-Serret frame plays more powerful roles when the parametrization is properly reparametrized.
\begin{defn}
A parametrization $\a$ of a regular curve is called \emph{unit speed curve} if it satisfies $\|\a'\|=1$.
\end{defn}
\begin{thm}
Every regular curve may be assumed to have unit speed.
Precisely, for every regular curve, there is a parametrization $\a$ such that $\|\a'\|=1$.
\end{thm}
\begin{pf}
Suppose we have a parametrization $\beta:I_t\to\R^d$ for a given regular curve.

Define $\tau:I_t\to I_s$ such that
\[\tau(t_0):=\int_0^{t_0}\|\beta'(t)\|\,dt.\]
Since $\tau$ is smooth and $\tau'>0$ everywhere so that $\tau$ is strictly increasing, the inverse $\tau^{-1}:I_s\to I_t$ is smooth by the inverse function theorem.
Define $\a:I_s\to\R^d$ by $\a:=\beta\circ\tau^{-1}$.
Then, by the chain rule,
\[\a'=\dd{\a}{s}=\dd{\beta}{t}\dd{\tau^{-1}}{s}=\beta'\left(\dd{\tau}{t}\right)^{-1}=\frac{\beta'}{\|\beta'\|}.\qedhere\]
\end{pf}



\subsubsection{Differentiation of Frenet-Serret frame}

\begin{defn}
Let $\a$ be a unit speed nondegenerate curve.
The \emph{curvature} and \emph{torsion} are defined by:
\[\kappa(t):=\<\rT'(t),\rN(t)\>,\quad\tau(t):=-\<\rB'(t),\rN(t)\>.\]
Note that $\kappa>0$ cannot vanish by definition of nondegenerate curve.
\end{defn}

\begin{thm}[Frenet-Serret formula]
Let $\a$ be a unit speed nondegenerate curve.
Then,
\[\begin{pmatrix}\rT'\\\rN'\\\rB'\end{pmatrix}=\begin{pmatrix}0&\kappa&0\\-\kappa&0&\tau\\0&-\tau&0\end{pmatrix}\begin{pmatrix}\rT\\\rN\\\rB\end{pmatrix}.\]
\end{thm}
\begin{pf}
Note that $\{\rT,\rN,\rB\}$ is an orthonormal basis.

\Step{1}[Show that $\rT',\rB',\rN$ are parallel]
Two vectors $\rT'$ and $\rN$ are parallel by definition.
Since $\<\rT,\rB\>=0$ and $\<\rB,\rB\>=1$ are constant, we have
\[\<\rB',\rT\>=\<\rB,\rT\>'-\<\rB,\rT'\>=0,\qquad\<\rB',\rB\>=\tfrac12\<\rB,\rB\>'=0,\]
which show $\rB'$ and $\rN$ are parallel.
By the definition of $\kappa$ and $\tau$, we have
\[\rT'=\kappa\rN,\qquad\rB'=-\tau\rB.\]

\Step{2}[Describe $\rN'$]
Since
\begin{align*}
\<\rN',\rT\>&=-\<\rN,\rT'\>=-\kappa,\\
\<\rN',\rN\>&=\tfrac12\<\rN,\rN\>'=0,\\
\<\rN',\rB\>&=-\<\rN,\rB'\>=\tau,
\end{align*}
we have
\[\rN'=-\kappa\rT+\tau\rB.\qedhere\]
\end{pf}
\begin{rmk}
Skew-symmetricity in the Frenet-Serret formula is not by chance.
Let $\rX(t)$ be the curve of orthogonal matrices $(\rT(t),\rN(t),\rB(t))^T$.
Then, the Frenet-Serret formula reads
\[\rX'(t)=A(t)\rX(t)\]
for a matrix curve $A(t)$.
Since $\rX(t+h)=R_t(h)\rX(t)$ for a family of orthogonal matrices $\{R_t(h)\}_h$ with $R_t(0)=I$, we can describe $A(t)$ as 
\[A(t)=\left.\dd{R_t}{h}\right\rvert_{h=0}.\]
By differentiating the relation $R_t^T(h)R_t(h)=I$ with respect to $h$, we get to know that $A(t)$ is skew-symmetric for all $t$.
In other words, the tangent space $T_I\SO(3)$ forms a skew symmetric matrix.
\end{rmk}

\begin{prop}
Let $\a$ be a nondegenerate curve.
\[\kappa=\frac{\|\a'\x\a''\|}{\|\a'\|^3},\qquad\tau=\frac{\a'\x\a''\cdot\a'''}{\|\a'\x\a''\|}.\]
\end{prop}
\begin{pf}
If we let $s=\|\a'\|$, then
\begin{align*}
\a'&=s\rT,\\
\a''&=s'\rT+s^2\kappa\rN,\\
\a'''&=(s''-s^3\kappa^2)\rT+(3ss'\kappa+s^2\kappa')\rN+(s^3\kappa\tau)\rB.
\end{align*}
Now the formulas are easily derived.
\end{pf}




\subsection{Problems}

Let $\a$ be a nondegenerate unit speed space curve, and let $\{\rT,\rN,\rB\}$ be the Frenet-Serret frame for $\a$.
Consider a diagram as follows:
\begin{cd}
\<\a,\rT\>=\ ?\ar{r}\ar{d} & \<\a,\rN\>=\ ? \ar{l}\ar{d}\ar{r} & \<\a,\rB\>=\ ? \ar{l}\ar{d} \\
\<\a',\rT\>=1 & \<\a',\rN\>=0 &\<\a',\rB\>=0.
\end{cd}
Here the arrows indicate which information we are able to get by differentiation.
For example, if we know a condition
\[\<\a(t),\rT(t)\>=f(t),\]
then we can obtain by differentiating it
\[\<\a(t),\rN(t)\>=\frac{f'(t)-1}{\kappa(t)}\]
since we have known $\<\a',\rT\>$ but not $\<\a,\rN\>$, and further
\[\<\a(t),\rB(t)\>=\frac{\left(\frac{f'(t)-1}{\kappa(t)}\right)'+\kappa(t)f(t)}{\tau(t)}\]
since we have known $\<\a,\rT\>$ and $\<\a',\rN\>$ but not $\<\a,\rB\>$.
Thus, $\<\a,\rT\>=f$ implies
\[\a(t)=f(t)\cdot\rT+\frac{f'(t)-1}{\kappa(t)}\cdot\rN+\frac{\left(\frac{f'(t)-1}{\kappa(t)}\right)'+\kappa(t)f(t)}{\tau(t)}\cdot\rB,\]
when given $\tau(t)\ne0$.

Suggested a strategy for space curve problems:
\begin{itemize}
\item Formulate the assumptions of the problem as the form
\[\<\ \text{(interesting vector)},\ \text{(Frenet-Serret basis)}\ \>\ =\ \text{(some function)}.\]
\item Aim for finding the coefficients of the position vector in the Frenet-Serret frame, and obtain relations of $\kappa$ and $\tau$ by comparing with assumptions.
\item Heuristically find a constant vector and show what you want directly.
\end{itemize}
Here we give example solutions of several selected problems.
Always $\a$ denote a reparametrized unit speed nondegenerate curve.

\begin{prb}
A space curve whose normal lines always pass through a fixed point lies in a circle.
\end{prb}
\begin{sol}
\Step{1}[Formulate conditions]
By the assumption, there is a constant point $p\in\R^3$ such that the vectors $\a-p$ and $\rN$ are parallel so that we have
\[\<\a-p,\rT\>=0,\qquad\<\a-p,\rB\>=0.\]
Our goal is to show that $\|\a-p\|$ is constant and there is a constant vector $v$ such that $\<\a-p,v\>=0$.

\Step{2}[Collect information]
Differentiate $\<\a-p,\rT\>=0$ to get
\[\<\a-p,\rN\>=-\frac1\kappa.\]
%\[\<\a-p,\rN\>=\frac1\kappa\<\a-p,\rT'\>=\frac1\kappa(\<\a-p,\rT\>'-\<\a',\rT\>)=-\frac1\kappa.\]
Differentiate $\<\a-p,\rB\>=0$ to get
\[\tau=0.\]

\Step{3}[Complete proof]
We can deduce that $\|\a-p\|$ is constant from
\[(\|\a-p\|^2)'=\<\a-p,\a-p\>'=2\<\a-p,\rT\>=0.\]
Also, if we heuristically define a vector $v:=\rB$, then $v$ is constant since
\[v'=-\tau\rN=0,\]
and clearly $\<\a-p,v\>=0$
\end{sol}


\begin{prb}
A sphere curve of constant curvature lies in a circle.
\end{prb}
\begin{sol}
\Step{1}[Formulate conditions]
The condition that $\a$ lies on a sphere can be given as follows: for a constant point $p\in\R^3$,
\[\|\a-p\|=\const.\]
Also we have
\[\kappa=\const.\]

\Step{2}[Collect information]
Differentiate $\|\a-p\|^2=\const$ to get
\[\<\a-p,\rT\>=0.\]
Differentiate $\<\a-p,\rT\>=0$ to get
\[\<\a-p,\rN\>=-\frac1\kappa.\]
Differentiate $\<\a-p,\rN\>=-1/\kappa=\const$ to get
\[\tau\<\a-p,\rB\>=0.\]

There are two ways to show that $\tau=0$.

\emph{Method 1}:
Assume that there is $t$ such that $\tau(t)\ne0$.
By the continuity of $\tau$, we can deduce that $\tau$ is locally nonvanishing.
In other words, we have $\<\a-p,\rB\>=0$ on an open interval containing $t$.
Differentiate $\<\a-p,\rB\>=0$ at $t$ to get $\<\a-p,\rN\>=0$ near $t$, which is a contradiction.
Therefore, $\tau=0$ everywhere.

\emph{Method 2}:
Since $\<\a-p,\rB\>$ is continuous and
\[\<\a-p,\rB\>=\pm\sqrt{\|\a-p\|^2-\<\a-p,\rT\>^2-\<\a-p,\rN\>^2}=\pm\const,\]
we get $\<\a-p,\rB\>=\const$.
Differentiate to get $\tau\<\a-p,\rN\>=0$.
Finally we can deduce $\tau=0$ since $\<\a-p,\rN\>\ne0$.

\Step{3}[Complete proof]
The zero torsion implies that the curve lies on a plane.
A planar curve in a sphere is a circle.
\end{sol}

\begin{prb}
A curve such that $\tau/\kappa=(\kappa'/\tau\kappa^2)'$ lies on a sphere.
\end{prb}
\begin{sol}
\Step{1}[Find the center heuristically]
If we assume that $\a$ is on a sphere so that we have $\|\a-p\|=r$ for constants $p\in\R^3$ and $\r>0$, then by the routine differentiations give
\[\<\a-p,\rT\>=0,\qquad\<\a-p,\rN\>=-\frac1\kappa,\qquad\<\a-p,\rB\>=-\left(\frac1\kappa\right)'\frac1\tau,\]
that is,
\[\a-p=-\frac1\kappa\rN-\left(\frac1\kappa\right)'\frac1\tau\rB.\]

\Step{2}[Complete proof]
Let us get started the proof.
Define
\[p:=\a+\frac1\kappa\rN+\left(\frac1\kappa\right)'\frac1\tau\rB.\]
We can show that it is constant by differentiation.
Also we can show that
\[\<\a-p,\a-p\>\]
is constant by differentiation.
So we are done.
\end{sol}

\begin{prb}
A curve with more than one Bertrand mates is a circular helix.
\end{prb}
\begin{sol}
\Step{1}[Formulate conditions]

\Step{2}[Collect information]

\Step{3}[Complete proof]
\end{sol}

Here are additional representative problem sets.
\begin{prb}[Plane curves]
Let $\a$ be a nondegenerate curve in $\R^3$.
TFAE:
\begin{cond}
\item the curve $\a$ lies on a plane,
\item $\tau=0$,
\item the osculating plane constains a fixed point.
\end{cond}
\end{prb}

\begin{prb}[Helices]
Let $\a$ be a nondegenerate curve in $\R^3$.
TFAE:
\begin{cond}
\item the curve $\a$ is a helix,
\item $\tau/\kappa=\const$,
\item normal lines are parallel to a plane.
\end{cond}
\end{prb}

\begin{prb}[Sphere curves]
Let $\a$ be a nondegenerate curve in $\R^3$.
TFAE:
\begin{cond}
\item the curve $\a$ lies on a sphere,
\item $(1/\kappa)^2+((1/\kappa)'/\tau)^2=\const$,
\item $\tau/\kappa=(\kappa'/\tau\kappa^2)'$,
\item normal planes contain a fixed point.
\end{cond}
\end{prb}

\begin{prb}[Bertrand mates]
Let $\a$ be a nondegenerate curve in $\R^3$.
TFAE:
\begin{cond}
\item the curve $\a$ has a Bertrand mate,
\item there are two constants $\lambda\ne0,\mu$ such that $1/\lambda=\kappa+\mu\tau$.
\end{cond}
\end{prb}

% evolutes











\section{Local theory of surfaces}
\subsection{Theory}
\subsubsection{Gauss map}

\begin{defn}
Let $\a$ be a regular surface.
The \emph{Gauss map} or \emph{normal unit vector} $\nu:U\to\R^3$ is a $T_p\R^3$-valued vector field on $\a$ defined by:
\[\nu:=\frac{\a_x\times \a_y}{\|\a_x\times \a_y\|}.\]
The set of vector fields $\{\a_x,\a_y,\nu\}$ form a basis of $T_p\R^3$ at each point $p$ on $\a$.
\end{defn}
\begin{prop}
The Gauss map is independent on parametrizations up to sign.
\end{prop}

\subsubsection{Reparametrization}
% first fundamental form


\subsubsection{Differentiation of tangent vectors}
% christoffel symbol


\subsubsection{Differentiation of normal vector}
% normal curvature
% second fundamental form
% shape operator, weingarten map
% principal curvature
% mean curvature, gaussian curvature

% curvature tensor?

\end{document}