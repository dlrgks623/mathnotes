\documentclass{../exp}
\usepackage{../../ikany}

\def\a{\alpha}

\title{Classical differential geometry}

\begin{document}
\maketitle
\tableofcontents


% 길이 넓이, 방향
% 벡터를 접벡터로 미분하기: 아무 설명 없는 경우 바깥의 공변미분


% Local theory
% -이론
%   좌표의 설정
%    곡선: 프레네세레,
%    곡면: 가우스맵과 접벡터 두 개
%         선형독립    벡터들이 한점에서 주어졌을 때 -> 
%         선형독립    벡터장이 근방에서 주어졌을 때 -> 일반적으론 2차원에서만
%         선형독립 가환벡터장이 근방에서 주어졌을 때 -> n차원 다돼
%         선형독립    직교벡터들이 한점에서 주어졌을 때
%         선형독립    직교벡터장이 근방에서 주어졌을 때
%         선형독립 직교가환벡터장이 근방에서 주어졌을 때
%             곡률의 선, 점근곡선, 측지좌표
%   측정량과 구조의 의존성
%    기본적으로 -> 기저 벡터를 미분하면 어떻게 되느냐
%    곡선: 카파, 타우
%    곡면: 곡률들, 기본형식 항들, 크리스토펠, 가우스곡률; 좌표의존성과 바깥의존성
% -계산
%   평면곡선, 구면곡선, 헬릭스, 버트런드
%   프리이미지, 그래프, 접선쓸기, 회전체
%   뿔, 엘립소이드, 파라볼로이드, 쌍곡면, 헬리코이드; 곡률 부호
%    룰드 곡면, 데벨로퍼블
% -문제
%   곡선: 각 점에 대한 정보 -> 모양
%   곡면: 

% Global theory
%  곡선: 등주부등식, 네 꼭짓점, 볼록성/오발, 펜첼/페리-밀너
%  곡면: 최소곡면, 컴팩트곡면분류, 임베딩문제, 가우스-보네

% 평행수송, 공변미분, 측지선, 측지꼬임/곡률, 측지완비/호프리노프, 지수사상, 야코비장, 아다마르

\section{Parametrizations and Coordinates}
\subsection{Parametrizations}
\begin{defn}
An \emph{$m$-dimensional parametrization} is a smooth map $\a:U\to\R^n$ such that
\begin{cond}
\item $U\subset\R^m$ is open,
\item $\a$ is one-to-one (optional),
\item $d\a$ is nondegenerate; $\{\pd_i\a\}_{i=1}^m$ is linearly independent.
\end{cond}
\end{defn}
The third condition is important; in language of manifolds, the third condition defines what we call \emph{immersed submanifolds}.
We will see that the second condition is not important at all.

For each text on classical differential geometry, the definitions of regular curves and surfaces frequently vary.
In this note, we define as follows.
\begin{defn}
A \emph{regular curve} is a subset of $\R^n$ that can be given as the image of a one-dimensional parametrization.
A \emph{regular surface} is a subset of $\R^n$ that can be given as the image of a two-dimensional parametrization.
\end{defn}
With this definition, the regular surface is not appropriate to deal with the global nature of, for example, spheres or tori.

Reparametrization is just a choice of another parametrization for the same curve or surface.



\subsection{Coordinates}
\begin{defn}
Let $M\subset\R^3$ be a regular curve or a regular surface.
The inverse $\f:M\to U$ of a parametrization is called a \emph{coordinate map}.
\end{defn}


The choice of coordinate(parametrization) is extremely important in differential geometry.




\section{Curves in a space}
\begin{thm}
For every regular curve, there is a parametrization $\a$ such that $\|\a'\|=1$.
\end{thm}
\begin{pf}
Suppose we have a parametrization $\beta:I_t\to\R^d$.
Define $\tau:I_t\to I_s$ such that
\[\tau(t_0):=\int_0^{t_0}\|\beta'(t)\|\,dt.\]
Then, $s$ is a diffeomorphism.
Define $\a:I_s\to\R^d$ by $\a:=\beta\circ\tau^{-1}$.
Then, by the chain rule,
\[\a'=\dd{\a}{s}=\dd{\beta}{t}\dd{\tau^{-1}}{s}=\beta'\left(\dd{\tau}{t}\right)^{-1}=\frac{\beta'}{\|\beta'\|}.\qedhere\]
\end{pf}



\subsection{Frenet-Serret theory}

\begin{defn}
We call a curve parametrized as $\a:I\to\R^3$ is \emph{nondegenerate} if the normalized tangent vector $\a'/\|\a'\|$ is never locally constant everywhere.
In other words, $\a$ is nowhere straight.
\end{defn}

\begin{defn}[Frenet-Serret frame]
Let $\a$ be a nondegenerate curve.
We define \emph{tangent unit vector}, \emph{normal unit vector}, \emph{binormal unit vector} by:
\[\bT(t):=\frac{\a'(t)}{\norm{\a'(t)}},\qquad\bN(t):=\frac{\bT'(t)}{\|\bT'(t)\|},\qquad\bB(t):=\bT(t)\times\bN(t),\]
and \emph{curvature} and \emph{torsion} by:
\[\kappa(t):=\<\bT'(t),\bN(t)\>,\quad\tau(t):=-\<\bB'(t),\bN(t)\>.\]
Note that $\kappa$ cannot vanish by definition.
\end{defn}

\begin{thm}[Frenet-Serret formula]
Let $\a$ be a unit speed nondegenerate curve.
\[\begin{pmatrix}\bT'\\\bN'\\\bB'\end{pmatrix}=\begin{pmatrix}0&\kappa&0\\-\kappa&0&\tau\\0&-\tau&0\end{pmatrix}\begin{pmatrix}\bT\\\bN\\\bB\end{pmatrix}.\]
\end{thm}
\begin{pf}
Note that $\{\bT,\bN,\bB\}$ is an orthonormal basis.

\Step{1}[Show that $\bT',\bB',\bN$ are parallel]
Two vectors $\bT'$ and $\bN$ are parallel by definition.
Since $\<\bT,\bB\>=0$ and $\<\bB,\bB\>=1$ are constant, we have
\[\<\bB',\bT\>=\<\bB,\bT\>'-\<\bB,\bT'\>=0,\qquad\<\bB',\bB\>=\tfrac12\<\bB,\bB\>'=0,\]
which show $\bB'$ and $\bN$ are parallel.
By the definition of $\kappa$ and $\tau$, we have
\[\bT'=\kappa\bN,\qquad\bB'=-\tau\bB.\]

\Step{2}[Describe $\bN'$]
Since
\begin{align*}
\<\bN',\bT\>&=-\<\bN,\bT'\>=-\kappa,\\
\<\bN',\bN\>&=\tfrac12\<\bN,\bN\>'=0,\\
\<\bN',\bB\>&=-\<\bN,\bB'\>=\tau,
\end{align*}
we have
\[\bN'=-\kappa\bT+\tau\bB.\qedhere\]
\end{pf}
\begin{rmk}
Skew-symmetricity in the Frenet-Serret formula is not by chance.
Let $\bX(t)$ be the curve of orthogonal matrices $(\bT(t),\bN(t),\bB(t))^T$.
Then, the Frenet-Serret formula reads
\[\bX'(t)=A(t)\bX(t)\]
for a matrix curve $A(t)$.
Since $\bX(t+h)=R_t(h)\bX(t)$ for a family of orthogonal matrices $\{R_t(h)\}_h$ with $R_t(0)=I$, we can describe $A(t)$ as 
\[A(t)=\left.\dd{R_t}{h}\right\rvert_{h=0}.\]
By differentiating the relation $R_t^T(h)R_t(h)=I$ with respect to $h$, we get to know that $A(t)$ is skew-symmetric for all $t$.
In other words, the tangent space $T_I\SO(3)$ forms a skew symmetric matrix.
\end{rmk}

\begin{prop}
Let $\a$ be a nondegenerate space curve.
\[\kappa=\frac{\|\a'\x\a''\|}{\|\a'\|^3},\qquad\tau=\frac{\a'\x\a''\cdot\a'''}{\|\a'\x\a''\|}.\]
\end{prop}
\begin{pf}
If we let $s=\|\a'\|$, then
\begin{align*}
\a'&=s\bT,\\
\a''&=s'\bT+s^2\kappa\bN,\\
\a'''&=(s''-s^3\kappa^2)\bT+(3ss'\kappa+s^2\kappa')\bN+(s^3\kappa\tau)\bB.
\end{align*}
Now the formulas are easily derived.
\end{pf}


\subsection{Problems}

Let $\a$ be a nondegenerate unit speed space curve, and let $\{\bT,\bN,\bB\}$ be the Frenet-Serret frame for $\a$.
Consider a diagram as follows:
\begin{cd}
\<\a,\bT\>=\ ?\ar{r}\ar{d} & \<\a,\bN\>=\ ? \ar{l}\ar{d}\ar{r} & \<\a,\bB\>=\ ? \ar{l}\ar{d} \\
\<\a',\bT\>=1 & \<\a',\bN\>=0 &\<\a',\bB\>=0.
\end{cd}
Here the arrows indicate which information we are able to get by differentiation.
For example, if we know a condition
\[\<\a(t),\bT(t)\>=f(t),\]
then we can obtain by differentiating it
\[\<\a(t),\bN(t)\>=\frac{f'(t)-1}{\kappa(t)}\]
since we have known $\<\a',\bT\>$ but not $\<\a,\bN\>$, and further
\[\<\a(t),\bB(t)\>=\frac{\left(\frac{f'(t)-1}{\kappa(t)}\right)'+\kappa(t)f(t)}{\tau(t)}\]
since we have known $\<\a,\bT\>$ and $\<\a',\bN\>$ but not $\<\a,\bB\>$.
Thus, $\<\a,\bT\>=f$ implies
\[\a(t)=f(t)\cdot\bT+\frac{f'(t)-1}{\kappa(t)}\cdot\bN+\frac{\left(\frac{f'(t)-1}{\kappa(t)}\right)'+\kappa(t)f(t)}{\tau(t)}\cdot\bB,\]
when given $\tau(t)\ne0$.

Suggested a strategy for space curve problems:
\begin{itemize}
\item Formulate the assumptions of the problem as the form
\[\<\ \text{(interesting vector)},\ \text{(Frenet-Serret basis)}\ \>\ =\ \text{(some function)}.\]
\item Aim for finding the coefficients of the position vector in the Frenet-Serret frame, and obtain relations of $\kappa$ and $\tau$ by comparing with assumptions.
\item Heuristically find a constant vector and show what you want directly.
\end{itemize}
Here we give an example solution of several problems.
Always $\a$ denote a reparametrized unit speed curve.

\begin{ex}
A space curve whose normal lines always pass through a fixed point lies in a circle.
\end{ex}
\begin{pf}
\Step{1}[Formulate conditions]
By the assumption, there is a constant point $p\in\R^3$ such that the vectors $\a-p$ and $\bN$ are parallel so that we have
\[\<\a-p,\bT\>=0,\qquad\<\a-p,\bB\>=0.\]
Our goal is to show that $\|\a-p\|$ is constant and there is a constant vector $v$ such that $\<\a-p,v\>=0$.

\Step{2}[Collect information]
Differentiate $\<\a-p,\bT\>=0$ to get
\[\<\a-p,\bN\>=-\frac1\kappa.\]
%\[\<\a-p,\bN\>=\frac1\kappa\<\a-p,\bT'\>=\frac1\kappa(\<\a-p,\bT\>'-\<\a',\bT\>)=-\frac1\kappa.\]
Differentiate $\<\a-p,\bB\>=0$ to get
\[\tau=0.\]

\Step{3}[Complete proof]
We can deduce that $\|\a-p\|$ is constant from
\[(\|\a-p\|^2)'=\<\a-p,\a-p\>'=2\<\a-p,\bT\>=0.\]
Also, if we heuristically define a vector $v:=\bB$, then $v$ is constant since
\[v'=-\tau\bN=0,\]
and clearly $\<\a-p,v\>=0$
\end{pf}


\begin{ex}
A sphere curve of constant curvature lies in a circle.
\end{ex}
\begin{pf}
\Step{1}[Formulate conditions]
The condition that $\a$ lies on a sphere can be given as follows: for a constant point $p\in\R^3$,
\[\|\a-p\|=\const.\]
Also we have
\[\kappa=\const.\]

\Step{2}[Collect information]
Differentiate $\|\a-p\|^2=\const$ to get
\[\<\a-p,\bT\>=0.\]
Differentiate $\<\a-p,\bT\>=0$ to get
\[\<\a-p,\bN\>=-\frac1\kappa.\]
Differentiate $\<\a-p,\bN\>=-1/\kappa=\const$ to get
\[\tau\<\a-p,\bB\>=0.\]

There are two ways to show that $\tau=0$.

\emph{Method 1}:
Assume that there is $t$ such that $\tau(t)\ne0$.
By the continuity of $\tau$, we can deduce that $\tau$ is locally nonvanishing.
In other words, we have $\<\a-p,\bB\>=0$ on an open interval containing $t$.
Differentiate $\<\a-p,\bB\>=0$ at $t$ to get $\<\a-p,\bN\>=0$ near $t$, which is a contradiction.
Therefore, $\tau=0$ everywhere.

\emph{Method 2}:
Since $\<\a-p,\bB\>$ is continuous and
\[\<\a-p,\bB\>=\pm\sqrt{\|\a-p\|^2-\<\a-p,\bT\>^2-\<\a-p,\bN\>^2}=\pm\const,\]
we get $\<\a-p,\bB\>=\const$.
Differentiate to get $\tau\<\a-p,\bN\>=0$.
Finally we can deduce $\tau=0$ since $\<\a-p,\bN\>\ne0$.

\Step{3}[Complete proof]
The zero torsion implies that the curve lies on a plane.
A planar curve in a sphere is a circle.
\end{pf}

\begin{ex}
A curve such that $\tau/\kappa=(\kappa'/\tau\kappa^2)'$ lies on a sphere.
\end{ex}
\begin{pf}
\Step{1}[Find center heuristically]
If we assume that $\a$ is on a sphere so that we have $\|\a-p\|=r$ for constants $p\in\R^3$ and $\r>0$, then by the routine differentiations give
\[\<\a-p,\bT\>=0,\qquad\<\a-p,\bN\>=-\frac1\kappa,\qquad\<\a-p,\bB\>=-\left(\frac1\kappa\right)'\frac1\tau,\]
that is,
\[\a-p=-\frac1\kappa\bN-\left(\frac1\kappa\right)'\frac1\tau\bB.\]

\Step{2}[Complete proof]
Let us get started the proof.
Define
\[p:=\a+\frac1\kappa\bN+\left(\frac1\kappa\right)'\frac1\tau\bB.\]
We can show that it is constant by differentiation.
Also we can show that
\[\<\a-p,\a-p\>\]
is constant by differentiation.
So we are done.
\end{pf}

\begin{ex}
A curve with more than one Bertrand mates is a circular helix.
\end{ex}
\begin{pf}
\Step{1}[Formulate conditions]
\Step{2}[Collect information]
\Step{3}[Complete proof]
\end{pf}

Here are representative problem sets.
\begin{ex}[Plane curves]
Let $\a$ be a nondegenerate curve in $\R^3$.
TFAE:
\begin{cond}
\item the curve $\a$ lies on a plane,
\item $\tau=0$,
\item the osculating plane constains a fixed point.
\end{cond}
\end{ex}

\begin{ex}[Helices]
Let $\a$ be a nondegenerate curve in $\R^3$.
TFAE:
\begin{cond}
\item the curve $\a$ is a helix,
\item $\tau/\kappa=\const$,
\item normal lines are parallel to a plane.
\end{cond}
\end{ex}

\begin{ex}[Sphere curves]
Let $\a$ be a nondegenerate curve in $\R^3$.
TFAE:
\begin{cond}
\item the curve $\a$ lies on a sphere,
\item $(1/\kappa)^2+((1/\kappa)'/\tau)^2=\const$,
\item $\tau/\kappa=(\kappa'/\tau\kappa^2)'$,
\item normal planes contain a fixed point.
\end{cond}
\end{ex}

\begin{ex}[Bertrand mates]
Let $\a$ be a nondegenerate curve in $\R^3$.
TFAE:
\begin{cond}
\item the curve $\a$ has a Bertrand mate,
\item there are two constants $\lambda\ne0,\mu$ such that $1/\lambda=\kappa+\mu\tau$.
\end{cond}
\end{ex}









\section{Surfaces in a space}
\[\nu_x=S(\a_x)=\kappa_1\a_x\]


\section{Curves on a surface}


\end{document}