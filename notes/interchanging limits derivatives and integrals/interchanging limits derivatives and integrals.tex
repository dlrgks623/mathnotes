\documentclass{../exp}
\usepackage{../../ikany}

\title{Interchanging Limits, Derivatives, and Integrals}

\begin{document}
\maketitle

\section{Limit and derivative}
\begin{thm}
$f_n$ pointwisely, $Df_n$ uniformly
\end{thm}
\begin{cor}
If $f_n\to f$ in $C^1$, then $Df_n\to Df$.
\end{cor}






\section{Limit and integral}
We want to find a criterion for 
This question asks the convergence
\[f_n\to f\quad\text{in}\quad L^1.\]

For a sequence of measurable functions $f_n:(X,\mu)\to\R$, define the maximal function
\[Mf(x):=\sup_n|f_n(x)|.\]
\begin{thm}[LDCT]
If $\|Mf\|_{L^1}<\oo$ and $f_n\to f$ a.e., then $f_n\to f$ in $L^1$.
\end{thm}

continuity application






\begin{thm}[Scheffe]
Let $\{f_n\}_n$ be a sequence of nonnegative functions in $L^1$.
Suppose it converges to $f$ pointwisely.
Then,
\[\lim_{n\to\oo}\|f_n\|_1=\|f\|_1\impl\lim_{n\to\oo}\|f_n-f\|_1=0.\]
\end{thm}



\section{Derivative and integral}

Define the Newton quotient as
\[D_kf(t,x):=\frac{f(t+k,x)-f(t,x)}k\]
for $k\ne0$.
We mainly recognize $D_k$ as an operator that maps $f(0,x)$ to a function of $x$.
Then, we can say that the partial derivative $\pd_tf(0,x)$ is well-defined a.e. $x$ if and only if
\[\lim_{k\to0}D_kf(0,x)=\pd_tf(0,x)\qquad\text{a.e. }x.\]

We may ask about conditions for the following to hold:
\[\lim_{h\to0}D_kf(0,x)=\pd_tf(0,x)\qquad\text{in }L_x^1(X).\]
This question naturally arise because it implies the commutability
\[\dd{t}\int f(t,x)\,dx=\int\pd{t}f(t,x)\,dx\]
at $t=0$.
As necessary conditions to formalize the statement, we must basically assume that $f(t,x)\in L_x^1$ for $|t|<\e$, and $\pd_tf(0,x)\in L_x^1$.
Above this, if we give a stronger condition $\esssup_{|t|<\e}|\pd_tf(t,x)|\in L_x^1$ than $\pd_tf(0,x)\in L_x^1$, then the $L_x^1$ convergence is obtained.


\begin{thm}[Leibniz rule]
Let $f:(-\e,\e)\times X\to\R$ be a curve of integrable functions such that $f(t,x)$ is absolutely continuous in $t$ for a.e. $x$.
If $\pd_tf\in L_x^1(L_t^\infty)$, then $D_kf(0,x)\to\pd_tf(0,x)$ in $L_x^1$.
\end{thm}
\begin{pf}
Our strategy is to apply the Lebesgue dominated convergence theorem.
In order to do this, we should control $D_kf(0,x)$ uniformly on $k$.

The fundamental theorem of calculus for absolute continuous functions implies
\[D_kf(0,x)=\frac1k\int_0^k\pd_tf(t,x)\,dt,\]
so we have
\[|D_kf(0,x)|\le\frac1k\int_0^k|\pd_tf(t,x)|\,dt\le\|\pd_tf(t,x)\|_{L_t^\infty}.\]
Since the right hand side does not depend on $k$, the main condition for LDCT is satisfied.

The pointwise convergence (in a.e. sense) is satisfied due to the absolute continuity.
By the Lebesgue dominated convergence theorem, we get the desired result.
\end{pf}










\end{document}

$F$ is absolutely continuous,
\[\pd_tF=f\iff F(x,t)=\int_c^tf(x,s)\,dx.\]
Then, for
\[T_hf(x,0):=\frac1h\int_0^hf(x,s)\,ds,\]

For $\|f\|_{L_x^1L_t^\oo}=\|\sup_t|f(x,t)|\|_{L_x^1}<\oo$
we have
\begin{align*}
|T_hf(x,0)|&\le\frac1h\int_0^h|f(x,s)|\,ds\\
&\le\left[\frac1h\int_0^h\,ds\right]\cdot\sup_t|f(x,t)|\\
&=\sup_t|f(x,t)|.
\end{align*}
Thus,
\[Mf(x,0)=\sup_h|T_hf(x,0)|\le\sup_t|f(x,t)|\in L_x^1.\]

Since $f(x,0)\in L_x^1$, by the Lebesgue differentiation theorem, we get
\[\lim_{h\to0}T_hf(x,0)=f(x,0)\]
for a.e. $x$.


