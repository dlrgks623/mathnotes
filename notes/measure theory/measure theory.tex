\documentclass{../exp}
\usepackage{../../ikany}

\title{Measure theory}

\begin{document}
\maketitle


\section{Measure construction}
\begin{defn}
Let $X$ be a set.
A \emph{ring of sets} is a family of subsets of $X$ that is closed under finite union and finite relative complement; in other words, $\cR\subset\cP(X)$ is called a ring of sets if the following two conditions are satisfied:
\begin{cond}
\item if $A,B\in\cR$, then $A\cup B\in\cR$,
\item if $A,B\in\cR$, then $A\setminus B\in\cR$.
\end{cond}
\end{defn}

\begin{prop}
Let $X$ be a set and $\cR\in\cP(X)$.
Then, the followings are equivalent:
\begin{cond}
\item $\cR$ is a ring of sets,
\item $\cR$ is closed under symmetric difference and finite intersection,
\item $\cR$ is a ring,
\item $\cR$ is a Boolean ring.
\end{cond}
For the ring structure, we take the symmetric difference as addition and the intersection as multiplication.
\end{prop}

\begin{prop}
A ring of sets is a distributive lattice.
\end{prop}


If a ring of sets contains a multiplicative identity, the entire set, then we call the ring of sets as follows:
\begin{defn}
An \emph{algebra of sets} is a ring of sets with the entire set.
\end{defn}
\begin{prop}
Let $X$ be a set and $\cR\in\cP(X)$.
Then, the followings are equivalent:
\begin{cond}
\item $\cR$ is an algebra of sets,
\item $\cR$ is closed under finite union, finite intersection, and complement,
\item $\cR$ is a Boolean algebra.
\end{cond}
\end{prop}
An algebra of sets is sometimes called a field of sets.

\begin{thm}[Carath\'eodory's extension theorem]
Let $\cR$ be a ring of sets over $X$.
Let $\sigma(\cR)$ be the $\sigma$-algebra generated by $\cR$.
A set function $\mu:\cR\to[0,\oo]$ is extended to a measure on $\sigma(\cR)$ if and only if it is a premeasure.
\end{thm}


\section{Measures for probability theory}

Since $\{f_n(x)\}_n$ diverges if and only if
\[\exists k>0,\quad\forall n_0>0,\quad\exists n>n_0:\quad|f_n(x)-f(x)|>\tfrac1k,\]
we have
\begin{align*}
\{x:\{f_n(x)\}_n\text{ diverges}\}
&=\bigcup_{k>0}\bigcap_{n_0>0}\bigcup_{n>n_0}\{x:|f_n-f|>\tfrac1k\}\\
&=\bigcup_{k>0}\limsup_n\{x:|f_n-f|>\tfrac1k\}.
\end{align*}
Since for every $k$ we have
\begin{align*}
\limsup_n\{x:|f_n-f|>\tfrac1k\}
&\subset\limsup_{n>k}\{x:|f_n-f|>\tfrac1n\}\\
&=\limsup_n\{x:|f_n-f|>\tfrac1n\},
\end{align*}
we have
\[\{x:\{f_n(x)\}_n\text{ diverges}\}\subset\limsup_n\{x:|f_n-f|>\tfrac1n\}.\]




\begin{thm}
Let $(X,\mu)$ be a measure space.
Let $f_n$ be a sequence of measurable functions.
If $f_n$ converges to $f$ in measure, then $f_n$ has a subsequence that converges to $f$ $\mu$-a.e.
\end{thm}
\begin{pf}
We can extract a subsequence $f_{n_k}$ such that
\[\mu(\{x:|f_{n_k}-f|>\tfrac1k\})>\tfrac1{2^k}.\]
Since
\[\sum_{k=1}^\infty\mu(\{x:|f_{n_k}-f|>\tfrac1k\})<\infty,\]
by the Borel-Canteli lemma, we get
\[\mu(\limsup_k\{x:|f_{n_k}-f|>\tfrac1k\})=0.\]
Therefore, $f_{n_k}$ converges $\mu$-a.e.
\end{pf}




\section{Topological measures}

\subsection{Regular measures}

\begin{thm}
Let $\mu$ be a Borel measure.
Then, the followings are equivalent:
\begin{cond}
\item$\mu$ is inner regular on $\sigma$-bounded sets
\item $\mu$ is outer regular on $\sigma$-bounded sets.
\end{cond}
\end{thm}



\subsection{Radon measures}

Locally compact Hausdorff spaces have at least two important applications in abstract analysis related to measure theory: one is locally compact groups and the associated Harr measures in abstract harmonic analysis, the other is the Gelfand-Naimark theorem which states every commutative $C^*$-algebra can be represented as a function space on a locally compact Hausdorff space.
In the set of this section, we assume every base space $X$ is locally compact Hausdorff.

Note that locally finite measures are compact finite but the converser holds only if in locally compact Hausdorff spaces.
We want to consider locally finite Borel measures as the minimally compatible measures with a given topology on $X$.
For locally finite Borel measures, a set is finite-measured if and only if it is contained in a compact set.
\begin{defn}
(Folland's)A \emph{Radon measure} is a Borel measure on $X$ which satisfies the following three conditions:
\begin{cond}
\item locally finite,
\item outer regular on all Borel sets,
\item inner regular on all open sets.
\end{cond}
\end{defn}

Radon measures are rather simply characterized when the base space $X$ is $\sigma$-compact.
The following proposition proves the equivalence between regularity and Radonness of locally finite Borel measure on a $\sigma$-compact space.
\begin{prop}
A Radon measure is inner regular on all $\sigma$-finite Borel sets.(Folland's)
\end{prop}
\begin{pf}
First we approximate Borel sets of finite measure, with compact sets.
Let $E$ be a Borel set with $\mu(E)<\oo$ and $U$ be an open set containing $E$.
By outer regularity, there is an open set $V\supset U-E$ such that
\[\mu(V)<\mu(U-E)+\frac\e2.\]
By inner regularity, there is a compact set $K\subset U$ such that
\[\mu(K)>\mu(U)-\frac\e2.\]
Then, we have a compact set $K-V\subset K-(U-E)\subset E$ such that
\begin{align*}
\mu(K-V)&\ge\mu(K)-\mu(V)\\
&>\left(\mu(U)-\frac\e2\right)-\left(\mu(U-E)+\frac\e2\right)\\
&\ge\mu(E)-\e.
\end{align*}
It implies that a Radon measure is inner regular on Borel sets of finite measures.

Suppose $E$ is a $\sigma$-finite Borel set so that $E=\bigcup_{n=1}^\oo E_n$ with $\mu(E_n)<\oo$.
We may assume $E_n$ are pairwise disjoint.
Let $K_n$ be a compact subset of $E_n$ such that
\[\mu(K_n)>\mu(E_n)-\frac\e{2^n},\]
and define $K=\bigcup_{n=1}^\oo K_n\subset E$.
Then,
\[\mu(K)=\sum_{n=1}^\oo\mu(K_n)>\sum_{n=1}^\oo\left(\mu(E_n)-\frac\e{2^n}\right)=\mu(E)-\e.\]
Therefore, a Radon measure is inner regular on all $\sigma$-finite Borel sets.
\end{pf}
We get a corollary:
\begin{cor}
If $X$ is $\sigma$-compact, then a locally finite Borel measure is Radon if and only if it is regular.
\end{cor}

\begin{thm}
If every open set in $X$ is $\sigma$-compact(i.e. Borel sets and Baire sets coincide), then every locally finite Borel measure is regular.
\end{thm}
\begin{prop}
In a second countable space, every open set is $\sigma$-compact(i.e. Borel sets and Baire sets coincide).
\end{prop}


Two corollaries are presented as follows:
\begin{rd}[column sep={120pt,between origins}]
\parbox{7em}{\centering locally finite \\ Borel regular} \ar{r} &
\parbox{5em}{\centering Radon}   \ar{r} \rds{l}{$X$ is $\sigma$-compact} &
\parbox{7em}{\centering locally finite \\ Borel}  \lds{ll}{$X$ is second countable}
\end{rd}


Many applications assume $X$ is an open subset of a Euclidean space, so $X$ is usually second countable.
In this case, the followings will be synonym: A measure is
\begin{cond}
\item 
\end{cond}



\[L_{\text{loc}}^1=\text{absolutely continuous measures}\subset\text{Radon measures}\subset\cD'.\]


\begin{thm}
Every finite Radon measure is regular.
\end{thm}







\section{Riesz-Markov-Kakutani representation theorem}
In this section, we always assume $X$ is a locally compact Hausdorff space.
Hence we can use the Urysohn lemma in the following way: If a compact subset $K$ and a closed subset $F$ are disjoint, then by applying the Urysohn lemma on a compact neighborhood of $K$, we can find a continuous function $f:X\to[0,1]$ such that $f|_K=1$ and $f|_F=0$.
In particular, there always exists a ``continuous characteristic function'' $\phi\in C_c(X)$ with $\phi|_K=1$.

There are two Riesz-Markov-Kakutani theorems: the first theorem describes the positive elements in $C_c(X)^*$ as Radon measures when the natural colimit topology is assumed, and the second theorem describes $C_c(X)^*$ as the space of finite Radon measures when uniform topology is assumed.

\subsection{Positive linear functional}
Positivity of linear functional itself implies a rather strong continuity property.
\begin{thm}
Let $X$ be LCH.
A positive linear functional on $C_c(X)$ is continuous with respect to the inductive topology.
\end{thm}
\begin{pf}
Let $I$ be a positive linear functional on $C_c(X)$.
We want to show every restriction of $I$ on $C_c(U)\emb C_c(X)$ for $U\Subset X$ is continuous for uniform norm.

Choose a nonnegative $\phi\in C_c(X)$ such that $\phi|_{\cl{U}}\equiv1$ using the Urysohn lemma.
Then,
\[|I(f)|\le I(|f|)=I(\phi|f|)\le I(\phi\|f\|)=I(\phi)\|f\|\les_U\|f\|\]
for $f\in C_c(U)$.
\end{pf}
% 우리손 렘마에 의한 캐릭터리스틱 함수 구성 가능하다










\begin{thm}[The Riesz-Markov-Kakutani representation theorem]
Let $X$ be LCH.
Let $I$ be a positive linear functional on $C_c(X)$.
Then, there is a unique Radon measure $\mu$ on $X$ such that
\[I(f)=\int f\,d\mu.\]
\end{thm}
\begin{pf}
Uniqueness follows from the fact that Radon measures are outer regular on all Borel sets so that they are determined by open sets.
% 정칙성은 유일성을 위해 필요한 것!!

Define a set function $\mu$ on Borel $\sigma$-algebra by
\[\mu(U):=\|I|_{C_c(U)}\|_{C_c^*(U)}\]
for open $U$ and
\[\mu(E):=\inf\{\,\mu(U):E\subset U,\,U\text{ is open}\,\}\]
for Borel sets $E$.
Since outer regularity is satisfied automatically by definition, we need to show countable additivity and inner regularity on open sets for justification of the measure $\mu$.

\Step{1}[Countable additivity]
We use the Carath\'eodory extension theorem.

\Step{4}[Realization as integration]

\end{pf}


A regular measure of an open set $U$ can be realized by the norm of integration functional on $C_c(U)$.
\begin{lem}
Let $\mu$ be a Borel measure on a LCH $X$.
If $\mu$ is inner regular on open sets, then
\[\mu(U)=\|\mu\|_{C_c^*(U)}\]
for every open $U$ in $X$.
\end{lem}
\begin{pf}
($\ge$)
For $f\in C_c(U)$, we have
\[|\int f\,d\mu|=|\int_Uf\,d\mu|\le\mu(U)\,\|f\|.\]

($\le$)
Since $\mu$ is inner regular on $U$, there is a compact set $K\subset U$ such that $\mu(U\setminus K)<\e$.
We can find a nonnegative function $f\in C_c(U)$ with $f|_K \equiv 1$ and $f\le1$ by the construction of Uryshon.
Then, for all $\e>0$ we have
\[\mu(U)<\mu(K)+\e\le\int f\,d\mu+\e\le\|\mu\|_{C_c^*(U)}+\e.\qedhere\]
\end{pf}


Topologies on $C_c(X)$ for LCH $X$: weaker to stronger
\begin{cond}
\item Topology of compact convergence: $\cl{C_c(X)}=C(X)$.
\item Topology of uniform convergence: $\cl{C_c(X)}=C_0(X)$, $C_c(X)^*=M(X)$.
\item Inductive topology: $\cl{C_c(X)}=C_c(X)$.
\end{cond}


\end{document}